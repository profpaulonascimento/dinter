\section*{Lista 02}


\begin{exercise}
Determinar se está registrado em alguma biografia ou em algum relato de Galileo uma frase equivalente à citação acima.
\end{exercise}


\begin{exercise}
Leia o artigo de \href{https://www.pnas.org/content/pnas/117/42/25963.full.pdf}{P. Buhlmann} (mencionado acima) e a resenha de \href{https://www.ams.org/journals/notices/201907/rnoti-p1093.pdf}{L. Goldberg} sobre o livro de Pearl-Mackenzie e faça sua resenha de 20 linhas sobre o tema. 
\end{exercise}

{\color{red}
%
\section*{Rumo à causalidade e melhoria da validade externa}


    ``\textit{Felix, qui potuit rerum cognoscere causas}'', do poeta latino Virgílio (1), traduzido literalmente como ``Afortunado, que era capaz de conhecer as causas das coisas'', sugere a importância da causalidade desde muito tempo atrás. Em PNAS, Bates et al. (2) começa sua contribuição com a frase ``O objetivo final dos estudos de associação do genoma (GWAS) é identificar regiões do genoma contendo variantes que afetam causalmente um fenótipo de interesse'' e fornecem uma metodologia estatística original e altamente inovadora para fornecer respostas sólidas a este objetivo. Como argumentaremos, o problema de inferência causal é ambicioso e deve-se confiar em suposições. Os pressupostos na ref. 2 são fáceis de comunicar; a capacidade de comunicar suposições subjacentes torna sua abordagem transparente e, em nossa própria avaliação, suas suposições são muito plausíveis.

    Quando observamos correlação ou dependência entre algumas variáveis de interesse, uma questão central é sobre a direcionalidade: se uma variável é a causa ou o efeito de outra. Claro, pode acontecer que nenhum dos dois seja verdade, por causa de confusão oculta. Veja a Fig. 1 para uma visão esquemática onde todas as variáveis observadas estão exibindo dependência de associação entre si, mas estas são, em parte, decorrentes de fatores ocultos invisíveis. Se pudéssemos obter conhecimento da direcionalidade causal, obviamente, isso levaria a muitas melhorias na compreensão e interpretabilidade de um sistema subjacente. Na Fig. 1, isso significa inferir as relações causais direcionadas entre as variáveis observadas.


    Medidas de associação sozinhas, como correlação ou regressão (multivariada potencialmente não linear), com base nos chamados dados observacionais (dados do ``estado estacionário''), não podem fornecer respostas para a direcionalidade e, portanto, para a causalidade em geral; são necessárias suposições ou dados adicionais de outras configurações de projeto experimental. Um ensaio de controle randomizado (RCT) é um poderoso padrão ouro para inferir causalidade, graças ao seu desenho experimental muito especial (cf. ref. 3 e também Dados de perturbação como entrada). No entanto, infelizmente, esse método padrão-ouro costuma ser inviável ou antiético. Na ausência de RCTs, outra metodologia deve ser usada, sempre dependendo crucialmente de algumas suposições. Bates et al. (2) fornece uma abordagem altamente interessante com suposições plausíveis para inferência causal no campo particular de GWAS; ver abaixo. Antes de discutir isso, elaboramos brevemente de forma mais geral o propósito da causalidade.

\noindent
\begin{minipage}[!ht]{\columnwidth}
\epsfig{figure=figs/fig01bullman.png,width=\columnwidth}
\captionof{figure}{\scriptsize Sistema observado e verdadeiro em duas configurações diferentes (configuração A e B e configuração C e D). Variável de resposta Y (fenótipo) e covariáveis \(X_j (j = 1,2)\) (por exemplo, SNPs). (A e C) Variáveis observadas \(X_1, X_2, Y\) em azul. Uma borda não direcionada representa a associação entre as variáveis correspondentes, por exemplo, em termos de correlação ou de dependência de regressão (não linear) (correlação parcial) dadas todas as outras variáveis observadas. (B e D) Sistemas subjacentes verdadeiros, com variáveis observadas em azul e variável latente H oculta em vermelho. Uma aresta direcionada representa uma relação causal direta entre as variáveis correspondentes, com a cauda sendo a causa e a cabeça sendo o efeito (ou seja, a variável que é diretamente influenciada pela variável causadora). (A e B) Configuração onde todas as setas entre Xj a Y em B devem apontar para Y, como em (a maioria) GWAS. (C e D) A direção da seta em D entre \(X_j\) e Y pode ir para qualquer lado, como em situações gerais. Os verdadeiros sistemas subjacentes em B e D geram a dependência de associação em A e C, em termos de correlação ou dependência de regressão (não linear). A observação de tais associações leva a descobertas espúrias, ou seja, falsos positivos com relação à causalidade.}
\label{fig:01}
\end{minipage}

\section*{Escopo Principal de Causalidade}

     Além de ter melhorado a compreensão de um mecanismo, graças ao conhecimento causal, destacamos dois objetivos principais (adicionais) da inferência causal. Eles são frequentemente menos ambiciosos e mais realistas do que inferir toda a rede ou gráfico com pesos de aresta funcionais correspondentes, como na Fig. 1.

\section*{Prevendo intervenções específicas: efeito do tratamento}

     Um objetivo clássico de causalidade é a previsão de uma intervenção ou manipulação que não foi observada antes. A causalidade dá respostas quantitativas a perguntas como: O que aconteceria se tratássemos um paciente com um determinado medicamento (e a intervenção do tratamento ainda não tivesse sido feita)? O que aconteceria se eliminássemos um determinado gene (e a intervenção genética ainda não tenha sido realizada)? Assim, a causalidade dá uma resposta a uma pergunta do tipo ``e se eu fizer'' (4, 5). Em muitas aplicações, é altamente desejável ter previsões precisas para essas questões.

\section*{Robustez contra Perturbações Inespecíficas: Validade Externa}

    O problema abordado na ref. 2 talvez não esteja tão diretamente relacionado a intervenções específicas, uma vez que lida com polimorfismos de nucleotídeo único (SNPs) em GWAS, onde intervenções em SNPs não podem ser feitas. Como um experimento de pensamento, entretanto, ainda se pode pensar sobre o que aconteceria com o estado de uma doença se um certo SNP interviesse. Nossa mensagem é que, mesmo na ausência da possibilidade de fazer intervenções diretas, a inferência causal é altamente interessante (além da questão da interpretação mencionada acima). A principal razão é que a estrutura causal leva a certas invariâncias e robustez, como explicamos brevemente a seguir.

    A maioria dos estudos científicos afirmam que as descobertas e resultados se generalizam para outros indivíduos ou populações e objetivam a validade externa. Em outras palavras, o objetivo é a replicabilidade das descobertas: queremos inferir resultados estáveis em diferentes subpopulações, em que cada uma delas pode ser uma versão perturbada de uma referência. Curiosamente, essa estabilidade em diferentes subpopulações ou diferentes perturbações tem uma relação muito intrínseca com a causalidade: a regressão nas variáveis causais, a solução causal, exibe (alguma) robustez ou estabilidade contra perturbações decorrentes de diferentes subpopulações (6-8) e, portanto, uma solução causal com sua robustez leva a uma melhor replicabilidade e melhor validade externa (em novos estudos, para novos pacientes, etc.). Em nossa opinião, esta é uma grande vantagem da abordagem e das conclusões da ref. 2: A metodologia deles, por visar relações causais, melhora a validade externa!

\section*{Métodos de inferência causal}

    Inferir causalidade a partir dos dados é uma tarefa ambiciosa e depende crucialmente do planejamento de experimentos ou de suposições adicionais, muitas vezes não testáveis.

\section*{Dados de perturbação como entrada}

    Aprender a estrutura causal e os efeitos é mais fácil com o acesso a dados de diferentes perturbações do sistema de interesse. Como já mencionado, o padrão ouro é uma perturbação na forma de um RCT. Lá, o experimentador tem a capacidade de fazer uma intervenção em uma variável (sendo um candidato a ser causal) ou de atribuir um tratamento: A randomização quebra todas as dependências entre a variável interveniente e qualquer possível confusão oculta. A conclusão poderosa é que, após a randomização, se sobrar um efeito entre a variável intervencionada ou de tratamento e uma resposta de interesse, deve ser um efeito causal (total). Um RCT leva à estabilidade e validade externa dos efeitos (regressão ou comparação de grupo) para uma grande classe de perturbações. Este é exatamente o objetivo, digamos, do desenvolvimento de uma farmacoterapia robusta: o medicamento ou os efeitos do tratamento ativo devem ser ``sempre'' externamente válidos. Se um RCT for inviável, os dados de perturbação de intervenções específicas (não randomizadas) ou de mudanças inespecíficas de ambiente ainda são muito mais informativos do que ter apenas acesso a dados observacionais. A informação dos dados de perturbação leva a invariâncias e estabilidade de efeitos (regressão) que são induzidos pelos diferentes ambientes, mas onde não se tem realmente controle sobre a ``natureza'' das perturbações que são inofensivas ou prejudiciais para inferir efeitos (regressão). Porém, grosso modo, ao observar mais perturbações, pode-se identificar mais invariância, estabilidade e robustez e, eventualmente, a estrutura causal e os efeitos (8). Assim, o cenário mais desafiador para inferir efeitos causais acontece quando apenas dados observacionais do ``estado estacionário'' estão disponíveis.


\section*{A Abordagem de Bates et al. (2) usando apenas dados observacionais}

    O método na ref. (2) usa apenas dados observacionais como entrada. No entanto, duas premissas principais são exploradas. Primeiro, a direcionalidade é postulada naturalmente apontando de SNPs genéticos para o fenótipo; ou seja, se houver associação de regressão infundada entre um fenótipo \(Y\) e uma variável SNP \(X_j\), ela deve ser direcionada \(X_j \to Y\). Esta é a situação na Fig. 1 A e B. A mesma direcionalidade é assumida de haplótipos parentais para SNPs descendentes . Em segundo lugar, para inferir a associação de regressão não-fundada, isto é, a força da regressão que resta após o ajuste para confusão oculta em potencial, um assim chamado estudo de design de trio especial leva, de maneira elegante, a tais efeitos de regressão não-fundados. A suposição é que o mecanismo estocástico de SNPs condicional aos haplótipos parentais, ou seja, a distribuição condicional correspondente, é independente de outros potenciais confundidores ocultos, e isso, por sua vez, permite a conclusão de que uma associação de regressão (potencialmente não linear) entre um SNP e um fenótipo, dados todos os outros SNPs e os haplótipos parentais, devem implicar uma dependência causal. Isso é uma analogia exata com um RCT: o condicionamento dos haplótipos serve como um substituto para a randomização! Bates et al. (2) referem-se a isso como ``variação na herança como um experimento aleatório''. Ambas as suposições podem ser comunicadas com clareza e são muito plausíveis, o que torna as alegadas descobertas causais muito convincentes. Claro, ainda pode haver violações de suposições, e os autores mencionam SNPs não medidos ou viés de seleção, para citar dois exemplos proeminentes. No entanto, em geral, a metodologia na ref. (2) é um grande passo em frente para chegar mais perto da ``verdadeira causalidade subjacente''.

    Além da maneira como a metodologia lida com suposições fundamentais para causalidade, ela fornece garantias estatísticas de amostra finita sobre a descoberta falsa ou a taxa de erro familiar. A principal suposição aqui é que o modelo de Haldane (9) é considerado ``verdadeiro'' (ou seja, uma aproximação muito boa), e as técnicas de inferência construídas em belos trabalhos anteriores de simulação de falsos recursos sintéticos que servem para contar falsos positivos (10, 11).

    Particularmente fascinante é a possibilidade de incluir dados GWAS externos (projeto não tri) para melhorar a energia; estudos de desenho de trio são raros e de tamanho de amostra muito menor do que estudos GWAS padrão, que podem vir em grande escala. Conforme ilustrado na ref. (2), pode-se usar qualquer algoritmo de aprendizado de máquina em dados GWAS externos para melhorar potencialmente a potência, enquanto a garantia de amostra finita na detecção de falso positivo ainda é válida.


\section*{Pensamentos Adicionais}

    Bates et al. (2) demonstra bem o uso de dados externos para aumentar potencialmente o poder de detecção de SNPs causais em estudos de projeto de trio. Invertendo o papel de usar dados externos, pode-se, e talvez deva, também usar parte deles para validar os resultados (e não usá-los na fase de descoberta); veja também ref. (12). Conforme mencionado em Robustness against Inspecific Perturbations: External Validity, se a estrutura inferida for causal, ela deve exibir alguma validade externa em novos dados, idealmente, em alguns conjuntos de dados de diferentes ambientes ou subpopulações. Como proposta, pode-se inspecionar a estabilidade da distribuição condicional do fenótipo, dados os SNPs causais encontrados, por exemplo, testando a independência condicional do fenótipo e dos ambientes dados os fenótipos causais (13, 14). Em particular, isso poderia ser feito com conjuntos de dados externos GWAS de design não triplo padrão que estão disponíveis em várias plataformas.

    Na ausência de estudos de design de trio e na ausência de direcionalidade postulada (como em GWAS de SNPs para fenótipos), o problema de inferência causal é muito mais difícil. As Fig. 1 C e D indicam esta configuração, que inclui, por exemplo, transcriptômica ou proteômica em biologia, onde postular a direcionalidade é frequentemente difícil ou sujeito a erros. Os dados de perturbação desempenharão um papel crucial para fazer um progresso confiável no sentido de inferir estruturas causais e efeitos. Mesmo quando não é possível ter experimentos randomizados, as perturbações não randomizadas ajudam enormemente. Para campos como biologia molecular e muitos outros, priorizar bons candidatos com respeito a ser causal é muito valioso, mesmo quando declarações de confiança estatística estritas parecem fora do escopo (15). Claramente, tal priorização causal deve ser realizada por métodos de inferência causal, em vez de técnicas de associação pura, onde as últimas variam de correlação simples a regressão não linear avançada ou aprendizado de máquina de classificação.
    
\section*{Agradecimentos}

     A pesquisa foi apoiada pelo Conselho Europeu de Pesquisa sob o Acordo de Subvenção 786461 (CausalStats - ERC-2017-ADG).

\section*{Referências}

1 Virgil, Georgica (vers 490, Book II, 29 BC).

2 S. Bates, M. Sesia, C. Sabatti, E. Candès, Causal inference in genetic trio studies. Proc. Nat. Acad. Sci. U.S.A. 117, 24117–24126 (2020).

3 G. Imbens, D. Rubin, Causal Inference for Statistics, Social, and Biomedical Sciences (Cambridge University Press, 2015).

4 J. Pearl, Causality: Models, Reasoning and Inference (Cambridge University Press, ed. 2, 2009).

5 J. Pearl, D. Mackenzie, The Book of Why: The New Science of Cause and Effect (Basic, 2018).

6 T. Haavelmo, The statistical implications of a system of simultaneous equations. Econometrica 11, 1–12 (1943).

7 A. P. Dawid, V. Didelez, Identifying the consequences of dynamic treatment strategies: A decision-theoretic overview. Stat. Surv. 4, 184–231 (2010).

8 J. Peters, P. Bühlmann, N. Meinshausen, Causal inference using invariant prediction: Identification and confidence interval (with discussion). J. R. Stat. Soc. Ser. B
Stat. Methodol. 78, 947–1012 (2016).

9 J. B. S. Haldane, The combination of linkage values and the calculation of distances between the loci of linked factors. J. Genet. 8, 299–309 (1919).

10 R. F. Barber, E. Candès, Controlling the false discovery rate via knockoffs. Ann. Stat. 43, 2055–2085 (2015).

11 E. Candès, Y. Fan, L. Janson, J. Lv, Panning for gold: Model-X knockoffs for high dimensional controlled variable selection. J. R. Stat. Soc. Ser. B Stat. Methodol.
80, 551–577 (2018).

12 B. Yu, K. Kumbier, Veridical data science. Proc. Natl. Acad. Sci. U.S.A. 117, 3920–3929 (2020).

13 R. Shah, J. Peters, The hardness of conditional independence testing and the generalised covariance measure. Ann. Stat. 48, 1514–1538 (2020).

14 M. Azadkia, S. Chatterjee, A simple measure of conditional dependence. arXiv:1910.12327 (27 October 2019).

15 N. Meinshausen et al., Methods for causal inference from gene perturbation experiments and validation. Proc. Nat. Acad. Sci. U.S.A. 113, 7361–7368 (2016).

%\section*{O livro do porquê: uma crítica de Lisa R. Goldberg}



\noindent
\begin{minipage}[!ht]{0.25\columnwidth}
\epsfig{figure=figs/fig00goldberg.png,width=0.9\columnwidth}
%\captionof{figure}{\scriptsize }
%\label{fig:00}
\end{minipage}
\begin{minipage}[!ht]{0.75\columnwidth}\scriptsize
The Book of Why \\
The New Science of Cause and Effect \\
Judea Pearl and Dana Mackenzie \\
Basic Books, 2018 \\
432 pages \\
ISBN-13: 978-0465097609
\end{minipage}


Judea Pearl tem a missão de mudar a maneira como interpretamos os dados. Um eminente professor de ciência da computação, Pearl documentou suas pesquisas e opiniões em livros e artigos acadêmicos. Agora, ele tornou suas ideias acessíveis a um amplo público em O Livro do Porquê: A Nova Ciência de Causa e Efeito, em coautoria com a escritora científica Dana Mackenzie. Com o lançamento deste livro historicamente fundamentado e instigante, Pearl salta da torre de marfim para o mundo real.

    O Livro do Porquê visa as limitações percebidas dos estudos observacionais, cujos dados subjacentes são encontrados na natureza e não controlados por pesquisadores. Muitos acreditam que um estudo observacional pode elucidar a associação, mas não causa e efeito. Não pode te dizer por quê.

    Talvez o exemplo mais famoso diga respeito ao impacto do tabagismo na saúde. Em meados da década de 1950, os pesquisadores estabeleceram uma forte associação entre tabagismo e câncer de pulmão. Somente em 1984, no entanto, o governo dos Estados Unidos determinou a frase "fumar causa câncer de pulmão".

    O atraso era o espectro de um fator latente, talvez algo genético, que poderia causar câncer de pulmão e desejo por tabaco. Se o fator latente fosse responsável pelo câncer de pulmão, limitar o tabagismo não impediria a doença. Naturalmente, as empresas de tabaco gostavam dessa explicação, mas também foi defendida pelo proeminente estatístico Ronald A. Fisher, co-inventor do chamado padrão ouro de experimentação, o Randomized Controlled Trial (RCT).

    Os participantes de um ECR sobre tabagismo e câncer de pulmão teriam sido designados a fumar ou não no cara ou coroa. O estudo tinha o potencial de desqualificar um fator latente como a principal causa de câncer de pulmão e elevar os cigarros à posição de principal suspeito. Uma vez que um ECR de tabagismo seria antiético, no entanto, os pesquisadores se contentaram com estudos observacionais que mostravam associação e objetaram sobre a questão de causa e efeito por décadas.

    O problema era simplesmente que as ferramentas disponíveis nas décadas de 1950 e 1960 eram muito limitadas em escopo? Pearl aborda essa questão em sua escada de causalidade de três etapas, que organiza métodos inferenciais em termos dos problemas que podem resolver. O degrau inferior é para métodos estatísticos sem modelo que dependem estritamente de associação ou correlação. O degrau do meio é para intervenções que permitem a medição de causa e efeito. O degrau mais alto é para a análise contrafactual, a exploração de realidades alternativas.

    As primeiras investigações científicas sobre a relação entre tabagismo e câncer de pulmão baseavam-se em métodos estatísticos sem modelos, degraus inferiores, cujos análogos modernos dominam a análise dos estudos observacionais hoje. Em uma das muitas anedotas históricas maravilhosas do The Book of Why, a predominância desses métodos é atribuída ao trabalho de Francis Galton, que descobriu o princípio da regressão à média em uma tentativa de compreender o processo que impulsiona a hereditariedade das características humanas. A regressão à média envolve associação, e isso levou Galton e seu discípulo, Karl Pearson, a concluir que a associação era mais central para a ciência do que a causalidade.

    Pearl coloca o aprendizado profundo e outras ferramentas modernas de mineração de dados no degrau inferior da Escada da Causalidade. Os métodos inferiores incluem AlphaGo, o programa de aprendizado profundo que derrotou os melhores jogadores humanos de Go do mundo em 2015 e 2016 [1]. Para o benefício daqueles que se lembram dos tempos antigos antes da mineração de dados mudar tudo, ele explica,

\begin{quotation}
    Os sucessos do aprendizado profundo foram realmente notáveis e pegaram muitos de nós de surpresa. No entanto, o aprendizado profundo foi bem-sucedido principalmente ao mostrar que certas questões ou tarefas que pensávamos serem difíceis, na verdade não o são.
\end{quotation}

    A questão é que algoritmos, ao contrário de crianças de três anos, fazem o que mandam, mas para criar um algoritmo capaz de raciocínio causal, ... temos que ensinar o computador como quebrar seletivamente as regras da lógica. Os computadores não são bons em quebrar regras, uma habilidade na qual as crianças se destacam.



\noindent
\begin{minipage}[!ht]{\columnwidth}\centering
\epsfig{figure=figs/fig01goldberg.png,width=0.9\columnwidth}
\captionof{figure}{\scriptsize Modelo causal de relações presumidas entre fumo, câncer de pulmão e um gene do fumo.}
\label{fig:01}
\end{minipage}



    Métodos para extrair conclusões causais de estudos observacionais estão no degrau intermediário da Escada de Causalidade de Pearl e podem ser expressos em uma linguagem matemática que estende a estatística clássica e enfatiza os modelos gráficos.

\begin{quotation}
    Existem várias opções para modelos causais: diagramas causais, equações estruturais, afirmações lógicas e assim por diante. Estou fortemente convencido de diagramas causais para quase todas as aplicações, principalmente devido à sua transparência, mas também devido às respostas explícitas que fornecem a muitas das perguntas que desejamos fazer.
\end{quotation}

    O uso de modelos gráficos para determinar causa e efeito em estudos observacionais foi iniciado por Sewall Wright, cujo trabalho sobre os efeitos do peso ao nascer, tamanho da ninhada, duração do período de gestação e outras variáveis sobre o peso de uma cobaia de porco de 33 dias de idade está em [2]. Pearl relata a persistência de Wright em resposta à recepção fria que seu trabalho recebeu da comunidade científica.

\begin{quotation}
    Minha admiração pela precisão de Wright só perde para a minha admiração por sua coragem e determinação. Imagine a situação em 1921. Um matemático autodidata enfrenta sozinho a hegemonia do sistema estatístico. Eles dizem a ele ``Seu método é baseado em uma compreensão totalmente equivocada da natureza da causalidade no sentido científico.'' E ele retruca: ``Não é assim! Meu método é importante e vai além de qualquer coisa que você possa gerar.''
\end{quotation}

    Pearl define um \textit{modelo causal} como um gráfico acíclico direcionado que pode ser emparelhado com dados para produzir estimativas causais quantitativas. O gráfico incorpora as relações estruturais que um pesquisador assume que estão gerando resultados empíricos. A estrutura do modelo gráfico, incluindo a identificação de vértices como mediadores, confundidores ou aceleradores, pode guiar o projeto experimental por meio da identificação de conjuntos mínimos de variáveis de controle. As exposições modernas sobre modelos gráficos de causa e efeito são [3] e [4].

\noindent
\begin{minipage}[!ht]{\columnwidth}\centering
\epsfig{figure=figs/fig02goldberg.png,width=0.9\columnwidth}
\captionof{figure}{\scriptsize Modelo causal mutado que facilita o cálculo do efeito do tabagismo no câncer de pulmão. A seta do gene que confunde o fumo para o ato de fumar foi deletada.}
\label{fig:02}
\end{minipage}


    Dentro dessa estrutura, Pearl define o operador \(\operatorname{do}\), que isola o impacto de uma única variável de outros efeitos. A probabilidade de \(Y \operatorname{do} X\), \(P[Y|\operatorname{do}(X)]\), não é a mesma coisa que a probabilidade condicional de \(Y\) dada \(X\). Em vez disso, \(P[Y|\operatorname{do}(X)] \) é estimado em um modelo causal mutado, do qual as setas apontando para a causa assumida são removidas. \textit{Confundir} é a diferença entre \(P[Y|\operatorname{do}(X)]\) e \(P[Y|X]\). Na década de 1950, os pesquisadores estavam atrás do primeiro, mas só podiam estimar o último em estudos observacionais. Esse foi o ponto de Ronald A. Fisher.

    A Figura 1 mostra uma relação simplificada entre tabagismo e câncer de pulmão. As bordas direcionadas representam relações causais assumidas, e o gene do fumo é representado por um círculo vazio, indicando que a variável não era observável quando a conexão entre fumo e câncer estava em questão. Círculos preenchidos representam quantidades que podem ser medidas, como taxas de tabagismo e câncer de pulmão em uma população. A Figura 2 mostra o modelo causal mutado que isola o impacto do tabagismo no câncer de pulmão.

    A conclusão de que fumar causa câncer de pulmão foi finalmente alcançada sem recorrer a um modelo causal. Uma quantidade enorme de evidências, incluindo a poderosa análise de sensibilidade desenvolvida em [5], acabou influenciando a opinião. Pearl argumenta que seus métodos, se estivessem disponíveis, poderiam ter resolvido o problema mais cedo. Pearl ilustra seu ponto em um cenário hipotético em que fumar causa câncer apenas por depositar alcatrão nos pulmões. O diagrama causal correspondente é mostrado na Figura 3. Sua \textit{fórmula da porta da frente} corrige a confusão do gene do fumo não observável, sem nunca mencioná-lo. O impacto corrigido do preconceito do tabagismo, \(X\) no câncer de pulmão, \(Y\) pode ser expresso
    \[P[Y|\operatorname{do}(X)] = \sum_ {Z} P[Z|X] \sum_ {X'} P[Y|X',Z] P[X'].\]


\noindent
\begin{minipage}[!ht]{\columnwidth}\centering
\epsfig{figure=figs/fig03goldberg.png,width=0.9\columnwidth}
\captionof{figure}{\scriptsize A fórmula da porta da frente de Pearl corrige o viés devido a variáveis latentes em certos exemplos.}
\label{fig:03}
\end{minipage}

    \textit{O Livro do Porquê} extrai um corpo substancial da literatura acadêmica, que explorei a fim de obter um quadro mais completo do trabalho de Pearl. De uma perspectiva matemática, uma aplicação importante é o estudo de 2007 de Nicholas Christakis e James Fowler descrito em [6] argumentando que a obesidade é contagiosa. A alegação que chamou a atenção foi controversa porque o mecanismo de contágio social é difícil de definir e porque o estudo foi observacional. Em seu artigo, Christakis e Fowler atualizaram uma associação observada, grupos de indivíduos obesos em uma rede social, para a afirmação de que indivíduos obesos fazem com que seus amigos e amigos de seus amigos se tornem obesos. É difícil compreender a complexa rede de suposições, argumentos e dados que compõem este estudo. Também é difícil compreender suas refutações matizadas por Russell Lyons [7] e por Cosma Shalizi e Andrew Thomas [8], que surgiram em 2011. Há um momento de clareza, no entanto, no comentário de Shalizi e Thomas, quando eles citam o teorema de Pearl sobre a não identificabilidade em modelos gráficos particulares. Usando os resultados de Pearl, Shalizi e Thomas mostram que na rede social que Christakis e Fowler estudaram, é impossível separar o contágio, a propagação da obesidade por meio da amizade, das inclinações compartilhadas que levaram a amizade a se formar em primeiro lugar.

    O degrau mais alto da Escada da Causação diz respeito aos contrafatuais, que Michael Lewis chamou a atenção do mundo com seu livro mais vendido, The Undoing Project [9]. Lewis conta a história dos psicólogos israelenses Daniel Kahneman e Amos Tversky, especialistas em erro humano, que mudaram fundamentalmente nossa compreensão de como tomamos decisões. Pearl se baseia no trabalho de Kahneman e Tversky em The Book of Why, e a abordagem de Pearl para analisar contrafactuais pode ser melhor explicada em termos de uma questão que Kahneman e Tversky colocaram em seu estudo [10] de como exploramos realidades alternativas.

\begin{quotation}
    Quão perto os cientistas de Hitler chegaram de desenvolver a bomba atômica na Segunda Guerra Mundial? Se eles o tivessem desenvolvido em fevereiro de 1945, o resultado da guerra teria sido diferente?
    
        —A Simulação Heurística
\end{quotation}

    A resposta de Pearl a esta pergunta inclui a \textit{probabilidade de necessidade} para a Alemanha e seus aliados terem ganho o Mundo II se tivessem desenvolvido a bomba atômica em 1945, dado nosso conhecimento histórico de que eles não tinham uma bomba atômica em fevereiro de 1945 e perderam a guerra. Se \(Y\) denota a Alemanha ganhando ou perdendo a guerra (0 ou 1) e \(X\) denota a Alemanha tendo a bomba em 1945 ou não a tendo (0 ou 1), a probabilidade de necessidade pode ser expressa no linguagem dos resultados potenciais,
     \[P[Y_{X=0} = 0|X = 1, Y = 1].\]
    
    Dual a \textit{probabilidade de suficiência}, a probabilidade de necessidade reflete a noção legal de causalidade ``\textit{but-for}'' como em: se não fosse por seu fracasso em construir uma bomba atômica em fevereiro de 1945, a Alemanha provavelmente teria vencido a guerra. Pearl aplica o mesmo tipo de raciocínio para gerar declarações transparentes sobre as mudanças climáticas. O aquecimento global antropogênico foi responsável pela onda de calor de 2003 na Europa? Todos nós já ouvimos que, embora o aquecimento global devido à atividade humana tenda a aumentar a probabilidade de ondas de calor extremas, não é possível atribuir nenhum evento específico a essa atividade. De acordo com Pearl e uma equipe de cientistas do clima, a resposta pode ser enquadrada de forma diferente: há 90\% de chance de que a onda de calor de 2003 na Europa não teria ocorrido na ausência do aquecimento global antropogênico [11].

    Essa formulação do impacto do aquecimento global antropogênico na Terra é forte e clara, mas está correta? O princípio do \textit{``lixo-dentro e do lixo-fora''} nos diz que os resultados baseados em um modelo causal não são melhores do que suas suposições subjacentes. Essas suposições podem representar o conhecimento e a experiência de um pesquisador. No entanto, muitos estudiosos estão preocupados com o fato de que os pressupostos do modelo representam o viés do pesquisador ou simplesmente não são examinados. David Freedman enfatiza isso em [12], e como ele escreveu mais recentemente em [13],

\begin{quotation}
        As suposições por trás dos modelos raramente são articuladas, muito menos defendidas. O problema é exacerbado porque os periódicos tendem a favorecer um grau moderado de novidade nos procedimentos estatísticos. Modelagem, a busca por significância, a preferência por novidades e a falta de interesse em suposições - essas normas provavelmente geram uma enxurrada de resultados não reproduzíveis.
        
        —Oasis ou Mirage?
\end{quotation}

    Os modelos causais podem ser usados para retroceder a partir das conclusões que preferimos para as suposições de suporte. Nossa tendência de raciocinar a serviço de nossas crenças anteriores é um tópico favorito do psicólogo moral Jonathan Haidt, autor de The Righteous Mind [14], que escreveu sobre ``o cão emocional e sua cauda racional''. Ou como Udny Yule explicou em [15],

\begin{quotation}
        Agora, suponho que seja possível, com um pouco de engenhosidade e boa vontade, racionalizar quase tudo.
        
        — Discurso presidencial de 1926 na Royal Statistical Society
\end{quotation}

    A preocupação com o impacto de preconceitos e preconceitos em estudos empíricos está crescendo, e vem de fontes tão diversas como o Professor de Medicina John Ioannides, que explicou por que a maioria das descobertas de pesquisas publicadas são falsas [16]; o comediante John Oliver, que nos alertou para sermos céticos ao ouvirmos a frase ``estudos mostram'' [17]; e o ex-escritor nova-iorquino Jonah Lehrer, que escreveu sobre os problemas com a ciência empírica em [18], mas mais tarde foi desacreditado por representar coisas que inventou como fatos.

    A abordagem gráfica da inferência causal que Pearl favorece tem sido influente, mas não é a única abordagem. Muitos pesquisadores contam com o modelo de resultados potenciais de Neyman (ou Neyman-Rubin), que é discutido em [19], [20], [21] e [22]. Na linguagem dos ensaios clínicos randomizados, um pesquisador que usa esse modelo tenta quantificar a diferença de impacto entre o tratamento e o não tratamento em indivíduos de um estudo observacional. Os escores de propensão são combinados em uma tentativa de equilibrar as desigualdades entre assuntos tratados e não tratados. Uma vez que nenhum assunto pode ser tratado e não tratado, no entanto, a estimativa necessária do impacto às vezes é formulada como um problema de valor ausente, uma perspectiva que Pearl contesta veementemente.

    Em outra direção, o conceito de fixação, desenvolvido por Heckman em [23] e Heckman e Pinto em [24], se assemelha, pelo menos superficialmente, ao operador do que Pearl utiliza. Aqueles que gostam de disputas acadêmicas podem olhar para o blog de Andrew Gelman, [25] e [26], para troca de ideias entre os discípulos de Pearl e Rubin (o próprio Rubin não parece participar - naquele fórum, pelo menos) ou para o tributos escritos por Pearl [27] e Heckman e Pinto [24] ao solitário Prêmio Nobel, Trygve Haavelmo, que foi o pioneiro da inferência causal em economia na década de 1940 em [28] e [29]. Esses diálogos têm sido controversos às vezes e trazem à mente a lei de Sayre, que diz que a política acadêmica é a forma mais cruel e amarga de política porque os riscos são muito baixos. É a opinião deste revisor que as diferenças entre essas abordagens para inferência causal são muito menos importantes do que suas semelhanças. Suporte para isso inclui construções por Pearl em [3] e por Thomas Richardson e James Robins em [30] incorporando contrafactuais em modelos gráficos de causa e efeito, unificando assim vários tópicos da literatura de inferência causal.

    
\noindent
\begin{minipage}[!ht]{\columnwidth}\centering
\epsfig{figure=figs/fig04goldberg.png,width=0.9\columnwidth}
\captionof{figure}{\scriptsize Inspetores do National Transportation Safety Board examinando o Uber sem motorista que matou um pedestre em Tempe, Arizona, em 18 de março de 2018.}
\label{fig:04}
\end{minipage}



    No final de uma tarde de julho de 2018, a coautora de Pearl, Dana Mackenzie, falou sobre inferência causal no Simons Institute da UC Berkeley. Sua apresentação foi na primeira pessoa do singular da perspectiva de Pearl, a mesma voz usada em The Book of Why, e concluiu com uma imagem do primeiro carro que dirige sozinho a matar um pedestre. De acordo com um relatório [31] do National Transportation Safety Board (NTSB), o carro reconheceu um objeto em seu caminho seis segundos antes da colisão fatal. Com um tempo de avanço de um segundo e meio, o carro identificou o objeto como um pedestre. Quando o carro tentou engatar o sistema de frenagem de emergência, nada aconteceu. O relatório do NTSB afirma que os engenheiros desativaram o sistema em resposta a uma preponderância de falsos positivos nos testes.

    Os engenheiros estavam certos, é claro, que paradas frequentes e abruptas tornam um carro que dirige sozinho inútil. Mackenzie gentil e otimista sugeriu que dotar o carro com um modelo causal que pode fazer julgamentos matizados sobre a intenção do pedestre pode ajudar. Se isso levasse a carros autônomos mais seguros e inteligentes, não seria a primeira vez que as ideias de Pearl levariam a uma tecnologia melhor. Seu trabalho fundamental em redes bayesianas foi incorporado à tecnologia de telefone celular, filtros de spam, biomonitoramento e muitas outras aplicações de importância prática.

    A professora Judea Pearl nos deu uma teoria da causalidade elegante, poderosa e controversa. Como ele pode dar a sua teoria a melhor chance de mudar a maneira como interpretamos os dados? Não existe uma receita para fazer isso, mas formar uma parceria com a escritora de ciências e professora Dana Mackenzie, um estudioso por direito próprio, foi uma ideia muito boa.


ACKNOWLEDGMENT.

    Esta revisão se beneficiou de diálogos com David Aldous, Bob Anderson, Wachi Bandera, Jeff Bohn, Brad DeLong, Michael Dempster, Peng Ding, Tingyue Gan, Nate Jensen, Barry Mazur, Liz Michaels, LaDene Otsuki, Caroline Ribet, Ken Ribet, Stephanie Ribet, Cosma Shalizi, Alex Shkolnik, Philip Stark, Lee Wilkinson e os participantes do grupo de almoço social do Departamento de Estatística de Berkeley da Universidade da Califórnia. Agradeço a Nick Jewell por me informar sobre os estudos científicos sobre a relação entre exercícios e colesterol, o que aumentou minha apreciação do Livro do Porquê.

References

    [1] Silver D, Simonyan JSK, Antonoglou I, Huang A, Guez A, Hubert T, Baker L, Lai M, Bolton A, Chen Y, Lillicrap T, Hui F, Sifre L, van den Driessche G, Graepel T, Hassabis D. Mastering the game of Go without human knowledge, Nature, vol. 550, pp. 354–359, 2017.

    [2] Wright S. Correlation and causation, Journal of Agricultural Research, vol. 20, no. 7, pp. 557–585, 1921.
    
    [3] Pearl J. Causality: Models, Reasoning, and Inference. Cambridge University Press, second ed., 2009. MR2548166
    
    [4] Spirtes P, Glymour C, Scheines R. Causation, Prediction and Search. The MIT Press, 2000. MR1815675
    
    [5] Cornfield J, Haenszel W, Hammond EC, Shimkin MB, Wynder EL. Smoking and lung cancer: recent evidence and a discussion of some questions, Journal of the National Cancer Institute, vol. 22, no. 1, pp. 173–203, 1959.
    
    [6] Christakis NA, Fowler JH. The spread of obesity in a large social network over 32 years, The New England Journal of Medicine, vol. 357, no. 4, pp. 370–379, 2007.
    
    [7] Lyons R. The spread of evidence-poor medicine via flawed social-network analysis, Statistics, Politics, and Policy, vol. 2, no. 1, pp. DOI: 10.2202/2151–7509.1024, 2011.
    
    [8] Shalizi CR, Thomas AC. Homophily and contagion are generically confounded in observational social network studies, Sociological Methods \& Research, vol. 40, no. 2, pp. 211–239, 2011. MR2767833

    [9] Lewis M. The Undoing Project: A Friendship That Changed Our Minds. W.W. Norton and Company, 2016.
    
    [10] Kahneman D, Tversky A. The simulation heuristic, in Judgment under Uncertainty: Heurisitics and Biases (D. Kahneman, P. Slovic, and A. Tversky, eds.), pp. 201–208, Cambridge University Press, 1982.
    
    [11] Hannart A, Pearl J, Otto F, Naveu P, Ghil M. Causal counterfactural theory for the attribution of weather and climate-related events, Bulletin of the American Meterological Society, vol. 97, pp. 99–110, 2016.
    
    [12] Freedman DA. Statistical models and shoe leather, Sociological Methodology, vol. 21, pp. 291–313, 1991.
    
    [13] Freedman DA. Oasis or mirage? Chance, vol. 21, no. 1, pp. 59–61, 2009. MR2422783
    
    [14] Haidt J. The Righteous Mind: Why Good People Are Divided by Politics and Religion. Vintage, 2013.
    
    [15] Yule U. Why do we sometimes get nonsense-correlations between time-series?–a study insampling and the nature of time-series, Royal Statistical Society, vol. 89, no. 1, 1926.
    
    [16] Ionnidis JPA. Why most published research findings are false, PLoS Med, vol. 2, no. 8, p. https://doi.org/10.1371/journal.pmed.0020124, 2005. MR2216666

    [17] Oliver J. Scientific studies: Last week tonight with John Oliver (HBO), May 2016.
    
    [18] Lehrer J. The truth wears off, The New Yorker, December 2010.
    
    [19] Neyman J. Sur les applications de la theorie des probabilities aux experiences agricoles: Essaies des principes., Statistical Science, vol. 5, pp. 463–472, 1923, 1990. 1923 manuscript translated by D.M. Dabrowska and T.P. Speed. MR1092985
    
    [20] Rubin DB. Estimating causal effects of treatments in randomized and non-randomized studies, Journal of Educational Psychology, vol. 66, no. 5, pp. 688–701, 1974.
    
    [21] Rubin DB. Causal inference using potential outcomes, Journal of the American Statistical Association, vol. 100, no. 469, pp. 322–331, 2005. MR2166071
    
    [22] Sekhon J. The Neyman-Rubin model of causal inference and estimation via matching methods, in The Oxford Handbook of Political Methodology (J. M. Box-Steffensmeier, H. E. Brady, and D. Collier, eds.), Oxford Handbooks Online, Oxford University Press, 2008.
    
    [23] Heckman J. The scientific model of causality, Sociological Methodology, vol. 35, pp. 1–97, 2005.
    
    [24] Heckman J, Pinto R. Causal analysis after Haavelmo, Econometric Theory, vol. 31, no. 1, pp. 115–151, 2015. MR3303188
    
    [25] Gelman A. Resolving disputes between J. Pearl and D. Rubin on causal inference, July 2009.
    
    [26] Gelman A. Judea Pearl overview on causal inference, and more general thoughts on the reexpression of existing methods by considering their implicit assumptions, 2014.
    
    [27] Pearl J. Trygve Haavelmo and the emergence of causal calculus, Econometric Theory, vol. 31, no. 1, pp. 152–179, 2015. MR3303189
    
    [28] Haavelmo T. The statistical implications of a system of simultaneous equations, Econometrica, vol. 11, no. 1, pp. 1– 12, 1943. MR0007954
    
    [29] Haavelmo T. The probability approach in econometrics, Econometrica, vol. 12, no. Supplement, pp. iii–iv+1–115, 1944. MR0010953
    
    [30] Richardson TS, Robins JM. Single world intervention graphs (SWIGS): A unification of the counterfactual and graphical approaches to causality, April 2013.
    
    [31] NTSB, Preliminary report released for crash involving pedestrian, uber technologies, inc., test vehicle, May 2018.



}


\begin{exercise}
Se, no início do primeiro mês, dispomos de um coelho imaturo, quantos coelhos teremos em 12 meses?
\end{exercise}

{\color{red} \textbf{Solução} (Paulo Henrique):
\[\begin{array}{c|c|c|c|c|c|c|c|c|c|c|c|c|c}
t & 0 & 1 & 2 & 3 & 4 & 5 & 6 & 7 & 8 & 9 & 10 & 11 & 12 \\ \hline
M & 0 & 1 & 1 & 2 & 3 & 5 & 8 & 13 & 21 & 34 & 55 & 89 & 144 \\ \hline
N & 1 & 0 & 1 & 1 & 2 & 3 & 5 & 8 & 13 & 21 & 34 & 55 & 89 \\ \hline
\mbox{Total} & 1 & 1 & 2 & 3 & 5 & 8 & 13 & 21 & 34 & 55 & 89 & 144 & 233
\end{array}\]
onde \(t\) é o tempo em meses, \(M\) representa a quantidade de coelhos maduros, \(N\), de coelhos novos. Claramente, a linha de totais é uma sequência de Fibonacci e é definida pela fórmula recursiva:
\[F(n+2)=F(n+1)+F(n),\]
com \(n \ge 1\)  e \(F(1) = F(2) = 1.\)
}


\begin{exercise}
Represente a população descrita ``biologicamente'' por Fibonacci, agora por intermédio de um grafo (``árvore'').
\end{exercise}



\begin{exercise}
Verifique as três afirmações acima, i.e., sobre a veracidade
\begin{enumerate}
\item da Recursão como representativa do Fenômeno;
\item da Fórmula para representar a População e;
\item do comportamento assintótico exponencial/geométrico da Fórmula.
\end{enumerate}
\end{exercise}

\url{http://www.ime.unicamp.br/~apmat/um-problema-de-fibonacci/#:~:text=Em%201202%2C%20um%20matem%C3%A1tico%20italiano,nascem%20no%20in%C3%ADcio%20do%20ano.&text=Depois%20de%20dois%20meses%20de,ocorrem%20mortes%20durante%20o%20ano.}

{\color{red}
\textbf{Solução}: 
1. (Paulo Henrique)

Utilizando a fórmula recursiva
\[N(k+2) = N(k+1) + N(k),\]
encontramos a tabela a seguir:
\[\begin{array}{|c|c|c|c|} \hline
k & N(k+1) & N(k) & N(k+2) \\ \hline
0 & 2 & 1 & 3 \\ \hline
1 & 3 & 2 & 5 \\ \hline
2 & 5 & 3 & 8 \\ \hline
3 & 8 & 5 & 13 \\ \hline
4 & 13 & 8 & 21 \\ \hline
5 & 21 & 13 & 34 \\ \hline
6 & 34 & 21 & 55 \\ \hline
7 & 55 & 34 & 89 \\ \hline
8 & 89 & 55 & 144 \\ \hline
9 & 144 & 89 & 233 \\ \hline
10 & 233 & 144 & 377 \\ \hline
11 & 377 & 233 & 610 \\ \hline
12 & 610 & 377 & 987 \\ \hline
\end{array}\]
que verifica o obtido no exercício anterior.

2. (Laís Buzo)

Verifiquemos que a função $N(k)$ pode ser representada pela Fórmula Elementar. De fato, tomemos para $n\geq 1$,
		$$N(n+1)=N(n-1)+N(n).$$
		Consideremos a sequência de Fibonacci $w(n)$ que seja uma sequência geométrica com $w(1)=1$ e com razão $\lambda \neq 0$, logo
		$$w(n)=\lambda^{n-1}.$$
		Temos então que,
		$$w(n+1)=w(n-1)+w(n)\rightarrow\lambda^{n}=\lambda^{n-2}+\lambda^{n-1}$$
		que pode ser reduzida a
		$$\lambda^2-\lambda-1=0$$
		cujas raízes são da forma
		$$\lambda_1=\frac{1}{2}(1+\sqrt{5})$$
		e
		$$\lambda_2=\frac{1}{2}(1-\sqrt{5}).$$
		Construindo $v(n)$ e $w(n)$ tal que $v(n)=\lambda_1^{n-1}$ e $w(n)=\lambda_2^{n-1}$. Como $N(n)$ é uma combinação linear de $v(n)$ e $w(n)$ temos que, para $c_1$ e $c_2$,
		$$N(n)=c_1\frac{1}{2}(1+\sqrt{5})^{(n-1)}+c_2\frac{1}{2}(1-\sqrt{5})^{(n-1)}.$$


3.

}

\begin{exercise}
Mostre que a função \(N(k)\) pode ser representada por intermédio de uma quinta forma, isto é, como solução de uma equação de diferenças finitas da forma \(P(E) \varphi = 0\) ou \(Q(\Delta) \varphi = 0\) onde \(P, Q\) são polinômios e \(E\) é um operador de deslocamento e \(\Delta\) um operador de diferença, ambos aplicáveis no conjunto de funções discretas \(\mathcal{F} = \{\varphi: \mathbb{N} \to \mathbb{R}\}\). (Consulte Bassanezi-Ferreira (1988) a respeito).
\end{exercise}

{\color{red}
\textbf{Solução}: (Laís Buzo)

Primeiramente, definimos o operador $E$ (deslocamento, ou \textit{shift}) por $Ef(n)=f(n+1)$, e o operador $\Delta$ (operador de diferença à frente) por $\Delta f(n)=f(n+1)-f(n)=(E-I)f(n)$.

		Considere o problema de combinatória onde utilizaremos dígitos binários. Seja $f(n)$ a quantidade de sinais não confluentes (não tem 1's adjacentes) onde $n$ é a quantidade de dígitos. Logo,
		$$f(n)=f(n-1)+f(n-2).$$
		As condições iniciais são: $f(0)=1$, $f(1)=2$ (pois temos $(0;1)$) e $f(2)=3$ (pois $((0,0);(0,1);(1,0))$. Portanto temos o problema de valor inicial:\\
		$\begin{cases}
			P(E)f=0\\
			If=(f(0),f(1)=(1,2))
		\end{cases}$
		
		$P(\lambda)=\lambda^2-\lambda-1=(\lambda-\lambda_1)(\lambda-\lambda_2)$ com $P(E)=E^2-E-I=(E-\lambda_1)(E-\lambda_2)$, logo $\lambda_{1,2}=\frac{1\pm \sqrt{5}}{2}$.
		Para calcular $N(P(E))$ devemos determinar $N(P(E-\lambda_1))$ e $N(P(E-\lambda_2))$ em $V(\mathbb{Z},\mathbb{C})$.
		$$(E-\lambda_1)\nu=0 \iff E\nu=\lambda \nu \iff \nu_{k+1}=\lambda_1\nu_k \iff \nu_k=C_1\lambda_1^k$$
		$$(E-\lambda_2)\omega=0 \iff \omega_k=C_2\lambda_2^k$$
		Portanto a solução geral é 
		$$f_k=C_1\frac{1}{2}(1-\sqrt{5})^{k}+C_2\frac{1}{2}(1+\sqrt{5})^{k}.$$
	(Ref.: BASSANEZZI, FERREIRA)

}


\begin{exercise}
Avalie o número de sacas de 20kg de trigo e o número de caminhões que seriam necessários para transportar esta quantidade de trigo. Compare esta quantidade com a produção mundial anual deste cereal. (Ref. Maharajan, Weinstein).
\end{exercise}

{
\def\atrigo{0.30}%cm
\def\btrigo{0.25}%cm
\def\ctrigo{0.24}%cm

\url{https://www.embrapa.br/documents/1355008/0/Folder+cultivares+Trigo/9d0598ad-d738-4113-bb73-1ea85c95a127}

\color{red}
\textbf{Solução}: (Paulo Henrique)

Considere que:

(a) grãos de trigo se aproximem do formato de um elipsoide de equação
\[\dfrac{x^2}{a^2} + \dfrac{y^2}{b^2} + \dfrac{z^2}{c^2} = 1\]
e que, em média, temos os valores de \(a = \atrigo\ cm\), \(b = \btrigo\ cm\) e \(c = \ctrigo\ cm\).

Portanto, o volume médio de um grão de trigo é dado por:
\[V = \dfrac{4}{3}\pi abc = \ca{4/3*3.14*\atrigo*\btrigo*\ctrigo*10^{-6}}\ m^3\]
\def\vtrigo{.000000075359999999} % volume (m^3) de um grão de trigo
\def\dtrigo{770} % densidade (kg/m^3) de um grão de trigo - 770 não é o valor correto!

(b) a densidade média de um grão de trigo é próxima da densidade de um grão de soja \(\varrho = 770\ kg/m^3\) \href{https://ainfo.cnptia.embrapa.br/digital/bitstream/item/161751/1/246.pdf}{(BENASSI, 2017. p. 247)}. % Aqui não encontrei a densidade de um grão de trigo!

Portanto, a massa de um grão de trigo é
\[M = \ca{\vtrigo*\dtrigo}\ kg\]
\def\mtrigo{0.00005802719999923}

(c) a quantidade \(N\) de grãos de trigo que deveria ser paga ao servo hindu é:
\[N = 1+2+4+\ldots+2^{63},\]
ou seja, a soma dos \(n = 64\) termos de uma progressão geométrica de razão \(q = 2\). Dessa forma:
\[
N
= a_1 \dfrac{q^n-1}{q-1}
= 2^{64}-1 \approx 1,84\ \cdot 10^{19}
%= \ca{2^{64}-1}
\]

\def\ntrigo{18446744073709600000}
(d) a quantidade \(S\) de sacos de \(20\ kg\) é:
\[S = \dfrac{M \cdot N}{20} = \ca{\mtrigo/20*\ntrigo}\]
\def\strigo{535206453849.878854091621804}

\def\kcaminhao{500}
(e) um caminhão consiga transportar \(\kcaminhao\ kg\) de mercadoria. Então, seria necessário:
\[C = S/500 = \ca{\strigo/\kcaminhao}\]
caminhões para transportar todo o trigo.
}




\begin{exercise}
Variações sobre o tema de Fibonacci.
\end{exercise}

{\color{red}
\textbf{Resposta}: (Paulo Henrique)

A sequência de Fibonacci (Leonardo de Pisa (1175-1240)) apresenta vários fatos curiosos:

(a) através do quociente de um número com o seu antecessor, obtém-se uma sequência cujos termos tendem a constante 1,6180339887... (número áureo);

(b) foi utilizada por Da Vinci, que chamou a sequência de Divina Proporção, para fazer desenhos perfeitos;

(c) A partir dessa sequência, pode ser construído um retângulo, que é chamado de Retângulo de Ouro e ao desenhar um arco dentro desse retângulo, obtemos, por sua vez, a Espiral de Fibonacci;

(d) Pode ser encontrada em alguns elementos da natureza como:
nas folhas das cabeças das alfaces;
na couve-flor,
nas camadas das cebolas;
nos padrões de saliências dos ananases;
nas sementes das pinhas (oito irradiando no sentido horário e 13 no anti-horário);
na concha de Nautilus;
na folha de uma Bromélia;
na cauda de um camaleão;
nas presas de marfim de um elefante se crescessem infinitamente;
nas sementes de um girassol (Neste caso são dois conjuntos de espirais, 21 no sentido horário e 34 no sentido anti-horário).


(e) Na espirradeira ou na cevadilha, mostra os números da sucessão de Fibonacci nos seus pontos de crescimento. Quando tem um novo rebento, leva dois meses a crescer até que as ramificações fiquem suficientemente fortes. Se a planta ramifica todos os meses, depois disso, no ponto de ramificação, obtemos ramificações que correspondem aos números de Fibonacci.

(f) no número de abelhas em cada geração da árvore genealógica de um zângão. Um zângão tem apenas um dos pais (pois provém de um ovo não fertilizado), ao passo que a fêmea exige ambos os pais (pois provem de um ovo fertilizado).
}


\begin{exercise}
Consideremos uma população de organismos que se reproduzem individualmente. Suponhamos, como Fibonacci, que o tempo é medido discretamente e registrado em períodos uniformes, de tal forma que os indivíduos desta população tornam-se reprodutíveis após uma unidade de tempo e que, após esta maturação, cada um deles, enquanto vivo, produz \(\nu\) novos indivíduos em cada unidade de tempo. Suponhamos também que a cada unidade de tempo uma fração \(\mu\) de indivíduos desta população perece. (Este modelo de reprodução e de mortalidade é dito proporcional já que são diretamente proporcionais ao tempo).

\begin{enumerate}
\item Mostre que se \(\varphi: \mathbb{N} \to \mathbb{C},\ \varphi(k)\) representa o número de indivíduos desta população no instante \(k\), então a população acima descrita lhe induz a seguinte restrição (Equação Recursiva de segunda Ordem) \(P(E) \varphi = 0\), onde \(P(z) = a_0 + a_1 z + a_2 z^2\) é uma polinômio algébrico de segundo grau, isto é, \(a_0f(k) + a_1f(k+1) + a_2 f(k+2) = 0\). (Sugestão: Segundo Fibonacci, a população no instante \((k+2)\) é constituída da população sobrevivente do ano passado, mais os novos integrantes nascidos no período anterior, que são filhos de quem já estava presente na população dois períodos atrás e não morreu.)
\item Mostre, portanto, que a recursão é de ordem \(2\) e completamente determinada quando são conhecidos \(F(0)\) e \(F(1)\).
\item Obtenha condições nos parâmetros \(\mu\) e \(\nu\) para que a população apresente um crescimento exponencial
\item Obtenha condições nos parâmetros \(\mu\) e \(\nu\) para que a população apresente uma extinção em decrescimento exponencial.
\item Obtenha condições nos parâmetros \(\mu\) e \(\nu\) para que a população apresente uma oscilação e crescimento exponencial
\item Mostre que descrevendo esta população pela função \(\Phi: \mathbb{N} \to \mathbb{C}^2\), \(\Phi_1(k) =\) ``População de Férteis no instante \(k\)'', \(\Phi_2(k) =\) ``População de Imaturos no instante \(k\)'' o argumento de Fibonacci pode ser representado na forma recursiva de primeira ordem:
\[\Phi(k+1) = A \Phi(k),\] onde \(A\) é uma matriz \(2 \times 2\).
\item \textbf{Método de Fourier - Transformação}: Obtenha uma matriz \(P\) invertível tal que \(P^{-1}A P = D\) é diagonal, \(D = \operatorname{diag}\{\lambda_1, \lambda_2\}\). Obtenha a função \(\Phi(k) = P \operatorname{diag}\{\lambda_1^k, \lambda_2^k\} P^{-1}\).
\item \textbf{Método de Fourier - Espectral}: Se \(A v_j = \lambda_j v_j\) (onde \(v_j\) é autovetor e \(\lambda_j\) é autovalor), mostre que \(\Phi^{(j)}(k) = (\lambda_j)^{k} v_j\) são soluções da recorrência vetorial. Mostre como obter a solução da equação com condição inicial \(\Phi(0)\) dada.
\item* \textbf{Método das Funções Geradoras}: Represente os valores da função discreta \(\varphi(k)\) como coeficientes da expansão de uma função analítica \(f\), isto é, \(f(z) = \displaystyle \sum_{k=0}^{\infty} \varphi(k) z^k\), obtenha uma equação funcional para \(f\) com base nas propriedades de \(\varphi\) e obtenha uma representação elementar para a função \(f(z)\). Com base nesta sintetização dos valores da função discreta em uma ``Função Geradora'', mostre como recuperar os valores de seus coeficientes utilizando a Análise Complexa.
\item Generalize o Modelo de Fibonacci considerando fertilidade variável, \(\nu_j =\) ``Numero de descendentes produzidos por indivíduo com idade \(j\) em uma unidade de tempo'' e também mortalidade variável, \(\mu_j =\) ``Fração de indivíduos de uma população de idade \(j\) que falecem em uma unidade de tempo''.
\end{enumerate}
\end{exercise}



\begin{exercise}
Verifique, historicamente, se Euler utilizou ideias semelhantes em algum contexto de sua volumosa Opera Omnia. (Ref. Biografias de Euler, R. Graham \& D. Knuth).
\end{exercise}

\begin{exercise}
Consulte uma referencia \textit{Baliza} sobre o Método de Quadrados Mínimos, sua origem e generalizações e faça um resenha de 20 linhas a respeito. (A generalização das ideias da
representação linear de Gauss para funções de variáveis vetoriais com o ajuste de hiperplanos a uma nuvem de pontos \(P^k = x_1^{(k)}, \ldots, x_n^{(k)} \in \mathbb{R}^n\) é um dos Métodos contemporâneos mais importantes na busca de uma representação funcional para Tabelas de dados discretos multivariados e se constitui em um instrumento computacional
indispensável na computação científica contemporânea (G. Strang - Introduction to Linear Algebra, Wellesley, J. N. Kutz -...).
\end{exercise}



\begin{exercise}
Mostre que duas funções assintoticamente equivalentes no infinito não implica necessariamente que a diferença entre elas tende a zero, e que pode até mesmo se tornar ilimitada.

Mostre que essa equivalência é uma aproximação no sentido relativo (ou seja, um erro da ordem de 1km na distância entre a Terra e a Lua não é ``a mesma coisa'' que o mesmo erro na distância entre a Unicamp e Campinas).
\end{exercise}

\begin{exercise}
Mostre, na verdade, que duas funções são assintoticamente equivalentes no infinito se na escala logarítmica os seus valores se aproximam.
\end{exercise}

\begin{exercise}
Em particular, mostre que se \(f(k) \sim e^{\gamma k}\), para \(k \sim \), então há uma ``aproximação linear na escala logarítmica da função f'' no infinito, ou seja,
\[\displaystyle\lim_{k \to \infty} \{\ln(f(k)) - (\ln(A)+\gamma k)\} = 0.\]
\end{exercise}

\begin{exercise}\(\ast\)
Analise uma tabela de números primos (v. Abramowitz \& Stegun) e repita o argumento de Gauss.
\end{exercise}

\begin{exercise}[Fórmula de Stirling]*
Verifique, experimentalmente, utilizando as Tabelas de Abramowitz \& Stegun que a importante Função Gama de Euler \(\Gamma: \mathbb{N} \to \mathbb{C}\), definida recursivamente por \(\Gamma(k+1) = (k+1) \Gamma(k),\ \Gamma(1) = 1\) é assintoticamente equivalente à função elementar (transcendental) \(g(k) = \sqrt{2\pi} k^{k+\frac{1}{2}} e^{-k}\) logaritmizando as duas variáveis, \(u = \ln(\Gamma(k))\) e \(v = \ln(k)\) e comparando os pontos com o gráfico de {\color{red} ?????}

\textbf{Observação}: Ao contrário da Função de Fibonacci e tal como a função densidade de números primos de Gauss, a função Gama não é representável por funções Elementares em todo o seu domínio, mas é assintoticamente equivalente a uma Função Elementar, um importantíssimo resultado que é fundamental para a Análise Combinatória (Flajolet \& Sedgewick), Teoria de Computação (Graham \& Knuth), Teoria de Probabilidade (Gnedenko), Física Estatística (van Kampen) e etc. Nenhum estudante sério de Matemática (Aplicada ou não) pode desconhecer este resultado que será abordado com maiores detalhes no capítulo sobre Métodos Assintóticos).
\end{exercise}

\begin{exercise}
Dada uma ``grande'' Tabela de dados \(P^{(k)} = (x_{1}^{(k)}, x_{2}^{(k)})\) representada cartesianamente na forma de uma ``nuvem de pontos'' no plano imagine uma forma de representá-la sinteticamente (e aproximadamente) da melhor maneira possível com apenas \textbf{duas} informações numéricas independentes (dois parâmetros reais). Convença-se de que esta questão pode ser geometricamente resolvida, obtendo-se uma reta \(r(a, b)\) (\(r = \{(x, y); y = ax+b\}\)) que minimiza a soma dos quadrados das distâncias entre \(P^{(k)}\) e \(r\) medidas das seguintes maneiras:
\begin{description}
\item (a) \(d(P^{(k)}, r) = \min\{(x_{1}^{(k)}, y), r\}\) (distância vertical) e
\item (b) \(d\ast(P^{(k)}, r) = \min\{P^{(k)}, (x, y) \in r\}\) (distância ortogonal).
\end{description}

Resolva matematicamente este problema.
\end{exercise}

\begin{exercise}
Mostre que a População produzida por um Modelo de Fibonacci \(F(k)\) é assintoticamente exponencial e determine esta função exponencial \(A e^{\gamma k}\).
\end{exercise}

\begin{exercise}
Consulte a Linearização assintótica de inúmeros dados biológicos, físicos e geométricos apresentados em G. West e, particularmente, a de Suzana Herculano-Houzel, fazendo uma resenha a respeito desta última..
\end{exercise}

\begin{exercise}
Obtenha uma tabela de censo demográfico do Brasil em sua época de maior crescimento populacional (excetuando imigração) e utilize o Método de Linearização para determinar um modelo proporcional de população de Euler para a sua representação.
\end{exercise}

\begin{exercise}
Faça o mesmo para o início de crescimento de mortalidade pela COVID-19 em 2020 no Brasil e alguns países da Europa e EUA.
\end{exercise}



\begin{exercise}
Discuta as afirmações acima e apresentando argumentos que as justifiquem.
\end{exercise}




No exemplo a seguir, veremos como Christiaan Huygens, um matemático de enorme habilidade do século XVII ao examinar as tabelas de mortalidade de Graunt e Neumann, foi incapaz de identificar a função que as representaria por desconhecer a função exponencial, cujo estudo foi levado a efeito somente em anos mais tarde por Euler. (Euler-Introductio in Analysin Inifinitorum, 1748 (trad. Introduction to Analysis of the Infinite - Springer). Entretanto, não escapou à Huygens a propriedade geométrica fundamental que caracteriza esta função e mais tarde foi utilizada pelo próprio Euler para representar analiticamente a curva de mortalidade.

\begin{exercise}
Comentar a frase acima.
\end{exercise}




\begin{exercise}
Mostre que a regra de Huygens, aplicada a funções diferenciáveis positivas, determina unicamente a função exponencial.
\end{exercise}





\begin{exercise}
Considere uma população Malthusiana que inicia sua história com \(P_0 = P(0)\) indivíduos (vivos!) ``colonizadores'' submetida a uma taxa específica de mortalidade \(\mu\) e de natalidade \(\nu\). Determine o número total de nascimentos \(N(t)\) e o de óbitos \(M(t)\) durante o período \([0,t]\) nesta população.
\end{exercise}

\begin{exercise}
Mostre que a subpopulação de sobreviventes dentre os indivíduos colonizadores \(P_0\) no instante \(t\) é \(e^{-\mu t} = P(0)\) e que, em geral, os sobreviventes no futuro \(t+T\) da população \(P(t)\) existente no instante \(t\) é \(e^{-\mu T} P(t)\), e que os não sobreviventes são \((1-e^{-\mu T}) P(t)\).
\end{exercise}




\begin{exercise}

\begin{description}
\item (a) Realize as seguintes experiências REAIS: \(N\sim 20\) lançamentos sucessivos de uma moeda seguidos da anotação do resultado em uma Tabela concreta (computador ou papel) onde constam o número de experimentos, \(1 \le n \le N\), o tempo \(T(n)\) necessário para executá-los, o número de caras e o de coroas até a referida etapa \(n\), e obtenha uma representação linear (aproximada \(T(n) = \alpha n\)) entre estas variáveis. Extrapole o resultado para avaliar o tempo necessário para executar \(10^{23}\) experimentos. Compare com a idade da Terra, que segundo alguns físicos contemporâneos é da ordem de \(\simeq 6 \cdot 10^{9}\) anos (H. Fritzsch). (Sugestão: Método de Mínimos Quadrados de Gauss explicado).

{\color{red}
\textbf{Solução}: (Paulo Henrique)

Seja

\(N^t = \left[\begin{array}{cccc} n_1 & n_2 & \ldots & n_N \end{array}\right]^t\) (Pontos de entrada)

\(T^t = \left[\begin{array}{cccc} t_1 & t_2 & \ldots & t_N\end{array}\right]^t\) (Pontos de saída)

Queremos obter um \(Z \sim T\). Se \(Z\) é linear, temos:
\begin{eqnarray*}
Z
= \theta_1 N + \theta_2
= \left[\begin{array}{c} n_1\theta_1+\theta_2 \\ n_2\theta_1+\theta_2 \\ \vdots \\ n_N\theta_1+\theta_2 \end{array}\right]
= \underbrace{\left[\begin{array}{cc} n_1 & 1 \\ n_2 & 1 \\ \vdots \\ n_N & 1 \end{array}\right]}_{\overline{N}}
\underbrace{\left[\begin{array}{c} \theta_1 \\ \theta_2 \end{array}\right]}_{\Theta}
\end{eqnarray*}

Vamos minimizar a função
\[E(\Theta)
= (T-Z)^2
= (T-\overline{N}\Theta)^2
\]
e, para tal, determinemos:
\[
\dfrac{\partial E}{\partial \Theta}
= \dfrac{\partial E}{\partial Z}\ \dfrac{\partial Z}{\partial \Theta}.
\]

Mas
\[
\dfrac{\partial E}{\partial Z}
= \dfrac{\partial}{\partial Z} (T-Z)^2 = -2 \underbrace{(T-Z)}_{N \times 1}
\]
e
\[
\dfrac{\partial E}{\partial \Theta}
= \dfrac{\partial}{\partial \Theta} (\overline{N}\Theta) =  \underbrace{\overline{N}^t}_{2 \times N}.
\]

Segue que
\[0
= \dfrac{\partial E}{\partial \Theta}
= -2 \overline{N}^t (T-Z)
= -2 \overline{N}^t (T-\overline{N}\Theta),
\]
ou seja,
\[
\overline{N}^t T = \overline{N}^t \overline{N} \Theta
\]
implicando em
\[
\Theta = (\overline{N}^t\ \overline{N})^{-1} \overline{N}^t T.
\]
(ver \href{https://www.youtube.com/watch?v=txnrFZG7Ugs&ab_channel=LeonardoOlivi}{Youtube})


Os valores de \(\theta_1\) e \(\theta_2\), após algumas contas, são dados por:
\[\begin{array}{rcl}
\theta_1 &=& \dfrac{\displaystyle\sum_{i=1}^{N} n_i \sum_{i=1}^{N} t_i - N \sum_{i=1}^{N} n_i\ t_i}{\displaystyle\left(\sum_{i=1}^{N} n_i\right)^2 - N \sum_{i=1}^{N} n_i^2} \\
\theta_2 &=& \dfrac{\displaystyle\sum_{i=1}^{N} t_i - \theta_1 \sum_{i=1}^{N} n_i}{N}
\end{array}
\]

Os experimentos executados, bem como alguns valores necessários para a obtenção dos coeficientes da reta de regressão são apresentados na tabela a seguir e foram obtidos utilizando uma planilha eletrônica (Excel):
\[\scriptsize
\begin{array}{c|c|c|c|c|c|c|c|c}
n & 1 & f(n) & t_n & T(n) & K & C & n^2 & n\ t \\ \hline
1 & 1 & 0 & 24 & 24 & 1 & 0 & 1 & 24 \\ \hline
2 & 1 & 1 & 36 & 60 & 1 & 1 & 4 & 120 \\ \hline
3 & 1 & 1 & 27 & 87 & 1 & 2 & 9 & 261 \\ \hline
4 & 1 & 1 & 17 & 104 & 1 & 3 & 16 & 416 \\ \hline
5 & 1 & 0 & 44 & 148 & 2 & 3 & 25 & 740 \\ \hline
6 & 1 & 0 & 24 & 172 & 3 & 3 & 36 & 1032 \\ \hline
7 & 1 & 1 & 35 & 207 & 3 & 4 & 49 & 1449 \\ \hline
8 & 1 & 0 & 15 & 222 & 4 & 4 & 64 & 1776 \\ \hline
9 & 1 & 1 & 49 & 271 & 4 & 5 & 81 & 2439 \\ \hline
10 & 1 & 0 & 24 & 295 & 5 & 5 & 100 & 2950 \\ \hline
11 & 1 & 1 & 21 & 316 & 5 & 6 & 121 & 3476 \\ \hline
12 & 1 & 1 & 17 & 333 & 5 & 7 & 144 & 3996 \\ \hline
13 & 1 & 0 & 39 & 372 & 6 & 7 & 169 & 4836 \\ \hline
14 & 1 & 1 & 19 & 391 & 6 & 8 & 196 & 5474 \\ \hline
15 & 1 & 1 & 42 & 433 & 6 & 9 & 225 & 6495 \\ \hline
16 & 1 & 1 & 32 & 465 & 6 & 10 & 256 & 7440 \\ \hline
17 & 1 & 1 & 24 & 489 & 6 & 11 & 289 & 8313 \\ \hline
18 & 1 & 0 & 23 & 512 & 7 & 11 & 324 & 9216 \\ \hline
19 & 1 & 1 & 16 & 528 & 7 & 12 & 361 & 10032 \\ \hline
20 & 1 & 0 & 27 & 555 & 8 & 12 & 400 & 11100 \\ \hline
21 & 1 & 0 & 33 & 588 & 9 & 12 & 441 & 12348 \\ \hline
22 & 1 & 0 & 22 & 610 & 10 & 12 & 484 & 13420 \\ \hline
23 & 1 & 1 & 24 & 634 & 10 & 13 & 529 & 14582 \\ \hline
24 & 1 & 0 & 18 & 652 & 11 & 13 & 576 & 15648 \\ \hline
25 & 1 & 0 & 30 & 682 & 12 & 13 & 625 & 17050
\end{array}
\]
onde
\begin{description}
\item \(K\) representa o n\textordmasculine\ de caras obtidas até o \(n\)-ésimo lançamento;
\item \(C\) representa o n\textordmasculine\ de coroas obtidas até o \(n\)-ésimo lançamento.
\end{description}

Os valores obtidos a seguir, também com o auxílio do Excel, são utilizados para determinar os valores dos coeficientes da reta de regressão:
\[
\begin{array}{c|c}
S_n & 325 \\ \hline
S_T & 9150 \\ \hline
S_{nT} & 154633 \\ \hline
S_{n^2} & 5525 \\ \hline
N & 25
\end{array}
\]
e, os coeficientes a determinar, são:
\[
\begin{array}{c|c}
\theta_1 & 27,44846154 \\ \hline
\theta_2 & 9,17
\end{array}
\]

Para \(n = 10^{23}\), obtemos: \(T_n = 2,74485 \cdot 10^{24}\) segundos, ou seja, \(8,70385 \cdot 10^{16}\) anos. Comparando com a idade da Terra, temos que \(T_n\) é, aproximadamente, \(14.506.416\) vezes maior.
}


\item (b) No exercício anterior, você fez poucos lançamentos e não se cansou muito, o que resultou em uma reta bem representativa do resultado. Agora repita, sem descansar ou diversificar a atenção durante o experimento, \(100\) lançamentos sucessivos e verifique que a curva é ascendente. Interprete o resultado com relação ao exercício anterior. Utilize a escala logarítmica e argumente se a curva é assintoticamente exponencial ou polinomial. (Esta questão visa determinar o seu comportamento e é representativo de muitas experiências em Biologia. Este exercício pode ser feito em grupos em que apenas duas pessoas realizam a experiência para efeito de comparação).

\item (c) Antes de se cansar, é possível que ocorra intermediariamente um ``aprendizado'' que tornará o procedimento mais rápido. Mas este aprendizado é saturado em pouco tempo. Em uma etapa posterior, ``vence o cansaço'' e o tempo necessário para cumprir a tarefa começa a se alongar. Analise estas fases do procedimento em termos da Tabela anotada e de um re-escalonamento logarítmico.

\item (d) Suponha que você tenha vida quase-eterna. Utilizando o resultado anterior (com cansaço) calcule o tempo necessário para realizar os \(10^{23}\) experimentos, mas observe que o ``índice de cansaço", que pode ser medido pela curvatura do gráfico deve ser variável, ou seja, o cansaço é cumulativo, como sabemos). Supondo (especulativamente) que sua atenção seja exponencialmente decrescente, analise a questão acima.
\end{description}
\end{exercise}

\begin{exercise}
Aproveite a sua tabela e obtenha uma representação assintótica da função \(p_k: \mathbb{N} \to \mathbb{N}\), definida como, \(p_k(n) = p\), onde \(p\) representa o número de vezes que em \(n\) experiências produziu \(k\) sucessivos''. ``Logaritmize'' esta Tabela e conclua que ela sugere um comportamento assintótico da função \(p_k(n)\) exponencial e analise como variam os coeficientes exponenciais em dependência de \(k\) (Se crescem ou decrescem).
\end{exercise}

\begin{exercise}
Registre (mentalmente) o resultado de cada sequência de \(n\) experimentos na forma de um ``sinal'' (ou ``palavra'') de comprimento \(n\) constituído de \(0's\) e \(1's\), onde \(0\) corresponde a coroa e \(1\) corresponde a cara da moeda. Seja, então, \(F_2(n) = f\), onde \(f\) é o número de possíveis ``sinais'' de comprimento \(n\) que não tem \(0's\) sucessivos''. Mostre que esta função satisfaz à recursão \(F(n+2) = F(n+1) + F(n)\) com dados iniciais \(F(1) = 2\), \(F(2) = 3\). Calcule uma Tabela para esta Função e mostre via linearização logarítmica que ela tem comportamento assintótico exponencial \(Ae^{\gamma n}\) e, aproximadamente, o seu coeficiente \(\gamma\).
\end{exercise}


\section*{Lista 03}

\begin{exercise}
Leia o \href{https://www.hendrix.edu/uploadedFiles/Admission/GarrettHardinArticle.pdf}{artigo} de Garret Hardin  e algumas referências posteriores sobre o tema ``\textit{Tragedy of Commons}''. Consulte o texto de Simon Levin e visite a página deste importante autor. Consulte também o artigo de Luwig - Walter e Holling no primeiro número da revista (online - acesso livre) Ecology and Society de 1997. Faça um resumo comentado de aproximadamente 20 linhas sobre o tema.
\end{exercise}


\begin{exercise}
Considere \(n\) populações com tamanhos \(N_k\) que subsistem com a exploração peculiar de um portfólio de recursos com taxas de consumo basal distintas. Adapte o argumento acima para obter um sistema de \(n\) equações diferenciais para as funções \(N_k(t)\) que represente este Modelo de Malthus-Verhulst com Saturação. (Sistemas deste tipo tem sido empregados por Martin Nowak para a descrição de competições e evolução de populações. Ref: M. Nowak - Evolutionary Dynamics, Harvard UP 2010).
\end{exercise}

\begin{exercise}
Obtenha uma solução explícita em termos de funções elementares para a equação diferencial de Riccatti \(\dfrac{du}{dt} = au + bu^2\) dividindo por \(u^2\) e escrevendo uma equação linear para \(v = \frac{1}{u}\) que pode ser resolvida explicitamente.
\end{exercise}

\begin{exercise}
Escreva o Modelo de Malthus-Verhulst na forma adimensional e explique o resultado. 
\end{exercise}

\begin{exercise}
Considere um conjunto de indivíduos \(P_0\) ``fundadores'' de uma população no instante \(t=0\) cuja dinâmica é regida por um Modelo de Malthus-Verhulst. Considere que para a dinâmica da população resultante o termo \(-\alpha N\) designa a mortalidade per capita da população. Determine o tempo médio de sobrevivência da população fundadora desta dinâmica. (\textbf{Sugestão}: Considere a subpopulação \(P(t)\) dos ``fundadores'', \(P(0) = P_0\) e verifique que \(\frac{dP}{dt} = -\alpha NP\). Refaça o argumento utilizado para determinar o tempo médio de sobrevivência do Modelo de Malthus para esta subpopulação, mas observe que para isto será necessário dispor da função \(N(t)\) que estabelece a população total - deles e de seus descendentes - em que eles ``vivem'').
\end{exercise}

\begin{exercise}
\begin{description}
\item (a) Faça um esboço geométrico para a dinâmica da equação que descreve o Modelo de Malthus-Verhulst interpretando \(N(t)\) como uma trajetória que se move com velocidade \(\frac{dN}{dt}\) sobre a reta coordenada horizontal (chamada reta de fase) segundo um campo de velocidades pré-estabelecido sobre a reta (chamada espaço de fase). Este campo de velocidades é determinado pelo gráfico cartesiano da função \[F(N) = r_0\left(1-\dfrac{N}{K}\right)N\]
representado acima da reta de fase e a velocidade do ponto ocupando a posição \(N\) é dada pelo valor de \(F(N)\): Para a direita se positivo e para a esquerda se for negativo com intensidades respectivas. Observe que há duas posições estacionárias (isto é, posições onde a velocidade é nula) e que uma delas (a origem) é repulsiva (isto é, as velocidades em posições próximas tendem a afastar as trajetórias da origem, dita instável) e a outra (\(N = k\)) é atrativa (i.e., as velocidades em posições próximas tendem a dirigir as trajetórias de volta à posição estacionária original, dita estável).
\item (b) Com base nesta dinâmica na reta de fase faça um esboço dos gráficos cartesiano das curvas \(N(t) ( (t, N(t)) )\) observando que o formato da curva depende do ponto de partida \(N(0) = 0\). Determine quais delas são ``logísticas'' isto é, tem forma de ``S'' enquanto que as outras tem forma de ``C''. Se uma população com esta dinâmica inicia com \(N(0) = N_0 < K\) determine o tempo \(t(N)\) que ela leva para atingir os valores \(N > N_0\). Mostre que ela leva um tempo infinito para atingir \(N = K\) e, portanto, jamais atingirá um valor \(N > K\). Explique!
\item Analise uma dinâmica na reta determinada por um campo de velocidades da forma \[\dfrac{dn}{dt} = \cos(n) \sqrt[4]{1-\cos(n)},\]
determine as posições estacionárias atrativas e repulsivas e mostre que o tempo de percurso até a origem de uma trajetória que se inicia em seu campo de atração é finito.
\end{description}
\end{exercise}

\begin{exercise}
\begin{description}
\item (a) Mostre que a dinâmica ``Caricatura de Malthus''
\[\dfrac{dN}{dt} =  F(N),\]
onde o gráfico cartesiano da função \[F(N)\] consiste de uma reta \(y = r_0 N\) para \(0 le N \le K\) e outra reta \(y =-\lambda N + (r_0 + \lambda) K\), para \(N \ge K\), também produz uma dinâmica de saturação como o Modelo de Malthus, mas é linear por pedaços, o que pode ser uma vantagem na hora de obter uma solução explicita, mesmo que por pedaços. Analise esta questão. (Ideia inventada por Joseph B. Keller e seu aluno John Rinzel para analisar ondas em sistemas neurais na década de 1970).
\item (b) Analise o modelo ainda mais caricatural em que \(F(N) = a > 0\), para \(N < K\) e \(F(N) = b < 0\), para \(N > K\) e \(F(K) = 0\). Mostre que este modelo exibe um efeito de saturação e, portanto, é da classe de Malthus-Verhulst, mas que não é mais parcimonioso do que o modelo quadrático comparando o número de parâmetros necessários para defini-los adimensionalmente.
\end{description}
\end{exercise}

\begin{exercise}
Considere uma população cuja mortalidade se dá na forma (não Malthusiana) chamada dinâmica de Monod-Holling II:
\[\dfrac{dN}{dt} = -\mu \dfrac{N}{A+N}, N(0) = N_0.\]
(Segundo o bioquímico Jacques L. Monod (1910-1976) e o ecólogo Crawford S. Holling (1920-2019)). Mostre que o tempo de vida média dos indivíduos desta população depende do valor inicial \(N_0\) o que significa uma interferência mútua entre eles. Observe que a taxa de mortalidade per capita deste modelo, \(\frac{1}{N} \frac{dN}{dt}\) é decrescente com a densidade populacional, ao contrário do modelo Malthusiano. O que se pode concluir deste fato que, neste caso, a interação entre seus indivíduos, é (em suma) prejudicial.
\end{exercise}

 
 

\begin{exercise}
Analise a questão de sobrevivência de duas populações que utilizam um único Recurso Comum. Se uma das populações burlar consistentemente um ``acordo"de compartilhamento, analise os cenários que levam à extinção populacional de ambas.
\end{exercise}




\begin{exercise}
Considere a seguinte modificação natural do Modelo de Fibonacci: A população é constituída de ``casais'' (o que pode ser interpretado como o número de fêmeas da população supondo-se que haja um equilíbrio entre as populações dos dois sexos regulamentado de alguma forma pela biologia reprodutiva da espécie). Considere que em cada período entre dois ``censos'', \(k\) e \(k + 1\), a reprodução obedeça a uma taxa \(\alpha\), isto é, se \(M\) for o número de casais maduro no censo \(k\) então eles produzirão \(\alpha M(k)\) casais no censo \(k + 1\). Suponha também que ocorra uma mortalidade que será a uma taxa de \(\mu_1\) para os imaturos e \(\mu_2\) para os casais maduros. Por exemplo, se \(M(k)\) for o número de casais maduros no censo \(k\), então, a proporção destes que sobreviverão para o próximo censo \(k + 1\) será de \((1 - \mu_2)\). 
\begin{description}
\item (a) Escreva, argumentando, um Modelo recursivo para esta população
\item (b) Escreva a solução desta recursão em termos de funções elementares. (Sugestão: Estude a seção no texto Bassanezi-Ferreira 1988 - relativo a equações de recursão e o Método Operacional para resolve-las explicitamente)
\item (c) Analise a possibilidade de crescimento exponencial desta população, ou de extinção (isto é, de \(\displaystyle\lim_{k \to \infty} P(k) = 0)\).
\end{description}
\end{exercise}

\begin{exercise}
Modelo de Plantas anuais. Considere uma espécie de plantas que a cada ano \(k\) tem \(P(k)\) exemplares durante a primavera. Estas plantas produzirão durante o verão seguinte \(s P(k)\) sementes que serão lançadas ao solo durante o outono, quando as plantas secarão. Estas sementes lançadas hibernarão durante o inverno e apenas uma fração \(f_1\) delas sobreviverá e destas apenas uma fração \(g_1\) germinará produzindo plantas na próxima primavera no ano \(k + 1\). (Neste caso, as sementes que não germinaram morrem).

\begin{description}
\item (a) Escreva um modelo para a dinâmica da população P(k) destas plantas na primavera do ano \(k\).
\item (b) Generalize o Modelo para sementes que tem a capacidade de hibernar dois invernos seguidos (com mortalidades específicas para o primeiro e segundo inverno) para germinarem apenas no segundo ano seguinte (com taxa especifica). Neste caso, dentre as sementes que não germinaram no primeiro inverno, algumas morrem e outras hibernarão no próximo inverno.
\item (c) Escreva um modelo com muitas hibernações.
\textbf{Observação}: A hibernação alongada (isto é, a não germinação de todas as sementes na primeira primavera seguinte) é uma ``estratégia'' de sobrevivência para épocas de estiagem que certamente ocorrerão durante alguns anos e que poderiam levar à extinção da espécie que não ``guardasse'' sementes para o próximo ano.
\end{description}
\end{exercise}

\begin{exercise}
Considere diversos ``graus de maturidade'' de uma população de ``casais'' generalizando o Modelo de Fibonacci onde \(P(k)\) é esta população. Neste caso considere, digamos \(100\) ``faixas etárias'' de maturação cada uma com duração de um ano e com taxas (``frações'') específicas de mortalidade e reprodutibilidade.

\item (a) Escreva um Modelo recursivo para a dinâmica desta População que, a proposito é semelhante ao Modelo de Euler de 1760.
\item (b) Escreva a solução elementar desta recursão em termos de funções elementares.(Considere, ``ingenuamente'', que a obtenção de raízes de polinômios seja uma tarefa elementar e simples!).
\item (c) Convencido/a de que a obtenção dos valores de raízes de polinômios de grau superior não é tarefa simples, reduza suas ambições e Analise teoricamente apenas a possibilidade de explosão populacional e de extinção em termos matemáticos com a informação da localização destas raízes com relação ao disco unitário no plano complexo. (Consulte Bassanezi-Ferreira a respeito e o Método de Mikhailov).
\item (d) É historicamente interessante registrar que, embora o artigo de Euler tenha sido publicado nos Anais da Academia de Ciências de Berlim e precedido à publicação do livro de Malthus por 40 anos, o seu impacto foi quase nulo, tanto que o próprio Darwin, somente faz referência rápida a este trabalho em um ligeiro comentário na pg. 428 da 5a. edição do ``\textit{Origin of Species}''. Darwin revisou infatigavelmente o seu famoso texto inúmeras vezes desde sua publicação inicial em 1859 até a sexta edição de 1876 e não sendo muito afeito à Matemática (isto, por palavras dele próprio!- [autobiografia]) é possível que a referência ao artigo de Euler tenha sido resultado de alguma de suas muitas correspondências, alguma delas, talvez originárias de matemáticos. Esta é uma boa questão histórica para ser resolvida analisando o enorme arquivo de correspondência de Darwin. (Darwin[...]). Verifique a correção desta afirmação histórica com subsídios a favor ou contra ela.
\item (e) É de se ressaltar que àquela época o conceito (pelo menos implícito) de Modelo Matemático que descreve hipóteses e conclusões em linguagem matemática era já bem conhecido como consequência das bem sucedidas e inúmeras aplicações do Cálculo de Newton e Leibniz, notadamente à Mecânica (Celeste ou Terrestre), em cuja arte Leonhard Euler foi um mestre insuperável. Assim, não seria por falta de bons exemplos a imitar que Malthus deixou de formular um Modelo mais preciso para a sua teoria. De qualquer forma, desde o princípio do século XIX o crescimento geométrico, ou exponencial, de uma população tornou-se sinônimo de um processo ``Malthusiano''.
\end{exercise}




\begin{exercise}
Considere uma População (não homogênea segundo a dinâmica Malthusiana), com \(N(t)\) indivíduos no instante \(t\) constituída de subpopulações homogêneas, \(N_k(t)\), \(N(t) = \sum_{k} N_k(t)\), cujas dinâmicas são descritas pelos Modelos Malthusianos:
\[\dfrac{dN_k}{dt} = -\mu_k N_k.\]
Mostre que o tempo de vida médio da população total é
\[\tau = \sum \tau_k = \sum \dfrac{N_k(0)}{N(0)}\dfrac{1}{\mu_k}.\]
Conclua que uma ``homogeneização'' (este é o termo técnico utilizado para designar estes procedimentos) da população total segundo o Modelo Malthusiano deve utilizar a constante de Malthus
\[\mu_{eff} = \left(\dfrac{N_k(0)}{N(0)}\dfrac{1}{\mu_k}\right)^{-1}\]
obtida pela \textbf{média harmônica} das constantes das subpopulações. O modelo Malthusiano
\[\dfrac{dN_{eff}}{dt} = -\mu_{eff} N_{eff},\]
é chamado ``Modelo Efetivo''.
\end{exercise}

\begin{exercise}
Analise a aproximação das soluções do modelo efetivo com respeito à sua solução exata:
\[N(t) = \sum N_k(0) e^{-\mu_k t}\]
e as condições para que seja uma boa aproximação. 
\end{exercise}

\begin{exercise}\(\ast\)
Considere uma população contínua que seja distribuída continuamente segundo uma característica \(x\) na forma \(N(t, x)\), onde o ``Número de indivíduos com característica \(x_1 \le x \le x_2\) no instante \(t\)'' é dado por:
\[\int_{x_1}^{x_2} N(t, x)  dx\]
e suponha que a mortalidade Malthusiana varie na forma \(\mu(\epsilon x)\) lentamente (isto é, \(\left|\dfrac{d\mu(s)}{ds}\right| \approx 1\) e \(0 < \epsilon \ll 1\) (bem pequeno). Obtenha o Modelo Efetivo para esta população.
\end{exercise}



\begin{exercise}
\begin{description}
\item (a) Descreva as Matrizes \(M\) e \(\overline{M}\) para sistemas de reações autocatalíticas, respectivamente, dos casos finito e infinito. 
\item (b) Considere um processo de decaimento radioativo de uma amostra de Carbono 14. O \(C_{14}\) é um isótopo instável com um excesso de dois nêutrons no núcleo com relação ao Carbono 12, \(C_{12}\), que tem seis prótons e seis nêutrons e que é nuclearmente muito mais estável e mais comum. Apesar desta diferença nuclear, eles são quimicamente idênticos e, portanto, são utilizados sem distinção em processos bioquímicos de organismos vivos. O elemento carbono é um dos constituintes fundamentais no metabolismo dos organismos, junto com hidrogênio, oxigênio, fósforo, e outros. O \(C_{14}\) da biosfera (Terra e sua atmosfera), é produzido a uma taxa que pode ser considerada constante como resultado do bombardeio de raios cósmicos sobre os átomos de nitrogênio abundantes na alta atmosfera. Uma amostra típica de \(C_{14}\), um mol, \((14g)\), contém uma população em torno de \(10^{25}\) átomos. O processo de decaimento de um átomo \(C_{14}\) consiste em transmutar-se espontaneamente, ``sem essa nem aquela'', de volta ao átomo de nitrogênio, ou seja, de maneira totalmente ``aleatória'', o que significa ser imprevisível determinar (prática e mesmo teoricamente) quando isto ocorrerá com um átomo individual. Entretanto, observa-se que em uma grande população deles o processo coletivo se dá tal de acordo com o Modelo de Rutherford, ou seja, Malthusianamente: Se \(C(t)\) for a população de Carbono 14 no instante \(t\), então,
\[\dfrac{dC}{dt} = -\mu C,\]
onde \(k\) é a constante de decaimento.
\end{description}
\end{exercise}

\begin{exercise}
Em Físico-Química a tradição histórica ainda domina, razão porque é mais comum utilizar o conceito de ``meia-vida'', \(T_{1/2}\), para caracterizar esta dinâmica, e que é definido da seguinte maneira: ``A Meia vida do C14 é o tempo (característico desta substancia) necessário para que uma amostra (qualquer) dela se reduza à sua metade por conta do decaimento radioativo''. 

\begin{description}
\item (a) Mostre que esta é uma boa definição, ou seja, que não depende do tamanho da amostra, e relacione o parâmetro ``meia vida'' \(T_{1/2}\) de \(C_{14}\) com a sua ``vida média'' \(\mu^{-1}\).

\item (b) Obtenha a ``meia vida'' do \(C_{14}\) na literatura e determine a sua constante \(\mu\) de decaimento e a sua vida média em horas.

\item (c) Considere a dinâmica do sistema de Carbono \(14\) total da biosfera \(C(t)\), produzido a uma taxa constante pelo processo já descrito e continuamente em decaimento. Mostre que esta dinâmica pode ser descrita na forma
\[\dfrac{dC}{dt} = p - \mu C,\]
e mostre que um equilíbrio é atingido.

\item (d) Com base neste equilíbrio, (que depois de milhões de anos de `vida' da Terra, para todos os efeitos já deve ter sido atingido!), cujo valor você pode, e deve, descobrir qual seja na literatura, determine a sua taxa de produção \(p\) na alta atmosfera.

Neste equilíbrio participam também todos os seres vivos, já que estão em contínuo intercâmbio com a matéria orgânica do ambiente. Após a morte de um organismo, este intercambio é interrompido, enquanto a concentração de \(C_{14}\) em seus restos mortais continua decaindo. Portanto, analisando a concentração \(c_0\) de \(C_{14}\) de restos arqueológicos de matéria orgânica, pode-se avaliar o tempo \(T\) decorrido desde sua morte. 

\item (e) Descreva como medir o tempo com a concentração \(c_0\), ou seja, obtenha a função \(T(c_0)\).

É necessário ressaltar que o conceito de ``tempo médio de vida, ou de sobrevivência, ou de permanência'' dos indivíduos em uma população em extinção \(\mu^{-1}\), é diferente do conceito de ``tempo de meia-vida'' , este último utilizado no modelo de decaimento radioativo, cujo significado é: ``período de tempo necessário para que ocorra o decaimento da metade da `população'''. Relacione os dois conceitos.
\end{description}
\end{exercise}
 

 

\begin{exercise}
Escreva a matriz \(S\) para as quatro situações indicadas e determine se, de fato, ela é sempre simétrica. 
\end{exercise}

\begin{exercise}
Considere um sistema periódico, isto é, de tal forma que uma extremidade é conectada à outra. Neste caso o compartimento \(P_0\) terá \(P_N\) à sua esquerda e \(P_N\) terá \(P_0\) à sua direita. Mostre que ainda assim podemos escrever o modelo na forma \(\frac{dX}{dt} = SX\), com \(S\) simétrica.
\end{exercise}




\begin{exercise}
Desenhe um gráfico com 5 componentes e conexões múltiplas e estabeleça um Modelo de Difusão para este caso representando- na forma vetorial-matricial.
\end{exercise}

Anteriormente, mostramos que o Modelo Populacional Malthusiano pode ser encarado como uma grande quantidade de experimentos individuais, simultâneos e independentes e, portanto, passível de ser interpretado como um processo probabilístico para cada partícula individualmente. É natural portanto que estendamos o mesmo argumento para este modelo de Difusão. Por exemplo, podemos interpretar o parâmetro \(\mu^{-1}\) como o tempo médio de permanência de uma partícula em qualquer um dos compartimentos.

%Exercícios:

\begin{exercise}
Determine a probabilidade de que uma partícula do compartimento \(P_k\) se transfira para o compartimento adjacente \(P_{k-1}\) (ou para \(P_{k+1}\)) durante o intervalo de tempo de comprimento \(T\).
\end{exercise}

\begin{exercise}
Considere um sistema finito de compartimentos \(P_k(t), 0 \le k \le 2m+1\), conectados segundo um modelo de Difusão, sendo que as extremidade à esquerda de \(P_0\) e à direita de \(P_{2m+1}\) sejam absorventes. Supondo que no instante zero, exista apenas um único indivíduo e este ocupa o compartimento central \(P_m\), qual a probabilidade deste indivíduo se perder por uma das extremidades até o instante \(t\)? Ou seja, qual o tempo médio de permanência deste individuo no sistema?
\end{exercise}




\begin{exercise}
Mostre que a quantidade total de partículas é mantida no modelo de Difusão infinito,e também naqueles em que as condições laterais são de reflexão (Neumann) ou periódica. 
\end{exercise}

\begin{exercise}
Mostre que se pelo menos uma das condições laterais é de Absorção (Dirichlet), então a quantidade total de partículas do sistema decresce exponencialmente. 
\end{exercise}




\begin{exercise} \quad

\begin{description}
\item (a) Considere uma população Malthusiana que inicia sua história com \(P_0 = P(0)\) indivíduos (vivos!) ``colonizadores'' submetida a uma taxa específica de mortalidade \(\mu\) e de natalidade \(\nu\). Determine o número total de nascimentos \(N(t)\) e o de óbitos \(M(t)\) durante o período \([0,t]\) nesta população.

\item (b) Mostre que a subpopulação de sobreviventes dentre os indivíduos colonizadores \(P_0\) no instante \(t\) é \(e^{-\mu t} P(0)\) e que, em geral, os sobreviventes no futuro \(t+T\) da população \(P(t)\) existente no instante \(t\) é \(e^{-\mu T} P(t)\), e que os não sobreviventes são \((1-e^{-\mu T}) P(t)\).
\end{description}
\end{exercise}

\begin{exercise}
Considere uma população malthusiana com \textbf{altíssima} taxa de mortalidade, sem procriação. Argumente, matemática e demograficamente, porque um sistema com estas características é denominado como de ``\textit{Memória Curta}''. Se este sistema for alimentado por uma fonte externa \(f(t)\) de valor limitado, \((|f(t)| \le M)\) mostre que o ``grosso'' da \(t\) população
\[P(t) e^{-\mu t} P(0) + \int_{0}^{t} e^{-\mu(t-s)} f(s)\ ds,\]
para \(t\) distante da origem é descrito pela expressão: \(P(t) \approx \mu^{-1} f(t)\) a menos de erro exponencialmente pequeno. (\textbf{Sugestão}: Como \(1 \lll \mu\), valores da forma \(e^{-\mu t} P(0)\) são exponencialmente pequenos e na integral apenas a parte próxima de \(t\) (isto é, para \(s \sim t\)) o integrando \(e^{-\mu(t-s)} f(s)\) efetivamente contribui. Diz-se que os valores da integral estão ``condensados'' na extremidade \(t\). Com esta argumentação substitui-se a função \(f(s)\) por sua expansão de Taylor nas proximidades de \(t: f(s) = f(t)+(s-t) f'(t)+o(s-t)^2\) e avaliamos que apenas o termo \(f(t)\) terá contribuição não exponencialmente pequena). Esta argumentação faz parte de um conjunto de técnicas matemáticas importantes denominados ``\textbf{Métodos Assintóticos}'' a serem tratados com maiores detalhes em outro capítulo.
\end{exercise}

\begin{exercise}
Obtenha uma expressão ``Fatorada'' para o operador
\[\left(\dfrac{d}{dt} + \mu(t)\right),\]
onde \(\mu(t)\) é uma função de \(t\) e, com isso, obtenha uma fórmula de Green para o Modelo Malthusiano com influência externa
\[\dfrac{dN}{dt} = -\mu(t)N + f(t).\]
\end{exercise}


