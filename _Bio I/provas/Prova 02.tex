\section*{PROVA 02: MS680-MT624- II Sem 2020}

\begin{quote}
\textbf{POSTADA}: 22 de Dezembro de 2020 (Terça-feira)

\textbf{RECEBIMENTO}: 03 de Janeiro de 2021 (Domingo)

\textbf{ATENÇÃO}: ESCOLHA (apenas) 06 DENTRE AS 121 QUESTÕES DA LISTA ABAIXO.

1 - As Questões devem ser encaradas como oportunidades para demonstrar conhecimento não como perguntas.

\textbf{Precisão} e \textbf{Concisão} serão qualidades avaliadas.

2 - A \textbf{Redação} de cada Prova deve apresentar a forma de um depoimento \textbf{pessoal} distinto. Caso ocorram, todas as cópias envolvidas serão invalidadas.

3 - Cada Questão resolvida deve ser precedida de seu respectivo Enunciado Original completo.

4 - A Resolução deve ser \textbf{digitalizada} em um \textbf{único documento pdf} (\textit{Manuscritos} \textbf{NÃO} serão aceitos!)

5 - O documento pdf da Resolução deve ser enviado no \textbf{Anexo} de uma mensagem com título ``\textbf{PROVA 02}'' para o endereço eletrônico: wilson@unicamp.br

6 - \textbf{Antes} das 24h do dia 03 de janeiro. (Sugestão: Não deixe para a última hora e evite ser responsabilizado por acidentes)
\end{quote}


\section*{Questão 01}
\addcontentsline{toc}{section}{Questão 01}


A Psicologia da Matematização: Ockham (séc. 13) \& Kanizsa (séc. 20), Galileo (séc. 17) \& Newton (séc. 17-18)

    1a - Descreva o ``\textit{Efeito de Completamento (Interpolação) Visual}'' (``\textit{Efeito Kanizsa}'') em poucas linhas e exemplifique-o com o famoso triângulo de Kanizsa e especialmente com a visualização de formas sugeridas por uma sequência de pontos.

    1b - Argumente com base no ``\textit{Efeito Kanizsa}'' sobre a motivação cognitiva da representação contínua para dinâmicas de grandes populações. Como se explica evolutivamente a preferência cognitiva da espécie humana por registrar informações discretas em termos (reduzidos) como ``\textit{formas geométricas}''?

    1c - Descreva a Metodologia funcional de Galileo e justifique-a em termos do que foi discutido em 1a-b.

    1d - Descreva o grande aperfeiçoamento da Metodologia de Galileo realizada por Newton. (Sugestão: Biblioteca de funções)

    1e - Descreva o ``Princípio de Parcimônia de Ockham'' e discuta a sua conexão com a cognição humana, especialmente com o item 1b.

    1f - Exemplifique os itens 1b-c com dados de mortalidade da COVID19 em 2020 para uma grande comunidade durante aproximadamente 1 ano.






\section*{Questão 02}
\addcontentsline{toc}{section}{Questão 02}



Escala Logarítmica na Aproximação Assintótica: Princípio Sensorial (``Lei'') de
Weber-Fechner (séc. 19)

    2a - Descreva o ``\textit{Princípio Sensorial} (``\textit{Lei}'') de Weber-Fechner'' para a percepção visual, auditiva, táctil, olfativa e de cardinalidade.

    2b - Argumente com base no ``Princípio de Weber-Fechner'' sobre a conveniência cognitiva da escala logarítmica para variáveis com ``grandes'' valores.
    
    2c - Aplique a escala logarítmica para o registro numérico da população do exemplo citado no item 1f acima e caracterize os períodos de tempo em que o comportamento é linear (Malthusiano).

    2c - Mostre que, para duas sequencias de números positivos, \(\{a_k \to \infty\}\) e \(\{b_k \to \infty\}\), então valem as seguintes implicações para a aproximação assintótica em escala logarítmica
    \[\log a_k - \log b_k \to 0 \Leftrightarrow \log \dfrac{a_k}{b_k} \to 0 \Leftrightarrow \dfrac{a_k}{b_k} \to 1\]

    2d - Mostre que a aproximação assintótica na escala logarítmica  não implica necessariamente na aproximação assintótica em escala normal (isto é, \(a_k - b_k \to 0\), mas vale a implicação inversa. (Sugestão: Analise a igualdade \(a_k - b_k  a_k \left(1 - \dfrac{a_k}{b_k}\right)\) e observe que \(a_k - b_k \to 0 \Leftrightarrow \dfrac{a_k}{b_k}\) se aproxima de 1 com um erro de \(o\left(\dfrac{1}{a_k}\right)\), isto é, de ``ordem menor do que \(\dfrac{1}{a_k}\)''. Assim, para sequencias que convergem para \(\infty\) é mais interessante analisar a aproximação assintótica logarítmica , pois ela é mais abrangente e tem um fundo cognitivo. Além disso, para dois ``\textit{trens em alta velocidade uma aproximação na escala simples é extremamente perigosa}''!)

%\solucao{}



\section*{Questão 03}
\addcontentsline{toc}{section}{\textcolor{green}{Questão 03}}


Linearização logarítmica  Assintótica

Definições:

    1 - Diz-se que um Modelo Populacional, \(P: \mathbb{N} \to \mathbb{C}\), é \textbf{Malthusiano} se para algum \(A\) e \(\gamma\), se tem \(\dfrac{P(k)}{Ae^{\gamma k}} = 1\), para todo \(k\), ou, equivalentemente, se \(P(k) = Ae^{\gamma k}\).

    2 - Diz-se que um Modelo Populacional é \textbf{Assintoticamente Malthusiano} se para algum \(A\) e \(\gamma\), se tem \(\dfrac{P(k)}{Ae^{\gamma k}} \to 1\), para \(k \to \infty\), ou, equivalentemente, \(P(k) = Ae^{\gamma k} (1+\epsilon(k)) \to 1\), para \(\epsilon(k) \to 0\).

    3 - Diz-se que uma função \(P: \mathbb{N} \to \mathbb{C}\), é \textbf{Assintoticamente Linearizada na escala logarítmica} se
    \(\displaystyle \lim_{k \to \infty} \{\log|P(k)| - (\alpha +\gamma k)\} = 0\), para algum \(\alpha, \gamma\).

    4 - Diz-se que uma Relação funcional \(V = f(X)\) pode ser \textbf{Linearizada} (exatamente) se existirem funções inversíveis \(v = \psi(V)\) e \(x = \varphi(X)\) de tal forma que \(v = ax + b\), em algum domínio.

    5 – Diz-se que uma Relação funcional \(v = f(x)\) pode ser \textbf{Linearizada assintótica e localmente} nas vizinhanças de \(x = 0\) se \(v = a + bx + o(x)\) para algum \(a, b\). (Obs: Segundo Leibniz, uma função \(h(x)\) é dita um infinitésimo de ordem menor do que \(x\), e escreve-se, \(o(x)\) se for possível representá-la na forma \(h(x) = x \epsilon(x)\), onde \(\displaystyle \lim_{x \to 0} \epsilon(x) = 0\).

    3a - Considere uma Tabela de dados demográficos representada pela função \(P: \mathbb{N} \to \mathbb{C}\), cuja população quando medida na escala logarítmica na forma \(p(k) = \log(P(k))\), exibe um gráfico aproximadamente linear (isto é, \(p(k) = (\alpha + \beta k) + \epsilon\), com \(\epsilon \approx 0\), para alguma faixa de valores de \(k\)). Mostre como esta Dinâmica Populacional pode ser considerada aproximadamente Malthusiana nesta faixa de valores de \(k\).

    3b - Descreva o Método Numérico de Gauss (``mínimos quadrados'') comumente utilizado para determinar a reta que ``melhor aproxima'' uma Tabela de dados e descreva como este Método pode ser utilizado para a formulação de um Modelo Malthusiano.

    3c - Considere uma População medida na escala logarítmica \(\log P(k) = p(k)\). Mostre que uma aproximação linear \textbf{assintótica} na escala logarítmica  de uma população (isto é, \(\log P(k) - (\gamma k + \beta) \to 0\), para \(k \to \infty\)) \textbf{não} implica em um Modelo Malthusiano, mas apenas um Modelo Assintoticamente Malthusiano. (Sugestão: veja o próximo exercício).

    3d - Mostre \textbf{quando} uma população \(P(k)\) descrita pelo Modelo de Fibonacci é Malthusiana e \textbf{quando} ela é \textit{apenas} assintoticamente Malthusiana. (Sugestão: Analise as possíveis soluções a depender das condições iniciais).

    3e - Considere uma função ``\textbf{racional bilinear}'' \(V = \dfrac{AX}{CX+D}\). Mostre que é possível ''linearizar exatamente'' a relação entre as variáveis \(V\) e \(X\) tomando transformações \(v = \dfrac{1}{V}\) e \(x = \dfrac{1}{X}\), de tal forma que entre as ``novas variáveis'' resulte uma relação funcional de primeiro grau (\(v = a + bx\) (``\textit{linear}'').

    3f - Mostre que qualquer função diferenciável nas vizinhanças da origem pode ser localmente linearizada e vice-versa.

\solucao{


\subsection*{3a}

Considere a função \(p(k) = \log(P(k))\), onde \(P(k)\) é, também, uma função que associa dados de uma tabela demográfica \(k \mapsto P(k)\).

Como \(p(k) = \log(P(k)) \Rightarrow P(k) = e^{p(k)}\).

Tomando \(p(k) = \alpha + \gamma k + \epsilon\), uma função cujo gráfico é aproximadamente uma reta, temos:
\[P(k)
= e^{p(k)}
= e^{\alpha + \gamma k + \epsilon}
\]

Fazendo \(\epsilon \to 0\), temos:
\[P(k) 
\approx e^{\alpha + \gamma k}
= e^{\alpha} \cdot e^{\gamma k}
= A \cdot e^{\gamma k}
\]

\subsection*{3b}


Seja $d_{k} = f(x_{k})-g(x_{k})$ o desvio existente entre as imagens de $f$ e $g$ em $x_{k}$.

O método dos mínimos quadrados consiste em escolher os coeficientes $\alpha_j$, $j= 1, \ldots, m$ de tal forma que a soma dos quadrados dos desvios seja mínima, isto é:
$$\displaystyle\sum_{k=1}^{n} d_{k}^2 = \displaystyle\sum_{k=1}^{n} [f(x_{k}) - g(x_{k})]^2 \mbox{ é mínimo}.$$

Assim, os coeficientes $\alpha_j$, que fazem com que $g(x)$ se aproxime ao máximo de $f(x)$, são os que minimizam a função:
$$F(\alpha_1, \ldots, \alpha_m) = \displaystyle\sum_{k=1}^{n} [f(x_{k}) - g(x_{k})]^2 = \displaystyle\sum_{k=1}^{n} \left[f(x_{k}) - \displaystyle\sum_{j=1}^{m} \alpha_j g_j(x_{k})\right]^2.$$

Para isto, é necessário que as $m$ derivadas parciais de $F$ de primeira ordem se anulem, ou seja:
$$\dfrac{\partial F}{\partial \alpha_j}(\alpha_1, \ldots, \alpha_m) = 0, j = 1, \ldots, m,$$
ou seja,
$$\dfrac{\partial F}{\partial \alpha_j}(\alpha_1, \ldots, \alpha_j) = 2 \cdot \displaystyle\sum_{k=1}^{n} \left[f(x_{k}) - \displaystyle\sum_{j=1}^{m} \alpha_j g_j(x_{k})\right] \cdot \left[-g_j(x_{k})\right] = 0,\ j = 1, 2, \ldots, m.$$


Considere, agora,
\[\begin{array}{rcl}
N^t = \left[\begin{array}{cccc} n_1 & n_2 & \ldots & n_N \end{array}\right]^t \mbox{ (Pontos de entrada) e } \\
T^t = \left[\begin{array}{cccc} t_1 & t_2 & \ldots & t_N\end{array}\right]^t \mbox{ (Pontos de saída)}
\end{array}\]
%de um sistema \(T = N\theta\)

Queremos obter um \(Z \sim T\) e, se \(Z\) é linear, temos:
\begin{eqnarray*}
Z
= \theta_1 N + \theta_2 
= \left[\begin{array}{c} n_1\theta_1+\theta_2 \\ n_2\theta_1+\theta_2 \\ \vdots \\ n_N\theta_1+\theta_2 \end{array}\right] 
= \underbrace{\left[\begin{array}{cc} n_1 & 1 \\ n_2 & 1 \\ \vdots \\ n_N & 1 \end{array}\right]}_{\overline{N}}
\underbrace{\left[\begin{array}{c} \theta_1 \\ \theta_2 \end{array}\right]}_{\Theta}
\end{eqnarray*}

Vamos minimizar a função
\[E(\Theta)
= (T-Z)^2 = (T-\overline{N}\Theta)^2
\]
e, para tal, determinemos:
\[
\dfrac{\partial E}{\partial \Theta}
= \dfrac{\partial E}{\partial Z}\ \dfrac{\partial Z}{\partial \Theta}.\]

Mas
\[
\dfrac{\partial E}{\partial Z}
= \dfrac{\partial}{\partial Z} (T-Z)^2 = -2 \underbrace{(T-Z)}_{N \times 1}
\]
e
\[
\dfrac{\partial E}{\partial \Theta}
= \dfrac{\partial}{\partial \Theta} (\overline{N}\Theta) =  \underbrace{\overline{N}^t}_{2 \times N}.
\]

Segue que
\[0
= \dfrac{\partial E}{\partial \Theta}
= -2 \overline{N}^t (T-Z)
= -2 \overline{N}^t (T-\overline{N}\Theta),
\]
ou seja,
\[
\overline{N}^t T = \overline{N}^t \overline{N} \Theta
\]
implicando em
\[
\Theta = (\overline{N}^t\ \overline{N})^{-1} \overline{N}^t T.
\]
%(ver \href{https://www.youtube.com/watch?v=txnrFZG7Ugs&ab_channel=LeonardoOlivi}{Youtube})


Os valores de \(\theta_1\) e \(\theta_2\), após algumas contas, são dados por:
\[\begin{array}{rcl}
\theta_1 &=& \dfrac{\displaystyle\sum_{k=1}^{N} n_{k} \sum_{k=1}^{N} t_{k} - N \sum_{k=1}^{N} n_{k}\ t_{k}}{\displaystyle\left(\sum_{k=1}^{N} n_{k}\right)^2 - N \sum_{k=1}^{N} n_{k}^2} \\
\theta_2 &=& \dfrac{\displaystyle\sum_{k=1}^{N} t_{k} - \theta_1 \sum_{k=1}^{N} n_{k}}{N}
\end{array}
\]

Assim, ao tomarmos \(\log(Z) = \theta_1N+\theta_2 \Rightarrow Z = A \exp(\theta_1N)\)


\subsection*{3c}

\subsection*{3d}

\subsection*{3e}
Seja
\[V = \dfrac{AX}{CX+D}.\]

Efetuando as mudanças de variáveis \(V = \dfrac{1}{v}\) e \(X = \dfrac{1}{x}\), obtemos:
\[\dfrac{1}{v} = \dfrac{A\dfrac{1}{x}}{C\dfrac{1}{x}+D}
\Rightarrow
v = \dfrac{D}{A} x + \dfrac{C}{A}, A \ne 0
\Rightarrow
v = a x + b
.\]


\subsection*{3f}

Se \(f\) é diferenciável, pelo Teorema do Valor Médio, existe \(c \in V_0 = (-\delta, \delta)\) tal que
\[f'(c) = \dfrac{f(x) - f(0)}{x}, x \in V_0.\]
Logo,
\[f(x) = f'(c)x+f(0).\]

Por outro lado, se \(g\) é uma reta secante ao gráfico de \(f\), com pontos de interseção \(A(-\delta, f(-\delta))\) e \(B(\delta, f(\delta))\), então:
\[g(x) = m(x+\delta)+f(-\delta).\]

Como \(f\) é diferenciável, portanto, contínua, e \(0 \in (-\delta, \delta)\), temos:
\[\displaystyle \lim_{\delta \to 0} |(0,f(0)) - (0,g(0))|
= \lim_{\delta \to 0} f(0)-g(0)
= \lim_{\delta \to 0} f(0)-(-m\delta-f(-\delta))
= 0.
\]

Como queríamos demonstrar.
}

\section*{Questão 04}
\addcontentsline{toc}{section}{\textcolor{green}{Questão 04}}


Tempo Médio (\textit{Aritmético}) de Sobrevivência

    4a - Defina Média Aritmética Ponderada \(M_A(a_1, \ldots, a_N)\) para uma sequência de dados numéricos \(a_k > 0\). Discuta a razão de se dizer que uma Média Aritmética \(A\) é \textbf{uma única} informação numérica \textbf{populacional} que substitui (reduzindo) um conjunto (Tabela) de \textbf{várias} informações numéricas \textbf{individuais}, \(a_k\). Argumente com base nesta distinção sobre a (usual) insensatez de se afirmar que um \textbf{casal} brasileiro tem em média, por exemplo, \(1,44\) filhos.

    4b - Segundo um Teorema de Kolmogorov-Nagumo (1933) todas as ``Médias'' sobre uma sequência de dados numéricos \(a_k > 0\) ({\small conceito que pode ser facilmente definido por algumas poucas propriedades bem características}) são da forma \(M_\phi(a_1, \ldots, a_N) = \phi^{-1}(M_A(\phi(a_1), \ldots, \phi(a_N)))\), onde \(\phi\) é uma função real estritamente convexa inversível e \(M_A\) é uma Média Aritmética. Mostre a veracidade desta afirmação com respeito às médias, Aritmética, Harmônica, Geométrica e Quadrática.

    4c - Interprete o Método de Quadrados Mínimos de Gauss em termos de uma Média Quadrática.

    4d - Dada uma sequência de números positivos \(a = \{a_k\}\) obtenha, argumentando geometricamente, uma relação de ordem entre suas Médias Aritmética, \(M_A(a)\), Harmônica, \(M_h(a)\), Geométrica, \(M_g(a)\) e Quadrática, \(M_2(a)\). (Utilize uma sequência de apenas dois números para seus argumentos).

    4e - Mostre que, a depender da escolha da média de Kolmogorov-Nagumo, pode-se dizer que a média de filhos de um casal brasileiro pode ser qualquer número real entre \(m = \min \{a_k\}\) e \(M = \max\{a_k\}\), onde \(a_k = \mbox{``Número de casais com } k \mbox{ filhos''}\).


\solucao{

\subsection*{4a - }

Considere um conjunto de dados numéricos
\[A = \{a_i; i =1, 2, \ldots, n\},\]
em que cada \(a_i \in A\) possui frequência \(f_i\).

Se a característica a ser mantida quando substituímos cada valor \(a_i \in A\) por \(M_A\) é a soma dos elementos de \(A\), obtemos a \textbf{média aritmética}.

A média aritmética \(M_A\) é um valor tal que
\[\begin{array}{rcl}
a_1 + \ldots + a_1 + a_2 + \ldots + a_2 + \ldots + a_n + \ldots + a_n &=& M_A + \ldots + M_A \\
\underbrace{a_1 + \ldots + a_1}_{\times f_1} + \underbrace{a_2 + \ldots + a_2}_{\times f_2} + \ldots + \underbrace{a_n + \ldots + a_n}_{\times f_n} &=& \underbrace{M_A + \ldots + M_A}_{\times (f_1 + f_2 + \ldots + f_n)},
\end{array}\]
ou seja,
$$f_1 \cdot a_1 + f_2 \cdot a_2 + \ldots + f_n \cdot a_n = (f_1 + f_2 + \ldots + f_n) \cdot M_A.$$

Segue que
\begin{equation}\label{map}
M_A
= \dfrac{f_1 \cdot a_1 + f_2 \cdot a_2 + \ldots + f_n \cdot a_n}{f_1 + f_2 + \ldots + f_n}
= \dfrac{\displaystyle \sum_{i=1}^{n} \{a_i \cdot f_i\}}{\displaystyle \sum_{i=1}^{n} f_i}.
\end{equation}

%Em certas situações, os dados numéricos que queremos sintetizar possuem diferentes graus de importância. Utiliza-se, portanto,

Pode-se entender a frequência \(f_i\) como um ``peso'' (ou ponderação) ao valor do elemento \(a_i\), ou seja, quando os dados aparecem na forma de uma distribuição de frequências, os ponderadores são as frequências absolutas.

\textbf{Observação}: Esta média aritmética é também chamada aritmética ponderada. As frequências com que aparecem determinados elementos de um conjunto (pesos ou ponderações) assumem um grau de ``importância'' para cada valor.


%Podemos observar que a relação da equação ({\ref{map}}) é válida para dados tabulados não agrupados em classes.

Caso \(f_1= \ldots = f_n = 1\), temos que a \textbf{média aritmética} para o conjunto \(A\) é:

\begin{equation}\label{ma}
M_A = \dfrac{a_1 + a_2 + \ldots + a_n}{n} = \dfrac{\displaystyle \sum_{i=1}^{n} a_i}{n}
\end{equation}


Se o \textbf{produto} dos elementos de \(A\) é a característica a ser mantida, obtemos a \textbf{média geométrica}.

Seja \(f_i\) a frequência atribuída ao respectivo valor que a variável \(a_i \in A\) assume, \(a_i \in \mathbb{R}_+^\ast\). A \textbf{média geométrica} dos \(n\) números positivos do conjunto \(A\) é um valor positivo \(M_g\) tal que
\[a_1^{f_1} \cdot a_2^{f_2} \cdot \ldots \cdot a_k^{f_k} = M_g \cdot M_g \cdot \ldots \cdot M_g = M_g^n, \mbox{ em que } n = \displaystyle \sum_i^k f_i.\]
Logo,
\begin{equation}\label{mgp}
M_g = \sqrt[n]{a_1^{f_1} \cdot a_2^{f_2} \cdot \ldots \cdot a_k^{f_k}} = \sqrt[n]{\prod_{i=1}^{k} a_i^{f_i}}
\end{equation}

Podemos entender a frequência com que cada elemento aparece, como sendo um grau de importância para a variável.

Caso \(f_1= \ldots = f_n = 1\), a \textbf{média geométrica} dos \(n\) números positivos e não nulos do conjunto \(A\) é um valor positivo \(M_g\) tal que
$$a_1 \cdot a_2 \cdot \ldots \cdot a_n = M_g \cdot M_g \cdot \ldots \cdot M_g = M_g^n.$$
Logo,
\begin{equation}
M_g = \sqrt[n]{a_1 \cdot a_2 \cdot \ldots \cdot a_n} = \sqrt[n]{\prod_{i=1}^{n} a_i}
\end{equation}


Se a soma dos inversos dos elementos de \(A\) é a característica a ser conservada, obteremos a \textbf{média harmônica}.

Seja \(f_i\) o peso atribuído ao respectivo valor que a variável positiva e não nula \(a_i \in A\) assume. A \textbf{média harmônica} dos \(n\) números positivos do conjunto \(A\) é um valor positivo \(M_h\) tal que
\[\dfrac{f_1}{a_1} + \dfrac{f_2}{a_2} + \ldots + \dfrac{f_n}{a_n} = \dfrac{1}{M_h} + \dfrac{1}{M_h} + \ldots + \dfrac{1}{M_h} = \dfrac{\displaystyle \sum_{i=1}^{n} f_i}{M_h}.\]
Logo,
\begin{equation}
M_h = \dfrac{\displaystyle \sum_{i=1}^{n} f_i}{\dfrac{f_1}{a_1} + \dfrac{f_2}{a_2} + \ldots + \dfrac{f_n}{a_n}} = \dfrac{\displaystyle \sum_{i=1}^{n} f_i}{\displaystyle \sum_{i=1}^{n} \dfrac{f_i}{a_i}}.
\end{equation}


\textbf{Observação}: A possibilidade de não existirem as médias geométrica e harmônica é evitada, uma vez que estas só foram definidas para números positivos.

%Considere o conjunto \(A = \{a_1, a_2, \ldots, a_n\}\), em que cada \(a_i \neq 0\) aparece \(f_i\) vezes. Então,

A \textbf{média quadrática} é um valor \(M_2\) tal que
\[
a_1^2 \cdot f_1 + a_2^2 \cdot f_2 + \ldots + a_n^2 \cdot f_n
= \underbrace{M_2^2 + M_2^2 + \ldots + M_2^2}_{\times (f_1+f_2+\ldots+f_n)}
= M_2^2 \cdot (f_1+f_2+\ldots+f_n).\]
Logo,
\[
M_2^2
= \dfrac{a_1^2 \cdot f_1 + a_2^2 \cdot f_2 + \ldots + a_n^2 \cdot f_n}{f_1+f_2+\ldots+f_n}
= \dfrac{\displaystyle \sum_{i=1}^{n} a_i^2 \cdot f_i}{\displaystyle \sum_{i=1}^{n} f_i}.
\]
Portanto,
\begin{equation}\label{m2}
M_2
= \sqrt{\dfrac{\displaystyle \sum_{i=1}^{n} a_i^2 \cdot f_i}{\displaystyle \sum_{i=1}^{n} f_i}}.
\end{equation}


%{\red continuar}


\subsection*{4b - }

Seja \(\mathcal{A} = \{a_1, a_2, \ldots, a_N\}\), com \(a_k>0, \forall\ k = 1, \ldots, N\). O que devemos mostrar é que existe uma função estritamente convexa inversível \(\phi\) tal que
\[M_\phi(\mathcal{A}) = \phi^{-1}\left(M_A(\phi(a_1), \ldots, \phi(a_N))\right)\]
é válida para as médias aritmética \(M_A\), harmônica \(M_H\), Geométrica \(M_G\) e quadrática \(M_2\).

Para a média aritmética \(M_A\) de \(\mathcal{A}\), temos:
\[
M_A(\mathcal{A})
=  \dfrac{1}{N} \displaystyle\sum_{k=1}^{N} a_{k}
=  \displaystyle\sum_{k=1}^{N} \dfrac{a_{k}}{N}
\]

Se fizermos \(\phi(a_k) = a_k\), temos \(\phi^{-1}(a_k) = a_k\). O que nos leva a:
\[\begin{array}{rcl}
M_A(\mathcal{A})
&=& \displaystyle\sum_{k=1}^{N} \dfrac{a_{k}}{N} \\
&=& \displaystyle\sum_{k=1}^{N} \dfrac{\phi(a_{k})}{N} \\
&=& M_A(\phi(a_1), \ldots, \phi(a_N)) \\
&=& \phi^{-1}(M_A(\phi(a_1), \ldots, \phi(a_N)) \\
%&=& M_\phi(\mathcal{A})
\end{array}\]

\textbf{Observação}: A função identidade é inversível, estritamente monótona e convexa.

No caso da média harmônica \(M_H\) de \(\mathcal{A}\), temos:
\[
M_H(\mathcal{A})
=  \left(\dfrac{1}{N} \displaystyle\sum_{k=1}^{N} \dfrac{1}{a_{k}}\right)^{-1}.
\]

Se fizermos \(\phi(a_k) = \dfrac{1}{a_k}\), temos \(\phi^{-1}(a_k) = \dfrac{1}{a_k}\). O que nos leva a:
\[\begin{array}{rcl}
M_H(\mathcal{A})
&=& \left(\dfrac{1}{N} \displaystyle\sum_{k=1}^{N} \phi(a_{k})\right)^{-1} \\
&=& M_A\left(\phi(a_1), \ldots, \phi(a_k)\right)^{-1} \\
&=& \phi^{-1}\left(M_A(\phi(a_1), \ldots, \phi(a_k))\right)
\end{array}\]

\textbf{Observação}: A função \(\phi\) é inversível, estritamente monótona e estritamente convexa.




No caso da média geométrica \(M_G\) de \(\mathcal{A}\), temos:
\[
M_G(\mathcal{A})
= \displaystyle\sqrt[N]{\prod_{k=1}^{N} a_{k}}.
\]

Se fizermos \(\phi(a_k) = \ln(a_k)\), temos \(\phi^{-1}(a_k) = \exp(a_k)\). O que nos leva a:
\[\begin{array}{rcl}
M_G(\mathcal{A})
&=& \displaystyle \sqrt[N]{\prod_{k=1}^{N} \exp(\phi(a_{k}))} \\
&=& \displaystyle \exp\left(\dfrac{1}{N} \sum_{k=1}^{N} \phi(a_k)\right) \\
&=& \phi^{-1}(M_A(\phi(a_1), \ldots, \phi(a_N)))
\end{array}\]

\textbf{Observação}: A função \(\phi\) é inversível, estritamente monótona, mas não é convexa.


No caso da média quadrática \(M_2\) de \(\mathcal{A}\), temos:
\[
M_2(\mathcal{A})
= \displaystyle \sqrt{\dfrac{1}{N}\sum_{k=1}^{N} a_{k}^2}.
\]

Se fizermos \(\phi(a_k) = a_k^2\), temos \(\phi^{-1}(a_k) = \sqrt{a_k}\). O que nos leva a:
\[\begin{array}{rcl}
M_2(\mathcal{A})
&=& \displaystyle \sqrt{\dfrac{1}{N}\sum_{k=1}^{N} \phi(a_{k})} \\
&=& \displaystyle \sqrt{M_A(\phi(a_1), \ldots, \phi(a_N))} \\
&=& \phi^{-1}(M_A(\phi(a_1), \ldots, \phi(a_N)))
\end{array}\]

\textbf{Observação}: A função \(\phi\) é inversível, estritamente monótona e estritamente convexa.





\subsection*{4c - } %{\red fazer}


Seja \(d_i = f(a_i)-g(a_i)\) o desvio existente entre as imagens de \(f\) e \(g\) em \(a_i\).

O método dos mínimos quadrados consiste em escolher os coeficientes \(\alpha_j\), \(j = 1, \ldots, m\), de tal forma que a soma dos quadrados dos desvios seja mínima, isto é:
\begin{equation}\label{eq:somaquadraticadosdesvios}
\displaystyle\sum_{i=1}^{n} d_i^2 = \displaystyle\sum_{i=1}^{n} [f(a_i) - g(a_i)]^2 \mbox{ é mínimo}.
\end{equation}

Assim, os coeficientes \(\alpha_j\), que fazem com que \(g(a)\) se aproxime ao máximo de \(f(a)\), são os que minimizam a função:
\[F(\alpha_1, \ldots, \alpha_m) = \displaystyle\sum_{i=1}^{n} [f(a_i) - g(a_i)]^2 = \displaystyle\sum_{i=1}^{n} \left[f(a_i) - \displaystyle\sum_{j=1}^{m} \alpha_j g_j(a_i)\right]^2.\]

Para isto, é necessário que as \(m\) derivadas parciais de \(F\) de primeira ordem se anulem, ou seja:
\begin{equation}\label{eq:derivadadeordemumanulada}
\dfrac{\partial F}{\partial \alpha_j}(\alpha_1, \ldots, \alpha_m) = 0, j = 1, \ldots, m,
\end{equation}
ou seja,
\[\dfrac{\partial F}{\partial \alpha_j}(\alpha_1, \ldots, \alpha_j) = 2 \cdot \displaystyle\sum_{i=1}^{n} \left[f(a_i) - \displaystyle\sum_{j=1}^{m} \alpha_j g_j(a_i)\right] \cdot \left[-g_j(a_i)\right] = 0,\ j = 1, 2, \ldots, m.\]

Observa-se que ao multiplicarmos o primeiro membro da equação \eqref{eq:somaquadraticadosdesvios} por \(\left(\displaystyle\sum_{i=1}^{N} f_i\right)^{-1}\), em que \(f_i\) é a frequência com que \(a_i\) aparece no conjunto \(A\), obtendo-se:
\begin{equation}\label{eq:mediaquadraticadosdesvios}
\dfrac{\displaystyle \sum_{i=1}^{N} d_i^2}{\displaystyle \sum_{i=1}^{N}f_i} = \displaystyle \sum_{i=1}^{N} \dfrac{d_i^2}{\displaystyle \sum_{i=1}^{N} f_i},
\end{equation}
a média dos desvios quadráticos, em nada se altera a condição em \eqref{eq:derivadadeordemumanulada}.









\subsection*{4d - Relação entre as Médias}

Se \(a_1, a_2, \ldots, a_n\) são \(n\) números positivos e \(M_h\), \(M_g\), \(M_A\) e \(M_2\) são suas médias harmônica, geométrica, aritmética e quadrática, respectivamente, então
\[M_h \le M_g \le M_h \le M_2.\]

Além disso, duas quaisquer dessas médias serão iguais se, e somente se, \(a_1 = a_2 = \ldots = a_n\).



\subsubsection*{Análise 1: Geométrica}

Considere o triângulo \(\triangle ABP\), inscrito numa semicircunferência de um círculo de centro na origem do sistema cartesiano e de raio \(r\). Considere, ainda, que (sem perda da generalidade) a semicircunferência seja a que possui pontos nos I e II quadrantes, as coordenadas dos vértices sejam: \(A(-r,0)\), \(B(r,0)\) e \(P(x,y)\) um ponto arbitrário (Ver figura).


\begin{center}
\captionof{figure}{Interpretação geométrica das médias}

{\centering
\SpecialCoor

\psset{unit=0.8cm}
\begin{pspicture}(-8,-1)(8,8)
\pnode(0,0){O}
\pnode(-6.5,0){A}
\pnode(6.5,0){B}
\pnode(0,6.5){C}
\pnode(4.576923077,4.615384615){P}
\pnode(4.576923077,0){P1}
\pnode(2.269311568,2.288381413){P2}
%
\pscircle*[linecolor=red](O){0.15}\uput[dl](O|,-0.25){\(O\)}
\pscircle*[linecolor=red](A){0.15}\uput[dl](A){\(A\)}
\pscircle*[linecolor=red](B){0.15}\uput[dr](B){\(B\)}
\pscircle*[linecolor=red](C){0.15}\uput[ul](C){\(C\)}
\pscircle*[linecolor=red](P){0.15}\uput[ur](P){\(P(x,y)\)}
\pscircle*[linecolor=red](P1){0.15}\uput[d](P1|,-0.25){\(P_1\)}
\pscircle*[linecolor=red](P2){0.15}\uput[ul](P2){\(P_2\)}
%
\psaxes[Dx=10,Dy=10,linecolor=red]{->}(0,0)(-8,-1)(8,8)
%
\psarc{-}(O){6.5}{0}{180}
%
\psarc{<->}(A){1.5}{0}{22.6}\rput(A|,0){\rput(2;11.3){\(\alpha\)}}
\psarc{<->}(B){0.65}{112.6}{180}\rput(B|,0){\rput(0.9;141.3){\(\beta\)}}
%
\psarc{<->}(P){1.55}{270}{292.6}\rput(P){\rput(1.95;280.3){\(\alpha\)}}
\psarc{<->}(P){1.55}{202.6}{270}\rput(P){\rput(1.95;235.3){\(\beta\)}}
%
\psarc{*-*}(P){0.55}{223.1}{270}\rput(P){\rput(0.85;245.3){\(\gamma\)}}
%
\psarc{*-*}(O){0.55}{0}{42}\rput(O){\rput(0.85;21){\(\theta\)}}
%
\psline(A)(P)(B)
\psline(P)(P1)
\psline(P)(O)
\psline(P1)(P2)
\psline(P1)(C)
%
\def\marcacaoperp{\psline(0,0.5)(0.5,0.5)(0.5,0)
\pscircle*(0.25,0.25){0.075}
}
%
\rput(O){\psrotate(0,0){90}{\marcacaoperp}}
\rput(P2){\psrotate(0,0){-135}{\marcacaoperp}}
\rput(P1){\marcacaoperp}
%
\psbrace[
%singleline,
%linestyle=dashed,
linewidth=.75pt,
braceWidth=.75pt,
braceWidthOuter=5pt,
braceWidthInner=5pt,
rot=0,
ref=tC,
nodesepB=3pt,
](A)(P1){$a$}
%
\psbrace[
%singleline,
%linestyle=dashed,
linewidth=.75pt,
braceWidth=.75pt,
braceWidthOuter=5pt,
braceWidthInner=5pt,
rot=0,
ref=tC,
nodesepB=3pt
](P1)(B){$b$}
\end{pspicture}
}





\fonte{Figura elaborada pelo autor}
\end{center}

Os triângulos \(\triangle AP_1P\) e \(\triangle PP_1B\) são semelhantes (Caso \(LAA_0\)). Dessa forma,
\[B\hat{P}P_1 = P\hat{A}B = \alpha \mbox{ e } P_1\hat{P}A = P\hat{B}A = \beta.\]
Além disso, \(\alpha + \beta = 90^\circ\) (Pela soma dos ângulos internos de um triângulo), implicando que o triângulo \(\triangle APB\) é retângulo em \(P\).

Verifica-se, facilmente, que
\[2r = a+b \Rightarrow r = \dfrac{a+b}{2},\]
a \textbf{média arimética} entre as medidas \(a\) e \(b\) dos respectivos comprimentos das projeções dos catetos \(PA\) e \(PB\) sobre a hipotenusa do triângulo \(\triangle APB\).

Da semelhança entre os triângulos \(\triangle AP_1P\) e \(\triangle PP_1B\), podemos também extrair:
\[\dfrac{\overline{AP_1}}{\overline{PP_1}} = \dfrac{\overline{PP_1}}{\overline{P_1B}}
\Rightarrow
\dfrac{a}{y} = \dfrac{y}{b}
\Rightarrow
y = \sqrt{a \cdot b}
\]
a \textbf{média geométrica} entre as medidas \(a\) e \(b\).% dos respectivos comprimentos das projeções dos catetos \(\overline{PA}\) e \(\overline{PB}\) sobre a hipotenusa do triângulo \(\triangle APB\).


Considere, agora, o ponto \(P_2 \in OP\) de modo que \(P_2P_1 \perp OP\). Da semelhança entre os triângulos \(\triangle PP_2P_1\) e \(\triangle PP_1O\) (caso \(LAA_O\)), temos:
\[\dfrac{\overline{AP_2}}{\overline{PP_1}} = \dfrac{\overline{PP_1}}{\overline{OP}}
\Rightarrow
\dfrac{\overline{AP_2}}{y} = \dfrac{y}{r}
\Rightarrow
\overline{AP_2} = \dfrac{a \cdot b}{\dfrac{a+b}{2}} = \dfrac{1+1}{\dfrac{1}{a}+\dfrac{1}{b}}
\]
a \textbf{média harmônica} entre as medidas \(a\) e \(b\).


Já, pelo triângulo \(\triangle P_1OC\), retângulo em \(O\), as medidas dos seus catetos em função de \(a\) e \(b\) são obtidas a seguir:
\[\overline{OC} = r = \dfrac{a+b}{2} \mbox{ e } \overline{P_1O} = r - b = \dfrac{a+b}{2}-b = \dfrac{a-b}{2}\]

Aplicando em \(\triangle P_1OC\) o Teorema de Pitágoras, temos:
\[\begin{array}{rcl}
\overline{P_1C}^2
&=& \overline{P_1O}^2+\overline{OC}^2 \\
&=& \left(\dfrac{a-b}{2}\right)^2+\left(\dfrac{a+b}{2}\right)^2 \\
&=& \dfrac{a^2+b^2}{2},
\end{array}\]
implicando em
\[\overline{P_1C} = \sqrt{\dfrac{a^2+b^2}{2}}\]
a \textbf{média quadrática} entre as medidas \(a\) e \(b\).



\subsubsection*{Análise 2: Gráfica}


Considere o conjunto \(A = \{2, 8\}\). Suas médias harmônica, geométrica, aritmética e quadrática, respectivamente, são:
\[\begin{array}{rcl}
M_h &=& \dfrac{1+1}{\dfrac{1}{2}+\dfrac{1}{8}} = \ca{2/(1/2+1/8)} \\
M_g &=& \sqrt{2 \cdot 8} = 4 \\ %\ca{(2*8)^(1/2)} \\
M_A &=& \dfrac{\ca{2+8}}{2} = \ca{(2+8)/2} \\[0.3cm]
M_2 &=& \sqrt{\dfrac{2^2+8^2}{1+1}} \approx \ca{((2^2+8^2)/2)^(1/2)}
\end{array}\]

Graficamente, temos:

\begin{minipage}[!h]{\textwidth}\centering
\psset{yunit=0.7cm}
\begin{pspicture}(-1,-1)(6,8)
\psaxes[Dx=10,Dy=10,linecolor=red]{->}(0,0)(-1,-1)(6,7)
\uput[d](6,0){\scriptsize Médias}
\uput[l](0,7){\scriptsize Valores}

\psline[linewidth=4pt](1,0)(1,3.2)
\uput[d](1,0){\(M_h\)}\uput[l](0,3.2){\(\scriptstyle 3,20\)}

\psline[linewidth=4pt](2,0)(2,4)
\uput[d](2,0){\(M_g\)}\uput[l](0,4){\(\scriptstyle 4,00\)}

\psline[linewidth=4pt](3,0)(3,5)
\uput[d](3,0){\(M_A\)}\uput[l](0,5){\(\scriptstyle 5,00\)}

\psline[linewidth=4pt](4,0)(4,5.83)
\uput[d](4,0){\(M_2\)}\uput[l](0,5.83){\(\scriptstyle 5,83\)}

\end{pspicture}
\end{minipage}


\subsubsection*{4e}



Seja \(\mathcal{A} = \{a_k\},\ k = 1, 2, \ldots, N\), onde \(a_k = \mbox{ ``Número de casais com } k \mbox{ filhos''}\). Considere, ainda, \(M_\phi\) uma média de Kolmogorov-Nagumo dos elementos de \(\mathcal{A}\) onde \(\phi\) é uma função invertível, estritamente crescente.


Suponha que, para todo \(k\), temos \(a_k < M_\phi\). Dessa forma,
\[\begin{array}{rcl}
\phi(a_k) < \phi(M_\phi)
&\Rightarrow&
\displaystyle\sum_{k=1}^{N}\phi(a_k) < \sum_{k=1}^{N}\phi(M_\phi) = N \phi(M_\phi) \\
&\Rightarrow&
\dfrac{1}{N} \displaystyle\sum_{k=1}^{N}\phi(a_k) < \phi(M_\phi) \\
&\Rightarrow&
\phi^{-1}\left(\dfrac{1}{N} \displaystyle\sum_{k=1}^{N}\phi(a_k)\right) < M_\phi \\
&\Rightarrow&
\phi^{-1}\left(M_A(\phi(a_1), \ldots, \phi(a_N))\right) < M_\phi \\
&\Rightarrow&
M_\phi < M_\phi \mbox{ (absurdo!)}
\end{array}\]

Logo, \(\exists a_{\max} = \max\{a_k\} = M\) tal que \(M_\phi < M\).

De maneira análoga, mostramos que \(\exists a_{\min} = \min\{a_k\} = m\) tal que \(m < M_\phi\).

Portanto, para \(M_\phi\) podemos ter que a média de filhos de um casal brasileiro pode ser qualquer número real entre \(m\) e \(M\).


}

\section*{Questão 05}
\addcontentsline{toc}{section}{\textcolor{green}{Questão 05}}


Tempo Médio (Aritmético) de Sobrevivência de uma População

    \textbf{Definição}: Dado um Modelo populacional especificamente de mortalidade \(N(t)\) tal que \(\dfrac{dN}{dt} < 0\) e \(\displaystyle \lim_{t \to \infty} N(t) = 0\), diz-se que o valor (finito ou infinito) da integral \(\dfrac{1}{N_0} \displaystyle\int_{0}^{\infty} -t \dfrac{dN}{dt}\ dt\) é denominado Tempo Médio (Aritmético) de Sobrevivência da População.

    5a - \textbf{Argumente} sobre a motivação para que a expressão
    \(\dfrac{1}{N_0} \displaystyle\int_{0}^{\infty} -t \dfrac{dN}{dt}\ dt = \dfrac{1}{N_0} \displaystyle\int_{0}^{N_0} t \ dN\), que se refere a uma dinâmica \(N(t)\) decrescente de uma (Grande) população (sem natalidade e migração) inicialmente com \(N(0) = N_0\) indivíduos, possa \textbf{ser interpretada} como o tempo médio (aritmético) de sobrevivência desta população.

    5b - \textbf{Calcule} o Tempo Médio (Aritmético) de sobrevivência de uma população Malthusiana (isto é, descrita segundo o Modelo Newtoniano \(\dfrac{1}{N} \dfrac{dN}{dt} = -\mu,\ N(0) = N_0\)) e mostre que este valor \textbf{independe} de \(N_0\). \textbf{Discuta} o significado biológico deste resultado.

    5c - \textbf{Calcule} o tempo médio (aritmético) de sobrevivência de uma população cuja dinâmica de Mortalidade é descrita por uma função quase-polinomial \(N(t) = q(t)e^{-\mu t}\), onde \(q(t) = N_0 + \displaystyle \sum_{k = 1}^{m} a_k t^k\) é um polinômio e \(\mu > 0\). (Sugestão: Calcule explicitamente as integrais \(I(n) = \int_{0}^{\infty} t^n e^{-\mu t}\ dt\) recursivamente em \(n\) e utilizando integrações por partes)

    5d - O mesmo para \(N(t) = \dfrac{N_0}{t+1}\).


\solucao{

\subsection*{5a}

\subsection*{5b}

Considere o modelo Malthusiano de Mortalidade
\begin{equation}\label{eq:modelomathusianodemortalidade}
\dfrac{1}{N} \dfrac{dN}{dt} = -\mu,
\end{equation}
com condição inicial \(N(0) = N_0\).

A solução de \eqref{eq:modelomathusianodemortalidade} é obtida ao separar as variáveis, integrar indefinidamente o resultado e utilizar a sua condição inicial. A seguir, as passagens como citadas.

Separando as variáveis:
\[\dfrac{1}{N} \dfrac{dN}{dt} = -\mu \Rightarrow 
\dfrac{dN}{N} = -\mu dt.
\]

Integrando indefinidamente:
\[\displaystyle \int \dfrac{dN}{N} = -\int \mu dt
\Rightarrow
\log(N) = -\mu t + C \Rightarrow N(t) = e^{-\mu t}\ e^Ç.
\]

Utilizando a condição inicial:
\[N(0) = N_0 \Rightarrow e^C = N_0.\]

Portanto, temos:
\begin{equation}\label{eq:solucaomalthusianamortalidade}
N(t) = N_0\ e^{-\mu t}.
\end{equation}

O tempo médio de sobrevivência da população \(T_M\) é dado por:
\[
T_M = \dfrac{1}{N_0} \displaystyle\int_{0}^{\infty} -t \dfrac{dN}{dt}\ dt
\]

Para determinar \(T_M\), substituímos em sequência, as equações \eqref{eq:modelomathusianodemortalidade} e \eqref{eq:solucaomalthusianamortalidade}, na fórmula de \(T_M\), cancelamos a constante \(N_0\) e resolvemos uma integral imprópria, ou seja:
\begin{eqnarray*}
T_M
&=& \dfrac{1}{N_0} \displaystyle\int_{0}^{\infty} -t (-\mu N)\ dt \\
&=& \dfrac{\mu}{N_0} \displaystyle\int_{0}^{\infty} t\ 
N_0\ e^{-\mu t}\ dt \\
&=& 
\mu \displaystyle\int_{0}^{\infty} t\ e^{-\mu t}\ dt \\
&=& 
\mu \displaystyle \lim_{a \to \infty} \int_{0}^{a} t\ e^{-\mu t}\ dt \\
&=& \mu \displaystyle \lim_{a \to \infty} \left(t\ \dfrac{e^{-\mu t}}{-\mu} - \dfrac{1}{\mu} \left(\dfrac{e^{-\mu t}}{-\mu}\right) \right|_{0}^{a} \\
&=& \mu \displaystyle \lim_{a \to \infty} \left.\left(t + \dfrac{1}{\mu}\right) \dfrac{e^{-\mu t}}{-\mu} \right|_{0}^{a} \\
&=& \mu \displaystyle \lim_{a \to \infty} \left[\left(a + \dfrac{1}{\mu}\right) \dfrac{e^{-\mu a}}{-\mu} + \dfrac{1}{\mu^2} \right] \\
&=& \mu \left(\dfrac{1}{\mu^2} \right) = \dfrac{1}{\mu}
\end{eqnarray*}
onde a integral na quarta igualdade foi obtida utilizando o método de integração por partes.



\subsection*{5c}

O tempo médio \(T_M\) de sobrevivência de uma população cuja dinâmica de mortalidade é dada por:
\begin{equation}\label{eq:modelomortalidade}
N(t) = \left(N_0+\displaystyle\sum_{k=1}^{m} a_{k} t^{k}\right)\ e^{-\mu t},\ \mu > 0
\end{equation}
é dada por:
\begin{equation}\label{eq:temposobrevivencia}
T_M = \dfrac{1}{N_0} \displaystyle\int_{0}^{\infty} -t\ \dfrac{dN}{dt}\ dt
\end{equation}


A seguir, mostraremos o processo do cálculo da derivada de \eqref{eq:modelomortalidade} com respeito à variável \(t\).

\[\begin{array}{rcl}
\dfrac{dN}{dt}
&=& \dfrac{d}{dt}\left[\left(N_0+\displaystyle\sum_{k=1}^{m} a_{k} t^{k}\right)\ e^{-\mu t}\right] \\
&=& \dfrac{d}{dt}\left(N_0+\displaystyle\sum_{k=1}^{m} a_{k} t^{k}\right) \cdot e^{-\mu t} + \left(N_0+\displaystyle\sum_{k=1}^{m} a_{k} t^{k}\right) \cdot \dfrac{d}{dt}\left(e^{-\mu t}\right) \\
&=& \left(\displaystyle\sum_{k=2}^{m} k a_{k} t^{k-1}\right) \cdot e^{-\mu t} + \left(N_0+\displaystyle\sum_{k=1}^{m} a_{k} t^{k}\right) \cdot (-\mu) e^{-\mu t} \\
&=& e^{-\mu t} \cdot \left[
a_1-\mu\ N_0 + \left(\displaystyle\sum_{k=2}^{m} (k a_{k}-\mu a_{k-1}) t^{k-1}\right) - \mu a_m t^m,
\right]
\end{array}\]
implicando em
\[\begin{array}{rcl}
-t \dfrac{dN}{dt}
&=& e^{-\mu t} \cdot \left[
(\mu\ N_0-a_1) t + \left(\displaystyle\sum_{k=2}^{m} (\mu a_{k-1}-k a_{k}) t^{k}\right) + \mu a_m t^{m+1}\right]
\end{array}\]

Portanto,
\begin{eqnarray}
\nonumber
T_M
&=& \dfrac{1}{N_0} \displaystyle\int_{0}^{\infty} 
e^{-\mu t} \cdot \left[
(\mu\ N_0-a_1) t + \left(\displaystyle\sum_{k=2}^{m} (\mu a_{k-1}-k a_{k}) t^{k}\right) + \mu a_m t^{m+1}\right]
\ dt
\\ \nonumber
&=& 
\left(\mu -\dfrac{a_1}{N_0}\right) \displaystyle\int_{0}^{\infty} t e^{-\mu t}\ dt +
\dfrac{1}{N_0} \displaystyle\sum_{k=2}^{m} (\mu a_{k-1}-k a_{k}) \int_{0}^{\infty} t^{k} e^{-\mu t}\ dt \\ \label{eq:tmintegraisrecursivas}
&& +
\mu a_m \dfrac{1}{N_0}\displaystyle\int_{0}^{\infty} t^{m+1}\ e^{-\mu t}\ dt
\end{eqnarray}

Constatamos em \eqref{eq:tmintegraisrecursivas} que para determinar \(T_m\) é necessário encontrar integrais do tipo:
\begin{equation}\label{eq:integralIn}
\mathcal{I}(n) = \displaystyle\int_{0}^{\infty} t^{n}\ e^{-\mu t}\ dt. 
\end{equation}

Vamos provar que o valor de \(\mathcal{I}(n) = \dfrac{n!}{\mu^{n+1}}\) utilizando a indução sobre \(n\).

Para \(n=0\),
\begin{eqnarray}
\nonumber
\mathcal{I}(0)
&=& \displaystyle\int_{0}^{\infty} e^{-\mu t}\ dt \\ \nonumber
&=& \lim_{b_0 \to \infty} \displaystyle\int_{0}^{b_0} e^{-\mu t}\ dt \\ \nonumber
&=& \lim_{b_0 \to \infty}  \left.\dfrac{-1}{\mu} e^{-\mu t}\right|_{0}^{b_0} \\ \nonumber
&=& \lim_{b_0 \to \infty}  \dfrac{-1}{\mu} \left(e^{-\mu b_0} -1 \right) \\
\label{eq:resoltado_zero}
&=& \dfrac{1}{\mu} = \dfrac{0!}{\mu^{0+1}}.
\end{eqnarray}

Suponhamos que \(\mathcal{I}(n) = \dfrac{n!}{\mu^{n+1}}\). Vamos provar que \(\mathcal{I}(n+1) = \dfrac{(n+1)!}{\mu^{n+2}}\). De fato,
\begin{eqnarray}
\label{eq:passagem_I}
\mathcal{I}(n+1)
&=& \displaystyle\int_{0}^{\infty} t^{n+1}\ e^{-\mu t}\ dt \\
\label{eq:passagem_II}
&=& \displaystyle\lim_{b_{n+1} \to \infty}\int_{0}^{b_{n+1}} t^{n+1}\ e^{-\mu t}\ dt \\
\label{eq:passagem_III}
&=& 
\displaystyle\lim_{b_{n+1} \to \infty} \left(\left.t^{n+1} \dfrac{e^{-\mu t}}{-\mu}\right|_{0}^{b_{n+1}} - \int_{0}^{b_{n+1}} (n+1) t^{n} \dfrac{e^{-\mu t}}{-\mu}\ dt\right) \\
\label{eq:passagem_IV}
&=& 
\underbrace{\displaystyle\lim_{b_{n+1} \to \infty}  \dfrac{(b_{n+1})^{n+1}}{-\mu e^{\mu b_{n+1}}}}_{\mbox{tende a } 0}
+
\dfrac{n+1}{\mu} \cdot
\underbrace{\lim_{b_{n+1} \to \infty} \int_{0}^{b_{n+1}} t^{n} e^{-\mu t}\ dt}_{\mathcal{I}(n)} \\
\label{eq:passagem_V}
&=& \dfrac{n+1}{\mu} \cdot \dfrac{n!}{\mu^{n+1}} \\
\label{eq:passagem_VI}
&=& \dfrac{(n+1)!}{\mu^{n+2}},
\end{eqnarray}
onde, na passagem de \eqref{eq:passagem_I} para \eqref{eq:passagem_II} utilizamos a definição de integração imprópria. Na de \eqref{eq:passagem_II} para \eqref{eq:passagem_III}, integração por partes. Na de \eqref{eq:passagem_III} para \eqref{eq:passagem_IV}, propriedade de limites onde foi verificada a existência do limite. Na \eqref{eq:passagem_IV}, constatamos que o primeiro limite tende a zero utilizando \(n+1\) vezes a regra de L'Hospital.

Retornando ao cálculo do tempo médio de sobrevivência, temos:
\begin{eqnarray}
\nonumber
T_M
&=& 
\left(\mu -\dfrac{a_1}{N_0}\right) \dfrac{1}{\mu^2} +
\dfrac{1}{N_0} \displaystyle\sum_{k=2}^{m} (\mu a_{k-1}-k a_{k}) \dfrac{k!}{\mu^{k+1}} +
\mu a_m \dfrac{1}{N_0} \dfrac{(m+1)!}{\mu^{m+2}} \\
&=& 
\left(\dfrac{1}{\mu} -\dfrac{a_1}{N_0\mu^2}\right) +
\dfrac{1}{N_0} \displaystyle\sum_{k=2}^{m} (\mu a_{k-1}-k a_{k}) \dfrac{k!}{\mu^{k+1}} + \dfrac{a_m}{N_0} \dfrac{(m+1)!}{\mu^{m+1}}
\end{eqnarray}





\subsection*{5d}

A dinâmica de mortalidade da população é dada por:
\[N(t) = \dfrac{N_0}{t+1}\]

Portanto,
\[
\dfrac{d}{dt} N(t)
= \dfrac{d}{dt}\left(\dfrac{N_0}{t+1}\right)
= \dfrac{-N_0}{(t+1)^{2}}
\]


O tempo médio \(T_M\) de sobrevivência é dado por:
\[\begin{array}{rcl}
T_M
&=&
\dfrac{1}{N_0} \displaystyle\int_{0}^{\infty} -t\ \dfrac{dN}{dt}\ dt \\[0.3cm]
&=&
\dfrac{1}{N_0} \displaystyle\int_{0}^{\infty} -t\ \dfrac{-N_0}{(t+1)^{2}}\ dt \\[0.3cm]
&=&
\displaystyle\int_{0}^{\infty}  \dfrac{t}{(t+1)^{2}}\ dt
\end{array}\]

Essa integral imprópria é resolvida a seguir:
\[\begin{array}{rcl}
\displaystyle\int_{0}^{\infty}  \dfrac{t}{(t+1)^{2}}\ dt
&=&
\displaystyle\lim_{b \to \infty} \displaystyle\int_{0}^{b}  \dfrac{t}{(t+1)^{2}}\ dt \\[0.3cm]
&=&
\displaystyle\lim_{b \to \infty} \displaystyle\int_{-1}^{b-1}  \dfrac{t-1}{t^{2}}\ dt \\[0.3cm]
&=&
\displaystyle\lim_{b \to \infty} \left.\ln|t|+\dfrac{1}{t}\right|_{-1}^{b-1} \\
&=& 
\displaystyle\lim_{b \to \infty} \ln|b-1|+\dfrac{1}{b-1}-\ln(1)+1 \\
&=& \infty
\end{array}\]

}

\section*{Questão 06}
\addcontentsline{toc}{section}{Questão 06}


    Mortalidade por Predação Periférica e \textbf{Efeito de Rebanho Egoísta}:
    
    \begin{citacao}
    (Dois ``amigos'' em um campo de cerrado e uma onça esfomeada. Um deles, para e toma seu tempo para amarrar bem o calçado. O outro, apressado, lhe repreende:''Vamos correr logo que a onça é mais rápida do que nós!''. O Amigo (da onça): ``Eu não preciso correr mais do que a onça, eu preciso correr mais do que você!''. Ditado caboclo: ``Mingau quente, se come pelas beiradas''.
    \end{citacao}

    Considere uma população distribuída uniformemente em uma região delimitada no plano descrita por uma função diferenciável \(N(t)\) cuja mortalidade é causada unicamente por uma predação ``periférica'' da forma \(p(N) = -\mu \sqrt{N}\), caracterizada matematicamente segundo a Metodologia Newtoniana pela equação diferencial: \(\dfrac{dN}{dt} = - \mu \sqrt{N}\). (A justificativa da função de mortalidade na forma \(p(N) = -\mu \sqrt{N}\) para predação ``periférica'' se deve ao fato de que um grupo uniformemente distribuído em uma região delimitada do plano é predado apenas na fronteira, cuja extensão tem medida da ordem da dimensão linear da região, enquanto que a área, que é proporcional à população, é da ordem do quadrado da medida linear e, portanto, a fronteira é da ordem de \(N^{\frac{1}{2}}\). O formato da região pode ser considerado aproximadamente um disco (2D) ou uma esfera (3D) porque estas são as formas que apresentam menor extensão de fronteira para um mesmo conteúdo populacional.(Por exemplo, sapos na beira da lagoa diante da ameaça de cobras, ou rebanho de ovelhas diante de lobos).

    \textbf{Definição}: Diz-se que uma Dinâmica de mortalidade apresenta o ``\textbf{Efeito de Rebanho Egoísta}''(*)quando a mortalidade especifica (``\textit{per capita}'' \(\dfrac{1}{N} \dfrac{dN}{dt} = f(N))\) \textbf{diminui} com o aumento do tamanho do grupo, em outros termos, um individuo se sente particularmente mais ''protegido'' em um grupo maior; por isso ele se junta aos vencedores.. (*) Termo introduzido por W. Hamilton no antológico artigo: - \textbf{The Selfish Herd}, J. Theor.Biol, 1970).

    6a - Argumente como o conceito de ``Efeito Rebanho Egoísta'' pode ser interpretado em termos do Tempo Médio de Sobrevivência.

    6b - Mostre que não há ``Efeito Rebanho Egoista'' em uma população cuja mortalidade é unicamente Malthusiana.

    6c - Descreva uma Dinâmica Adimensional de mortalidade por predação periférica para um grupo populacional que ocupa uma região delimitada do espaço físico \textbf{tridimensional}. (Por exemplo, um cardume de Sardinhas e Baleias) e verifique se esta dinâmica apresenta um ``Efeito Rebanho Egoista'' e é dizimada em tempo finito.

    6d - Considere uma população com predação per capita tipo Holling II: \(p(N) = \dfrac{A}{B+N}\). Adimensionalize a equação e verifique se ocorre um ``Efeito Rebanho'' nesta dinâmica.

    6e - Discuta o comportamento individual das presas em termos de uma proteção por agrupamento com base na percepção de cardinalidade segundo a ``\textbf{Lei de Weber-Fechner}''.


%\solucao{}

\section*{Questão 07}
\addcontentsline{toc}{section}{Questão 07}


    7a – \textbf{Utilizando o Método Operacional} explicado no texto, obtenha uma expressão explícita (em termos de integrais) da solução da Equação de (Euler-Malthus) Verhulst \(\dfrac{1}{N} \dfrac{dN}{dt} = r(t) - \lambda(t)N\), onde \(r(t)\) e \(\lambda(t)\) são funções reais positivas. (Sugestão: Utilize a transformação linearizadora \(m = \dfrac{1}{N}\) seguida pelo Método Operacional).

    7b - Apresente um cenário biológico que indique a utilização desta equação como Modelo Matemático para uma Dinâmica Populacional.

    7c - Considere uma população cujo tamanho \(N(t)\) é regulado pelo chamado Modelo de Euler-Verhulst, \(\dfrac{1}{N} \dfrac{dN}{dt} = r - \lambda N\) (isto é, com taxa de natalidade Malthusiana (\textit{per capita}) \(r\) e mortalidade (\textit{per capita}) \(\lambda N,\ r, \lambda > 0\) constantes) que se inicia com uma população ``\textbf{colonizadora}'' de \(N_0 = N(0)\) indivíduos. Considere a decrescente população \(n(t)\) dos indivíduos colonizadores (\(n(0) = N_0\)) submetidos à taxa de mortalidade ambiente. ({\small Os descendentes de colonizadores não são colonizadores mas fazem parte da população ambiente!}). Obtenha uma expressão para a dinâmica desta população \(n(t)\) de colonizadores e mostre que o tempo médio de sobrevivência neste caso, apresenta uma dependência do tamanho da população inicial \(N_0\), indicando um fenômeno interativo no processo de mortalidade.


%\solucao{}

\section*{Questão 08}
\addcontentsline{toc}{section}{\textcolor{green}{Questão 08}}

    Sistemas Malthusianos com Acoplamento Sequencial

    \[\ldots
    A_1 \ \substack{\mu_1 \\ \longrightarrow} \
    A_2 \ \substack{\mu_2 \\ \longrightarrow} \
    A_3 \ \substack{\mu_3 \\ \longrightarrow} \
    \ldots
    A_n \ \substack{\mu_n \\ \longrightarrow} \
    A_{n+1} \ldots\]
    
    Considere um sistema de compartimentos sequencialmente acoplados com dinâmicas Malthusianas.

    8a - Supondo uma sequencia com \(N\) compartimentos, \(1 \le k \le N\), com \(\mu_N = 0\), escreva o Modelo deste sistema na forma de Equações Diferenciais Ordinárias (acopladas) \(\dfrac{dA}{dt} = DA = MA\), e Operacional \((D - M)A = 0\) identificando a matriz \textbf{numérica} \(M\), e a matriz \textbf{operacional} \(m(D) = D - A\). \(\left(\dfrac{d}{dt} \equiv D\right)\)

    8b - Se \(N = 3\) e \(\mu_k = \mu > 0\) obtenha as expressões analíticas elementares para as soluções \(A(t) = (A_1 \ A_2 \ A_3)^t = \left(\begin{array}{c} A_1 \\ A_2 \\ A_3 \end{array}\right)\), resolvendo antes as equações desacopladas \(\det m(D) x = 0\). (Refer. Bassanezi-Ferreira).

    8c - Mostre que, em geral, \(A_k(t) \to 0\) exponencialmente, como \(t^2 e^{-\mu t}\), isto é, \(\displaystyle \lim_{t \to \infty} \dfrac{A_k(t)}{t^2 e^{-\mu t}} = c \neq 0\).

    8d - Determine o tempo médio (aritmético) que estas partículas/organismos permanecem no sistema de compartimentos se inicialmente todas elas estão no primeiro no primeiro compartimento \(A_1(0) = 1, A_k(0) = 0,\ k > 1\).

    8e - Determine a relação entre o tempo médio (aritmético) de permanência destas partículas/organismos no sistema em termos da sua distribuição inicial, \(A_k(0) = A_{k0}\).


\solucao{

\subsection*{8a-}


Temos que a população total \(A\) é formada por \(N\) subpopulações sequencialmente acopladas. Então, esse modelo é regido por:
\[\begin{array}{rcl}
\dfrac{dA_{1}}{dt} &=& -\mu_{1} A_{1} \\[0.3cm]
\dfrac{dA_{k}}{dt} &=& \mu_{k-1} A_{k-1} - \mu_{k} A_{k},\ k = 2, \ldots, N \\[0.3cm]
\end{array}\]

Se fizermos \(\mathcal{A} = [A_1 \ A_2 \ \ldots\ A_N]^t\) e
\[M
= \left[
\begin{array}{cccccc}
-\mu_{1} & 0 & 0 & \cdots & 0 & 0 \\
\mu_{1} & -\mu_{2} & 0 & \cdots & 0 & 0 \\
0 & \mu_{2} & -\mu_{3} & \cdots & 0 & 0 \\
\vdots & \vdots & \vdots & \ddots & \vdots & \vdots\\
0 & 0 & 0 & \cdots & -\mu_{N-1} & 0 \\
0 & 0 & 0 & \cdots & \mu_{N-1} & \mu_N
\end{array}\right]
\]
teremos
\[\dfrac{dA}{dt} = MA \Rightarrow DA = MA \Rightarrow DA - MA = \mathbf{0},\]
sendo \(\mathbf{0}\) matriz nula de ordem \(N \times 1\). Segue que
\[(D-M) A = \mathbf{0},\]
com a matriz operacional
\[m(D) = D-A
= \left[
\left(\frac{d}{dt}-A_1\right) \ \ 
\left(\frac{d}{dt}-A_2\right) \ \
\left(\frac{d}{dt}-A_3\right)
\right]^t
%\left[
%\begin{array}{cccccc}
%\frac{d}{dt}-\mu_{1} & 0 & 0 & \cdots & 0 & 0 \\
%\mu_{1} & \frac{d}{dt}+\mu_{2} & 0 & \cdots & 0 & 0 \\
%0 & -\mu_{2} & \frac{d}{dt}+\mu_{3} & \cdots & 0 & 0 \\
%\vdots & \vdots & \vdots & \ddots & \vdots & \vdots\\
%0 & 0 & 0 & \cdots & \frac{d}{dt}+\mu_{N-1} & 0 \\
%0 & 0 & 0 & \cdots & \mu_{N-1} & \frac{d}{dt}-\mu_N
%\end{array}\right]
\]


\subsection*{8b-}

Considere o sistema 
\[
A_1 \ \substack{\mu_1 \\ \longrightarrow} \
A_2 \ \substack{\mu_2 \\ \longrightarrow} \
A_3
\]
de compartimentos sequencialmente acoplados, de três subpopulações com dinâmicas malthusianas. Então, esse modelo é regido por:
\[\begin{array}{rcl}
\dfrac{dA_{1}}{dt} &=& -\mu A_{1} \Rightarrow A_1(t) = K_1 e^{-\mu t} \\[0.3cm]
\dfrac{dA_{2}}{dt} &=& \mu(A_1 - A_{2}) \\[0.3cm]
\dfrac{dA_{3}}{dt} &=& \mu(A_2 - A_{3}).
\end{array}\]


\subsection*{8c-}

\subsection*{8d-}


O tempo médio \(T_M\) de sobrevivência da população total \(A\) é dado por:
\[
T_M
= \dfrac{1}{A_0} \displaystyle\int_{0}^{\infty} -t \dfrac{dA}{dt}\ dt
\]

Uma vez que ela está concentrada na subpopulação \(A_1\) malthusiana, com \(A_1(0) = 1\), temos:
\[
T_M
= \dfrac{1}{A_1(0)} \displaystyle\int_{0}^{\infty} -t \dfrac{dA_1}{dt}\ dt
= \displaystyle\int_{0}^{\infty} -t \dfrac{dA_1}{dt}\ dt
= \displaystyle\int_{0}^{\infty} -t e^{-\mu t}\ dt
= \dfrac{1}{\mu}
\]

}



\section*{Questão 09}
\addcontentsline{toc}{section}{\textcolor{green}{Questão 09}}


Modelos Efetivos

    9a - Considere uma população de indivíduos não interativos formada por uma mistura de subpopulações Malthusianas (homogêneas e não interativas) \(A_k\), sendo \(T_k\) seu respectivo tempo médio (aritmético) de sobrevivência. Considere agora a população total \(A(t) = \displaystyle \sum A_k(t)\) que obviamente decresce. Mostre que o tempo médio (aritmético) de sobrevivência da população misturada \(A\) é dado pela Média (aritmética) ponderada de \(T_k\).

    9b - Considere agora uma descrição da dinâmica de uma população ``Malthusianamente heterogênea'' por um ``Modelo Malthusiano Efetivo'', isto é, da forma \(\dfrac{dA}{dt} = -\mu A\). Argumente sobre o fato de que neste caso a ``melhor escolha'' para o coeficiente \(\mu\) do Modelo diferencial seria a \textbf{média harmônica} dos coeficientes \(\mu_k\).

    9c - Analise a mesma questão supondo que a população total é formada por \(N\) subpopulações sequencialmente acopladas na forma
    \[
    A_1 \ \substack{\mu_1 \\ \longrightarrow} \
    A_2 \ \substack{\mu_2 \\ \longrightarrow} \
    A_3 \
    \ \ldots \
    A_{N-1} \ \substack{\mu_{N-1} \\ \longrightarrow} \
    A_{N},\ \mu_N = 0.\]



\solucao{

\subsection*{9a-}

Temos uma população \(A\) de indivíduos não interativos formada por uma mistura de subpopulações Malthusianas \(A_k\) (homogêneas e não interativas). Logo, para cada \(1 \le k \le N\), temos:
\[\dfrac{1}{A_k} \dfrac{A_k}{dt} = -\mu_k,\ A_{k}(0) = A_{0_{k}} \Rightarrow A_{k}(t) = A_{0_{k}} e^{-\mu_{k}t}.\]

Segue que
\[A = \displaystyle\sum_{k=1}^{N} A_{k}(t) = \displaystyle\sum_{k=1}^{N} A_{0_{k}} e^{-\mu_{k}t}
\Rightarrow
\dfrac{dA}{dt} = \displaystyle\sum_{k=1}^{N} -\mu_{k} A_{0_{k}} e^{-\mu_{k}t}.
\]
Claramente, como visto nesta última equação, é bem provável que \(A\) seja não malthusiana, uma vez que isso só ocorrerá caso \(\mu_1 = \mu_2 = \ldots = \mu_N\). Entretanto, o tempo médio de sobrevivência \(T_M\) pode ser calculado como a seguir:
\[\begin{array}{rcl}
T_M
&=& \dfrac{1}{A_0} \displaystyle\int_{0}^{\infty} -t\ \dfrac{dA}{dt}\ dt \\
&=& \dfrac{1}{A_0} \displaystyle\int_{0}^{\infty} -t\ \left(\sum_{k=1}^{N} -\mu_{k} A_{0_{k}} e^{-\mu_{k}t}\right) dt\\
&=& \dfrac{1}{A_0} \displaystyle\left[\sum_{k=1}^{N} \mu_{k} A_{0_{k}} \underbrace{\left(\int_{0}^{\infty} t e^{-\mu_{k}t}\ dt \right)}_{\mathcal{I}(1)}\right] \\
&=& \dfrac{1}{A_0} \displaystyle\left[\sum_{k=1}^{N} \mu_{k} A_{0_{k}} \left(\dfrac{1}{\mu_{k}^{2}}\right) \right] \\
&=& \dfrac{1}{A_0} \displaystyle \sum_{k=1}^{N} \dfrac{A_{0_{k}}}{\mu_{k}},
\end{array}\]
onde o resultado da integral \(\mathcal{I}(1)\) foi o obtido na Questão 05.

Considerando que \(\displaystyle \sum_{k=1}^{N} A_{0_{k}} = A_0\), temos:
\[T_M = \dfrac{1}{\displaystyle \sum_{k=1}^{N} \mu_{k}}.\]

Por outro lado, a média aritmética (ponderada) de \(T_{M_{k}}\) é:
\[
M\left(T_{M_{k}}\right)
= \dfrac{1}{N} \displaystyle\sum_{k=1}^{N} T_{M_{k}}
= \dfrac{1}{N} \displaystyle\sum_{k=1}^{N} \dfrac{1}{\mu_{k}}
= \dfrac{1}{N} \dfrac{\displaystyle\sum_{k=1}^{N} 1}{\displaystyle\sum_{k=1}^{N} \mu_{k}}
= \dfrac{1}{N} \dfrac{N}{\displaystyle\sum_{k=1}^{N} \mu_{k}}
= \dfrac{1}{\displaystyle\sum_{k=1}^{N} \mu_{k}}
= T_M.
\]
Como queríamos mostrar.


\subsection*{9b-}

Seja \(\bar{A}(t) = \displaystyle\sum_{k=1}^{N} A_{0_{k}} e^{-\mu_k t}\) e \(A(t) = A_{0} e^{-\mu t}\), com \(\mu = \dfrac{N}{\displaystyle\sum_{k=1}^{N} \dfrac{1}{\mu_k}}\).

Considerando a norma euclidiana, analisemos a proximidade entre esses modelos de população.
\[\begin{array}{rcl}
\left|A(t)-\bar{A}(t)\right|
&=& \left|\displaystyle\sum_{k=1}^{N} A_{0_{k}} e^{-\mu_k t}-A_{0} e^{-\mu t}\right| \\
&=& \left|\displaystyle\sum_{k=1}^{N} A_{0_{k}} e^{-\mu_k t}-e^{-\mu t} \sum_{k=1}^{N} A_{0_{k}} \right| \\
&=& \left|\displaystyle\sum_{k=1}^{N} A_{0_{k}} (e^{-\mu_k t}-e^{-\mu t}) \right| \\
&\le& \displaystyle\sum_{k=1}^{N} A_{0_{k}} \left|e^{-\mu_k t}-e^{-\mu t}\right|.
\end{array}\]

Considere, agora, \(\min\{\mu_k\} = \mu_{\min}\) e \(\max\{\mu_k\} = \mu_{\max}\). Pelo Teorema do Valor Médio, temos que:
\(\left|e^{-\mu_k t}-e^{-\mu t}\right| \le t e^{-\mu_{\min} t} |\mu -\mu_k|.\)

Como \(\mu\) é a média harmônica dos \(\mu_k\), temos:
\[
|\mu -\mu_k| \le \mu_{\max} - \mu_{\min},\ \forall\ k.
\]

Portanto,
\[\begin{array}{rcl}
\left|A(t)-\bar{A}(t)\right|
&\le& \displaystyle\sum_{k=1}^{N} A_{0_{k}} t\ e^{-\mu_{\min} t} (\mu_{\max} - \mu_{\min}) \\
&=& A_{0} t\ e^{-\mu_{\min} t} (\mu_{\max} - \mu_{\min}).
\end{array}\]

Sendo assim
\[\begin{array}{rcl}
\displaystyle\lim_{t\to \tau} \left|A(t)-\bar{A}(t)\right|
&\le& 
A_{0} (\mu_{\max} - \mu_{\min}) \displaystyle\lim_{t\to \tau} \dfrac{t}{e^{\, \mu_{\min} t}},
\end{array}\]
e, portanto, teremos uma boa aproximação entre os modelos se:

(a) \(\tau\) crescem indefinidamente;
(b) os valores de \(\tau\) são pequenos e a amplitude entre os tempos médios de sobrevivência máximo e mínimo for pequeno e; (c) o tempo médio de sobrevivência mínimo for grande.  


\subsection*{9c-}

Temos que a população total \(A\) é formada por \(N\) subpopulações sequencialmente acopladas. Então, esse modelo é regido por:
\[\begin{array}{rcl}
\dfrac{dA_{1}}{dt} &=& -\mu_{1} A_{1} \\[0.3cm]
\dfrac{dA_{k}}{dt} &=& \mu_{k-1} A_{k-1} - \mu_{k} A_{k},\ k = 2, \ldots, N \\[0.3cm]
%\dfrac{dA_{N}}{dt} &=& \mu_{N} A_{N}
\end{array}\]

Se fizermos \(\mathcal{A} = [A_1 \ A_2 \ \ldots\ A_N]^t\), \(\mathbf{1} = (1 \ \ 1 \ \ldots\ \ 1)_{1 \times N}\) e
\[\Phi
=\left[
\begin{array}{cccccc}
-\mu_{1} & 0 & 0 & \cdots & 0 & 0 \\
\mu_{1} & -\mu_{2} & 0 & \cdots & 0 & 0 \\
0 & \mu_{2} & -\mu_{3} & \cdots & 0 & 0 \\
\vdots & \vdots & \vdots & \ddots & \vdots & \vdots\\
0 & 0 & 0 & \cdots & -\mu_{N-1} & 0 \\
0 & 0 & 0 & \cdots & \mu_{N-1} & \mu_N
\end{array}\right]
\]
teremos
\[\displaystyle\sum_{k=1}^{N} \dfrac{dA_{k}}{dt}
= \mathbf{1} \cdot \Phi \cdot \mathcal{A}
= -\mu_{N} \dfrac{dA_{N}}{dt}.\]



O tempo médio de sobrevivência da população total é então dado por:
\[
T_M
= \dfrac{1}{A_0} \displaystyle\int_{0}^{\infty} -t \displaystyle\sum_{k=1}^{N} \dfrac{dA_{k}}{dt}\ dt
= \dfrac{1}{A_0} \displaystyle\int_{0}^{\infty} t\ \mu_{N}\ \dfrac{dA_{N}}{dt}\ dt
= 0,
\]
uma vez que \(\mu_N = 0\).

Dessa forma, se cada subpopulação \(A_{k}, k = 1, \ldots, N\) de uma população \(A\) segue um modelo malthusiano com tempo médio de sobrevivência muito pequeno, o modelo da população total formada por \(N\) subpopulações sequencialmente acopladas possui o tempo médio de sobrevivência muito próximos.
}

\section*{Questão 10}
\addcontentsline{toc}{section}{Questão 10}


    Sistemas Malthusianos com Acoplamentos Multilaterais (Difusivos)

    Considere um sistema de \(N\) compartimentos \(\{A_k\}_{1 \le k \le N}\) conectados sequencialmente e simetricamente por dinâmicas Malthusianas bilaterais como no seguinte esquema
    \[A_0 \substack{\mu \\ \longleftarrow \\ \longrightarrow \\ \mu_0} 
    A_1 \substack{\mu \\ \longleftrightarrow}
    A_2 \substack{\mu \\ \longleftrightarrow}
    A_3 \substack{\mu \\ \longleftrightarrow}
    \ldots
    A_{N-1} \substack{\mu \\ \longleftrightarrow}
    A_N \substack{\mu_N \\ \longleftrightarrow}
    A_{N+1}\]


    10a - Mostre que um compartimento interior \(A_k , 2 \le k \le N-1\), é regido pela seguinte equação: \(\dfrac{dA_k}{dt} = -2\mu A_k + \mu A_{k-1} + \mu A_{k+1}\).

    10b - Obtenha a dinâmica do compartimento \(A_N\) que está obstruído à direita (isto é, não perde nem ganha população de \(A_{N+1}\)). Diz-se também que é reflexivo, ou que \(\mu_N = 0\).

    10c - Obtenha a dinâmica do compartimento \(A_1\) que somente perde indivíduos para o compartimento \(A_0\) e não recebe nada do mesmo, isto é, \(\mu_0 = 0\), Interprete este fato como a existência de um ``deserto'' em \(A_0\).

    10d - Escreva a dinâmica de todo o sistema acoplado na forma matricial \(\dfrac{dA}{dt} = S\ A\), \(A = A_1, \ldots, A_N)^t\) e mostre que a matriz \(S\) é simétrica e tridiagonal.

    10d - Mostre que \(n(t) = \displaystyle \sum_{k=1}{N} A_k\) é monotonicamente decrescente, isto é, \(\dfrac{dn}{dt} < 0\).
    \textbf{Sugestão}: \(\dfrac{dn}{dt} = \dfrac{d}{dt} \langle A, 1 \rangle = \langle \dfrac{dA}{dt}, 1 \rangle = \langle S\ A, 1 \rangle < 0\).

    10f - Na verdade, mostre que \(n(t)\) é exponencialmente decrescente, isto é, existe \(\lambda > 0\) tal que \(\displaystyle \lim_{t \to \infty} \dfrac{n(t)}{e^{-\lambda t}} = c > 0\).

    10g - Utilize o Método de Fourier para mostrar que a solução geral do sistema acima pode ser escrito na forma: \(A(t) = \displaystyle \sum_{k=1}^{N} c_k e^{\lambda_k t}\), verificando como determinar algebricamente os coeficientes \(c_k\) e os parâmetros \(\lambda_k\) e mostrando que \(\lambda_k < 0, 1 \le k \le N\).

    10h - Argumente sobre a propriedade ``homogeneizadora'' desta dinâmica no sentido de que todos \(A_k(t)\) convergem para uma média.

    {\small Obs: O Método de Fourier resolve completamente o Sistema de EDOs utilizando combinações lineares de soluções básicas \(e^{\lambda t} v\) para o Sistema, onde \(\lambda\) é autovalor da matriz simétrica \(S\) e \(v\) o seu autovetor correspondente, \(Sv = \lambda v\). Lembre-se do Teorema Espectral para matrizes simétricas que garante a expansão de qualquer vetor \(u\) na forma, \(u = \displaystyle \sum a_k v^{(k)}\) onde \(v^{(k)}\) é base ortonormal de autovetores de \(S\).}


%\solucao{}

\section*{Questão 11}
\addcontentsline{toc}{section}{Questão 11}


Acoplamento Difusivo de Dinâmicas Malthusianas

    Considere um Grafo com \(4\) vértices, \(3\) localizados nas quinas de um triângulo e \(1\) deles, \(A_0\), no seu centro. Cada vértice \(A_k\) das quinas é conectado bilateralmente aos seus dois adjacentes, \(A_{k-1}\) e \(A_{k+1}\) e todos, da mesma forma, ao centro \(A_0\) por uma dinâmica Malthusiana com o mesmo parâmetro \(\mu\) em todas as direções.

    11a - Escreva a dinâmica do sistema \(A = (A_1, A_2, A_3, A_0)^t\) na forma matricial,
    \(\dfrac{dA}{dt} = S\ A\)

    11b - Mostre que \(S\) é simétrica e tem autovalor nulo com autovetor \(1 =(1, \ldots, 1)\).

    11c - Mostre que \(n(t) = \displaystyle \sum_{k=0}^{4} A_k(t)\) é constante e que \(\displaystyle \lim_{t \to \infty} A(t) = n(0) (1, \ldots, 1)\).


%\solucao{}

\section*{Questão 12}
\addcontentsline{toc}{section}{Questão 12}



    12a - Mostre que em uma dinâmica Malthusiana com parâmetros \(\mu, \nu\) constantes a operação funcional ``Multiplicação de \(N(t)\) por \(e^{-\mu T}\)'' produz um resultado (função) que representa ``O número de sobreviventes dos indivíduos \(N(t)\) após um intervalo de tempo \(T\)''. Interprete analogamente as operações funcionais ``Multiplicação por \(1-e^{-\mu T}\)'' e ''Multiplicação por \(e^{\nu T}\)''.

    12b - Interprete probabilisticamente a operação sobre uma dinâmica Malthusiana \(N(t)\) definida pela operação funcional resultante da multiplicação por \(\dfrac{(1-e^{-\mu T})}{N(t)}\)

    12c - Considere uma população estruturada em duas faixas etárias, como no problema de Fibonacci, uma delas imatura, \(A_1(t)\) com dinâmica contínua de mortalidade Malthusiana e outra reprodutiva, \(A_2(t)\), com dinâmica contínua de mortalidade e reprodutividade também Malthusiana.

    \textbf{Argumente convincentemente} sobre o significado da expressão de cada termo e parâmetro do Modelo Matemático para a Dinâmica desta população expresso segundo o sistema de Equações Diferenciais Ordinárias com retardamento:
    \[\begin{array}{rcl}
    \dfrac{dA_1(t)}{dt} &=& \nu A_2(t) - \mu_1 A_1(t) - e^{-\mu_1 T_0} \nu A_2(t-T_0) \\[0.3cm]
    \dfrac{dA_2(t)}{dt} &=& e^{-\mu_1 T_0} \nu A_2(t-T_0) - \mu_2 A_2(t)
    \end{array}\]

    12d - Suponha que \(T_0\) seja ``muito pequeno'' comparado com as outras unidades intrínsecas de tempo \(\left(\dfrac{1}{\nu}, \dfrac{1}{\mu_k}\right)\) do Modelo Malthusiano com retardamento e substitua a expressão \(A_2(t-T_0)\) por sua aproximação de Taylor: \(A_2(t-T_0) = A_2(t) - T_0 A_2'(t) + \dfrac{T_0^2}{2} A_2''(t)\) obtendo um sistema de equações diferenciais ordinárias (não retardadas).

    12e - Utilize o Método Operacional e reescreva o Sistema de EDO obtido anteriormente na forma matricial operacional \(P(D) \left(\begin{array}{c} A_1 \\ A_2 \end{array}\right) = 0\) e obtenha expressões elementares gerais para as funções \(A_j(t)\) soluções do Sistema.


%\solucao{}

\section*{Questão EXTRA}
\addcontentsline{toc}{section}{\textcolor{green}{Questão EXTRA}}



Hipótese (Gauss - Legendre ~1796) - Teorema (Hadamard - de la Vallé-Poussin 1896) sobre a densidade dos Números Primos.

    A - Considere o Teorema de Distribuição Assintótica (da densidade da População) de Números Primos em \(\mathbb{N}\), descrita pela função \(\rho(n) = \dfrac{\pi(n)}{n}\), onde \(\pi(n) = \# \{\mbox{Números primos } p \le n\}\) em termos de uma linearização logarítmica assintótica utilizando uma Tabela de Números Primos (encontrada, por exemplo, no valioso M. Abramowitz \& I. Stegun - Handbook of Mathematical Functions with Formulas, Graphs, and Mathematical Tables - online). (Sugestão: Como é fácil ver pela tabela que \(\dfrac{1}{\rho(n)} = \dfrac{n}{\pi(n)} \to \infty\) mais ``lentamente'' do que \(n \to \infty\) re-escale a variável ``independente'' \(n\) ``logaritmizando-a'' e analise o gráfico de \(\dfrac{1}{\rho(n)}\) em função de \(\log n\)).

    B - Interprete a questão em termos do Princípio Sensorial (``Lei'') de Weber-Fechner.


\solucao{

\subsection*{13A}

A análise tem como ponto de partida a Tabela \ref{tab:milnumerosprimos}:


\captionof{table}{Sequência crescente dos primeiros 1000 números primos}
\label{tab:milnumerosprimos}
\begin{minipage}[!h]{0.45\textwidth}\centering
\tiny
\begin{longtable}{cccc} \hline
\(n\) & \(\pi(n)\) & \(\log(n)\) & \(\pi(n)/n\) \\ \hline
1 & 0 & 0 & 0 \\ \hline
2 & 0 & 0,301029996 & 0 \\ \hline
3 & 1 & 0,477121255 & 0,333333333 \\ \hline
4 & 2 & 0,602059991 & 0,5 \\ \hline
5 & 2 & 0,698970004 & 0,4 \\ \hline
6 & 3 & 0,77815125 & 0,5 \\ \hline
7 & 3 & 0,84509804 & 0,428571429 \\ \hline
8 & 4 & 0,903089987 & 0,5 \\ \hline
9 & 4 & 0,954242509 & 0,444444444 \\ \hline
10 & 4 & 1 & 0,4 \\ \hline
11 & 4 & 1,041392685 & 0,363636364 \\ \hline
12 & 5 & 1,079181246 & 0,416666667 \\ \hline
13 & 5 & 1,113943352 & 0,384615385 \\ \hline
14 & 6 & 1,146128036 & 0,428571429 \\ \hline
15 & 6 & 1,176091259 & 0,4 \\ \hline
16 & 6 & 1,204119983 & 0,375 \\ \hline
17 & 6 & 1,230448921 & 0,352941176 \\ \hline
18 & 7 & 1,255272505 & 0,388888889 \\ \hline
19 & 7 & 1,278753601 & 0,368421053 \\ \hline
20 & 8 & 1,301029996 & 0,4 \\ \hline
21 & 8 & 1,322219295 & 0,380952381 \\ \hline
22 & 8 & 1,342422681 & 0,363636364 \\ \hline
23 & 8 & 1,361727836 & 0,347826087 \\ \hline
24 & 9 & 1,380211242 & 0,375 \\ \hline
25 & 9 & 1,397940009 & 0,36 \\ \hline
26 & 9 & 1,414973348 & 0,346153846 \\ \hline
27 & 9 & 1,431363764 & 0,333333333 \\ \hline
28 & 9 & 1,447158031 & 0,321428571 \\ \hline
29 & 9 & 1,462397998 & 0,310344828 \\ \hline
30 & 10 & 1,477121255 & 0,333333333 \\ \hline
31 & 10 & 1,491361694 & 0,322580645 \\ \hline
32 & 11 & 1,505149978 & 0,34375 \\ \hline
33 & 11 & 1,51851394 & 0,333333333 \\ \hline
34 & 11 & 1,531478917 & 0,323529412 \\ \hline
35 & 11 & 1,544068044 & 0,314285714 \\ \hline
36 & 11 & 1,556302501 & 0,305555556 \\ \hline
37 & 11 & 1,568201724 & 0,297297297 \\ \hline
38 & 12 & 1,579783597 & 0,315789474 \\ \hline
39 & 12 & 1,591064607 & 0,307692308 \\ \hline
40 & 12 & 1,602059991 & 0,3 \\ \hline
41 & 12 & 1,612783857 & 0,292682927 \\ \hline
42 & 13 & 1,62324929 & 0,30952381 \\ \hline
43 & 13 & 1,633468456 & 0,302325581 \\ \hline
44 & 14 & 1,643452676 & 0,318181818 \\ \hline
45 & 14 & 1,653212514 & 0,311111111 \\ \hline
46 & 14 & 1,662757832 & 0,304347826 \\ \hline
47 & 14 & 1,672097858 & 0,29787234 \\ \hline
48 & 15 & 1,681241237 & 0,3125 \\ \hline
49 & 15 & 1,69019608 & 0,306122449 \\ \hline
50 & 15 & 1,698970004 & 0,3 \\ \hline
\end{longtable}
\end{minipage}
\begin{minipage}[!h]{0.45\textwidth}\centering
\tiny
\begin{longtable}{cccc} \hline
\(n\) & \(\pi(n)\) & \(\log(n)\) & \(\pi(n)/n\) \\ \hline
51 & 15 & 1,707570176 & 0,294117647 \\ \hline
52 & 15 & 1,716003344 & 0,288461538 \\ \hline
53 & 15 & 1,72427587 & 0,283018868 \\ \hline
54 & 16 & 1,73239376 & 0,296296296 \\ \hline
55 & 16 & 1,740362689 & 0,290909091 \\ \hline
56 & 16 & 1,748188027 & 0,285714286 \\ \hline
57 & 16 & 1,755874856 & 0,280701754 \\ \hline
58 & 16 & 1,763427994 & 0,275862069 \\ \hline
59 & 16 & 1,770852012 & 0,271186441 \\ \hline
60 & 17 & 1,77815125 & 0,283333333 \\ \hline
61 & 17 & 1,785329835 & 0,278688525 \\ \hline
62 & 18 & 1,792391689 & 0,290322581 \\ \hline
63 & 18 & 1,799340549 & 0,285714286 \\ \hline
64 & 18 & 1,806179974 & 0,28125 \\ \hline
65 & 18 & 1,812913357 & 0,276923077 \\ \hline
66 & 18 & 1,819543936 & 0,272727273 \\ \hline
67 & 18 & 1,826074803 & 0,268656716 \\ \hline
68 & 19 & 1,832508913 & 0,279411765 \\ \hline
69 & 19 & 1,838849091 & 0,275362319 \\ \hline
70 & 19 & 1,84509804 & 0,271428571 \\ \hline
71 & 19 & 1,851258349 & 0,267605634 \\ \hline
72 & 20 & 1,857332496 & 0,277777778 \\ \hline
73 & 20 & 1,86332286 & 0,273972603 \\ \hline
74 & 21 & 1,86923172 & 0,283783784 \\ \hline
75 & 21 & 1,875061263 & 0,28 \\ \hline
76 & 21 & 1,880813592 & 0,276315789 \\ \hline
77 & 21 & 1,886490725 & 0,272727273 \\ \hline
78 & 21 & 1,892094603 & 0,269230769 \\ \hline
79 & 21 & 1,897627091 & 0,265822785 \\ \hline
80 & 22 & 1,903089987 & 0,275 \\ \hline
81 & 22 & 1,908485019 & 0,271604938 \\ \hline
82 & 22 & 1,913813852 & 0,268292683 \\ \hline
83 & 22 & 1,919078092 & 0,265060241 \\ \hline
84 & 23 & 1,924279286 & 0,273809524 \\ \hline
85 & 23 & 1,929418926 & 0,270588235 \\ \hline
86 & 23 & 1,934498451 & 0,26744186 \\ \hline
87 & 23 & 1,939519253 & 0,264367816 \\ \hline
88 & 23 & 1,944482672 & 0,261363636 \\ \hline
89 & 23 & 1,949390007 & 0,258426966 \\ \hline
90 & 24 & 1,954242509 & 0,266666667 \\ \hline
91 & 24 & 1,959041392 & 0,263736264 \\ \hline
92 & 24 & 1,963787827 & 0,260869565 \\ \hline
93 & 24 & 1,968482949 & 0,258064516 \\ \hline
94 & 24 & 1,973127854 & 0,255319149 \\ \hline
95 & 24 & 1,977723605 & 0,252631579 \\ \hline
96 & 24 & 1,982271233 & 0,25 \\ \hline
97 & 24 & 1,986771734 & 0,24742268 \\ \hline
98 & 25 & 1,991226076 & 0,255102041 \\ \hline
99 & 25 & 1,995635195 & 0,252525253 \\ \hline
100 & 25 & 2 & 0,25 \\ \hline
\end{longtable}
\end{minipage}




\begin{minipage}[!h]{0.45\textwidth}\centering
\tiny
\begin{longtable}{cccc} \hline
\(n\) & \(\pi(n)\) & \(\log(n)\) & \(\pi(n)/n\) \\ \hline
101 & 25 & 2,004321374 & 0,247524752 \\ \hline
102 & 26 & 2,008600172 & 0,254901961 \\ \hline
103 & 26 & 2,012837225 & 0,252427184 \\ \hline
104 & 27 & 2,017033339 & 0,259615385 \\ \hline
105 & 27 & 2,021189299 & 0,257142857 \\ \hline
106 & 27 & 2,025305865 & 0,254716981 \\ \hline
107 & 27 & 2,029383778 & 0,252336449 \\ \hline
108 & 28 & 2,033423755 & 0,259259259 \\ \hline
109 & 28 & 2,037426498 & 0,256880734 \\ \hline
110 & 29 & 2,041392685 & 0,263636364 \\ \hline
111 & 29 & 2,045322979 & 0,261261261 \\ \hline
112 & 29 & 2,049218023 & 0,258928571 \\ \hline
113 & 29 & 2,053078443 & 0,256637168 \\ \hline
114 & 30 & 2,056904851 & 0,263157895 \\ \hline
115 & 30 & 2,06069784 & 0,260869565 \\ \hline
116 & 30 & 2,064457989 & 0,25862069 \\ \hline
117 & 30 & 2,068185862 & 0,256410256 \\ \hline
118 & 30 & 2,071882007 & 0,254237288 \\ \hline
119 & 30 & 2,075546961 & 0,25210084 \\ \hline
120 & 30 & 2,079181246 & 0,25 \\ \hline
121 & 30 & 2,08278537 & 0,247933884 \\ \hline
122 & 30 & 2,086359831 & 0,245901639 \\ \hline
123 & 30 & 2,089905111 & 0,243902439 \\ \hline
124 & 30 & 2,093421685 & 0,241935484 \\ \hline
125 & 30 & 2,096910013 & 0,24 \\ \hline
126 & 30 & 2,100370545 & 0,238095238 \\ \hline
127 & 30 & 2,103803721 & 0,236220472 \\ \hline
128 & 31 & 2,10720997 & 0,2421875 \\ \hline
129 & 31 & 2,11058971 & 0,240310078 \\ \hline
130 & 31 & 2,113943352 & 0,238461538 \\ \hline
131 & 31 & 2,117271296 & 0,236641221 \\ \hline
132 & 32 & 2,120573931 & 0,242424242 \\ \hline
133 & 32 & 2,123851641 & 0,240601504 \\ \hline
134 & 32 & 2,127104798 & 0,23880597 \\ \hline
135 & 32 & 2,130333768 & 0,237037037 \\ \hline
136 & 32 & 2,133538908 & 0,235294118 \\ \hline
137 & 32 & 2,136720567 & 0,233576642 \\ \hline
138 & 33 & 2,139879086 & 0,239130435 \\ \hline
139 & 33 & 2,1430148 & 0,237410072 \\ \hline
140 & 34 & 2,146128036 & 0,242857143 \\ \hline
141 & 34 & 2,149219113 & 0,241134752 \\ \hline
142 & 34 & 2,152288344 & 0,23943662 \\ \hline
143 & 34 & 2,155336037 & 0,237762238 \\ \hline
144 & 34 & 2,158362492 & 0,236111111 \\ \hline
145 & 34 & 2,161368002 & 0,234482759 \\ \hline
146 & 34 & 2,164352856 & 0,232876712 \\ \hline
147 & 34 & 2,167317335 & 0,231292517 \\ \hline
148 & 34 & 2,170261715 & 0,22972973 \\ \hline
149 & 34 & 2,173186268 & 0,228187919 \\ \hline
150 & 35 & 2,176091259 & 0,233333333 \\ \hline
\end{longtable}
\end{minipage}
\begin{minipage}[!h]{0.45\textwidth}\centering
\tiny
\begin{longtable}{cccc} \hline
\(n\) & \(\pi(n)\) & \(\log(n)\) & \(\pi(n)/n\) \\ \hline151 & 35 & 2,178976947 & 0,231788079 \\ \hline
152 & 36 & 2,181843588 & 0,236842105 \\ \hline
153 & 36 & 2,184691431 & 0,235294118 \\ \hline
154 & 36 & 2,187520721 & 0,233766234 \\ \hline
155 & 36 & 2,190331698 & 0,232258065 \\ \hline
156 & 36 & 2,193124598 & 0,230769231 \\ \hline
157 & 36 & 2,195899652 & 0,229299363 \\ \hline
158 & 37 & 2,198657087 & 0,234177215 \\ \hline
159 & 37 & 2,201397124 & 0,232704403 \\ \hline
160 & 37 & 2,204119983 & 0,23125 \\ \hline
161 & 37 & 2,206825876 & 0,229813665 \\ \hline
162 & 37 & 2,209515015 & 0,228395062 \\ \hline
163 & 37 & 2,212187604 & 0,226993865 \\ \hline
164 & 38 & 2,214843848 & 0,231707317 \\ \hline
165 & 38 & 2,217483944 & 0,23030303 \\ \hline
166 & 38 & 2,220108088 & 0,228915663 \\ \hline
167 & 38 & 2,222716471 & 0,22754491 \\ \hline
168 & 39 & 2,225309282 & 0,232142857 \\ \hline
169 & 39 & 2,227886705 & 0,230769231 \\ \hline
170 & 39 & 2,230448921 & 0,229411765 \\ \hline
171 & 39 & 2,23299611 & 0,228070175 \\ \hline
172 & 39 & 2,235528447 & 0,226744186 \\ \hline
173 & 39 & 2,238046103 & 0,225433526 \\ \hline
174 & 40 & 2,240549248 & 0,229885057 \\ \hline
175 & 40 & 2,243038049 & 0,228571429 \\ \hline
176 & 40 & 2,245512668 & 0,227272727 \\ \hline
177 & 40 & 2,247973266 & 0,225988701 \\ \hline
178 & 40 & 2,250420002 & 0,224719101 \\ \hline
179 & 40 & 2,252853031 & 0,223463687 \\ \hline
180 & 41 & 2,255272505 & 0,227777778 \\ \hline
181 & 41 & 2,257678575 & 0,226519337 \\ \hline
182 & 42 & 2,260071388 & 0,230769231 \\ \hline
183 & 42 & 2,26245109 & 0,229508197 \\ \hline
184 & 42 & 2,264817823 & 0,22826087 \\ \hline
185 & 42 & 2,267171728 & 0,227027027 \\ \hline
186 & 42 & 2,269512944 & 0,225806452 \\ \hline
187 & 42 & 2,271841607 & 0,22459893 \\ \hline
188 & 42 & 2,274157849 & 0,223404255 \\ \hline
189 & 42 & 2,276461804 & 0,222222222 \\ \hline
190 & 42 & 2,278753601 & 0,221052632 \\ \hline
191 & 42 & 2,281033367 & 0,219895288 \\ \hline
192 & 43 & 2,283301229 & 0,223958333 \\ \hline
193 & 43 & 2,285557309 & 0,222797927 \\ \hline
194 & 44 & 2,28780173 & 0,226804124 \\ \hline
195 & 44 & 2,290034611 & 0,225641026 \\ \hline
196 & 44 & 2,292256071 & 0,224489796 \\ \hline
197 & 44 & 2,294466226 & 0,223350254 \\ \hline
198 & 45 & 2,29666519 & 0,227272727 \\ \hline
199 & 45 & 2,298853076 & 0,226130653 \\ \hline
200 & 46 & 2,301029996 & 0,23 \\ \hline
\end{longtable}
\end{minipage}

\begin{minipage}[!h]{0.45\textwidth}\centering
\tiny
\begin{longtable}{cccc} \hline
\(n\) & \(\pi(n)\) & \(\log(n)\) & \(\pi(n)/n\) \\ \hline201 & 46 & 2,303196057 & 0,228855721 \\ \hline
202 & 46 & 2,305351369 & 0,227722772 \\ \hline
203 & 46 & 2,307496038 & 0,226600985 \\ \hline
204 & 46 & 2,309630167 & 0,225490196 \\ \hline
205 & 46 & 2,311753861 & 0,224390244 \\ \hline
206 & 46 & 2,31386722 & 0,223300971 \\ \hline
207 & 46 & 2,315970345 & 0,222222222 \\ \hline
208 & 46 & 2,318063335 & 0,221153846 \\ \hline
209 & 46 & 2,320146286 & 0,220095694 \\ \hline
210 & 46 & 2,322219295 & 0,219047619 \\ \hline
211 & 46 & 2,324282455 & 0,218009479 \\ \hline
212 & 47 & 2,326335861 & 0,221698113 \\ \hline
213 & 47 & 2,328379603 & 0,220657277 \\ \hline
214 & 47 & 2,330413773 & 0,219626168 \\ \hline
215 & 47 & 2,33243846 & 0,218604651 \\ \hline
216 & 47 & 2,334453751 & 0,217592593 \\ \hline
217 & 47 & 2,336459734 & 0,216589862 \\ \hline
218 & 47 & 2,338456494 & 0,21559633 \\ \hline
219 & 47 & 2,340444115 & 0,214611872 \\ \hline
220 & 47 & 2,342422681 & 0,213636364 \\ \hline
221 & 47 & 2,344392274 & 0,212669683 \\ \hline
222 & 47 & 2,346352974 & 0,211711712 \\ \hline
223 & 47 & 2,348304863 & 0,210762332 \\ \hline
224 & 48 & 2,350248018 & 0,214285714 \\ \hline
225 & 48 & 2,352182518 & 0,213333333 \\ \hline
226 & 48 & 2,354108439 & 0,212389381 \\ \hline
227 & 48 & 2,356025857 & 0,211453744 \\ \hline
228 & 49 & 2,357934847 & 0,214912281 \\ \hline
229 & 49 & 2,359835482 & 0,213973799 \\ \hline
230 & 50 & 2,361727836 & 0,217391304 \\ \hline
231 & 50 & 2,36361198 & 0,216450216 \\ \hline
232 & 50 & 2,365487985 & 0,215517241 \\ \hline
233 & 50 & 2,367355921 & 0,214592275 \\ \hline
234 & 51 & 2,369215857 & 0,217948718 \\ \hline
235 & 51 & 2,371067862 & 0,217021277 \\ \hline
236 & 51 & 2,372912003 & 0,216101695 \\ \hline
237 & 51 & 2,374748346 & 0,215189873 \\ \hline
238 & 51 & 2,376576957 & 0,214285714 \\ \hline
239 & 51 & 2,378397901 & 0,213389121 \\ \hline
240 & 52 & 2,380211242 & 0,216666667 \\ \hline
241 & 52 & 2,382017043 & 0,215767635 \\ \hline
242 & 53 & 2,383815366 & 0,219008264 \\ \hline
243 & 53 & 2,385606274 & 0,218106996 \\ \hline
244 & 53 & 2,387389826 & 0,217213115 \\ \hline
245 & 53 & 2,389166084 & 0,216326531 \\ \hline
246 & 53 & 2,390935107 & 0,215447154 \\ \hline
247 & 53 & 2,392696953 & 0,214574899 \\ \hline
248 & 53 & 2,394451681 & 0,213709677 \\ \hline
249 & 53 & 2,396199347 & 0,212851406 \\ \hline
250 & 53 & 2,397940009 & 0,212 \\ \hline
\end{longtable}
\end{minipage}
\begin{minipage}[!h]{0.45\textwidth}\centering
\tiny
\begin{longtable}{cccc} \hline
\(n\) & \(\pi(n)\) & \(\log(n)\) & \(\pi(n)/n\) \\ \hline
251 & 53 & 2,399673721 & 0,211155378 \\ \hline
252 & 54 & 2,401400541 & 0,214285714 \\ \hline
253 & 54 & 2,403120521 & 0,213438735 \\ \hline
254 & 54 & 2,404833717 & 0,212598425 \\ \hline
255 & 54 & 2,40654018 & 0,211764706 \\ \hline
256 & 54 & 2,408239965 & 0,2109375 \\ \hline
257 & 54 & 2,409933123 & 0,210116732 \\ \hline
258 & 55 & 2,411619706 & 0,213178295 \\ \hline
259 & 55 & 2,413299764 & 0,212355212 \\ \hline
260 & 55 & 2,414973348 & 0,211538462 \\ \hline
261 & 55 & 2,416640507 & 0,210727969 \\ \hline
262 & 55 & 2,418301291 & 0,209923664 \\ \hline
263 & 55 & 2,419955748 & 0,209125475 \\ \hline
264 & 56 & 2,421603927 & 0,212121212 \\ \hline
265 & 56 & 2,423245874 & 0,211320755 \\ \hline
266 & 56 & 2,424881637 & 0,210526316 \\ \hline
267 & 56 & 2,426511261 & 0,209737828 \\ \hline
268 & 56 & 2,428134794 & 0,208955224 \\ \hline
269 & 56 & 2,42975228 & 0,208178439 \\ \hline
270 & 57 & 2,431363764 & 0,211111111 \\ \hline
271 & 57 & 2,432969291 & 0,210332103 \\ \hline
272 & 58 & 2,434568904 & 0,213235294 \\ \hline
273 & 58 & 2,436162647 & 0,212454212 \\ \hline
274 & 58 & 2,437750563 & 0,211678832 \\ \hline
275 & 58 & 2,439332694 & 0,210909091 \\ \hline
276 & 58 & 2,440909082 & 0,210144928 \\ \hline
277 & 58 & 2,442479769 & 0,209386282 \\ \hline
278 & 59 & 2,444044796 & 0,212230216 \\ \hline
279 & 59 & 2,445604203 & 0,211469534 \\ \hline
280 & 59 & 2,447158031 & 0,210714286 \\ \hline
281 & 59 & 2,44870632 & 0,209964413 \\ \hline
282 & 60 & 2,450249108 & 0,212765957 \\ \hline
283 & 60 & 2,451786436 & 0,212014134 \\ \hline
284 & 61 & 2,45331834 & 0,214788732 \\ \hline
285 & 61 & 2,45484486 & 0,214035088 \\ \hline
286 & 61 & 2,456366033 & 0,213286713 \\ \hline
287 & 61 & 2,457881897 & 0,212543554 \\ \hline
288 & 61 & 2,459392488 & 0,211805556 \\ \hline
289 & 61 & 2,460897843 & 0,211072664 \\ \hline
290 & 61 & 2,462397998 & 0,210344828 \\ \hline
291 & 61 & 2,463892989 & 0,209621993 \\ \hline
292 & 61 & 2,465382851 & 0,20890411 \\ \hline
293 & 61 & 2,46686762 & 0,208191126 \\ \hline
294 & 62 & 2,46834733 & 0,210884354 \\ \hline
295 & 62 & 2,469822016 & 0,210169492 \\ \hline
296 & 62 & 2,471291711 & 0,209459459 \\ \hline
297 & 62 & 2,472756449 & 0,208754209 \\ \hline
298 & 62 & 2,474216264 & 0,208053691 \\ \hline
299 & 62 & 2,475671188 & 0,20735786 \\ \hline
300 & 62 & 2,477121255 & 0,206666667 \\ \hline
\end{longtable}
\end{minipage}

\begin{minipage}[!h]{0.45\textwidth}\centering
\tiny
\begin{longtable}{cccc} \hline
\(n\) & \(\pi(n)\) & \(\log(n)\) & \(\pi(n)/n\) \\ \hline
301 & 62 & 2,478566496 & 0,205980066 \\ \hline
302 & 62 & 2,480006943 & 0,205298013 \\ \hline
303 & 62 & 2,481442629 & 0,204620462 \\ \hline
304 & 62 & 2,482873584 & 0,203947368 \\ \hline
305 & 62 & 2,484299839 & 0,203278689 \\ \hline
306 & 62 & 2,485721426 & 0,202614379 \\ \hline
307 & 62 & 2,487138375 & 0,201954397 \\ \hline
308 & 63 & 2,488550717 & 0,204545455 \\ \hline
309 & 63 & 2,489958479 & 0,203883495 \\ \hline
310 & 63 & 2,491361694 & 0,203225806 \\ \hline
311 & 63 & 2,492760389 & 0,202572347 \\ \hline
312 & 64 & 2,494154594 & 0,205128205 \\ \hline
313 & 64 & 2,495544338 & 0,204472843 \\ \hline
314 & 65 & 2,496929648 & 0,207006369 \\ \hline
315 & 65 & 2,498310554 & 0,206349206 \\ \hline
316 & 65 & 2,499687083 & 0,205696203 \\ \hline
317 & 65 & 2,501059262 & 0,205047319 \\ \hline
318 & 66 & 2,50242712 & 0,20754717 \\ \hline
319 & 66 & 2,503790683 & 0,206896552 \\ \hline
320 & 66 & 2,505149978 & 0,20625 \\ \hline
321 & 66 & 2,506505032 & 0,205607477 \\ \hline
322 & 66 & 2,507855872 & 0,204968944 \\ \hline
323 & 66 & 2,509202522 & 0,204334365 \\ \hline
324 & 66 & 2,51054501 & 0,203703704 \\ \hline
325 & 66 & 2,511883361 & 0,203076923 \\ \hline
326 & 66 & 2,5132176 & 0,202453988 \\ \hline
327 & 66 & 2,514547753 & 0,201834862 \\ \hline
328 & 66 & 2,515873844 & 0,201219512 \\ \hline
329 & 66 & 2,517195898 & 0,200607903 \\ \hline
330 & 66 & 2,51851394 & 0,2 \\ \hline
331 & 66 & 2,519827994 & 0,19939577 \\ \hline
332 & 67 & 2,521138084 & 0,201807229 \\ \hline
333 & 67 & 2,522444234 & 0,201201201 \\ \hline
334 & 67 & 2,523746467 & 0,200598802 \\ \hline
335 & 67 & 2,525044807 & 0,2 \\ \hline
336 & 67 & 2,526339277 & 0,199404762 \\ \hline
337 & 67 & 2,527629901 & 0,198813056 \\ \hline
338 & 68 & 2,5289167 & 0,201183432 \\ \hline
339 & 68 & 2,530199698 & 0,200589971 \\ \hline
340 & 68 & 2,531478917 & 0,2 \\ \hline
341 & 68 & 2,532754379 & 0,19941349 \\ \hline
342 & 68 & 2,534026106 & 0,198830409 \\ \hline
343 & 68 & 2,53529412 & 0,198250729 \\ \hline
344 & 68 & 2,536558443 & 0,197674419 \\ \hline
345 & 68 & 2,537819095 & 0,197101449 \\ \hline
346 & 68 & 2,539076099 & 0,196531792 \\ \hline
347 & 68 & 2,540329475 & 0,195965418 \\ \hline
348 & 69 & 2,541579244 & 0,198275862 \\ \hline
349 & 69 & 2,542825427 & 0,197707736 \\ \hline
350 & 70 & 2,544068044 & 0,2 \\ \hline
\end{longtable}
\end{minipage}
\begin{minipage}[!h]{0.45\textwidth}\centering
\tiny
\begin{longtable}{cccc} \hline
\(n\) & \(\pi(n)\) & \(\log(n)\) & \(\pi(n)/n\) \\ \hline
351 & 70 & 2,545307116 & 0,199430199 \\ \hline
352 & 70 & 2,546542663 & 0,198863636 \\ \hline
353 & 70 & 2,547774705 & 0,198300283 \\ \hline
354 & 71 & 2,549003262 & 0,200564972 \\ \hline
355 & 71 & 2,550228353 & 0,2 \\ \hline
356 & 71 & 2,551449998 & 0,199438202 \\ \hline
357 & 71 & 2,552668216 & 0,198879552 \\ \hline
358 & 71 & 2,553883027 & 0,198324022 \\ \hline
359 & 71 & 2,555094449 & 0,197771588 \\ \hline
360 & 72 & 2,556302501 & 0,2 \\ \hline
361 & 72 & 2,557507202 & 0,199445983 \\ \hline
362 & 72 & 2,558708571 & 0,198895028 \\ \hline
363 & 72 & 2,559906625 & 0,198347107 \\ \hline
364 & 72 & 2,561101384 & 0,197802198 \\ \hline
365 & 72 & 2,562292864 & 0,197260274 \\ \hline
366 & 72 & 2,563481085 & 0,196721311 \\ \hline
367 & 72 & 2,564666064 & 0,196185286 \\ \hline
368 & 73 & 2,565847819 & 0,198369565 \\ \hline
369 & 73 & 2,567026366 & 0,197831978 \\ \hline
370 & 73 & 2,568201724 & 0,197297297 \\ \hline
371 & 73 & 2,56937391 & 0,196765499 \\ \hline
372 & 73 & 2,57054294 & 0,196236559 \\ \hline
373 & 73 & 2,571708832 & 0,195710456 \\ \hline
374 & 74 & 2,572871602 & 0,197860963 \\ \hline
375 & 74 & 2,574031268 & 0,197333333 \\ \hline
376 & 74 & 2,575187845 & 0,196808511 \\ \hline
377 & 74 & 2,57634135 & 0,196286472 \\ \hline
378 & 74 & 2,5774918 & 0,195767196 \\ \hline
379 & 74 & 2,57863921 & 0,19525066 \\ \hline
380 & 75 & 2,579783597 & 0,197368421 \\ \hline
381 & 75 & 2,580924976 & 0,196850394 \\ \hline
382 & 75 & 2,582063363 & 0,196335079 \\ \hline
383 & 75 & 2,583198774 & 0,195822454 \\ \hline
384 & 76 & 2,584331224 & 0,197916667 \\ \hline
385 & 76 & 2,58546073 & 0,197402597 \\ \hline
386 & 76 & 2,586587305 & 0,196891192 \\ \hline
387 & 76 & 2,587710965 & 0,196382429 \\ \hline
388 & 76 & 2,588831726 & 0,195876289 \\ \hline
389 & 76 & 2,589949601 & 0,195372751 \\ \hline
390 & 77 & 2,591064607 & 0,197435897 \\ \hline
391 & 77 & 2,592176757 & 0,196930946 \\ \hline
392 & 77 & 2,593286067 & 0,196428571 \\ \hline
393 & 77 & 2,59439255 & 0,195928753 \\ \hline
394 & 77 & 2,595496222 & 0,195431472 \\ \hline
395 & 77 & 2,596597096 & 0,194936709 \\ \hline
396 & 77 & 2,597695186 & 0,194444444 \\ \hline
397 & 77 & 2,598790507 & 0,19395466 \\ \hline
398 & 78 & 2,599883072 & 0,195979899 \\ \hline
399 & 78 & 2,600972896 & 0,195488722 \\ \hline
400 & 78 & 2,602059991 & 0,195 \\ \hline
\end{longtable}
\end{minipage}

\begin{minipage}[!h]{0.45\textwidth}\centering
\tiny
\begin{longtable}{cccc} \hline
\(n\) & \(\pi(n)\) & \(\log(n)\) & \(\pi(n)/n\) \\ \hline
401 & 78 & 2,603144373 & 0,194513716 \\ \hline
402 & 79 & 2,604226053 & 0,196517413 \\ \hline
403 & 79 & 2,605305046 & 0,196029777 \\ \hline
404 & 79 & 2,606381365 & 0,195544554 \\ \hline
405 & 79 & 2,607455023 & 0,195061728 \\ \hline
406 & 79 & 2,608526034 & 0,194581281 \\ \hline
407 & 79 & 2,609594409 & 0,194103194 \\ \hline
408 & 79 & 2,610660163 & 0,193627451 \\ \hline
409 & 79 & 2,611723308 & 0,193154034 \\ \hline
410 & 80 & 2,612783857 & 0,195121951 \\ \hline
411 & 80 & 2,613841822 & 0,194647202 \\ \hline
412 & 80 & 2,614897216 & 0,194174757 \\ \hline
413 & 80 & 2,615950052 & 0,1937046 \\ \hline
414 & 80 & 2,617000341 & 0,193236715 \\ \hline
415 & 80 & 2,618048097 & 0,192771084 \\ \hline
416 & 80 & 2,619093331 & 0,192307692 \\ \hline
417 & 80 & 2,620136055 & 0,191846523 \\ \hline
418 & 80 & 2,621176282 & 0,19138756 \\ \hline
419 & 80 & 2,622214023 & 0,190930788 \\ \hline
420 & 81 & 2,62324929 & 0,192857143 \\ \hline
421 & 81 & 2,624282096 & 0,19239905 \\ \hline
422 & 82 & 2,625312451 & 0,194312796 \\ \hline
423 & 82 & 2,626340367 & 0,193853428 \\ \hline
424 & 82 & 2,627365857 & 0,193396226 \\ \hline
425 & 82 & 2,62838893 & 0,192941176 \\ \hline
426 & 82 & 2,629409599 & 0,192488263 \\ \hline
427 & 82 & 2,630427875 & 0,192037471 \\ \hline
428 & 82 & 2,631443769 & 0,191588785 \\ \hline
429 & 82 & 2,632457292 & 0,191142191 \\ \hline
430 & 82 & 2,633468456 & 0,190697674 \\ \hline
431 & 82 & 2,63447727 & 0,19025522 \\ \hline
432 & 83 & 2,635483747 & 0,19212963 \\ \hline
433 & 83 & 2,636487896 & 0,191685912 \\ \hline
434 & 84 & 2,63748973 & 0,193548387 \\ \hline
435 & 84 & 2,638489257 & 0,193103448 \\ \hline
436 & 84 & 2,639486489 & 0,19266055 \\ \hline
437 & 84 & 2,640481437 & 0,19221968 \\ \hline
438 & 84 & 2,641474111 & 0,191780822 \\ \hline
439 & 84 & 2,64246452 & 0,191343964 \\ \hline
440 & 85 & 2,643452676 & 0,193181818 \\ \hline
441 & 85 & 2,644438589 & 0,192743764 \\ \hline
442 & 85 & 2,645422269 & 0,192307692 \\ \hline
443 & 85 & 2,646403726 & 0,191873589 \\ \hline
444 & 86 & 2,64738297 & 0,193693694 \\ \hline
445 & 86 & 2,648360011 & 0,193258427 \\ \hline
446 & 86 & 2,649334859 & 0,192825112 \\ \hline
447 & 86 & 2,650307523 & 0,192393736 \\ \hline
448 & 86 & 2,651278014 & 0,191964286 \\ \hline
449 & 86 & 2,652246341 & 0,191536748 \\ \hline
450 & 87 & 2,653212514 & 0,193333333 \\ \hline
\end{longtable}
\end{minipage}
\begin{minipage}[!h]{0.45\textwidth}\centering
\tiny
\begin{longtable}{cccc} \hline
\(n\) & \(\pi(n)\) & \(\log(n)\) & \(\pi(n)/n\) \\ \hline
451 & 87 & 2,654176542 & 0,192904656 \\ \hline
452 & 87 & 2,655138435 & 0,192477876 \\ \hline
453 & 87 & 2,656098202 & 0,19205298 \\ \hline
454 & 87 & 2,657055853 & 0,191629956 \\ \hline
455 & 87 & 2,658011397 & 0,191208791 \\ \hline
456 & 87 & 2,658964843 & 0,190789474 \\ \hline
457 & 87 & 2,6599162 & 0,190371991 \\ \hline
458 & 88 & 2,660865478 & 0,192139738 \\ \hline
459 & 88 & 2,661812686 & 0,191721133 \\ \hline
460 & 88 & 2,662757832 & 0,191304348 \\ \hline
461 & 88 & 2,663700925 & 0,190889371 \\ \hline
462 & 89 & 2,664641976 & 0,192640693 \\ \hline
463 & 89 & 2,665580991 & 0,192224622 \\ \hline
464 & 90 & 2,666517981 & 0,193965517 \\ \hline
465 & 90 & 2,667452953 & 0,193548387 \\ \hline
466 & 90 & 2,668385917 & 0,193133047 \\ \hline
467 & 90 & 2,669316881 & 0,192719486 \\ \hline
468 & 91 & 2,670245853 & 0,194444444 \\ \hline
469 & 91 & 2,671172843 & 0,194029851 \\ \hline
470 & 91 & 2,672097858 & 0,193617021 \\ \hline
471 & 91 & 2,673020907 & 0,193205945 \\ \hline
472 & 91 & 2,673941999 & 0,19279661 \\ \hline
473 & 91 & 2,674861141 & 0,192389006 \\ \hline
474 & 91 & 2,675778342 & 0,191983122 \\ \hline
475 & 91 & 2,67669361 & 0,191578947 \\ \hline
476 & 91 & 2,677606953 & 0,191176471 \\ \hline
477 & 91 & 2,678518379 & 0,190775681 \\ \hline
478 & 91 & 2,679427897 & 0,190376569 \\ \hline
479 & 91 & 2,680335513 & 0,189979123 \\ \hline
480 & 92 & 2,681241237 & 0,191666667 \\ \hline
481 & 92 & 2,682145076 & 0,191268191 \\ \hline
482 & 92 & 2,683047038 & 0,190871369 \\ \hline
483 & 92 & 2,683947131 & 0,19047619 \\ \hline
484 & 92 & 2,684845362 & 0,190082645 \\ \hline
485 & 92 & 2,685741739 & 0,189690722 \\ \hline
486 & 92 & 2,686636269 & 0,189300412 \\ \hline
487 & 92 & 2,687528961 & 0,188911704 \\ \hline
488 & 93 & 2,688419822 & 0,19057377 \\ \hline
489 & 93 & 2,689308859 & 0,190184049 \\ \hline
490 & 93 & 2,69019608 & 0,189795918 \\ \hline
491 & 93 & 2,691081492 & 0,189409369 \\ \hline
492 & 94 & 2,691965103 & 0,191056911 \\ \hline
493 & 94 & 2,692846919 & 0,190669371 \\ \hline
494 & 94 & 2,693726949 & 0,190283401 \\ \hline
495 & 94 & 2,694605199 & 0,18989899 \\ \hline
496 & 94 & 2,695481676 & 0,189516129 \\ \hline
497 & 94 & 2,696356389 & 0,189134809 \\ \hline
498 & 94 & 2,697229343 & 0,18875502 \\ \hline
499 & 94 & 2,698100546 & 0,188376754 \\ \hline
500 & 95 & 2,698970004 & 0,19 \\ \hline
\end{longtable}
\end{minipage}

\begin{minipage}[!h]{0.45\textwidth}\centering
\tiny
\begin{longtable}{cccc} \hline
\(n\) & \(\pi(n)\) & \(\log(n)\) & \(\pi(n)/n\) \\ \hline
501 & 95 & 2,699837726 & 0,189620758 \\ \hline
502 & 95 & 2,700703717 & 0,189243028 \\ \hline
503 & 95 & 2,701567985 & 0,188866799 \\ \hline
504 & 96 & 2,702430536 & 0,19047619 \\ \hline
505 & 96 & 2,703291378 & 0,19009901 \\ \hline
506 & 96 & 2,704150517 & 0,18972332 \\ \hline
507 & 96 & 2,705007959 & 0,189349112 \\ \hline
508 & 96 & 2,705863712 & 0,188976378 \\ \hline
509 & 96 & 2,706717782 & 0,188605108 \\ \hline
510 & 97 & 2,707570176 & 0,190196078 \\ \hline
511 & 97 & 2,7084209 & 0,189823875 \\ \hline
512 & 97 & 2,709269961 & 0,189453125 \\ \hline
513 & 97 & 2,710117365 & 0,189083821 \\ \hline
514 & 97 & 2,710963119 & 0,188715953 \\ \hline
515 & 97 & 2,711807229 & 0,188349515 \\ \hline
516 & 97 & 2,712649702 & 0,187984496 \\ \hline
517 & 97 & 2,713490543 & 0,18762089 \\ \hline
518 & 97 & 2,71432976 & 0,187258687 \\ \hline
519 & 97 & 2,715167358 & 0,186897881 \\ \hline
520 & 97 & 2,716003344 & 0,186538462 \\ \hline
521 & 97 & 2,716837723 & 0,186180422 \\ \hline
522 & 98 & 2,717670503 & 0,187739464 \\ \hline
523 & 98 & 2,718501689 & 0,187380497 \\ \hline
524 & 99 & 2,719331287 & 0,188931298 \\ \hline
525 & 99 & 2,720159303 & 0,188571429 \\ \hline
526 & 99 & 2,720985744 & 0,188212928 \\ \hline
527 & 99 & 2,721810615 & 0,187855787 \\ \hline
528 & 99 & 2,722633923 & 0,1875 \\ \hline
529 & 99 & 2,723455672 & 0,187145558 \\ \hline
530 & 99 & 2,72427587 & 0,186792453 \\ \hline
531 & 99 & 2,725094521 & 0,186440678 \\ \hline
532 & 99 & 2,725911632 & 0,186090226 \\ \hline
533 & 99 & 2,726727209 & 0,185741088 \\ \hline
534 & 99 & 2,727541257 & 0,185393258 \\ \hline
535 & 99 & 2,728353782 & 0,185046729 \\ \hline
536 & 99 & 2,72916479 & 0,184701493 \\ \hline
537 & 99 & 2,729974286 & 0,184357542 \\ \hline
538 & 99 & 2,730782276 & 0,18401487 \\ \hline
539 & 99 & 2,731588765 & 0,183673469 \\ \hline
540 & 99 & 2,73239376 & 0,183333333 \\ \hline
541 & 99 & 2,733197265 & 0,182994455 \\ \hline
542 & 100 & 2,733999287 & 0,184501845 \\ \hline
543 & 100 & 2,73479983 & 0,184162063 \\ \hline
544 & 100 & 2,7355989 & 0,183823529 \\ \hline
545 & 100 & 2,736396502 & 0,183486239 \\ \hline
546 & 100 & 2,737192643 & 0,183150183 \\ \hline
547 & 100 & 2,737987326 & 0,182815356 \\ \hline
548 & 101 & 2,738780558 & 0,184306569 \\ \hline
549 & 101 & 2,739572344 & 0,183970856 \\ \hline
550 & 101 & 2,740362689 & 0,183636364 \\ \hline
\end{longtable}
\end{minipage}
\begin{minipage}[!h]{0.45\textwidth}\centering
\tiny
\begin{longtable}{cccc} \hline
\(n\) & \(\pi(n)\) & \(\log(n)\) & \(\pi(n)/n\) \\ \hline
551 & 101 & 2,741151599 & 0,183303085 \\ \hline
552 & 101 & 2,741939078 & 0,182971014 \\ \hline
553 & 101 & 2,742725131 & 0,182640145 \\ \hline
554 & 101 & 2,743509765 & 0,182310469 \\ \hline
555 & 101 & 2,744292983 & 0,181981982 \\ \hline
556 & 101 & 2,745074792 & 0,181654676 \\ \hline
557 & 101 & 2,745855195 & 0,181328546 \\ \hline
558 & 102 & 2,746634199 & 0,182795699 \\ \hline
559 & 102 & 2,747411808 & 0,182468694 \\ \hline
560 & 102 & 2,748188027 & 0,182142857 \\ \hline
561 & 102 & 2,748962861 & 0,181818182 \\ \hline
562 & 102 & 2,749736316 & 0,181494662 \\ \hline
563 & 102 & 2,750508395 & 0,181172291 \\ \hline
564 & 103 & 2,751279104 & 0,182624113 \\ \hline
565 & 103 & 2,752048448 & 0,182300885 \\ \hline
566 & 103 & 2,752816431 & 0,181978799 \\ \hline
567 & 103 & 2,753583059 & 0,181657848 \\ \hline
568 & 103 & 2,754348336 & 0,181338028 \\ \hline
569 & 103 & 2,755112266 & 0,181019332 \\ \hline
570 & 104 & 2,755874856 & 0,18245614 \\ \hline
571 & 104 & 2,756636108 & 0,182136602 \\ \hline
572 & 105 & 2,757396029 & 0,183566434 \\ \hline
573 & 105 & 2,758154622 & 0,183246073 \\ \hline
574 & 105 & 2,758911892 & 0,182926829 \\ \hline
575 & 105 & 2,759667845 & 0,182608696 \\ \hline
576 & 105 & 2,760422483 & 0,182291667 \\ \hline
577 & 105 & 2,761175813 & 0,181975737 \\ \hline
578 & 106 & 2,761927838 & 0,183391003 \\ \hline
579 & 106 & 2,762678564 & 0,183074266 \\ \hline
580 & 106 & 2,763427994 & 0,182758621 \\ \hline
581 & 106 & 2,764176132 & 0,182444062 \\ \hline
582 & 106 & 2,764922985 & 0,182130584 \\ \hline
583 & 106 & 2,765668555 & 0,181818182 \\ \hline
584 & 106 & 2,766412847 & 0,181506849 \\ \hline
585 & 106 & 2,767155866 & 0,181196581 \\ \hline
586 & 106 & 2,767897616 & 0,180887372 \\ \hline
587 & 106 & 2,768638101 & 0,180579216 \\ \hline
588 & 107 & 2,769377326 & 0,181972789 \\ \hline
589 & 107 & 2,770115295 & 0,181663837 \\ \hline
590 & 107 & 2,770852012 & 0,181355932 \\ \hline
591 & 107 & 2,771587481 & 0,181049069 \\ \hline
592 & 107 & 2,772321707 & 0,180743243 \\ \hline
593 & 107 & 2,773054693 & 0,180438449 \\ \hline
594 & 108 & 2,773786445 & 0,181818182 \\ \hline
595 & 108 & 2,774516966 & 0,181512605 \\ \hline
596 & 108 & 2,77524626 & 0,181208054 \\ \hline
597 & 108 & 2,775974331 & 0,180904523 \\ \hline
598 & 108 & 2,776701184 & 0,180602007 \\ \hline
599 & 108 & 2,777426822 & 0,180300501 \\ \hline
600 & 109 & 2,77815125 & 0,181666667 \\ \hline
\end{longtable}
\end{minipage}

\begin{minipage}[!h]{0.45\textwidth}\centering
\tiny
\begin{longtable}{cccc} \hline
\(n\) & \(\pi(n)\) & \(\log(n)\) & \(\pi(n)/n\) \\ \hline
601 & 109 & 2,778874472 & 0,181364393 \\ \hline
602 & 110 & 2,779596491 & 0,182724252 \\ \hline
603 & 110 & 2,780317312 & 0,182421227 \\ \hline
604 & 110 & 2,781036939 & 0,182119205 \\ \hline
605 & 110 & 2,781755375 & 0,181818182 \\ \hline
606 & 110 & 2,782472624 & 0,181518152 \\ \hline
607 & 110 & 2,783188691 & 0,18121911 \\ \hline
608 & 111 & 2,783903579 & 0,182565789 \\ \hline
609 & 111 & 2,784617293 & 0,18226601 \\ \hline
610 & 111 & 2,785329835 & 0,181967213 \\ \hline
611 & 111 & 2,78604121 & 0,181669394 \\ \hline
612 & 111 & 2,786751422 & 0,181372549 \\ \hline
613 & 111 & 2,787460475 & 0,181076672 \\ \hline
614 & 112 & 2,788168371 & 0,182410423 \\ \hline
615 & 112 & 2,788875116 & 0,182113821 \\ \hline
616 & 112 & 2,789580712 & 0,181818182 \\ \hline
617 & 112 & 2,790285164 & 0,181523501 \\ \hline
618 & 113 & 2,790988475 & 0,182847896 \\ \hline
619 & 113 & 2,791690649 & 0,182552504 \\ \hline
620 & 114 & 2,792391689 & 0,183870968 \\ \hline
621 & 114 & 2,7930916 & 0,183574879 \\ \hline
622 & 114 & 2,793790385 & 0,183279743 \\ \hline
623 & 114 & 2,794488047 & 0,182985554 \\ \hline
624 & 114 & 2,79518459 & 0,182692308 \\ \hline
625 & 114 & 2,795880017 & 0,1824 \\ \hline
626 & 114 & 2,796574333 & 0,182108626 \\ \hline
627 & 114 & 2,797267541 & 0,181818182 \\ \hline
628 & 114 & 2,797959644 & 0,181528662 \\ \hline
629 & 114 & 2,798650645 & 0,181240064 \\ \hline
630 & 114 & 2,799340549 & 0,180952381 \\ \hline
631 & 114 & 2,800029359 & 0,18066561 \\ \hline
632 & 115 & 2,800717078 & 0,181962025 \\ \hline
633 & 115 & 2,80140371 & 0,181674566 \\ \hline
634 & 115 & 2,802089258 & 0,181388013 \\ \hline
635 & 115 & 2,802773725 & 0,181102362 \\ \hline
636 & 115 & 2,803457116 & 0,18081761 \\ \hline
637 & 115 & 2,804139432 & 0,180533752 \\ \hline
638 & 115 & 2,804820679 & 0,180250784 \\ \hline
639 & 115 & 2,805500858 & 0,179968701 \\ \hline
640 & 115 & 2,806179974 & 0,1796875 \\ \hline
641 & 115 & 2,80685803 & 0,179407176 \\ \hline
642 & 116 & 2,807535028 & 0,180685358 \\ \hline
643 & 116 & 2,808210973 & 0,180404355 \\ \hline
644 & 117 & 2,808885867 & 0,181677019 \\ \hline
645 & 117 & 2,809559715 & 0,181395349 \\ \hline
646 & 117 & 2,810232518 & 0,181114551 \\ \hline
647 & 117 & 2,810904281 & 0,180834621 \\ \hline
648 & 118 & 2,811575006 & 0,182098765 \\ \hline
649 & 118 & 2,812244697 & 0,181818182 \\ \hline
650 & 118 & 2,812913357 & 0,181538462 \\ \hline
\end{longtable}
\end{minipage}
\begin{minipage}[!h]{0.45\textwidth}\centering
\tiny
\begin{longtable}{cccc} \hline
\(n\) & \(\pi(n)\) & \(\log(n)\) & \(\pi(n)/n\) \\ \hline
651 & 118 & 2,813580989 & 0,181259601 \\ \hline
652 & 118 & 2,814247596 & 0,180981595 \\ \hline
653 & 118 & 2,814913181 & 0,180704441 \\ \hline
654 & 119 & 2,815577748 & 0,181957187 \\ \hline
655 & 119 & 2,8162413 & 0,181679389 \\ \hline
656 & 119 & 2,816903839 & 0,181402439 \\ \hline
657 & 119 & 2,81756537 & 0,181126332 \\ \hline
658 & 119 & 2,818225894 & 0,180851064 \\ \hline
659 & 119 & 2,818885415 & 0,180576631 \\ \hline
660 & 120 & 2,819543936 & 0,181818182 \\ \hline
661 & 120 & 2,820201459 & 0,181543116 \\ \hline
662 & 121 & 2,820857989 & 0,182779456 \\ \hline
663 & 121 & 2,821513528 & 0,182503771 \\ \hline
664 & 121 & 2,822168079 & 0,182228916 \\ \hline
665 & 121 & 2,822821645 & 0,181954887 \\ \hline
666 & 121 & 2,823474229 & 0,181681682 \\ \hline
667 & 121 & 2,824125834 & 0,181409295 \\ \hline
668 & 121 & 2,824776462 & 0,181137725 \\ \hline
669 & 121 & 2,825426118 & 0,180866966 \\ \hline
670 & 121 & 2,826074803 & 0,180597015 \\ \hline
671 & 121 & 2,82672252 & 0,180327869 \\ \hline
672 & 121 & 2,827369273 & 0,180059524 \\ \hline
673 & 121 & 2,828015064 & 0,179791976 \\ \hline
674 & 122 & 2,828659897 & 0,181008902 \\ \hline
675 & 122 & 2,829303773 & 0,180740741 \\ \hline
676 & 122 & 2,829946696 & 0,180473373 \\ \hline
677 & 122 & 2,830588669 & 0,180206795 \\ \hline
678 & 123 & 2,831229694 & 0,181415929 \\ \hline
679 & 123 & 2,831869774 & 0,181148748 \\ \hline
680 & 123 & 2,832508913 & 0,180882353 \\ \hline
681 & 123 & 2,833147112 & 0,18061674 \\ \hline
682 & 123 & 2,833784375 & 0,180351906 \\ \hline
683 & 123 & 2,834420704 & 0,180087848 \\ \hline
684 & 124 & 2,835056102 & 0,18128655 \\ \hline
685 & 124 & 2,835690571 & 0,181021898 \\ \hline
686 & 124 & 2,836324116 & 0,180758017 \\ \hline
687 & 124 & 2,836956737 & 0,180494905 \\ \hline
688 & 124 & 2,837588438 & 0,180232558 \\ \hline
689 & 124 & 2,838219222 & 0,179970972 \\ \hline
690 & 124 & 2,838849091 & 0,179710145 \\ \hline
691 & 124 & 2,839478047 & 0,179450072 \\ \hline
692 & 125 & 2,840106094 & 0,180635838 \\ \hline
693 & 125 & 2,840733235 & 0,18037518 \\ \hline
694 & 125 & 2,84135947 & 0,180115274 \\ \hline
695 & 125 & 2,841984805 & 0,179856115 \\ \hline
696 & 125 & 2,84260924 & 0,179597701 \\ \hline
697 & 125 & 2,843232778 & 0,179340029 \\ \hline
698 & 125 & 2,843855423 & 0,179083095 \\ \hline
699 & 125 & 2,844477176 & 0,178826896 \\ \hline
700 & 125 & 2,84509804 & 0,178571429 \\ \hline
\end{longtable}
\end{minipage}

\begin{minipage}[!h]{0.45\textwidth}\centering
\tiny
\begin{longtable}{cccc} \hline
\(n\) & \(\pi(n)\) & \(\log(n)\) & \(\pi(n)/n\) \\ \hline
701 & 125 & 2,845718018 & 0,17831669 \\ \hline
702 & 126 & 2,846337112 & 0,179487179 \\ \hline
703 & 126 & 2,846955325 & 0,179231863 \\ \hline
704 & 126 & 2,847572659 & 0,178977273 \\ \hline
705 & 126 & 2,848189117 & 0,178723404 \\ \hline
706 & 126 & 2,848804701 & 0,178470255 \\ \hline
707 & 126 & 2,849419414 & 0,178217822 \\ \hline
708 & 126 & 2,850033258 & 0,177966102 \\ \hline
709 & 126 & 2,850646235 & 0,177715092 \\ \hline
710 & 127 & 2,851258349 & 0,178873239 \\ \hline
711 & 127 & 2,851869601 & 0,17862166 \\ \hline
712 & 127 & 2,852479994 & 0,178370787 \\ \hline
713 & 127 & 2,85308953 & 0,178120617 \\ \hline
714 & 127 & 2,853698212 & 0,177871148 \\ \hline
715 & 127 & 2,854306042 & 0,177622378 \\ \hline
716 & 127 & 2,854913022 & 0,177374302 \\ \hline
717 & 127 & 2,855519156 & 0,177126918 \\ \hline
718 & 127 & 2,856124444 & 0,176880223 \\ \hline
719 & 127 & 2,85672889 & 0,176634214 \\ \hline
720 & 128 & 2,857332496 & 0,177777778 \\ \hline
721 & 128 & 2,857935265 & 0,177531207 \\ \hline
722 & 128 & 2,858537198 & 0,177285319 \\ \hline
723 & 128 & 2,859138297 & 0,177040111 \\ \hline
724 & 128 & 2,859738566 & 0,17679558 \\ \hline
725 & 128 & 2,860338007 & 0,176551724 \\ \hline
726 & 128 & 2,860936621 & 0,17630854 \\ \hline
727 & 128 & 2,861534411 & 0,176066025 \\ \hline
728 & 129 & 2,862131379 & 0,177197802 \\ \hline
729 & 129 & 2,862727528 & 0,176954733 \\ \hline
730 & 129 & 2,86332286 & 0,176712329 \\ \hline
731 & 129 & 2,863917377 & 0,176470588 \\ \hline
732 & 129 & 2,864511081 & 0,176229508 \\ \hline
733 & 129 & 2,865103975 & 0,175989086 \\ \hline
734 & 130 & 2,86569606 & 0,177111717 \\ \hline
735 & 130 & 2,866287339 & 0,176870748 \\ \hline
736 & 130 & 2,866877814 & 0,176630435 \\ \hline
737 & 130 & 2,867467488 & 0,176390773 \\ \hline
738 & 130 & 2,868056362 & 0,176151762 \\ \hline
739 & 130 & 2,868644438 & 0,175913396 \\ \hline
740 & 131 & 2,86923172 & 0,177027027 \\ \hline
741 & 131 & 2,869818208 & 0,176788124 \\ \hline
742 & 131 & 2,870403905 & 0,176549865 \\ \hline
743 & 131 & 2,870988814 & 0,176312248 \\ \hline
744 & 132 & 2,871572936 & 0,177419355 \\ \hline
745 & 132 & 2,872156273 & 0,177181208 \\ \hline
746 & 132 & 2,872738827 & 0,1769437 \\ \hline
747 & 132 & 2,873320602 & 0,176706827 \\ \hline
748 & 132 & 2,873901598 & 0,176470588 \\ \hline
749 & 132 & 2,874481818 & 0,17623498 \\ \hline
750 & 132 & 2,875061263 & 0,176 \\ \hline
\end{longtable}
\end{minipage}
\begin{minipage}[!h]{0.45\textwidth}\centering
\tiny
\begin{longtable}{cccc} \hline
\(n\) & \(\pi(n)\) & \(\log(n)\) & \(\pi(n)/n\) \\ \hline
751 & 132 & 2,875639937 & 0,175765646 \\ \hline
752 & 133 & 2,876217841 & 0,176861702 \\ \hline
753 & 133 & 2,876794976 & 0,176626826 \\ \hline
754 & 133 & 2,877371346 & 0,176392573 \\ \hline
755 & 133 & 2,877946952 & 0,17615894 \\ \hline
756 & 133 & 2,878521796 & 0,175925926 \\ \hline
757 & 133 & 2,87909588 & 0,175693527 \\ \hline
758 & 134 & 2,879669206 & 0,176781003 \\ \hline
759 & 134 & 2,880241776 & 0,17654809 \\ \hline
760 & 134 & 2,880813592 & 0,176315789 \\ \hline
761 & 134 & 2,881384657 & 0,1760841 \\ \hline
762 & 135 & 2,881954971 & 0,177165354 \\ \hline
763 & 135 & 2,882524538 & 0,176933159 \\ \hline
764 & 135 & 2,883093359 & 0,176701571 \\ \hline
765 & 135 & 2,883661435 & 0,176470588 \\ \hline
766 & 135 & 2,88422877 & 0,176240209 \\ \hline
767 & 135 & 2,884795364 & 0,17601043 \\ \hline
768 & 135 & 2,88536122 & 0,17578125 \\ \hline
769 & 135 & 2,88592634 & 0,175552666 \\ \hline
770 & 136 & 2,886490725 & 0,176623377 \\ \hline
771 & 136 & 2,887054378 & 0,176394293 \\ \hline
772 & 136 & 2,8876173 & 0,176165803 \\ \hline
773 & 136 & 2,888179494 & 0,175937904 \\ \hline
774 & 137 & 2,888740961 & 0,177002584 \\ \hline
775 & 137 & 2,889301703 & 0,176774194 \\ \hline
776 & 137 & 2,889861721 & 0,176546392 \\ \hline
777 & 137 & 2,890421019 & 0,176319176 \\ \hline
778 & 137 & 2,890979597 & 0,176092545 \\ \hline
779 & 137 & 2,891537458 & 0,175866496 \\ \hline
780 & 137 & 2,892094603 & 0,175641026 \\ \hline
781 & 137 & 2,892651034 & 0,175416133 \\ \hline
782 & 137 & 2,893206753 & 0,175191816 \\ \hline
783 & 137 & 2,893761762 & 0,174968072 \\ \hline
784 & 137 & 2,894316063 & 0,174744898 \\ \hline
785 & 137 & 2,894869657 & 0,174522293 \\ \hline
786 & 137 & 2,895422546 & 0,174300254 \\ \hline
787 & 137 & 2,895974732 & 0,17407878 \\ \hline
788 & 138 & 2,896526217 & 0,175126904 \\ \hline
789 & 138 & 2,897077003 & 0,174904943 \\ \hline
790 & 138 & 2,897627091 & 0,174683544 \\ \hline
791 & 138 & 2,898176483 & 0,174462705 \\ \hline
792 & 138 & 2,898725182 & 0,174242424 \\ \hline
793 & 138 & 2,899273187 & 0,174022699 \\ \hline
794 & 138 & 2,899820502 & 0,173803526 \\ \hline
795 & 138 & 2,900367129 & 0,173584906 \\ \hline
796 & 138 & 2,900913068 & 0,173366834 \\ \hline
797 & 138 & 2,901458321 & 0,17314931 \\ \hline
798 & 139 & 2,902002891 & 0,174185464 \\ \hline
799 & 139 & 2,902546779 & 0,173967459 \\ \hline
800 & 139 & 2,903089987 & 0,17375 \\ \hline
\end{longtable}
\end{minipage}

\begin{minipage}[!h]{0.45\textwidth}\centering
\tiny
\begin{longtable}{cccc} \hline
\(n\) & \(\pi(n)\) & \(\log(n)\) & \(\pi(n)/n\) \\ \hline
801 & 139 & 2,903632516 & 0,173533084 \\ \hline
802 & 139 & 2,904174368 & 0,173316708 \\ \hline
803 & 139 & 2,904715545 & 0,173100872 \\ \hline
804 & 139 & 2,905256049 & 0,172885572 \\ \hline
805 & 139 & 2,90579588 & 0,172670807 \\ \hline
806 & 139 & 2,906335042 & 0,172456576 \\ \hline
807 & 139 & 2,906873535 & 0,172242875 \\ \hline
808 & 139 & 2,907411361 & 0,172029703 \\ \hline
809 & 139 & 2,907948522 & 0,171817058 \\ \hline
810 & 140 & 2,908485019 & 0,172839506 \\ \hline
811 & 140 & 2,909020854 & 0,172626387 \\ \hline
812 & 141 & 2,909556029 & 0,17364532 \\ \hline
813 & 141 & 2,910090546 & 0,173431734 \\ \hline
814 & 141 & 2,910624405 & 0,173218673 \\ \hline
815 & 141 & 2,911157609 & 0,173006135 \\ \hline
816 & 141 & 2,911690159 & 0,172794118 \\ \hline
817 & 141 & 2,912222057 & 0,172582619 \\ \hline
818 & 141 & 2,912753304 & 0,172371638 \\ \hline
819 & 141 & 2,913283902 & 0,172161172 \\ \hline
820 & 141 & 2,913813852 & 0,17195122 \\ \hline
821 & 141 & 2,914343157 & 0,171741778 \\ \hline
822 & 142 & 2,914871818 & 0,172749392 \\ \hline
823 & 142 & 2,915399835 & 0,17253949 \\ \hline
824 & 143 & 2,915927212 & 0,173543689 \\ \hline
825 & 143 & 2,916453949 & 0,173333333 \\ \hline
826 & 143 & 2,916980047 & 0,173123487 \\ \hline
827 & 143 & 2,91750551 & 0,172914148 \\ \hline
828 & 144 & 2,918030337 & 0,173913043 \\ \hline
829 & 144 & 2,918554531 & 0,173703257 \\ \hline
830 & 145 & 2,919078092 & 0,174698795 \\ \hline
831 & 145 & 2,919601024 & 0,174488568 \\ \hline
832 & 145 & 2,920123326 & 0,174278846 \\ \hline
833 & 145 & 2,920645001 & 0,174069628 \\ \hline
834 & 145 & 2,921166051 & 0,173860911 \\ \hline
835 & 145 & 2,921686475 & 0,173652695 \\ \hline
836 & 145 & 2,922206277 & 0,173444976 \\ \hline
837 & 145 & 2,922725458 & 0,173237754 \\ \hline
838 & 145 & 2,923244019 & 0,173031026 \\ \hline
839 & 145 & 2,923761961 & 0,172824791 \\ \hline
840 & 146 & 2,924279286 & 0,173809524 \\ \hline
841 & 146 & 2,924795996 & 0,173602854 \\ \hline
842 & 146 & 2,925312091 & 0,173396675 \\ \hline
843 & 146 & 2,925827575 & 0,173190985 \\ \hline
844 & 146 & 2,926342447 & 0,172985782 \\ \hline
845 & 146 & 2,926856709 & 0,172781065 \\ \hline
846 & 146 & 2,927370363 & 0,172576832 \\ \hline
847 & 146 & 2,92788341 & 0,172373081 \\ \hline
848 & 146 & 2,928395852 & 0,172169811 \\ \hline
849 & 146 & 2,92890769 & 0,17196702 \\ \hline
850 & 146 & 2,929418926 & 0,171764706 \\ \hline
\end{longtable}
\end{minipage}
\begin{minipage}[!h]{0.45\textwidth}\centering
\tiny
\begin{longtable}{cccc} \hline
\(n\) & \(\pi(n)\) & \(\log(n)\) & \(\pi(n)/n\) \\ \hline
851 & 146 & 2,92992956 & 0,171562867 \\ \hline
852 & 146 & 2,930439595 & 0,171361502 \\ \hline
853 & 146 & 2,930949031 & 0,17116061 \\ \hline
854 & 147 & 2,931457871 & 0,172131148 \\ \hline
855 & 147 & 2,931966115 & 0,171929825 \\ \hline
856 & 147 & 2,932473765 & 0,171728972 \\ \hline
857 & 147 & 2,932980822 & 0,171528588 \\ \hline
858 & 148 & 2,933487288 & 0,172494172 \\ \hline
859 & 148 & 2,933993164 & 0,172293364 \\ \hline
860 & 149 & 2,934498451 & 0,173255814 \\ \hline
861 & 149 & 2,935003151 & 0,173054588 \\ \hline
862 & 149 & 2,935507266 & 0,172853828 \\ \hline
863 & 149 & 2,936010796 & 0,172653534 \\ \hline
864 & 150 & 2,936513742 & 0,173611111 \\ \hline
865 & 150 & 2,937016107 & 0,173410405 \\ \hline
866 & 150 & 2,937517892 & 0,173210162 \\ \hline
867 & 150 & 2,938019097 & 0,173010381 \\ \hline
868 & 150 & 2,938519725 & 0,17281106 \\ \hline
869 & 150 & 2,939019776 & 0,172612198 \\ \hline
870 & 150 & 2,939519253 & 0,172413793 \\ \hline
871 & 150 & 2,940018155 & 0,172215844 \\ \hline
872 & 150 & 2,940516485 & 0,172018349 \\ \hline
873 & 150 & 2,941014244 & 0,171821306 \\ \hline
874 & 150 & 2,941511433 & 0,171624714 \\ \hline
875 & 150 & 2,942008053 & 0,171428571 \\ \hline
876 & 150 & 2,942504106 & 0,171232877 \\ \hline
877 & 150 & 2,942999593 & 0,171037628 \\ \hline
878 & 151 & 2,943494516 & 0,171981777 \\ \hline
879 & 151 & 2,943988875 & 0,171786121 \\ \hline
880 & 151 & 2,944482672 & 0,171590909 \\ \hline
881 & 151 & 2,944975908 & 0,171396141 \\ \hline
882 & 152 & 2,945468585 & 0,172335601 \\ \hline
883 & 152 & 2,945960704 & 0,17214043 \\ \hline
884 & 153 & 2,946452265 & 0,173076923 \\ \hline
885 & 153 & 2,946943271 & 0,172881356 \\ \hline
886 & 153 & 2,947433722 & 0,17268623 \\ \hline
887 & 153 & 2,94792362 & 0,172491545 \\ \hline
888 & 154 & 2,948412966 & 0,173423423 \\ \hline
889 & 154 & 2,948901761 & 0,173228346 \\ \hline
890 & 154 & 2,949390007 & 0,173033708 \\ \hline
891 & 154 & 2,949877704 & 0,172839506 \\ \hline
892 & 154 & 2,950364854 & 0,17264574 \\ \hline
893 & 154 & 2,950851459 & 0,172452408 \\ \hline
894 & 154 & 2,951337519 & 0,172259508 \\ \hline
895 & 154 & 2,951823035 & 0,172067039 \\ \hline
896 & 154 & 2,95230801 & 0,171875 \\ \hline
897 & 154 & 2,952792443 & 0,171683389 \\ \hline
898 & 154 & 2,953276337 & 0,171492205 \\ \hline
899 & 154 & 2,953759692 & 0,171301446 \\ \hline
900 & 154 & 2,954242509 & 0,171111111 \\ \hline
\end{longtable}
\end{minipage}

\begin{minipage}[!h]{0.45\textwidth}\centering
\tiny
\begin{longtable}{cccc} \hline
\(n\) & \(\pi(n)\) & \(\log(n)\) & \(\pi(n)/n\) \\ \hline
901 & 154 & 2,954724791 & 0,170921199 \\ \hline
902 & 154 & 2,955206538 & 0,170731707 \\ \hline
903 & 154 & 2,95568775 & 0,170542636 \\ \hline
904 & 154 & 2,95616843 & 0,170353982 \\ \hline
905 & 154 & 2,956648579 & 0,170165746 \\ \hline
906 & 154 & 2,957128198 & 0,169977925 \\ \hline
907 & 154 & 2,957607287 & 0,169790518 \\ \hline
908 & 155 & 2,958085849 & 0,170704846 \\ \hline
909 & 155 & 2,958563883 & 0,170517052 \\ \hline
910 & 155 & 2,959041392 & 0,17032967 \\ \hline
911 & 155 & 2,959518377 & 0,1701427 \\ \hline
912 & 156 & 2,959994838 & 0,171052632 \\ \hline
913 & 156 & 2,960470778 & 0,170865279 \\ \hline
914 & 156 & 2,960946196 & 0,170678337 \\ \hline
915 & 156 & 2,961421094 & 0,170491803 \\ \hline
916 & 156 & 2,961895474 & 0,170305677 \\ \hline
917 & 156 & 2,962369336 & 0,170119956 \\ \hline
918 & 156 & 2,962842681 & 0,169934641 \\ \hline
919 & 156 & 2,963315511 & 0,169749728 \\ \hline
920 & 157 & 2,963787827 & 0,170652174 \\ \hline
921 & 157 & 2,96425963 & 0,170466884 \\ \hline
922 & 157 & 2,964730921 & 0,170281996 \\ \hline
923 & 157 & 2,965201701 & 0,170097508 \\ \hline
924 & 157 & 2,965671971 & 0,16991342 \\ \hline
925 & 157 & 2,966141733 & 0,16972973 \\ \hline
926 & 157 & 2,966610987 & 0,169546436 \\ \hline
927 & 157 & 2,967079734 & 0,169363538 \\ \hline
928 & 157 & 2,967547976 & 0,169181034 \\ \hline
929 & 157 & 2,968015714 & 0,168998924 \\ \hline
930 & 158 & 2,968482949 & 0,169892473 \\ \hline
931 & 158 & 2,968949681 & 0,169709989 \\ \hline
932 & 158 & 2,969415912 & 0,169527897 \\ \hline
933 & 158 & 2,969881644 & 0,169346195 \\ \hline
934 & 158 & 2,970346876 & 0,169164882 \\ \hline
935 & 158 & 2,970811611 & 0,168983957 \\ \hline
936 & 158 & 2,971275849 & 0,168803419 \\ \hline
937 & 158 & 2,971739591 & 0,168623266 \\ \hline
938 & 159 & 2,972202838 & 0,169509595 \\ \hline
939 & 159 & 2,972665592 & 0,169329073 \\ \hline
940 & 159 & 2,973127854 & 0,169148936 \\ \hline
941 & 159 & 2,973589623 & 0,168969182 \\ \hline
942 & 160 & 2,974050903 & 0,16985138 \\ \hline
943 & 160 & 2,974511693 & 0,169671262 \\ \hline
944 & 160 & 2,974971994 & 0,169491525 \\ \hline
945 & 160 & 2,975431809 & 0,169312169 \\ \hline
946 & 160 & 2,975891136 & 0,169133192 \\ \hline
947 & 160 & 2,976349979 & 0,168954593 \\ \hline
948 & 161 & 2,976808337 & 0,169831224 \\ \hline
949 & 161 & 2,977266212 & 0,169652266 \\ \hline
950 & 161 & 2,977723605 & 0,169473684 \\ \hline
\end{longtable}
\end{minipage}
\begin{minipage}[!h]{0.45\textwidth}\centering
\tiny
\begin{longtable}{cccc} \hline
\(n\) & \(\pi(n)\) & \(\log(n)\) & \(\pi(n)/n\) \\ \hline
951 & 161 & 2,978180517 & 0,169295478 \\ \hline
952 & 161 & 2,978636948 & 0,169117647 \\ \hline
953 & 161 & 2,979092901 & 0,168940189 \\ \hline
954 & 162 & 2,979548375 & 0,169811321 \\ \hline
955 & 162 & 2,980003372 & 0,169633508 \\ \hline
956 & 162 & 2,980457892 & 0,169456067 \\ \hline
957 & 162 & 2,980911938 & 0,169278997 \\ \hline
958 & 162 & 2,981365509 & 0,169102296 \\ \hline
959 & 162 & 2,981818607 & 0,168925965 \\ \hline
960 & 162 & 2,982271233 & 0,16875 \\ \hline
961 & 162 & 2,982723388 & 0,168574402 \\ \hline
962 & 162 & 2,983175072 & 0,168399168 \\ \hline
963 & 162 & 2,983626287 & 0,168224299 \\ \hline
964 & 162 & 2,984077034 & 0,168049793 \\ \hline
965 & 162 & 2,984527313 & 0,167875648 \\ \hline
966 & 162 & 2,984977126 & 0,167701863 \\ \hline
967 & 162 & 2,985426474 & 0,167528438 \\ \hline
968 & 163 & 2,985875357 & 0,16838843 \\ \hline
969 & 163 & 2,986323777 & 0,168214654 \\ \hline
970 & 163 & 2,986771734 & 0,168041237 \\ \hline
971 & 163 & 2,98721923 & 0,167868177 \\ \hline
972 & 164 & 2,987666265 & 0,16872428 \\ \hline
973 & 164 & 2,98811284 & 0,168550874 \\ \hline
974 & 164 & 2,988558957 & 0,168377823 \\ \hline
975 & 164 & 2,989004616 & 0,168205128 \\ \hline
976 & 164 & 2,989449818 & 0,168032787 \\ \hline
977 & 164 & 2,989894564 & 0,167860798 \\ \hline
978 & 165 & 2,990338855 & 0,168711656 \\ \hline
979 & 165 & 2,990782692 & 0,168539326 \\ \hline
980 & 165 & 2,991226076 & 0,168367347 \\ \hline
981 & 165 & 2,991669007 & 0,168195719 \\ \hline
982 & 165 & 2,992111488 & 0,16802444 \\ \hline
983 & 165 & 2,992553518 & 0,16785351 \\ \hline
984 & 166 & 2,992995098 & 0,168699187 \\ \hline
985 & 166 & 2,99343623 & 0,168527919 \\ \hline
986 & 166 & 2,993876915 & 0,168356998 \\ \hline
987 & 166 & 2,994317153 & 0,168186424 \\ \hline
988 & 166 & 2,994756945 & 0,168016194 \\ \hline
989 & 166 & 2,995196292 & 0,167846309 \\ \hline
990 & 166 & 2,995635195 & 0,167676768 \\ \hline
991 & 166 & 2,996073654 & 0,167507568 \\ \hline
992 & 167 & 2,996511672 & 0,168346774 \\ \hline
993 & 167 & 2,996949248 & 0,168177241 \\ \hline
994 & 167 & 2,997386384 & 0,168008048 \\ \hline
995 & 167 & 2,997823081 & 0,167839196 \\ \hline
996 & 167 & 2,998259338 & 0,167670683 \\ \hline
997 & 167 & 2,998695158 & 0,167502508 \\ \hline
998 & 168 & 2,999130541 & 0,168336673 \\ \hline
999 & 168 & 2,999565488 & 0,168168168 \\ \hline
1000 & 168 & 3 & 0,168 \\ \hline
\end{longtable}
\end{minipage}

\fonte{https://pt.wikipedia.org/}


\textbf{Observação}: (a) Os valores na segunda coluna (valores da função \(\pi(n)\)) foram obtidos mediante ao uso de uma fórmula do Excel (a saber CONT.SES()). (b) À medida que os valores de \(n\) crescem, os valores de \(\rho(n)\) tendem a zero (deforma lenta)



\subsection*{Gráfico de dispersão da função \(\rho(n) = \dfrac{\pi(n)}{n}\) com valores de \(n\) logaritmizados}

\readdata{\data}{testZ.dat}
\begin{center}
\captionof{figure}{Gráfico de dispersão de \(\rho(n)\)}
\psset{xunit=3.6cm,yunit=10cm}% wirkt auch auf pspicture
\begin{pspicture}(-0.1,-0.1)(3.2,0.7)
\psaxes[
%axesstyle=frame,
%xlogBase=10,
%logLines=x, 
xticksize=0 0.5,
yticksize=0 3.2,
Dx=0.5,
Dy=0.1,
%subticksize=3.5,
%%%subticksize relativ
tickwidth=0.05pt,
%subtickwidth=0.25pt,
%subticks=5,
tickcolor=black!20
]{->}(0,0)(-0.099,-0.099)(3.2,0.6)
%\pstScalePoints(1,1){log}{}
\listplot[plotstyle=dots,dotscale=0.75,linecolor=red]{\data}
\end{pspicture}
\fonte{Elaborado pelo autor}
\end{center}

\subsection*{13B}


A lei de Weber-Fechner tenta descrever a relação existente entre a magnitude física de um estímulo e a intensidade do estímulo que é percebida. Pode ser enunciada como: ``a resposta a qualquer estímulo é proporcional ao logaritmo da intensidade do estímulo''. Esta lei aplica-se aos 5 sentidos, mas as suas implicações são mais bem entendidas quando se refere aos estímulos provocados pela luz e pelo som. É decorrente do fenômeno assim descrito, que as medidas de percepção da intensidade sonora pelo ouvido humano, e luminosa pelos órgãos de visão, são feitas por grandezas logarítmicas. É o caso do Decibel (dB) definido como 10 vezes o logaritmo decimal da intensidade sonora. A mesma grandeza logarítmica descreve também a intensidade luminosa percepcionada, sendo genericamente usada em óptica e engenharia. (WIKIPEDIA)

Ernst Heinrich Weber (1795–1878) foi um dos primeiros a fazer uma aproximação ao estudo da resposta do ser humano a um estímulo físico de uma maneira quantitativa. Gustav Theodor Fechner (1801–1887) mais tarde elaborou uma interpretação teórica elaborada sobre as descobertas de Weber. (Encyclopædia Britannica \textit{Online})


Com base nas afirmações anteriores, no que Gauss já havia constatado em suas observações feitas na tabela por ele construída (onde relacionava os valores de \(n = \mbox{ ``posição de um número na sequência crescente de primos''}\) e a densidade média \(\rho(n) = \dfrac{\pi(n)}{n}\), onde:
\[\pi(n) = \mbox{ ``quantidade de números primos existentes no intervalo } [0, n]''\]
e, posteriormente, substituindo os valores de \(n\) por \(\log(n)\)), e verificando-se, tabuladamente (tabela construída no Excel), constatamos que:
\[\displaystyle\lim_{n \to \infty} \dfrac{\rho(n)}{\dfrac{1}{\log(n)}} = \lim_{n \to \infty} \dfrac{\pi(n)}{n} \log(n) = 1.\]


}


