% Questão 01 ------------------------------------------------------------------
\question
A Psicologia da Matematização: Ockham~(sec.13) \& Kanizsa~(sec.20), 
Galileo~(sec.17) \& Newton~(sec.17-18).
\begin{parts}
  \part Descreva o \textit{Efeito de Completamento (Interpolação) Visual} 
   (\textit{Efeito Kanizsa}) em poucas linhas e exemplifique-o com o famoso 
   triângulo de Kanizsa e especialmente com a visualização de formas sugeridas 
   por uma sequência de pontos.
  \part Argumente com base no \textit{Efeito Kanizsa} sobre a motivação cognitiva 
   da representação contínua para dinâmicas de grandes populações. 
   Como se explica evolutivamente a preferência cognitiva da espécie humana por 
   registrar \textit{informações discretas} em termos (reduzidos) como 
   \textit{formas geométricas}?
  \part Descreva a Metodologia funcional de Galileo e justifique-a em termos do 
   que foi discutido em (a)-(b).
  \part Descreva o grande aperfeiçoamento da Metodologia de Galileo realizada 
   por Newton. 
   {
    \scriptsize 
    (
     \textbf{Sugestão}: Biblioteca de funções
    )
   }
  \part Descreva o \textit{Principio de Parcimônia de Ockham} e discuta a sua 
   conexão com a cognição humana, especialmente com o item (b).
  \part Exemplifique os itens (b)-(c) com dados de mortalidade da COVID19 em 
   2020 para uma grande comunidade durante aproximadamente 1 ano.
 \end{parts}
%------------------------------------------------------------------------------