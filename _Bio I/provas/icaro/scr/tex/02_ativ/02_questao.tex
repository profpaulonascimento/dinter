% Questão 02 ------------------------------------------------------------------
\question
Escala Logarítmica na Aproximação Assintótica: \textit{Princípio Sensorial 
(``Lei") de Weber-Fechner~(sec.19)}.
\begin{parts}
  \part Descreva o \textit{Princípio Sensorial (``Lei") de Weber-Fechner} para a 
   percepção visual, auditiva, táctil, olfativa e de cardinalidade.
  \part Argumente com base no \textit{Princípio de Weber-Fechner} sobre a 
   conveniência cognitiva da escala logarítmica para variáveis com ``grandes" 
   valores.
  \part Aplique a escala logarítmica para o registro numérico da população do 
   exemplo citado no item 1~(f) acima e caracterize os períodos de tempo em que 
   o comportamento é linear (Malthusiano).
  \part Mostre que, para duas sequencias de números positivos, 
   $\{\, a_k \to \infty \,\}$ e $\{\, b_k \to \infty \,\}$, então valem as 
   seguintes implicações para a aproximação assintótica em escala logarítmica:
   \[
    \log{a_k} - \log{b_k} \to 0 
    \Leftrightarrow 
    \log{\frac{a_k}{b_k}} \to 0
    \Leftrightarrow 
    \frac{a_k}{b_k} \to 1
   \]
  \part Mostre que a aproximação assintótica na escala logarítmica não implica 
   necessariamente na aproximação assintótica em escala normal (isto é,
   $ a_k - b_k \to 0 $) , mas vale a implicação inversa. 
   {
    \scriptsize 
    (
    \textbf{Sugestão:} 
    Analise a igualdade $ a_k - b_k = a_k\left(1 - \frac{a_k}{b_k}\right)$ e 
    observe que $ a_k - b_k \to 0 \Leftrightarrow \frac{a_k}{b_k} 
    \text{ se aproxima de } 1 \text{ com erro de } o\left(\frac{1}{a_k}\right)$, 
    isto é, de ``ordem menor do que $\frac{1}{a_k}$''.
    Assim, para sequencias que convergem para $ \infty $ é mais interessante 
    analisar a aproximação assintótica logarítmica, pois ela é mais abrangente e 
    tem um fundo cognitivo. 
    Além disso, para dois ``trens em alta velocidade uma aproximação na escala 
    simples é extremamente perigosa''!
   )
  }
\end{parts}
%------------------------------------------------------------------------------

