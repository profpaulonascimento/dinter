% Questão 07 ------------------------------------------------------------------
\question
\begin{parts}
  \part \textbf{Utilizando o Método Operacional} explicado no texto, obtenha uma 
   expressão explícita (em termos de integrais) da solução da Equação de 
   (Euler-Malthus) Verhulst 
   \[
    \frac{1}{N}\frac{\dd N}{\dd t} = r(t) -\lambda{(t)}N,
   \]
   onde $ r(t) $ e $ \lambda(t) $ são funções reais positivas. 
   (
    \textbf{Sugestão:} 
    Utilize a transformação linearizadora $ m = \frac{1}{N} $, seguida pelo 
    Método Operacional
   ).
  \part Apresente um cenário biológico que indique a utilização desta equação 
   como Modelo Matemático para uma Dinâmica Populacional.
  \part Considere uma população cujo tamanho $ N(t) $ é regulado pelo chamado 
  Modelo de Euler-Verhulst
   \[
    \frac{1}{N}\frac{\dd N}{\dd t} = r -\lambda N
   \]
   (isto é, com taxa de natalidade Malthusiana (\textit{per capita}) $ r $ e 
   mortalidade (\textit{per capita}) $ \lambda N, r, \lambda > 0 $ constantes) 
   que se inicia com uma população ``colonizadora'' de $ N_0 = N(0) $ 
   indivíduos. 
   Considere a decrescente população $ n(t) $ dos indivíduos colonizadores 
   ($ n(0) = N_0 $) submetidos à taxa de mortalidade ambiente (os descendentes 
   de colonizadores não são colonizadores mas fazem parte da população 
   ambiente!). 
   Obtenha uma expressão para a dinâmica desta população $ n(t) $  de 
   colonizadores e mostre que o tempo médio de sobrevivência neste caso, 
   apresenta uma dependência do tamanho da população inicial $ N_0 $, indicando 
   um fenômeno interativo no processo de mortalidade.
\end{parts}
%------------------------------------------------------------------------------