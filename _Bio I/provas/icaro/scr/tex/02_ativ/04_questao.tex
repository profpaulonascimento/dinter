% Questão 04 ------------------------------------------------------------------
\question
Tempo Médio (Aritmético) de Sobrevivência.
\begin{parts}
  \part Defina Média Aritmética Ponderada $ M_A{(a_1, a_2, \ldots, a_N)} $ para 
   uma sequência de dados numéricos $ a_k > 0 $. 
   Discuta a razão de se dizer que uma Média Aritmética $ A $ é 
   \textbf{uma única} informação numérica \textbf{populacional} que substitui 
   (reduzindo) um conjunto (Tabela) de \textbf{várias} informações numéricas 
   \textbf{individuais}, $ a_k $. 
   Argumente com base nesta distinção sobre a (usual) insensatez de se afirmar 
   que um \textbf{casal} brasileiro tem em média, por exemplo, $1.44$ filhos.
  \part Segundo um Teorema de Kolmogorov-Nagumo~(1933) todas as ``Médias'' sobre 
   uma sequência de dados numéricos $ a_K > 0 $ (conceito que pode ser 
   facilmente definido por algumas poucas propriedades bem caracteristicas) são 
   da forma 
   $ 
    M_\phi \left(a_1, \ldots a_N \right) = 
    \phi^{-1}\left(M_A(\phi(a_1)), \ldots, \phi(a_N)\right) 
   $, 
   onde $ \phi $  é uma função real estritamente convexa inversível e $ M_A $ é 
   uma Média Aritmética. 
   Mostre a veracidade desta afirmação com respeito às médias: 
   \textit{Aritmética}, \textit{Harmônica}, \textit{Geométrica} e 
   \textit{Quadrática}
  \part Interprete o Método de Quadrados Mínimos de Gauss em termos de uma Média 
   Quadrática.
  \part Dada uma sequência de números positivos $ a = \{\,a_k\,\} $ obtenha, 
   argumentando geometricamente, uma relação de ordem entre suas Médias 
   Aritmética, $ M_A{(a)} $, Harmônica, $ M_h{(a)} $, Geométrica, $ M_g{(a)} $
   e Quadrática, $ M_2{(a)} $ 
   (
    Utilize uma sequencia de apenas dois números para seus argumentos
   ).
  \part Mostre que, a depender da escolha da média de Kolmogorov-Nagumo, pode-se 
   dizer que a média de filhos de um casal brasileiro pode ser qualquer número 
   real entre $ m = \min\{a_k\} $ e $ M = \max\{a_k\}$, onde 
   $ a_k = \text{``Número de casais com $k$ filhos''} $.
\end{parts}
%------------------------------------------------------------------------------