% Questão 05 ------------------------------------------------------------------
\question
Tempo Médio (Aritmético) de Sobrevivência de uma População.
\paragraph{Definição.}
  Dado um Modelo populacional especificamente de mortalidade $ N(t) $ tal que 
  $ \frac{\dd N}{\dd t} < 0 $ e $ \lim_{t \to \infty} N(t) = 0 $, diz-se que o 
  valor (finito ou infinito) da integral
  \[
   \frac{1}{N_0} \int_{0}^{\infty} -t\frac{\dd N}{\dd t}\, \dd{t}
  \]
  é denominado Tempo Médio (Aritmético) de Sobrevivência da População.
  
  \begin{parts}
   \part \textbf{Argumente} sobre a motivação para que a expressão
    \[
     \frac{1}{N_0} \int_{0}^{\infty} -t\frac{\dd N}{\dd t}\, \dd{t} =
     \frac{1}{N_0} \int_{0}^{N_0} t\,\dd{N},
    \] 
    que se refere a uma dinâmica $ N(t) $ decrescente de uma (Grande) população 
    (sem natalidade e migração) inicialmente com $ N(0) = N_0 $ indivíduos, 
    possa \textbf{ser interpretada} como o tempo médio (aritmético) de 
    sobrevivência desta população.
   \part \textbf{Calcule} o Tempo Médio (Aritmético) de sobrevivência de uma 
    população Malthusiana (isto é, descrita segundo o Modelo Newtoniano
    $ \frac{1}{N} \frac{\dd N}{\dd t} = -\mu $, $ N(0) = N_0 $ e mostre que este 
    valor \textbf{independe} de $ N_0 $. 
    \textbf{Discuta} o significado biológico deste resultado.
   \part Calcule o tempo médio (aritmético) de sobrevivência de uma população 
    cuja dinâmica de Mortalidade é descrita por uma função quase-polinomial
    $ \displaystyle N(t) = q(t)e^{-\mu t} $, onde 
    $ \displaystyle q(t) = N_0 + \sum_{k = 1}^{m} a_k t^{k}$ é um polinômio e 
    $ \mu > 0 $.
    {
     \scriptsize
     (
     \textbf{Sugestão:}
     Calcule explicitamente as integrais 
     $ \displaystyle I(n) = \int_{0}^{\infty} t^{n}e^{-\mu t}\,\dd{t} $ 
     recursivamente em $ n $ e utilizando integrações por partes
     )
    }
   \part O mesmo para $ \displaystyle N(t) = \frac{N_0}{t + 1} $.
  \end{parts}
%------------------------------------------------------------------------------