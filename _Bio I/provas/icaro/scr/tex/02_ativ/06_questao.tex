% Questão 06 ------------------------------------------------------------------
\question
Mortalidade por Predação Periférica e Efeito de Rebanho Egoista.
 
\begin{quote}
Dois ``amigos'' em um campo de cerrado e uma onça esfomeada. 
Um deles, para e toma seu tempo para amarrar bem o calçado. 
O outro, apressado, lhe repreende: ``Vamos correr logo que a onça é mais rápida 
do que nós!''. 
O Amigo (da onça): ``Eu não preciso correr mais do que a onça, eu preciso correr 
mais do que você!''.
Ditado caboclo: ``Mingau quente, se come pelas beiradas''.
\end{quote}

Considere uma população distribuída uniformemente em uma região delimitada no
plano descrita por uma função diferenciável $ N(t) $ cuja mortalidade é causada 
unicamente por uma predação ``periférica'' da forma $ p(N) = -\mu \sqrt{N} $, 
caracterizada matematicamente segundo a Metodologia Newtoniana pela equação 
diferencial: $ \frac{\dd N}{\dd t} = -\mu\sqrt{N}$. 
(
  A justificativa da função de mortalidade na forma $ p(N) = -\mu \sqrt{N} $ 
  para predação ``periférica'' se deve ao fato de que um grupo uniformemente 
  distribuído em uma região delimitada do plano é predado apenas na fronteira, 
  cuja extensão tem medida da ordem da dimensão linear da região, enquanto que a 
  área, que é proporcional à população, é da ordem do quadrado da medida linear 
  e, portanto, a fronteira é da ordem de $ N^{\frac{1}{2}} $ . 
  O formato da região pode ser considerado aproximadamente um disco~(2D) ou uma 
  esfera~(3D) porque estas são as formas que apresentam menor extensão de 
  fronteira para um mesmo conteúdo populacional --- Por exemplo, sapos na beira 
  da lagoa diante da ameaça de cobras, ou rebanho de ovelhas diante de lobos.
)

\paragraph{Definição.}
 Diz-se que uma Dinâmica de mortalidade apresenta o 
 \textit{Efeito de Rebanho Egoista}
 \footnote
 {
  Termo introduzido por W.~Hamilton no antológico artigo: 
  \textit{The Selfish Herd}, J.Theor.Biol 1970
 }
 quando a mortalidade especifica (\textit{per capita} 
 $ \frac{1}{N}\frac{\dd{N}}{\dd t} = f(N) $) \textbf{diminui} com o aumento do 
 tamanho do grupo, em outros termos, um individuo se sente particularmente mais
 ``protegido'' em um grupo maior; por isso ele se junta aos vencedores\ldots
 
 \begin{parts}
  \part Argumente como o conceito de ``Efeito Rebanho Egoista'' pode ser 
   interpretado em termos do Tempo Médio de Sobrevivencia.
  \part Mostre que não há ``Efeito Rebanho Egoista'' em uma população cuja 
   mortalidade é unicamente Malthusiana.
  \part Descreva uma Dinâmica Adimensional de mortalidade por predação 
   periférica para um grupo populacional que ocupa uma região delimitada do 
   espaço físico \textbf{tridimensional} (Por exemplo, um cardume de Sardinhas e 
   Baleias) e verifique se esta dinâmica apresenta um ``Efeito Rebanho Egoista'' 
   e é dizimada em tempo finito.
  \part Considere uma população com predação per capita tipo 
   \textit{Holling II:} $ p(N) = \frac{A}{B + N} $. 
   Adimensionalize a equação e verifique se ocorre um ``Efeito Rebanho'' nesta 
   dinâmica.
  \part Discuta o comportamento individual das presas em termos de uma proteção 
   por agrupamento com base na percepção de cardinalidade segundo a ``Lei de 
   Weber-Fechner''.
 \end{parts}
%------------------------------------------------------------------------------