% Questão 03 ------------------------------------------------------------------
\question
\begin{parts}
 \part Determinar as unidades compostas de Pressão (P), Energia (E) e Potência
	 (W) a partir:
		 \begin{subparts}
		  \subpart Do conjunto (genérico) de unidades básicas $\{M, L, T\}$
				% Solução (i) =============================================================
				\begin{solution}
				Para estabelecermos as corretas relações, precisamos das definições de
				\textit{Pressão}, \textit{Energia} e \textit{Potência}.
				\begin{itemize}
					\item \textbf{Pressão}. 
					 É a grandeza física que relaciona a Força ($F$) 
					 pela área de uma superfície. 
					 Se definirmos $F = ma$, onde $m$ é a massa e $a$ a aceleração; e, $A$ a 
						área	de uma superfície, temos: $P = \frac{F}{A}$.
					 Assim,
					 \[
					  [P] = \left[\frac{F}{A}\right]
						 = \frac{[F]}{[A]} 
						 = \frac{MLT^{-2}}{L^2}
						 = ML^{-1}T^{-2}
							\qquad \therefore \qquad \ovalbox{$[P] = ML^{-1}T^{-2} $}
					 \]
					\item \textbf{Energia}.
					 Se considerarmos a Energia Cinética, podemos fazer $  E = \frac{mv^2}{2}$.
						Como $[1/2] = 1$, temos:
						\[
						 [E] = M (LT^{-1})^2 
							= M L^2 T^{-2}
						\qquad\therefore\qquad \ovalbox{$[E] = ML^{2}T^{-2} $}
						\]
					\item \textbf{Potência}.
					 Se a Potência for a rapidez da Energia em certo intervalo de tempo, ou 
						seja, $ W = \frac{E}{t}$, temos:
						\[
						 [W] = \left[\frac{E}{t}\right]
							= \frac{[E]}{[t]}
							= \frac{M L^2 T^{-2}}{T}
							= M L^2 T^{-3}
							\qquad\therefore\qquad \ovalbox{$[W] = M L^2 T^{-3} $}
						\]
				\end{itemize}
				\end{solution}
				%==========================================================================
		  \subpart De um outro conjunto $\{M_1 = \delta M, L_1 = bL, T_1 = cT\}$.
				 % Solução (i) ============================================================
					\begin{solution}
				  Notemos que $ M = \frac{1}{\delta} M_1$, $L = \frac{1}{b} L_1$ e	
						$ T = \frac{1}{c} T_1$.
					 Portanto:
					 \begin{itemize}
						 \item \textbf{Pressão}.
						  \begin{align*}
							  [P] &= M L^{-1} T^{-2}\\
									&= \left(\frac{1}{\delta} M_1\right)\left(\frac{1}{b} L_1\right)^{-1} \left(\frac{1}{c} T_1\right)^{-2}\\
									&= \frac{1}{\delta} b c^2 M_1 L_1^{-1} T_1^{-2}
							 \end{align*}
								Logo, \ovalbox{$[P] = \frac{1}{\delta} b c^2 M_1 L_1^{-1} T_1^{-2}$}.
						 \item \textbf{Energia}.
						  \begin{align*}
							  [E] &= M L^2 T^{-2}\\
									&= \left(\frac{1}{\delta}M_1\right)\left(\frac{1}{b} L_1\right)^{2}\left(\frac{1}{c} T_1\right)^{-2}\\
									&= \frac{1}{\delta}\frac{1}{b^2} c^2 M_1 L_1^2 T_1^{-2}
							 \end{align*}
								Logo, \ovalbox{$ [E] = \frac{c^2}{\delta b^2} M_1 L_1^{2} T_1^{-2} $}.
						 \item \textbf{Potência}.
						  \begin{align*}
							  [W] &= M L^2 T^{-3}\\
									&= \left(\frac{1}{\delta}M_1\right)\left(\frac{1}{b} L_1\right)^{2}\left(\frac{1}{c} T_1\right)^{-3}\\
									&= \frac{1}{\delta}\frac{1}{b^2} c^3 M_1 L_1^2 T_1^{-3}
							 \end{align*}
								Logo, \ovalbox{$ [W] = \frac{1}{\delta}\frac{1}{b^2} c^3 M_1 L_1^2 T_1^{-3} $}.
					 \end{itemize}
					\end{solution}
				 %=========================================================================
			\end{subparts}
	\part Obtenha as unidades derivadas das unidades básicas para as seguintes
	 medidas: Área, Volume, \sout{Pressão}, Densidade de Massa, Trabalho, 
		\sout{Potência}.
		% Solução (b)================================================================
		\begin{solution}
		 Das unidades expostas, lembremos que a \textit{Densidade de Massa}, $\rho$,
			pode ser dada por $\rho = \frac{m}{V}$; bem como o \textit{Trabalho}, 
			$\tau$ é o	produto de uma força $F$ pelo deslocamento $d$, ou seja, 
			$\tau = Fd$.
			Assim, denotando \textit{Área} e \textit{Volume} por $A$ e $V$, 
			respectivamente, temos: $[A] =L^2$,	$[V] = L^3$,	$[\rho]= M L^{-3}$,
				$[\tau] =[F][d] = (M L T^{-2})(L) = M L^2 T^{-2}$.
		\end{solution}
		%============================================================================
\end{parts}
%------------------------------------------------------------------------------