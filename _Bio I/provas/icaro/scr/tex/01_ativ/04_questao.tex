% Questão 04 ------------------------------------------------------------------
\question
Supondo que a função matemática $\varphi$ é continuamente diferenciável, obtenha
uma segunda aproximação para o período 
\[
  T_0 \approx
		\left(\,
		 \varphi(0)\ + \varphi^{\prime}(0)\,\frac{A}{l}
		\right)
		\sqrt{\frac{l}{g}},
\]
calculando $ \varphi^{\prime}(0) $ usando uma expansão no parâmetro 
$ \frac{A}{l} = \varepsilon_1 $ após adimensionar adequadamente o modelo 
diferencial da dinâmica do pêndulo (com suas condições inicais):
\[
  \begin{cases}
		 m\dfrac{\dd^{2}(l\theta)}{\dd t^2} = -mg\sen{\theta}\qquad \text{(segunda lei de Newton tangencial)}\\
			\dfrac{\dd \theta (0)}{\dd t} = 0\\
			l\theta(0) = A.
  \end{cases}
\]

%------------------------------------------------------------------------------

\begin{solution}
\paragraph{\small Adimensionalização.}

Para adimensionalizar corretamente a EDO em questão, vamos escolher as unidades
intrínsecas de massa $m = M$, de comprimento $\ell = L$ e de tempo 
$\sqrt{\frac{\ell}{g}} = T$.
Daí, fazendo $ \tau = \frac{t}{T} $, temos:
\begin{align*}
 m\dfrac{\dd^{2}(l\theta)}{\dd t^2} = -mg\sen{\theta} 
	&\Longrightarrow
	\frac{\ell}{T^2}\cdot \frac{\dd^2{\theta}}{\dd{\left(\frac{t}{T}\right)^2}} = -g\sen{\theta}\\
	&\Longrightarrow
	\frac{\dd^2{\theta}}{\dd{\tau^2}} = \frac{-g}{\ell}\cdot T^2\sen{\theta}\\
	&\Longrightarrow
	\frac{\dd^2{\theta}}{\dd{\tau^2}} = \frac{-g}{\ell}\cdot \left(\sqrt{\frac{\ell}{g}}\right)^2\sen{\theta}\\
	&\Longrightarrow
	\frac{\dd^2{\theta}}{\dd{\tau^2}} = -\sen{\theta}
\end{align*}

Além disso, 
\[
 \frac{\dd\theta(0)}{\dd{t}} = 0 
	\Longrightarrow \frac{1}{T}\frac{\dd\theta(0)}{\dd{\left(\frac{t}{T}\right)}} = 0
	\Longrightarrow \frac{\dd\theta(0)}{\dd{\tau}} = 0,
\]
bem como $\theta (0) =\frac{A}{\ell} = \varepsilon_1$.
Portanto, temos o seguinte sistema adimensionalizado:
\begin{equation}
 \begin{cases}
	 \dfrac{\dd^2{\theta}}{\dd{\tau^2}} = -\sen{\theta}\\
		\theta (0) = \varepsilon_1\\
		\dfrac{\dd\theta(0)}{\dd{\tau}} = 0.
	\end{cases}
\label{eq:edo-adim}
\end{equation}
%
\paragraph{\small Resolvendo a EDO.} %-----------------------------------------
%
Para resolvermos a EDO em questão, consideremos uma função 
$\sigma = \sigma(\theta)$, tal que 
\[
 \sigma = \frac{\dd\theta}{\dd\tau}.
\]
Assim,
\begin{enumerate}[(i)]
	\item $\sigma (0) = \dfrac{\dd\theta(0)}{\dd\tau}$ e, como 
	 $\frac{\dd\theta(0)}{\dd{\tau}} = 0$, temos $\sigma(0) =0$.
	\item Derivando $\sigma$ em relação a $\tau$, temos:
	 \begin{align*}
		 \frac{\dd\sigma}{\dd\tau} = \frac{\dd}{\dd\tau} \left(\frac{\dd{\theta}}{\dd\tau}\right) 
			&\Longrightarrow \frac{\dd\sigma}{\dd\theta}\cdot \textcolor{blue}{\frac{\dd\theta}{\dd\tau}} = \frac{\dd^2\theta}{\dd\tau^2}\\
			&\Longrightarrow \frac{\dd\sigma}{\dd\tau}\cdot \textcolor{blue}{\sigma} =  \frac{\dd^2\theta}{\dd\tau^2}.
		\end{align*}
\end{enumerate}
Substituindo (ii) em \eqref{eq:edo-adim}, temos:
\begin{align*}
 \sigma \cdot \frac{\dd\sigma}{\dd\theta} = - \sen\theta 
 &\Longrightarrow \sigma \dd{\sigma} = -\sen\theta\dd{\theta}\\
	&\Longrightarrow \int\sigma \dd{\sigma} = \int (-\sen\theta)\dd{\theta}\\
	&\Longrightarrow \frac{\sigma^2}{2} = \cos\theta + C_1\\
	&\Longrightarrow \sigma^2 = 2\cos\theta + C \tag{com $C=2C_1$}
\end{align*}
Agora, quando $\theta (0) = \varepsilon_1$, a velocidade é zero, ou seja, 
$\frac{\dd\theta(0)}{\dd{\tau}} = \sigma(0) =0$.
Logo,
\begin{align*}
 \left[\sigma(0)\right]^2 &= 2\cos(\varepsilon_1) + C\\
	0 &= 2\cos\varepsilon_1 + C\\
	C &= -2\cos\varepsilon_1
\end{align*}
Dessa forma,
\begin{align*}
 \sigma^2 &= 2\cos\theta - 2\cos\varepsilon_1\\
	\sigma   &= \sqrt{2\cos\theta - 2\cos\varepsilon_1}\\
	\frac{\dd\theta}{\dd\tau} &= \sqrt{2\cos\theta - 2\cos\varepsilon_1}\\
	\frac{1}{\sqrt{2\cos\theta - 2\cos\varepsilon_1}}\dd{\theta} &= \dd\tau\\
	\int_{0}^{\theta}\frac{1}{\sqrt{2\cos\theta - 2\cos\varepsilon_1}}\dd{\theta} &= \int_{0}^{\tau}\dd\tau\\
\end{align*}
Portanto, fazendo $ k = 2\cos{\varepsilon_1} $, temos:
\begin{equation}
 \tau = \int_{0}^{\theta}\frac{1}{\sqrt{2\cos\theta - k}}\dd{\theta}
\label{eq:eliptica-dim}
\end{equation}

\paragraph{Calculando o Período Adimensional.}
Se $\tau_0$ é o período no sistema adimensional, pela simetria do fenômeno,
quando $\theta = \frac{\pi}{2}$, temos $\tau = \frac{\tau_0}{4}$ (de fato, como
o período é o tempo necessário para que o sistema volte à posição inicial, ao
abandonar a massa de uma posição $\theta_\ast$, ela volta depois de percorrer
$\theta_\ast$ por quatro vezes: duas até a posição simétrica em relação à 
vertical e as outras duas voltando).
Assim:
\begin{align}
 \frac{\tau_0}{4} &= \int_{0}^{\pi/2}\frac{1}{\sqrt{2\cos\theta - k}}\dd{\theta}\notag\\
	\tau_0 &= 4 \int_{0}^{\pi/2}\frac{1}{\sqrt{2\cos\theta - k}}\dd{\theta}
\end{align}

\paragraph{Aproximando o Período Adimensional.}
Considere $\phi(\theta) = \dfrac{1}{\sqrt{2\cos\theta - k}} = \left(2\cos\theta - k\right)^{-1/2} $.
Então,
\begin{enumerate}
 \item[(iii)] $\displaystyle \phi (0) = (2\cos 0 - k)^{-1/2} = (2 - k)^{-1/2} = \frac{1}{\sqrt{2 - k}}$.
	\item[(iv)] $\displaystyle \phi^{\prime} (\theta) = -\dfrac{1}{2}\cdot \left(2\cos\theta - k\right)^{-3/2}\cdot \left(-2\sen{\theta}\right) = \dfrac{\sen\theta}{\sqrt{(2\cos\theta - k)^{3}}}$, o que 
	 resulta $\phi^{\prime}(0) = 0$.
\end{enumerate}
Assim, aproximando $\phi$ por Taylor, temos:
\begin{align*}
 \phi (\theta) &\approx \phi{(0)} + \phi^{\prime}(0)\theta\\
	&= \frac{1}{\sqrt{2 - k}} 
\end{align*}
Integrando em relação a $\theta$:
\begin{align*}
 \tau_0 &= 4 \int_{0}^{\pi/2}\frac{1}{\sqrt{2\cos\theta - k}}\dd{\theta}\\
	&\approx 4 \int_{0}^{\pi/2} \frac{1}{\sqrt{2 - k}}  \dd{\theta}\\
	&= 4\cdot \frac{1}{\sqrt{2 - k}} \cdot \frac{\pi}{2}\\
	&= \frac{2\pi}{\sqrt{2 - k}}
\end{align*}

\paragraph{Voltando à Dimensão Original.}
Como $\tau_0 = \frac{T_0}{T}$, podemos fazer:
\begin{align*}
 \frac{T_0}{T} &\approx \frac{2\pi}{\sqrt{2 - k}}\\
	T_0 & = T \cdot \frac{2\pi}{\sqrt{2 - k}}\\
	&= \sqrt{\frac{\ell}{g}}\frac{2\pi}{\sqrt{2 - k}}
\end{align*}

Assim,
\begin{equation}
 T_0 \approx \frac{2\pi}{\sqrt{2 - \cos\varepsilon_1}} \sqrt{\frac{\ell}{g}}
\label{eq:original}
\end{equation}

\paragraph{Observações.}
Obviamente, se considerássemos 
\[
 \phi(\theta) \approx \phi(0) + \phi^{\prime}(0) \theta + \frac{\phi^{\prime\prime}(0)}{2}\theta ^2,
\]
teríamos uma melhor aproximação para $T_0$, visto que
\begin{itemize}
	\item[(v)] $ \displaystyle \phi^{\prime\prime}(\theta) = \frac{\cos\theta}{\sqrt{(2-k)^3}} + \frac{3\sen^2\theta}{\sqrt{(2\cos\theta - k)^5}}$
	\item[(vi)] O que implica : $ \displaystyle \phi^{\prime\prime}(0) = \frac{1}{\sqrt{(2-k)^3}} $
	\item[(vii)] Daí, 
	 \begin{align*}
	  \phi(\theta) 
			&\approx \frac{1}{\sqrt{2-k}} + 0\cdot \theta + \frac{1}{\sqrt{(2-k)^3}} \theta^2\\
			&= \frac{1}{\sqrt{2-k}} + \frac{1}{\sqrt{(2-k)^3}} \theta^2
		\end{align*}
	\item[(viii)] Integrando, temos:
	 \begin{align*}
		 \tau_0 &\approx 4 \int_{0}^{\pi/2} \left(\frac{1}{\sqrt{2-k}} + \frac{1}{\sqrt{(2-k)^3}} \theta^2\right)\dd{\theta}\\
			&= 4 \cdot \left.\left(\frac{1}{\sqrt{2-k}} \theta + \frac{1}{\sqrt{(2-k)^3}} \frac{\theta^3}{3}\right)\right|_{0}^{\pi/2}\\
			&= 4 \cdot \left[\frac{\pi}{2\sqrt{2-k}} + \frac{\pi^3}{24\sqrt{(2-k)^3}}\right]\\
			&= \frac{2\pi}{\sqrt{2-k}} + \frac{\pi^3}{6\sqrt{(2-k)^3}}
		\end{align*}
\end{itemize}

E, portanto (lembrando que $ k = 2\cos\varepsilon_1 $ e $\tau_0 = \frac{T_0}{T}$):
	 \begin{equation}
		 T_0 \approx \left[\frac{2\pi}{\sqrt{2-2\cos\varepsilon_1}} + \frac{\pi^3}{6\sqrt{(2-2\cos\varepsilon_1)^3}}\right]\cdot \sqrt{\frac{\ell}{g}}
		\label{eq:periodo-2}
	 \end{equation}
		
 É interessante notarmos que, as aproximações tanto \eqref{eq:original}, quando
	em \eqref{eq:periodo-2} indicam que o período	depende, também, da	posição 
	$\varepsilon_1$, para uma melhor precisão.
	Além disso, estamos considerando $\vert\theta\vert < \pi$, visto que a série
	pode não convergir para as potências de $\theta$ fora desse intervalo.
\end{solution}



















