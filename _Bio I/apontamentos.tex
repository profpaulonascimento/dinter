\renewcommand\title{MODELOS MATEMÁTICOS} % Title

\begin{filecontents*}{testZ.dat}
0  0
0.301029996  0
0.477121255  0.333333333
0.602059991  0.5
0.698970004  0.4
0.77815125  0.5
0.84509804  0.428571429
0.903089987  0.5
0.954242509  0.444444444
1  0.4
1.041392685  0.363636364
1.079181246  0.416666667
1.113943352  0.384615385
1.146128036  0.428571429
1.176091259  0.4
1.204119983  0.375
1.230448921  0.352941176
1.255272505  0.388888889
1.278753601  0.368421053
1.301029996  0.4
1.322219295  0.380952381
1.342422681  0.363636364
1.361727836  0.347826087
1.380211242  0.375
1.397940009  0.36
1.414973348  0.346153846
1.431363764  0.333333333
1.447158031  0.321428571
1.462397998  0.310344828
1.477121255  0.333333333
1.491361694  0.322580645
1.505149978  0.34375
1.51851394  0.333333333
1.531478917  0.323529412
1.544068044  0.314285714
1.556302501  0.305555556
1.568201724  0.297297297
1.579783597  0.315789474
1.591064607  0.307692308
1.602059991  0.3
1.612783857  0.292682927
1.62324929  0.30952381
1.633468456  0.302325581
1.643452676  0.318181818
1.653212514  0.311111111
1.662757832  0.304347826
1.672097858  0.29787234
1.681241237  0.3125
1.69019608  0.306122449
1.698970004  0.3
1.707570176  0.294117647
1.716003344  0.288461538
1.72427587  0.283018868
1.73239376  0.296296296
1.740362689  0.290909091
1.748188027  0.285714286
1.755874856  0.280701754
1.763427994  0.275862069
1.770852012  0.271186441
1.77815125  0.283333333
1.785329835  0.278688525
1.792391689  0.290322581
1.799340549  0.285714286
1.806179974  0.28125
1.812913357  0.276923077
1.819543936  0.272727273
1.826074803  0.268656716
1.832508913  0.279411765
1.838849091  0.275362319
1.84509804  0.271428571
1.851258349  0.267605634
1.857332496  0.277777778
1.86332286  0.273972603
1.86923172  0.283783784
1.875061263  0.28
1.880813592  0.276315789
1.886490725  0.272727273
1.892094603  0.269230769
1.897627091  0.265822785
1.903089987  0.275
1.908485019  0.271604938
1.913813852  0.268292683
1.919078092  0.265060241
1.924279286  0.273809524
1.929418926  0.270588235
1.934498451  0.26744186
1.939519253  0.264367816
1.944482672  0.261363636
1.949390007  0.258426966
1.954242509  0.266666667
1.959041392  0.263736264
1.963787827  0.260869565
1.968482949  0.258064516
1.973127854  0.255319149
1.977723605  0.252631579
1.982271233  0.25
1.986771734  0.24742268
1.991226076  0.255102041
1.995635195  0.252525253
2  0.25
2.004321374  0.247524752
2.008600172  0.254901961
2.012837225  0.252427184
2.017033339  0.259615385
2.021189299  0.257142857
2.025305865  0.254716981
2.029383778  0.252336449
2.033423755  0.259259259
2.037426498  0.256880734
2.041392685  0.263636364
2.045322979  0.261261261
2.049218023  0.258928571
2.053078443  0.256637168
2.056904851  0.263157895
2.06069784  0.260869565
2.064457989  0.25862069
2.068185862  0.256410256
2.071882007  0.254237288
2.075546961  0.25210084
2.079181246  0.25
2.08278537  0.247933884
2.086359831  0.245901639
2.089905111  0.243902439
2.093421685  0.241935484
2.096910013  0.24
2.100370545  0.238095238
2.103803721  0.236220472
2.10720997  0.2421875
2.11058971  0.240310078
2.113943352  0.238461538
2.117271296  0.236641221
2.120573931  0.242424242
2.123851641  0.240601504
2.127104798  0.23880597
2.130333768  0.237037037
2.133538908  0.235294118
2.136720567  0.233576642
2.139879086  0.239130435
2.1430148  0.237410072
2.146128036  0.242857143
2.149219113  0.241134752
2.152288344  0.23943662
2.155336037  0.237762238
2.158362492  0.236111111
2.161368002  0.234482759
2.164352856  0.232876712
2.167317335  0.231292517
2.170261715  0.22972973
2.173186268  0.228187919
2.176091259  0.233333333
2.178976947  0.231788079
2.181843588  0.236842105
2.184691431  0.235294118
2.187520721  0.233766234
2.190331698  0.232258065
2.193124598  0.230769231
2.195899652  0.229299363
2.198657087  0.234177215
2.201397124  0.232704403
2.204119983  0.23125
2.206825876  0.229813665
2.209515015  0.228395062
2.212187604  0.226993865
2.214843848  0.231707317
2.217483944  0.23030303
2.220108088  0.228915663
2.222716471  0.22754491
2.225309282  0.232142857
2.227886705  0.230769231
2.230448921  0.229411765
2.23299611  0.228070175
2.235528447  0.226744186
2.238046103  0.225433526
2.240549248  0.229885057
2.243038049  0.228571429
2.245512668  0.227272727
2.247973266  0.225988701
2.250420002  0.224719101
2.252853031  0.223463687
2.255272505  0.227777778
2.257678575  0.226519337
2.260071388  0.230769231
2.26245109  0.229508197
2.264817823  0.22826087
2.267171728  0.227027027
2.269512944  0.225806452
2.271841607  0.22459893
2.274157849  0.223404255
2.276461804  0.222222222
2.278753601  0.221052632
2.281033367  0.219895288
2.283301229  0.223958333
2.285557309  0.222797927
2.28780173  0.226804124
2.290034611  0.225641026
2.292256071  0.224489796
2.294466226  0.223350254
2.29666519  0.227272727
2.298853076  0.226130653
2.301029996  0.23
2.303196057  0.228855721
2.305351369  0.227722772
2.307496038  0.226600985
2.309630167  0.225490196
2.311753861  0.224390244
2.31386722  0.223300971
2.315970345  0.222222222
2.318063335  0.221153846
2.320146286  0.220095694
2.322219295  0.219047619
2.324282455  0.218009479
2.326335861  0.221698113
2.328379603  0.220657277
2.330413773  0.219626168
2.33243846  0.218604651
2.334453751  0.217592593
2.336459734  0.216589862
2.338456494  0.21559633
2.340444115  0.214611872
2.342422681  0.213636364
2.344392274  0.212669683
2.346352974  0.211711712
2.348304863  0.210762332
2.350248018  0.214285714
2.352182518  0.213333333
2.354108439  0.212389381
2.356025857  0.211453744
2.357934847  0.214912281
2.359835482  0.213973799
2.361727836  0.217391304
2.36361198  0.216450216
2.365487985  0.215517241
2.367355921  0.214592275
2.369215857  0.217948718
2.371067862  0.217021277
2.372912003  0.216101695
2.374748346  0.215189873
2.376576957  0.214285714
2.378397901  0.213389121
2.380211242  0.216666667
2.382017043  0.215767635
2.383815366  0.219008264
2.385606274  0.218106996
2.387389826  0.217213115
2.389166084  0.216326531
2.390935107  0.215447154
2.392696953  0.214574899
2.394451681  0.213709677
2.396199347  0.212851406
2.397940009  0.212
2.399673721  0.211155378
2.401400541  0.214285714
2.403120521  0.213438735
2.404833717  0.212598425
2.40654018  0.211764706
2.408239965  0.2109375
2.409933123  0.210116732
2.411619706  0.213178295
2.413299764  0.212355212
2.414973348  0.211538462
2.416640507  0.210727969
2.418301291  0.209923664
2.419955748  0.209125475
2.421603927  0.212121212
2.423245874  0.211320755
2.424881637  0.210526316
2.426511261  0.209737828
2.428134794  0.208955224
2.42975228  0.208178439
2.431363764  0.211111111
2.432969291  0.210332103
2.434568904  0.213235294
2.436162647  0.212454212
2.437750563  0.211678832
2.439332694  0.210909091
2.440909082  0.210144928
2.442479769  0.209386282
2.444044796  0.212230216
2.445604203  0.211469534
2.447158031  0.210714286
2.44870632  0.209964413
2.450249108  0.212765957
2.451786436  0.212014134
2.45331834  0.214788732
2.45484486  0.214035088
2.456366033  0.213286713
2.457881897  0.212543554
2.459392488  0.211805556
2.460897843  0.211072664
2.462397998  0.210344828
2.463892989  0.209621993
2.465382851  0.20890411
2.46686762  0.208191126
2.46834733  0.210884354
2.469822016  0.210169492
2.471291711  0.209459459
2.472756449  0.208754209
2.474216264  0.208053691
2.475671188  0.20735786
2.477121255  0.206666667
2.478566496  0.205980066
2.480006943  0.205298013
2.481442629  0.204620462
2.482873584  0.203947368
2.484299839  0.203278689
2.485721426  0.202614379
2.487138375  0.201954397
2.488550717  0.204545455
2.489958479  0.203883495
2.491361694  0.203225806
2.492760389  0.202572347
2.494154594  0.205128205
2.495544338  0.204472843
2.496929648  0.207006369
2.498310554  0.206349206
2.499687083  0.205696203
2.501059262  0.205047319
2.50242712  0.20754717
2.503790683  0.206896552
2.505149978  0.20625
2.506505032  0.205607477
2.507855872  0.204968944
2.509202522  0.204334365
2.51054501  0.203703704
2.511883361  0.203076923
2.5132176  0.202453988
2.514547753  0.201834862
2.515873844  0.201219512
2.517195898  0.200607903
2.51851394  0.2
2.519827994  0.19939577
2.521138084  0.201807229
2.522444234  0.201201201
2.523746467  0.200598802
2.525044807  0.2
2.526339277  0.199404762
2.527629901  0.198813056
2.5289167  0.201183432
2.530199698  0.200589971
2.531478917  0.2
2.532754379  0.19941349
2.534026106  0.198830409
2.53529412  0.198250729
2.536558443  0.197674419
2.537819095  0.197101449
2.539076099  0.196531792
2.540329475  0.195965418
2.541579244  0.198275862
2.542825427  0.197707736
2.544068044  0.2
2.545307116  0.199430199
2.546542663  0.198863636
2.547774705  0.198300283
2.549003262  0.200564972
2.550228353  0.2
2.551449998  0.199438202
2.552668216  0.198879552
2.553883027  0.198324022
2.555094449  0.197771588
2.556302501  0.2
2.557507202  0.199445983
2.558708571  0.198895028
2.559906625  0.198347107
2.561101384  0.197802198
2.562292864  0.197260274
2.563481085  0.196721311
2.564666064  0.196185286
2.565847819  0.198369565
2.567026366  0.197831978
2.568201724  0.197297297
2.56937391  0.196765499
2.57054294  0.196236559
2.571708832  0.195710456
2.572871602  0.197860963
2.574031268  0.197333333
2.575187845  0.196808511
2.57634135  0.196286472
2.5774918  0.195767196
2.57863921  0.19525066
2.579783597  0.197368421
2.580924976  0.196850394
2.582063363  0.196335079
2.583198774  0.195822454
2.584331224  0.197916667
2.58546073  0.197402597
2.586587305  0.196891192
2.587710965  0.196382429
2.588831726  0.195876289
2.589949601  0.195372751
2.591064607  0.197435897
2.592176757  0.196930946
2.593286067  0.196428571
2.59439255  0.195928753
2.595496222  0.195431472
2.596597096  0.194936709
2.597695186  0.194444444
2.598790507  0.19395466
2.599883072  0.195979899
2.600972896  0.195488722
2.602059991  0.195
2.603144373  0.194513716
2.604226053  0.196517413
2.605305046  0.196029777
2.606381365  0.195544554
2.607455023  0.195061728
2.608526034  0.194581281
2.609594409  0.194103194
2.610660163  0.193627451
2.611723308  0.193154034
2.612783857  0.195121951
2.613841822  0.194647202
2.614897216  0.194174757
2.615950052  0.1937046
2.617000341  0.193236715
2.618048097  0.192771084
2.619093331  0.192307692
2.620136055  0.191846523
2.621176282  0.19138756
2.622214023  0.190930788
2.62324929  0.192857143
2.624282096  0.19239905
2.625312451  0.194312796
2.626340367  0.193853428
2.627365857  0.193396226
2.62838893  0.192941176
2.629409599  0.192488263
2.630427875  0.192037471
2.631443769  0.191588785
2.632457292  0.191142191
2.633468456  0.190697674
2.63447727  0.19025522
2.635483747  0.19212963
2.636487896  0.191685912
2.63748973  0.193548387
2.638489257  0.193103448
2.639486489  0.19266055
2.640481437  0.19221968
2.641474111  0.191780822
2.64246452  0.191343964
2.643452676  0.193181818
2.644438589  0.192743764
2.645422269  0.192307692
2.646403726  0.191873589
2.64738297  0.193693694
2.648360011  0.193258427
2.649334859  0.192825112
2.650307523  0.192393736
2.651278014  0.191964286
2.652246341  0.191536748
2.653212514  0.193333333
2.654176542  0.192904656
2.655138435  0.192477876
2.656098202  0.19205298
2.657055853  0.191629956
2.658011397  0.191208791
2.658964843  0.190789474
2.6599162  0.190371991
2.660865478  0.192139738
2.661812686  0.191721133
2.662757832  0.191304348
2.663700925  0.190889371
2.664641976  0.192640693
2.665580991  0.192224622
2.666517981  0.193965517
2.667452953  0.193548387
2.668385917  0.193133047
2.669316881  0.192719486
2.670245853  0.194444444
2.671172843  0.194029851
2.672097858  0.193617021
2.673020907  0.193205945
2.673941999  0.19279661
2.674861141  0.192389006
2.675778342  0.191983122
2.67669361  0.191578947
2.677606953  0.191176471
2.678518379  0.190775681
2.679427897  0.190376569
2.680335513  0.189979123
2.681241237  0.191666667
2.682145076  0.191268191
2.683047038  0.190871369
2.683947131  0.19047619
2.684845362  0.190082645
2.685741739  0.189690722
2.686636269  0.189300412
2.687528961  0.188911704
2.688419822  0.19057377
2.689308859  0.190184049
2.69019608  0.189795918
2.691081492  0.189409369
2.691965103  0.191056911
2.692846919  0.190669371
2.693726949  0.190283401
2.694605199  0.18989899
2.695481676  0.189516129
2.696356389  0.189134809
2.697229343  0.18875502
2.698100546  0.188376754
2.698970004  0.19
2.699837726  0.189620758
2.700703717  0.189243028
2.701567985  0.188866799
2.702430536  0.19047619
2.703291378  0.19009901
2.704150517  0.18972332
2.705007959  0.189349112
2.705863712  0.188976378
2.706717782  0.188605108
2.707570176  0.190196078
2.7084209  0.189823875
2.709269961  0.189453125
2.710117365  0.189083821
2.710963119  0.188715953
2.711807229  0.188349515
2.712649702  0.187984496
2.713490543  0.18762089
2.71432976  0.187258687
2.715167358  0.186897881
2.716003344  0.186538462
2.716837723  0.186180422
2.717670503  0.187739464
2.718501689  0.187380497
2.719331287  0.188931298
2.720159303  0.188571429
2.720985744  0.188212928
2.721810615  0.187855787
2.722633923  0.1875
2.723455672  0.187145558
2.72427587  0.186792453
2.725094521  0.186440678
2.725911632  0.186090226
2.726727209  0.185741088
2.727541257  0.185393258
2.728353782  0.185046729
2.72916479  0.184701493
2.729974286  0.184357542
2.730782276  0.18401487
2.731588765  0.183673469
2.73239376  0.183333333
2.733197265  0.182994455
2.733999287  0.184501845
2.73479983  0.184162063
2.7355989  0.183823529
2.736396502  0.183486239
2.737192643  0.183150183
2.737987326  0.182815356
2.738780558  0.184306569
2.739572344  0.183970856
2.740362689  0.183636364
2.741151599  0.183303085
2.741939078  0.182971014
2.742725131  0.182640145
2.743509765  0.182310469
2.744292983  0.181981982
2.745074792  0.181654676
2.745855195  0.181328546
2.746634199  0.182795699
2.747411808  0.182468694
2.748188027  0.182142857
2.748962861  0.181818182
2.749736316  0.181494662
2.750508395  0.181172291
2.751279104  0.182624113
2.752048448  0.182300885
2.752816431  0.181978799
2.753583059  0.181657848
2.754348336  0.181338028
2.755112266  0.181019332
2.755874856  0.18245614
2.756636108  0.182136602
2.757396029  0.183566434
2.758154622  0.183246073
2.758911892  0.182926829
2.759667845  0.182608696
2.760422483  0.182291667
2.761175813  0.181975737
2.761927838  0.183391003
2.762678564  0.183074266
2.763427994  0.182758621
2.764176132  0.182444062
2.764922985  0.182130584
2.765668555  0.181818182
2.766412847  0.181506849
2.767155866  0.181196581
2.767897616  0.180887372
2.768638101  0.180579216
2.769377326  0.181972789
2.770115295  0.181663837
2.770852012  0.181355932
2.771587481  0.181049069
2.772321707  0.180743243
2.773054693  0.180438449
2.773786445  0.181818182
2.774516966  0.181512605
2.77524626  0.181208054
2.775974331  0.180904523
2.776701184  0.180602007
2.777426822  0.180300501
2.77815125  0.181666667
2.778874472  0.181364393
2.779596491  0.182724252
2.780317312  0.182421227
2.781036939  0.182119205
2.781755375  0.181818182
2.782472624  0.181518152
2.783188691  0.18121911
2.783903579  0.182565789
2.784617293  0.18226601
2.785329835  0.181967213
2.78604121  0.181669394
2.786751422  0.181372549
2.787460475  0.181076672
2.788168371  0.182410423
2.788875116  0.182113821
2.789580712  0.181818182
2.790285164  0.181523501
2.790988475  0.182847896
2.791690649  0.182552504
2.792391689  0.183870968
2.7930916  0.183574879
2.793790385  0.183279743
2.794488047  0.182985554
2.79518459  0.182692308
2.795880017  0.1824
2.796574333  0.182108626
2.797267541  0.181818182
2.797959644  0.181528662
2.798650645  0.181240064
2.799340549  0.180952381
2.800029359  0.18066561
2.800717078  0.181962025
2.80140371  0.181674566
2.802089258  0.181388013
2.802773725  0.181102362
2.803457116  0.18081761
2.804139432  0.180533752
2.804820679  0.180250784
2.805500858  0.179968701
2.806179974  0.1796875
2.80685803  0.179407176
2.807535028  0.180685358
2.808210973  0.180404355
2.808885867  0.181677019
2.809559715  0.181395349
2.810232518  0.181114551
2.810904281  0.180834621
2.811575006  0.182098765
2.812244697  0.181818182
2.812913357  0.181538462
2.813580989  0.181259601
2.814247596  0.180981595
2.814913181  0.180704441
2.815577748  0.181957187
2.8162413  0.181679389
2.816903839  0.181402439
2.81756537  0.181126332
2.818225894  0.180851064
2.818885415  0.180576631
2.819543936  0.181818182
2.820201459  0.181543116
2.820857989  0.182779456
2.821513528  0.182503771
2.822168079  0.182228916
2.822821645  0.181954887
2.823474229  0.181681682
2.824125834  0.181409295
2.824776462  0.181137725
2.825426118  0.180866966
2.826074803  0.180597015
2.82672252  0.180327869
2.827369273  0.180059524
2.828015064  0.179791976
2.828659897  0.181008902
2.829303773  0.180740741
2.829946696  0.180473373
2.830588669  0.180206795
2.831229694  0.181415929
2.831869774  0.181148748
2.832508913  0.180882353
2.833147112  0.18061674
2.833784375  0.180351906
2.834420704  0.180087848
2.835056102  0.18128655
2.835690571  0.181021898
2.836324116  0.180758017
2.836956737  0.180494905
2.837588438  0.180232558
2.838219222  0.179970972
2.838849091  0.179710145
2.839478047  0.179450072
2.840106094  0.180635838
2.840733235  0.18037518
2.84135947  0.180115274
2.841984805  0.179856115
2.84260924  0.179597701
2.843232778  0.179340029
2.843855423  0.179083095
2.844477176  0.178826896
2.84509804  0.178571429
2.845718018  0.17831669
2.846337112  0.179487179
2.846955325  0.179231863
2.847572659  0.178977273
2.848189117  0.178723404
2.848804701  0.178470255
2.849419414  0.178217822
2.850033258  0.177966102
2.850646235  0.177715092
2.851258349  0.178873239
2.851869601  0.17862166
2.852479994  0.178370787
2.85308953  0.178120617
2.853698212  0.177871148
2.854306042  0.177622378
2.854913022  0.177374302
2.855519156  0.177126918
2.856124444  0.176880223
2.85672889  0.176634214
2.857332496  0.177777778
2.857935265  0.177531207
2.858537198  0.177285319
2.859138297  0.177040111
2.859738566  0.17679558
2.860338007  0.176551724
2.860936621  0.17630854
2.861534411  0.176066025
2.862131379  0.177197802
2.862727528  0.176954733
2.86332286  0.176712329
2.863917377  0.176470588
2.864511081  0.176229508
2.865103975  0.175989086
2.86569606  0.177111717
2.866287339  0.176870748
2.866877814  0.176630435
2.867467488  0.176390773
2.868056362  0.176151762
2.868644438  0.175913396
2.86923172  0.177027027
2.869818208  0.176788124
2.870403905  0.176549865
2.870988814  0.176312248
2.871572936  0.177419355
2.872156273  0.177181208
2.872738827  0.1769437
2.873320602  0.176706827
2.873901598  0.176470588
2.874481818  0.17623498
2.875061263  0.176
2.875639937  0.175765646
2.876217841  0.176861702
2.876794976  0.176626826
2.877371346  0.176392573
2.877946952  0.17615894
2.878521796  0.175925926
2.87909588  0.175693527
2.879669206  0.176781003
2.880241776  0.17654809
2.880813592  0.176315789
2.881384657  0.1760841
2.881954971  0.177165354
2.882524538  0.176933159
2.883093359  0.176701571
2.883661435  0.176470588
2.88422877  0.176240209
2.884795364  0.17601043
2.88536122  0.17578125
2.88592634  0.175552666
2.886490725  0.176623377
2.887054378  0.176394293
2.8876173  0.176165803
2.888179494  0.175937904
2.888740961  0.177002584
2.889301703  0.176774194
2.889861721  0.176546392
2.890421019  0.176319176
2.890979597  0.176092545
2.891537458  0.175866496
2.892094603  0.175641026
2.892651034  0.175416133
2.893206753  0.175191816
2.893761762  0.174968072
2.894316063  0.174744898
2.894869657  0.174522293
2.895422546  0.174300254
2.895974732  0.17407878
2.896526217  0.175126904
2.897077003  0.174904943
2.897627091  0.174683544
2.898176483  0.174462705
2.898725182  0.174242424
2.899273187  0.174022699
2.899820502  0.173803526
2.900367129  0.173584906
2.900913068  0.173366834
2.901458321  0.17314931
2.902002891  0.174185464
2.902546779  0.173967459
2.903089987  0.17375
2.903632516  0.173533084
2.904174368  0.173316708
2.904715545  0.173100872
2.905256049  0.172885572
2.90579588  0.172670807
2.906335042  0.172456576
2.906873535  0.172242875
2.907411361  0.172029703
2.907948522  0.171817058
2.908485019  0.172839506
2.909020854  0.172626387
2.909556029  0.17364532
2.910090546  0.173431734
2.910624405  0.173218673
2.911157609  0.173006135
2.911690159  0.172794118
2.912222057  0.172582619
2.912753304  0.172371638
2.913283902  0.172161172
2.913813852  0.17195122
2.914343157  0.171741778
2.914871818  0.172749392
2.915399835  0.17253949
2.915927212  0.173543689
2.916453949  0.173333333
2.916980047  0.173123487
2.91750551  0.172914148
2.918030337  0.173913043
2.918554531  0.173703257
2.919078092  0.174698795
2.919601024  0.174488568
2.920123326  0.174278846
2.920645001  0.174069628
2.921166051  0.173860911
2.921686475  0.173652695
2.922206277  0.173444976
2.922725458  0.173237754
2.923244019  0.173031026
2.923761961  0.172824791
2.924279286  0.173809524
2.924795996  0.173602854
2.925312091  0.173396675
2.925827575  0.173190985
2.926342447  0.172985782
2.926856709  0.172781065
2.927370363  0.172576832
2.92788341  0.172373081
2.928395852  0.172169811
2.92890769  0.17196702
2.929418926  0.171764706
2.92992956  0.171562867
2.930439595  0.171361502
2.930949031  0.17116061
2.931457871  0.172131148
2.931966115  0.171929825
2.932473765  0.171728972
2.932980822  0.171528588
2.933487288  0.172494172
2.933993164  0.172293364
2.934498451  0.173255814
2.935003151  0.173054588
2.935507266  0.172853828
2.936010796  0.172653534
2.936513742  0.173611111
2.937016107  0.173410405
2.937517892  0.173210162
2.938019097  0.173010381
2.938519725  0.17281106
2.939019776  0.172612198
2.939519253  0.172413793
2.940018155  0.172215844
2.940516485  0.172018349
2.941014244  0.171821306
2.941511433  0.171624714
2.942008053  0.171428571
2.942504106  0.171232877
2.942999593  0.171037628
2.943494516  0.171981777
2.943988875  0.171786121
2.944482672  0.171590909
2.944975908  0.171396141
2.945468585  0.172335601
2.945960704  0.17214043
2.946452265  0.173076923
2.946943271  0.172881356
2.947433722  0.17268623
2.94792362  0.172491545
2.948412966  0.173423423
2.948901761  0.173228346
2.949390007  0.173033708
2.949877704  0.172839506
2.950364854  0.17264574
2.950851459  0.172452408
2.951337519  0.172259508
2.951823035  0.172067039
2.95230801  0.171875
2.952792443  0.171683389
2.953276337  0.171492205
2.953759692  0.171301446
2.954242509  0.171111111
2.954724791  0.170921199
2.955206538  0.170731707
2.95568775  0.170542636
2.95616843  0.170353982
2.956648579  0.170165746
2.957128198  0.169977925
2.957607287  0.169790518
2.958085849  0.170704846
2.958563883  0.170517052
2.959041392  0.17032967
2.959518377  0.1701427
2.959994838  0.171052632
2.960470778  0.170865279
2.960946196  0.170678337
2.961421094  0.170491803
2.961895474  0.170305677
2.962369336  0.170119956
2.962842681  0.169934641
2.963315511  0.169749728
2.963787827  0.170652174
2.96425963  0.170466884
2.964730921  0.170281996
2.965201701  0.170097508
2.965671971  0.16991342
2.966141733  0.16972973
2.966610987  0.169546436
2.967079734  0.169363538
2.967547976  0.169181034
2.968015714  0.168998924
2.968482949  0.169892473
2.968949681  0.169709989
2.969415912  0.169527897
2.969881644  0.169346195
2.970346876  0.169164882
2.970811611  0.168983957
2.971275849  0.168803419
2.971739591  0.168623266
2.972202838  0.169509595
2.972665592  0.169329073
2.973127854  0.169148936
2.973589623  0.168969182
2.974050903  0.16985138
2.974511693  0.169671262
2.974971994  0.169491525
2.975431809  0.169312169
2.975891136  0.169133192
2.976349979  0.168954593
2.976808337  0.169831224
2.977266212  0.169652266
2.977723605  0.169473684
2.978180517  0.169295478
2.978636948  0.169117647
2.979092901  0.168940189
2.979548375  0.169811321
2.980003372  0.169633508
2.980457892  0.169456067
2.980911938  0.169278997
2.981365509  0.169102296
2.981818607  0.168925965
2.982271233  0.16875
2.982723388  0.168574402
2.983175072  0.168399168
2.983626287  0.168224299
2.984077034  0.168049793
2.984527313  0.167875648
2.984977126  0.167701863
2.985426474  0.167528438
2.985875357  0.16838843
2.986323777  0.168214654
2.986771734  0.168041237
2.98721923  0.167868177
2.987666265  0.16872428
2.98811284  0.168550874
2.988558957  0.168377823
2.989004616  0.168205128
2.989449818  0.168032787
2.989894564  0.167860798
2.990338855  0.168711656
2.990782692  0.168539326
2.991226076  0.168367347
2.991669007  0.168195719
2.992111488  0.16802444
2.992553518  0.16785351
2.992995098  0.168699187
2.99343623  0.168527919
2.993876915  0.168356998
2.994317153  0.168186424
2.994756945  0.168016194
2.995196292  0.167846309
2.995635195  0.167676768
2.996073654  0.167507568
2.996511672  0.168346774
2.996949248  0.168177241
2.997386384  0.168008048
2.997823081  0.167839196
2.998259338  0.167670683
2.998695158  0.167502508
2.999130541  0.168336673
2.999565488  0.168168168
3  0.168
\end{filecontents*}


\chapter{PRINCÍPIOS DE REPRESENTAÇÃO MATEMÁTICA: Modelos Matemáticos e Análise Dimensional}

\begin{citacao}
``Mathematical Models are Lies that help us see some Truth''.

\rightline{Lee A. Segel, Biomatemático.}

``All Mathematical Models are wrong, however, some are useful but the only way to know which when and how they are useful is by using them, which seems totally tautological, although it is not!''.

Parafraseando e estendendo:

\rightline{George Box (Estatístico).}
\end{citacao}

\section{Introdução: Representação Matemática e o Elogio à Simplicidade}

    O tema genérico a ser tratado nesta seção é o conceito de Modelo Matemático, um tema raramente apresentado na literatura, mesmo de Matemática Aplicada.

    Uma definição geral e preliminar deste conceito que abrange a maioria dos casos a serem estudados neste texto pode ser a seguinte:

    Um \textbf{Modelo Matemático} para um determinado Fenômeno Natural, é uma \textbf{Estrutura Matemática} associada a uma \textbf{interpretação} que a identifica com um \textbf{Conjunto de Aspectos} (``traços'') considerados relevantes e suficientes para a descrição do referido Fenômeno não matemático.

    Uma \textbf{Estrutura Matemática} é constituída {\color{orange} por} um conjunto de Elementos dotada de operações e relações entre eles. Por exemplo, os Números Naturais \(\mathbb{N}\) em que estão definidas {\color{orange} a adição e a multiplicação} e os números reais \(\mathbb{R}\) em que esta definida uma estrutura que consiste de operações algébricas ({\color{orange} adição, multiplicação}) uma ordem e a topologia de convergência.
    
    A demografia simples de populações consiste na observação dos seus respectivos tamanhos, os únicos ``traços'' que nos interessam descrever no caso (em que idade, sexo, peso, nacionalidade, grau de infecciosidade, etc. ,etc., etc., não fazem parte) e os tamanhos podem ser representados por um número natural.

    O ``Conjunto de aspectos considerados relevantes para a descrição de um Fenômeno Natural'', em qualquer circunstância, nunca poderá ser exaustivo e completo e deve ser necessariamente restrito (pode-se descrever muita coisa sobre uma população, mas não tudo!). Esta limitação não é apenas decorrente da linguagem utilizada nesta descrição (seja ela linguística ou matemática), mas também devido à limitação intelectual humana.

    Parafraseando Lee Segel (que parafraseou Ernst Gombrich com respeito à arte) ``\textit{Um Modelo é apenas uma caricatura que enfatiza traços de interesse do objeto em observação}'', ou ainda, ``\textit{Um Modelo Matemático é uma mentira que nos ajuda a entender parte da verdade}''.

    Portanto, uma das tarefas mais difíceis na construção de um Modelo Matemático é estabelecer quais informações são indispensáveis para produzir uma descrição desejável do Fenômeno, desprezando assim, explícita ou implicitamente toda a infinidade de outras informações, disponíveis ou não. (Segundo o matemático Mark Kac: ``É importante saber descartar corretamente as informações, ou seja, é importante saber como não lançar fora o bebê junto com a água do banho'').

    Há duas maneiras metodológicas para proceder neste empreendimento que, caracteristicamente, é processado em várias etapas.
    
    O primeiro método encara a questão de ``\textit{Cima para Baixo}'' (Método ``Top-Down'', segundo Segel) incorporando deliberadamente todas as informações disponíveis e imagináveis do Fenômeno estudado a um super Modelo para, posteriormente, descartar as informações supérfluas até que encontremos um Modelo minimalista e suficiente para os nossos propósitos. Este processo é, obviamente, impossível de ser estritamente realizado, se não por falta de conhecimento completo das informações sobre o Fenômeno, como também pelo enorme trabalho que envolveria a simplificação do mostrengo resultante. (A metodologia Top-Down poderia ser associado ao nome de Michelangelo a quem é atribuída a frase sobre sua escultura de David: ``\textit{A imagem já estava no bloco rústico de mármore que me foi entregue, a minha função foi apenas aquela de retirar as partes supérfluas}''). Em Matemática, Aplicada ou não, os Michelangelos são ainda mais raros do que na escultura! Todavia, é importante ressaltar que a metodologia de redução é indispensável para a simplificação de Modelos que, originalmente, não são completos, mas são excessivamente detalhistas para o objetivo em vista. Métodos de Redução serão abordados em um capítulo à parte e fazem parte essencial dos instrumentos matemáticos necessários para a construção de Modelos Matemáticos.

    Resta-nos, portanto, a segunda alternativa que encara a tarefa de construção de um Modelo Matemático de ``\textit{Baixo para Cima}'' (ou, ``\textit{Bottom-Up}'', como dizia Segel). Neste caso, partimos de um Modelo drástica e deliberadamente simplificado (às vezes chamados de ``Toy Models'') em que se preserva apenas uma semelhança fundamental com o Fenômeno em vista, para, posteriormente, acrescentarmos, passo a passo, se necessário, informações ao Modelo original tornando-o sucessivamente mais representativo, até o ponto que seja satisfatório para os fins a que se destina. Esta atitude é construtiva e permite um maior controle sobre cada passo uma vez que se inicia com Modelos que, em tese, são matematicamente analisáveis e cujas discrepâncias com a ``Realidade'' podem ser diagnosticadas, e resolvidas, com maior facilidade. Modelos extremamente simples, tanto pela sua estrutura matemática elementar quanto pela descrição deliberadamente caricatural podem ser de grande utilidade para a compreensão do fenômeno estudado se elaborados com perspicácia e conhecimento do tema em questão. James Murray, um dos iniciadores da moderna Biomatemática (e autor de um de seus textos mais influentes - \textbf{Mathematical Biology}, Springer - 1\textordfeminine\ ed. 1989) argumentou, de maneira convincente e com autoridade, sobre este fato em uma memorável conferência plenária da European Soc. for Theor. and Math. Biology - ESMTB, em 2005 (Dresden, Alemanha), posteriormente publicada na forma de artigos (Refer. de Leitura). Segundo ele:

\begin{quote}
    ``\(95\%\) or more of some phenomenon can be explained with very elementary mathematics, while adding more explanation of it requires an exponentially increasing amount of Mathematics which may not even be available''.
\end{quote}

    Neste ponto, é válido citar o famoso e antigo \textbf{Princípio de Parcimônia} do monge inglês William Ockham (séc. XIII) expresso pela imponente frase em latim ``\textit{Pluralitas non est ponenda sine neccesitate}'' que nos adverte sobre a importância de evitar a tentação em acrescentar complicações além do necessário. Em outras palavras, poderíamos dizer que a Natureza já é suficientemente complicada e não necessita de nossa ajuda voluntária neste sentido; o nosso papel é atuar exatamente no sentido oposto, ou nos depararemos rapidamente com uma barreira intransponível.

    Sendo assim, iniciaremos o estudo do tema deste capítulo com o exemplo mais simples possível dos modelos matemáticos.

\section{As Estruturas Numéricas}

\begin{quote}
    ``What is man that can understand the concept of number, and what is number that man can understand it''.

    \rightline{W. McCulloch.}
\end{quote}

    As estruturas matemáticas mais antigas e úteis nas aplicações foram, na verdade, originadas de sua própria utilidade prática cotidiana. Dentre todas as estruturas matemáticas utilizadas como Modelos Matemáticos, sem dúvida alguma a Aritmética dos números naturais é a mais fundamental e presente em todas as civilizações humanas. O conceito de número natural é uma característica cognitiva tão fundamental que é observada até mesmo no comportamento de alguns animais e aparentemente está impressa na fisiologia inata do \textit{homo sapiens} (S. Dehaene, A. Nieder).

    As estruturas matemáticas dos números naturais são utilizadas desde tempos imemoriais para a representação do conceito de \textbf{medida de tamanho} de Conjuntos (Cardinalidade) e a estrutura dos números reais para a medida de Comprimento e de Deslocamento linear.

    A Aritmética dos números naturais é a mais antiga e útil dentre todas as estruturas matemáticas e a primeira aprendida nas escolas. (Matematicamente, a estrutura de \(\mathbb{N}\) é o fundamento de todas as outras estruturas numéricas e de grande parte da Matemática. Segundo L. Kronecker: ``\textit{Os números naturais foram dádivas divinas e perfeitas, o resto são construções, imperfeitas, dos homens}'').

    O aprendizado da Aritmética na escola elementar se faz por intermédio de uma abstração inconsciente de processos concretos de contagem, digamos, de laranjas. As operações de soma e produto são interpretadas como manipulações específicas com ``populações de laranjas'' e se tornam de tal forma entranhadas que mal distinguimos o conceito de Modelo Matemático que se forma inconscientemente. De alguma forma, ainda misteriosa e intensamente estudada nos últimos anos (Dehaene, Nieder), a criança percebe que o procedimento de contagem não se restringe aos objetos concretos utilizados para representá-la (laranjas), mas pode ser abstraído na forma de uma estrutura matemática abstrata subjacente ao processo. Este salto de conceptualização é, certamente, um dos mais significativos na história da civilização e na Psicologia do aprendizado infantil, pois nos permite utilizar uma mesma estrutura abstrata para enumerar e ordenar objetos e situações que nunca foram, nem poderiam ser, mencionadas ou previstas durante o seu aprendizado. Em outros termos, com esta abstração, felizmente não precisamos aprender uma Aritmética para laranjas, outra para bolinhas de gude, outra para dinheiro, ..., e outra para contar quantas Aritméticas já aprendemos.

    Além disso, também não necessitamos de um manual de aplicação da Aritmética, pois imediatamente percebemos as circunstâncias em que podemos aplicá-la. A capacidade de síntese do Método de abstração se constitui na própria essência da Matemática e tem suas raízes profundas nas atividades práticas mais corriqueiras.

\section{A Análise Dimensional}

    A construção de um Modelo Matemático \textbf{quantitativo} (isto é, baseado na estrutura de números) é um procedimento que depende de uma escolha arbitrária, mas indispensável, que fica a cargo do observador: a \textbf{Unidade} Padrão de tamanho. Por exemplo, um lote (ou caixa) para a medida de Conjuntos, um Segmento de Reta para a medida de Comprimento linear, um Quadrado para a medida de área, o Peso para a medida de massa, um Período de oscilação de um pêndulo (ou, batida do pulso, o movimento do Sol ou uma oscilação de um átomo de Césio) para a medida de tempo e etc. Além desta escolha arbitrária inicial da Unidade padrão, o processo de mensuração exige a descrição de um procedimento ``prático'' para a comparação entre o objeto a ser medido e o objeto padrão. Estes dois fatos demonstram que a mesuração não é um processo estritamente matemático, mas é intimamente ligada ao fenômeno estudado e se encontra na fronteira entre as duas ciências envolvidas, no caso da Biomatemática, entre a Matemática e a Biologia.

    Explica-se assim, pelo menos parcialmente, porque este tema nunca é sequer mencionado em textos de Matemática. Entretanto, apesar de sua importância óbvia para a Matemática Aplicada, este tema nem sempre é se quer tratado em alguns textos do assunto e, frequentemente, tratado com menos ênfase e detalhes do que necessário. (Para uma notável exceção consulte C. C. Lin - L. A. Segel - Mathematics Applied to Natural Sciences, SIAM 1990).
    
    A \textbf{Análise Dimensional} é o estudo do processo de mensuração e, especialmente, do efeito que a escolha arbitrária de unidades tem na constituição de um Modelo Matemático quantitativo. O seu estudo é um pré-requisito indispensável para o entendimento de qualquer ciência quantitativa e será o tema central do presente capítulo.

    A Matemática utilizada na exposição da Análise Dimensional (Aritmética e elementos de Álgebra Linear) é considerada elementar e isto a torna, no julgamento de alguns, indigna de ser apresentada em textos mais sisudos. Entretanto, como veremos mais adiante, há uma série de resultados surpreendentes e profundos decorrentes de sua aplicação que por vezes se bastam para a compreensão de um Modelo Matemático (Barenblatt, West, MacMahon).

    Estruturas Matemáticas construídas diretamente a partir dos Números Reais, tais como Espaços Vetoriais \(\mathbb{R}^n\), Matrizes, Funções de variável real e valores reais e etc., são amplamente utilizados na construção de Modelos Matemáticos para a representação dos mais diversos fenômenos naturais e sociais. Dentre estas, certamente as Funções de variável e valores numéricos, constituem a mais importante classe de objetos matemáticos para a construção de Modelos Matemáticos, depois das estruturas numéricas.




\section{Funções como Modelos Matemáticos}

    As funções são objetos matemáticos cruciais na descrição de relações de causalidade e variação temporal cuja origem pode ser detectada nos trabalhos de Galileu.

    Consideremos, com Galileu, a descrição da queda livre de um objeto solto do alto da torre de Pisa, utilizando, para isso, uma tabela de duas colunas sendo: a primeira, constituída da anotação dos tempos \(t_k\) de observação (digamos, a cada batida de pulso) e; a segunda, da respectiva altura \(x_k\) do objeto em cada um daqueles instantes. Esta tabela é uma descrição do fato consumado, tal como a História humana que frequentemente é apresentada na forma de um grande tabela de ``efemérides'' impossível de ser memorizada, difícil de ser visualizada em contexto e, portanto, com uma utilidade muito reduzida.

    A grande ideia de Galileu foi buscar a representação desta enorme tabela de observações na forma de um Algoritmo que fosse capaz de produzir, ``à pedido'', qualquer valor da posição \(x(t)\) do objeto uma vez apresentado o respectivo instante \(t\) de queda, e não apenas para os arbitrários intervalos de tempo de uma tabela. Assim, em lugar de uma tabela explícita, mas inadministrável e sem uma estrutura matemática subjacente, teríamos uma sintética ``máquina analítica'' representada pela sua famosa \textbf{Fórmula} \(x = x_0 - \frac{1}{2}gt^2\) {\red (Aqui não está faltando \(+v_0t\))}. (O conceito de Algoritmo é mais antigo do que o conceito de Função e tem suas origens no século IX com o matemático iraquiano M. Al-Kwarismi de onde provém o termo (Knuth, Ershov). O italiano Fibonacci utilizou exatamente esta ideia de Algoritmo, no século XIII, para representar uma tabela de números naturais que descreveriam o tamanho da população de uma criação ``virtual'' de coelhos. Com isto, Fibonacci inaugurou, com um ``Toy Model'', uma das áreas mais importantes da Biomatemática que será desenvolvida em outro capítulo. O conceito de Fórmula e Algoritmo foi mais tarde extraordinariamente expandido com o advento de operações limite do Cálculo, especialmente as somas (séries).

    Entretanto, apesar da extraordinária síntese e simplificação que a descrição funcional representa com relação às tabelas de dados, este fato não esgotaria as vantagens deste procedimento.
    
    A grande vantagem oriunda da estratégia de Galileu é a possibilidade de definir e utilizar uma rica estrutura matemática no conjunto de funções o que foi proporcionado pela invenção do Cálculo Diferencial e Integral no século seguinte. Ou seja, as funções não seriam apenas ``máquinas de produzir tabelas, ou números'', mas \textbf{objetos matemáticos} com identidade matemática pertencentes a uma estrutura matemática, na qual é possível definir operações (adição, multiplicação, composição, derivação, integração e limites) que permitem aumentar enormemente a sua capacidade representativa. Esta é uma situação de todo análoga àquela que ocorreu com a invenção da Aritmética que transforma os números de símbolos inertes em elementos de uma estrutura matemática, com a qual é possível representar diversos procedimentos. (Em especial, com esta estrutura é possível representá-los e construí-los a partir de pequenos blocos; a expansão decimal).

    É imprescindível lembrar que invenção do Cálculo foi, na verdade, motivada pelo objetivo de Newton em representar o exaustivo conjunto de observações astronômicas de Tycho Brahe (já drasticamente sintetizadas pelas ``leis'' de Johannes Kepler) por intermédio de funções caracterizadas, não apenas por Fórmulas algébricas elementares (``explícitas'') à moda de Galileu, mas como soluções de equações diferenciais expressas com as operações do Cálculo, o que expandiu consideravelmente a classe de funções disponíveis.

    As grandes massas de dados (``\textit{Big Data}'') amplamente disponibilizadas pela moderna tecnologia informática para as mais diversas áreas do conhecimento instigou o ressurgimento de novos procedimentos que visam sintetizá-las compactamente e estruturalmente na forma de Funções. A Metodologia moderna não repete, todavia, exatamente o procedimento histórico Brahe-Newton, pois lança mão de diversas teorias e técnicas matemáticas surgidas apenas no último século. Este será o tema de um dos próximos capítulos destinado ao \textbf{Princípio de Reconstrução de Modelos}, denominado pelo descritivo acrônimo DDM (``\textit{Data Driven Models}'') (Kutz).


\section{Operadores Lineares e outros objetos Matemáticos}

    A teoria Quântica segundo a versão inventada pelos físicos W. Heisenberg, P. Dirac e formalizada pelo matemático J. von Neumann nas primeiras décadas do século XX, dá um passo adiante no conceito de Modelos Matemáticos, quando introduz a estrutura de operadores lineares em espaços de Hilbert (``matrizes infinitas'') para a representação de fenômenos físicos.

    O conceito clássico e Newtoniano de funções também foi ampliado durante a primeira metade do século XX segundo a teoria de funções generalizadas, ou distribuições, sugerido por Paul Dirac dentre outros, desenvolvido por K. O. Friedrichs, L. S. Sobolev e formalizado pelo russo I. M. Gelfand (``\textbf{Funções Generalizadas}'', ed. MIR e pelo francês Laurent Schwartz (``\textbf{Théorie des Distribuitions}'', Hermann, 1950).

    No presente texto, não abordaremos modelos matemáticos fundamentados na estrutura de operadores lineares e o conceito de função generalizada será mencionado de passagem no estudo de processos difusivos.

\section{Modelos Matemáticos Quantitativos: Conceito, Dimensionalização, Redução e Complexidade}

    Em grande maioria dos casos, mas não todos, os Aspectos (``Traços Caricaturais'') escolhidos para a descrição de um fenômeno, e que serão denominados Parâmetros Descritivos do Modelo, são representados matematicamente por intermédio de suas respectivas Medidas. Portanto, a representação matemática dos Parâmetros Descritivos é realizada por intermédio da mais importante estrutura matemática: os \textbf{Números Reais} \(\mathbb{R}\). O estudo dos fundamentos que norteiam esta representação, isto é, a Mensuração, é o tema central da Análise Dimensional.

    A \textbf{Análise Dimensional} contemporânea trata, especificamente, da maneira pela qual é possível atribuir Medidas Numéricas aos Parâmetros Descritivos que, em conjunto, determinam completamente as informações desejadas sobre o fenômeno estudado.

    A origem das ideias que levaram à Análise Dimensional pode ser estabelecida mais explicitamente nos estudos sobre Mecânica e Elasticidade iniciados por Galileu Galilei (séc. XVII) embora a enorme sombra intelectual de Arquimedes (séc. III A.C.) seja claramente perceptível nos seus argumentos. A sistematização moderna de seus conceitos básicos foi iniciada pelo físico escocês James Clerk Maxwell (séc. XIX) e, desde o século XX, se constitui em ferramenta indispensável para a Física, as Engenharias e, mais recentemente, para a aplicação da Matemática à Biologia.
    
    Apresentaremos, a seguir, alguns aspectos gerais e fundamentais da Teoria de Modelos Matemáticos Quantitativos que são intimamente ligados e dependentes dos conceitos da Análise Dimensional.

    A \textbf{Mensuração} de um Parâmetro Descritivo que compõe um Modelo Matemático somente pode ser realizada por comparação a uma unidade padrão, arbitrariamente escolhida, que assim determina os valores numéricos da medida.

    A mensuração de um Parâmetro Descritivo sempre pressupõe um procedimento experimental associado a ele, por intermédio do qual se estabelece a própria interpretação de seu significado. Esta dependência é raramente indicada na literatura, talvez por não se constituir em uma operação Matemática, o que se torna mais um motivo para ressaltá-la aqui.

    A estipulação do conjunto suficiente de Parâmetros Descritivos para a representação de um Modelo e a escolha de suas respectivas unidades é uma etapa inicial e indispensável na construção de um Modelo. Denominaremos esta etapa de Dimensionalização, que se fundamenta completamente no conhecimento experimental especifico sobre o fenômeno. A inclusão ou descarte de itens no conjunto de Observações de um Modelo e, consequentemente, de seus Parâmetros Descritivos, é um processo de tentativa e erro que não visa uma solução final ótima, mas apenas uma maior ou menor conveniência a depender dos objetivos do estudo. Este fato faz da Análise Dimensional um tópico essencial para entender as interfaces entre a Matemática e as suas diversas aplicações na forma de Modelos Quantitativos. Também, torna-se imediatamente claro, mais uma vez, que a Análise Dimensional \textbf{não} é uma disciplina da Matemática (dita ``\textit{pura}''), já que não se esgota dentro dela, mas sim, da \textit{Matemática} (dita) \textit{Aplicada} (ou, ``\textit{Aplicanda}'') especifica de alguma interface da Matemática com o seu ``exterior''.

    A Análise Dimensional trata de assuntos que à primeira vista parecem óbvios por serem corriqueiros, mas progressivamente ela nos surpreende pelos resultados sutis e profundos que são decorrentes da formalização de princípios simples que passam desapercebidos por quem se acostuma a operar com eles apenas automaticamente.

    Portanto, o \textbf{Primeiro Ingrediente} na construção de um Modelo Matemático é o conjunto \(\{P_1, \ldots, P_k\}\) de Parâmetros Descritivos, que deve ser inicialmente estipulado, assim como as suas interpretações extra-matemáticas, e as unidades que permitirão representá-los numericamente. Esta etapa inicial é apenas descritiva do fenômeno; resta tratar agora da segunda etapa, que é Preditiva.

    O \textbf{Segundo Ingrediente} na construção de um Modelo Quantitativo consiste em estabelecer um conjunto finito de \textbf{Funções Matemáticas} de \(k\) variáveis reais e valores reais, \(\varphi_j(\zeta_1, \ldots, \zeta_k),\ 1 \le j \le m,\ (\varphi_j: U_j \subset \mathbb{R}^k \to \mathbb{R})\) que determinarão uma interdependência entre as possíveis medidas dos Parâmetros Descritivos na forma de um sistema de equações implícitas: \(\varphi_j(p_1, \ldots, p_k) = 0,\ 1 \le j \le m\),

    Se alguns dentre os Parâmetros Descritivos apresentam especial interesse, as equações implícitas podem assumir uma forma explícita, que é mais frequente. Por exemplo, no caso \(k > 1\), o Modelo implícito da forma \(\varphi_j(p_1, \ldots, p_k) = 0\) pode, eventualmente, ser re-escrito em uma forma explícita: \(x = \psi_j(t, p_3, \ldots, p_k) = 0\), onde \(p_1 = x\) e \(p_2 = t\). Neste caso, \(p_3, \ldots, p_k\), são denominados \textbf{Parâmetros Constitutivos do Modelo}, enquanto \(x\) é o \textbf{Parâmetro} (ou, variável) \textbf{Dependente}, e \(t\) \textbf{Parâmetro} (ou,variável) \textbf{Independente}.

    A determinação das Funções Matemáticas \(\{\varphi_j\}_{1 \le j \le m}\) é a parte essencialmente matemática no processo de construção de um Modelo Quantitativo.

    Como já vimos, as expressões numéricas \(p_k\) dos respectivos Parâmetros Descritivos \(P_k\) dependem da \textbf{escolha} de unidades, que é um processo completamente arbitrário e não matemático.

    Assim, é natural questionar se, para diferentes sistemas de unidades, as condições funcionais impostas ao conjunto de medidas \(\{p_k\}\) se modificam, ou seja, se as Funções Matemáticas que definem o Modelo são vulneráveis a esta arbitrariedade. A resposta a esta questão não é demonstrável mas é tacitamente respondida com a Hipótese Fundamental na Teoria de Modelos Matemáticos Quantitativos (que é assumida tacitamente e quase nunca mencionada) a ser formulada da seguinte maneira:
    \begin{quotation}
    As \textbf{Funções Matemáticas} \(\varphi_j\) que descrevem o Modelo Matemático na forma de equações \(\varphi_j(p_1, \ldots, p_k) = 0,\ 1 \le j \le m\), são invariantes com relação ao sistema de unidades empregado para a mensuração dos parâmetros descritivos do Modelo.
    \end{quotation}

    O conjunto de valores numéricos das medidas \(p_1, \ldots, p_k\) pode variar enormemente (embora não de maneira totalmente arbitrária) com a modificação do sistema de unidades empregado para a sua mensuração. A Hipótese Fundamental elimina esta indeterminação aparente estabelecendo que as condições impostas pelas Equações permanecem iguais e independentes do sistema básico de unidades utilizado. Isto é, se \(p'_j\) são medidas dos mesmos respectivos Parâmetros Descritivos em outro sistema de unidades, então, necessariamente devemos ter: \(\varphi_j(p'_1, \ldots, p'_k) = 0,\ 1 \le j \le m\).

    A Hipótese Fundamental não é matematicamente demonstrável, pois não está totalmente inserida na Matemática, mas se apoia em argumentos e evidências exteriores à ela. Alguma reflexão sobre este assunto nos leva a crer, em geral, que uma dependência destas funções com respeito às unidades seria algo mais estranho do que a própria Hipótese; simplesmente significaria que o conjunto de parâmetros descritivos do fenômeno não é completo e algo dependente do sistema de unidades está implícito e ``escondido'' na definição da função. O termo ``Função Matemática'' tem por objetivo exatamente enfatizar que estas funções são puramente numéricas e não tem conexão alguma com o sistema de unidades.

    Tal como ocorre amplamente em Matemática Aplicada a única maneira de verificar a Hipótese Fundamental é consequencialista (ou seja, \textit{a posteriori}, constatando-se a sua utilidade) e não pela sua justificação lógica dedutível de outras hipóteses. Em ouros termos, esta é uma aplicação típica do princípio em que ``Os Fins Justificam os Meios'', desprezado tanto em Matemática Pura (onde os ``fins''-teoremas- é que são justificados pelos meios, i.e., axiomas e deduções-) quanto na Moral (ref. Stanford Encycl. Philosophy-online-: ``Consequentialism''). A própria Teoria de Modelos Matemáticos (assim como a Física) é completamente consequencialista em algum nível ou, parafraseando (e acrescentando) o estatístico George Box: ``\textit{All Mathematical Models are wrong, however, some are useful but the only way to know which when and how are useful is by using them, which seems totally tautological. However, it is not!}''.

    A determinação das Funções Matemáticas que representam um Modelo Quantitativo é uma etapa posterior à sua Dimensionalização e, em geral, realizada com a utilização da Metodologia Newtoniana em que estas Funções são obtidas como soluções de Equações Funcionais, (geralmente Diferenciais ou Integrais) ``deduzidas'' a partir de hipóteses sobre o fenômeno. Por esta razão, um Modelo ``Newtoniano'' é sempre identificado mais com as Equações Diferenciais que determinam as Funções definidoras do Modelo, do que com a própria Função.

    A Metodologia Newtoniana tem dominado a Matemática Aplicada em quase todas as suas manifestações desde que Isaac Newton inventou o Cálculo Diferencial e Integral, no século XVII, para este mesmo objetivo.

    É interessante ressaltar, todavia, a existência já mencionada, de outros métodos para a construção de Modelos Matemáticos, denominados DMKD (\textit{Data Mining and Knowledge Discovery}) e DMD (``\textit{Data Driven Models}'') em que as Funções Matemáticas que o definem são construídas a partir do arquivo de uma grande massa de Dados Numéricos experimentais ou observacionais sobre o fenômeno. Este tema será tratado em outro capítulo.

    A \textbf{Complexidade} de um Modelo Matemático pode ser \textit{provisoriamente} definida como sendo o número mínimo de Parâmetros necessários para descrevê-lo, o que se associa, de certa maneira, à um conceito de ``complexidade'' das Funções Matemáticas que deverão representá-lo. Interpretando uma função de duas variáveis \(\varphi(x, y)\) como uma família infinita (e contínua) de funções de uma variável \((x)\), \(\varphi_y(x)\), é imediatamente claro que, pelo menos em um sentido intuitivo, a ``complexidade'' de funções de duas variáveis é \textbf{muito maior} do que a ``complexidade'' de funções de apenas uma variável, \(\psi(x)\). E, progressivamente, mantendo outras características iguais, quanto maior o número de variáveis, maior a ordem de complexidade de uma classe de funções.

    Entretanto, a precariedade desta definição (provisória) é imediatamente aparente, a menos que seja demonstrada a unicidade do número de Parâmetros Descritivos e das Funções Matemáticas que descrevem o Modelo pois, caso contrário, a complexidade dependeria da representação e não seria uma propriedade intrínseca do Modelo Matemático.

    A Análise Dimensional nos mostrará que é possível determinar um número ``\textbf{míni\-mo efetivo}'' de Parâmetros Descritivos, \(\pi_1, \ldots, \pi_s,\ s \le k\) (denominados Parâmetros Adimensionais) definidos por expressões algébricas elementares da forma \(\pi_l = p_{1}^{n_1} \cdots p_{k}^{n_k}\) (onde \(n_s \in \mathbb{Z}\)) de tal maneira que as Funções Matemáticas que representam o Modelo \(\varphi_j(p_1, \ldots, p_k) = \Phi_j(\pi_1, \ldots, \pi_k)\), podem ser expressas em termos de Funções Matemáticas, \(\Phi_j(\eta_1, \ldots, \eta_s)\), com \(s \le k\) (eventualmente muito menor) de variáveis. Isto significa que a complexidade do Modelo é efetivamente dependente do número de \textbf{Parâmetros Adimensionais} (\(s\)) que é bem determinado e frequentemente menor do que o número \(k\) de Parâmetros dimensionais. A descrição Adimensional é, portanto, considerada \textbf{Reduzida} no sentido de que a construção do Modelo, nesta forma, exige a obtenção de funções com um número de variáveis efetivamente menor. Este resultado, que será obtido com argumentações simples e manipulações elementares, tem um enorme impacto para a compreensão do conceito de Modelo Matemático e para o seu tratamento analítico e numérico adequado.

    A obtenção do Modelo Matemático Adimensional (Reduzido) faz uso de duas hipóteses:
    \begin{enumerate}
    \item Liberdade que a escolha de unidades padrões nos confere e;
    \item da Invariância da representação funcional com relação às unidades padrões dos parâmetros descritivos.
    \end{enumerate}

    Embora a escolha inicial de unidades padrões seja completamente arbitrária (e todas são, em princípio, lícitas), uma vez discriminados os Parâmetros Descritivos do Modelo, distinguimos facilmente que o próprio Modelo explicita Unidades Intrínsecas para todas as suas medidas. A utilização destas unidades intrínsecas (também denominadas, Escalas Intrínsecas do Modelo) é a chave do processo de Adimensionalização e da obtenção de Modelos Reduzidos. A Redução dimensional de um Modelo Matemático tem uma relação imediata com o \{17\textordmasculine\ \} problema da lista de \(23\) problemas que David Hilbert propôs no Congresso Internacional de Matemática em Paris, \(1900\) como programa de pesquisa para a Matemática do século XX. A grosso modo, este problema pode ser enunciado da seguinte maneira:
    \begin{quotation}
    ``Dada uma função de muitas variáveis, em que situação é possível representá-la como composição de funções com um número menor de variáveis?''
    \end{quotation}
    Esta possibilidade indica vantagens óbvias como por exemplo, quando uma função originalmente determinada por uma Equação Diferencial Parcial (para funções de várias variáveis) possa ser determinada por uma, ou mais, Equações Diferenciais Ordinárias (para funções de uma variável). A solução geral deste problema foi obtida pelos matemáticos russos Andrei Kolmogorov e seu aluno Vladimir Arnold em 1954, mas a sua implementação computacional é assunto em aberto [ref. A. N. Kolmogorov-Selected Works,...]). É importante ressaltar que este problema abstrato foi sugerido a Hilbert pelo trabalhos de um engenheiro, Maurice d’Ocagne que desenvolveu um método gráfico e prático, chamado Nomografia, para a representação de Modelos Matemáticos. Esta técnica foi amplamente utilizada até o surgimento recente dos computadores digitais.

    A garantia da invariância da Função Matemática que representa o Modelo com respeito a mudanças de unidades nos permite modificar, e eventualmente simplificar o problema sem perda de soluções, aplicando a ele um grupo especifico de Transformações, no caso, mudança linear de unidades (ref. G. Barenblatt). A generalização matemática desta estrategia originada na Análise Dimensional deu origem à teoria de Grupos de Transformações de Equações Diferenciais desenvolvida inicialmente por Sophus Lie que admite o uso de Grupos muito mais gerais de transformações que possibilitam simplificar equações diferenciais sem que suas soluções sejam modificadas. Esta classe de Métodos de Lie tem particular importância para a obtenção de soluções explicitas de Equações Diferenciais em algumas áreas da Matemática Aplicada e será tratada na seção ``Métodos Matemáticos - Grupos de Transformação e Invariancia'' deste texto (ref. L. V. Ovsiannikov - Group Transformations of Differential EquationsAcad. Press 1980, ...). A origem histórica desta ideia de Transformações de Equações se encontra nos trabalhos de Galois sobre equações polinomiais e no Algoritmo de Gauss para resolução de Sistemas de Equações Lineares.

    O esclarecimento do processo de mensuração é portanto indispensável não somente para a compreensão dos Princípios Fundamentais da construção de Modelos Matemáticos Quantitativos, mas também para o desenvolvimento de diversos Métodos da Matemática Aplicada, tais como os Métodos de Redução, Métodos Assintóticos de Múltiplas Escalas, Métodos de Invariância e etc.

    Apresentaremos a seguir as ideias e Métodos da Análise Dimensional por intermédio de alguns exemplos ilustrativos simples, mas relevantes, com o objetivo de esclarecer a exposição mais abstrata acima.

\section{Modelos geométricos elementares}

    A maneira mais didática para a exposição dos conceitos e a sistematização das técnicas da Análise Dimensional se faz por \underline{intermédio de exemplos concretos} que são, de fato, a sua origem e finalidade e, que além disso, aproveita a intuição que adquirimos ao longo do tempo com a manipulação, ainda que ingênua, destas ideias.

    Observemos, inicialmente, que qualquer Modelo Matemático quantitativo faz uso da estrutura de números reais para a representação dos valores de medidas em um sistema de \textbf{unidades} pré-estabelecidas; sem estas unidades a matematização numérica de um modelo é inviável.

    A Geometria é a origem do conceito de números reais que surgiu na Matemática grega clássica com o objetivo de ``medir'' o comprimento (isto é, estabelecer uma ordem de tamanho) de segmentos de retas e, posteriormente, para medir o comprimento de segmentos de curvas, e as extensões de áreas e volumes. Para isto, instituiu-se o conceito de unidade padrão de comprimento \(u\) que poderia ser \textbf{qualquer} segmento retilíneo. Esta unidade seria então dividida em partes iguais (por construção de régua, compasso e semelhança de triângulos) de todas as ordens \(q\) produzindo sucessivamente sub-unidades \(u_q\) que poderiam ser menores do que qualquer comprimento. O processo de mensuração do comprimento de um segmento provem da associação de um número racional \(\dfrac{p}{q}\) cujo significado seria um um segmento obtido por \textbf{justaposição consecutiva} de \(p\) cópias de sub-segmentos \(u_q\) . Os gregos acreditavam que todos os comprimentos retilíneos poderiam ser exatamente medidos com um dos segmentos representados por números racionais \(\dfrac{p}{q}\).

    A maior crise da história da Matemática (que, muitos séculos depois motivou a origem de uma de suas mais profundas teorias) eclodiu com a demonstração de que a hipotenusa de um simples triângulo retângulo com catetos unitários era incomensurável com esta unidade, ou seja, não haveria nenhum número racional que o representasse. (Em termos modernos, \(2\) é irracional!). A partir deste momento os números reais existiriam apenas como segmentos de reta; a sua representação decimal infinita somente foi instituída séculos mais tarde com a invenção do Cálculo e dos conceitos de limite.

    Observe que a unidade de área \(a\) é definida pelo quadrado com lados unitários \(u\) e as subunidades \(a_q\) por quadrados com lados dados por subdivisões \(u_q\) da unidade linear. A medida de uma área também se faz por \textbf{justaposições} (como um quebra-cabeça) de subunidades de área até que se esgote o recobrimento da figura. A quadratura do círculo de raio unitário seria equivalente à incomensurabilidade da hipotenusa e os catetos.

    Consideremos, agora, o conjunto de todos os triângulos retângulos e o problema de determinar as suas áreas. Como um triângulo retângulo é completamente determinado por um cateto e por sua hipotenusa (verifique), estes serão os parâmetros descritivos do ``fenômeno''. (A cor do triângulo, sua localização no Universo e o material de que é feito são informações descartadas). Portanto, o Modelo Matemático para este ``fenômeno geométrico'' pode ser funcionalmente representado na forma \(A = f(c, h)\) em que \(A\) é o número que representa a área do retângulo, \(c\) o número que representa o comprimento do cateto e \(h\) o número que representa o comprimento da hipotenusa, todos eles medidos segundo uma mesma unidade qualquer, digamos \(u\), que, em princípio, nada tem a ver com este problema mas deve ser escolhida de princípio (são válidas, por exemplo a escolha do raio do átomo de hidrogênio, um Angstrom, ou um ano-luz, que é a distância que a luz percorre durante um ano de viagem no vácuo, a polegada, que é o comprimento da articulação de um vaidoso rei inglês ou o metro representado por uma barra fortemente guardada em um museu distante de Paris que pouca gente já visitou).

    Sob a hipótese de que esta função \(f\) não varia com a mudança de unidade de comprimento linear escolhida para mensurar o cateto, a hipotenusa e, consequentemente, a área do triângulo (ou seja, não depende dela), podemos, então, espertamente, escolher uma unidade padrão intrínseca do problema, por exemplo, o próprio cateto. Com esta unidade, o número que representa o comprimento do cateto será \(1\), e o número que representa o comprimento da hipotenusa será \(\dfrac{h}{c}\), enquanto que o número que representa a media da área será \(\dfrac{A}{c^2}\) de onde, teremos, necessariamente: \(\dfrac{A}{c^2} = f\left(1,\dfrac{h}{c}\right) = g\left(\dfrac{h}{c}\right)\) ou, \(A = c^2g\left(\dfrac{h}{c}\right)\). Portanto, para descrevermos completamente este Modelo Matemático, em vez de uma função matemática de duas variáveis, \(f(x, y)\), passamos a necessitar de uma função \(g(z)\) de apenas uma variável, o que é um ganho analítico considerável. Isto significa que o problema tem complexidade \(1\).

    Consideremos, agora, uma perpendicular à hipotenusa passando pelo vértice dos dois catetos o que determina dois novos triângulos retângulos. Cada um destes compartilha um ângulo agudo com o triângulo maior e, portanto ambos são semelhantes a ele e, consequentemente todos os três semelhantes entre si. Isto significa que a razão entre os comprimentos da hipotenusa e o cateto tem o mesmo valor para todos eles, ou seja, \(g\left(\dfrac{h}{c}\right)\) é o mesmo para os três triângulos. Identificando as hipotenusas dos três triângulos, utilizando a respectiva fórmula e cancelando o valor comum de \(g\left(\dfrac{h}{c}\right)\), obtemos o Teorema de Pitágoras: \(h^2 = c^2+b^2\) (Verifique!).

    Consideremos, agora, o \textbf{conjunto de todos os discos planos}. Fundamentados em nossa experiência geométrica do espaço, admitiremos que qualquer um deles é totalmente determinado pelo seu raio. Esta é uma hipótese Física e não matemática que, como veremos, pode ser disputada em contextos distintos. Dada uma unidade qualquer de comprimento linear \(u\), podemos então medir o comprimento \(R\) do raio (isto é o número que representa a medida do raio na unidade \(u\)) e da circunferência \(C\)(...) e, certamente, teríamos uma função \(C = f(R)\), em que, por hipótese, a função \(f\) \textbf{não depende} da unidade escolhida. Portanto, tomando a unidade intrínseca de comprimento linear para este ``fenômeno'' como sendo o próprio raio (o famoso círculo de raio unitário!) teríamos: \(\dfrac{C}{R} = f(1)\) ou seja, a razão entre os comprimentos de qualquer circunferência e o seu raio é sempre constante independente da unidade escolhida para medi-los. Este número foi chamado por Euler de \(2\pi\) e é dito ``adimensional'' (ou ``puro'') exatamente por ser invariante com relação à unidade de comprimento utilizada nas medidas do disco.

    Voltando à determinação completa do disco unicamente pelo seu raio, observemos que a hipótese de ``planitude'' do espaço em que ele se encontra é essencial para este argumento, mas não necessariamente correto em geral. Por exemplo, o disco pode estar apoiado sobre a superfície terrestre! A razão entre comprimento da circunferência e o raio para círculos sobre superfícies ligeiramente encurvadas \textbf{dependerá também} desta curvatura e teremos sempre \(\dfrac{C}{R} \le 2\pi\) com a igualdade sendo verificada somente no caso dito plano. Na verdade, este pode ser o próprio critério de ``planitude'' do espaço (Verifique!).



\subsection{O pêndulo de galileu}

    \begin{quote}
    ``Mathematical Models cannot be replicas of Nature, but are powerful tools that help us understand its phenomena.''
    
    \rightline{Peter Kareiva (Entomologista).}
    \end{quote}

    Um dos exemplos mais simples e didáticos para ilustrar os conceitos de Modelo Matemático e Análise Dimensional deve-se a Galileu e apresenta um dos fenômenos mais fundamentais da Mecânica, tanto sob o ponto de vista histórico como conceitual: o movimento oscilatório de um pêndulo clássico.(G. Baker - J. Blackburn - The Pendulum, Oxford UP 2008). Este pêndulo é um dispositivo simples que consiste em uma partícula (cuja massa é considerada pontual e concentrada), sobre a qual é exercida uma força vertical constante, a gravidade, e, por outro lado, é sustentada por uma haste não extensível, dotada de massa desprezível, presa por sua extremidade superior a um vínculo fixo. Supõe-se ainda que este vínculo não oferece atrito de roçamento e o movimento plano do pêndulo não sofra atrito viscoso do ar. A oscilação se inicia quando a massa em repouso é liberada a partir de uma posição deslocada de seu ponto de equilíbrio. O experimento, como se observa, resulta em uma oscilação periódica em um plano vertical entre duas posições extremas e simétricas com respeito ao ponto de equilíbrio.

    Observe que estamos tratando de um Modelo extremamente simplificado do fenômeno mas que guarda ainda uma considerável semelhança com a questão original.

    Digamos que o período de oscilação do pêndulo seja a observação de principal interesse do experimento. A construção do Modelo Matemático para este fenômeno será fundamentada na seguinte hipótese:
    \begin{quotation}
    ``O período \(T\) de oscilação do pêndulo é a observação de interesse que se supõe completamente determinado pelo seguinte conjunto de Parâmetros Descritivos (ditos constitutivos) do dispositivo: massa (\(m\)), força gravitacional (aceleração da gravidade, \(g\)), comprimento da haste (\(l\)), amplitude da oscilação, (\(A\))''.
    \end{quotation}

    Neste Modelo, assumimos deliberadamente que o Parâmetro Descritivo ``Período de Oscilação'' depende unicamente dos outros Parâmetros Descritivos, e esta hipótese é baseada inteiramente em argumentos físicos e experimentais. A hipótese assumida é uma ``hipótese de trabalho'' que será mais tarde justificada \textit{a posteriori} ou, não, pela coerência entres seus resultados e as observações (Isto é, ``Os Fins [resultados] justificarão, ou não, os meios [hipóteses]'').

    Esta abordagem enfatiza apenas a determinação explícita do período de oscilação do pêndulo como dependente dos outros Parâmetros Descritivos (Constitutivos) e, não a descrição completa do seu movimento temporal que será tema de outro modelo dinâmico a ser tratado mais adiante. A escolha do período como ``Variável Dependente'' (ou ``incógnita'') em função dos outros Parâmetros Descritivos (``conhecidos'') é um mero ponto de vista e traz um vício de linguagem, pois não provem de qualquer particularidade intrínseca do fenômeno. De qualquer maneira, seguindo a tradição, o Modelo será procurado explicitamente na forma de uma Função Matemática \(\varphi(\zeta_1, \zeta_2, \zeta_3, \zeta_4)\) de quatro variáveis, tal que \(T(m, l, g, A)\).

    A definição dos conceitos elementares de Mecânica (posição, velocidade, aceleração, massa, força) nos levam a utilizar três tipos de medidas ``independentes'' , ou, ``básicas'', necessárias para a formulação de Modelos matemáticos nesta disciplina: Tempo, Comprimento e Massa. A partir de suas unidades básicas, as próprias definições dos outros conceitos levarão às suas respectivas unidades. Por exemplo, a unidade (\textbf{composta}) de velocidade será determinada pelo movimento uniforme ao longo de uma unidade de comprimento durante uma unidade de tempo. Com isto, evita-se a definição de novas unidades, da mesma forma como utilizamos a unidade de comprimento para definir uma unidade (composta) de área construindo um quadrado de lados unitários como padrão. Esta figura é mais ``prática'' em geral para a mensuração de áreas de figuras poliédricas do que, por exemplo o disco com raio unitário. Observe-se que a definição de uma unidade sempre requer a explicitação de um processo de mensuração (que é um procedimento concreto e não matemático) entre o objeto a ser medido e a unidade estipulada. Quando duas medidas não não são comparáveis (isto é, não é possível medir uma em termos da outra) dizemos que tem \textbf{dimensões independentes}, como é o caso das medidas de comprimento e tempo. Com estas medidas padrões escolhidas a velocidade é uma medida com unidade composta. Entretanto, poderíamos utilizar o comprimento e a velocidade como dimensões básicas; neste caso o tempo seria uma medida composta.

    \begin{exercise}
    Determine o processo de mensuração de (a) velocidade e de (b) tempo com unidades padrão de comprimento e velocidade.
    \end{exercise}
    
    \textbf{Solução}:
    {\color{orange}
    Seja:
    \begin{itemize}
    \item \(t\) a medida de tempo;
    \item \(v\) a taxa de variação de um corpo obtida ao deslocar \(x\) unidades de comprimento, um corpo durante um intervalo de tempo \(t\);
    \end{itemize}

    Tomemos como unidades padrão de comprimento e velocidade as do conjunto \(\{L, V\}\). Sendo assim, temos:
    
    \begin{description}
    \item (a) \([v] = [x t^{-1}] = V\) 
    \item (b) \([t] = [x v^{-1}] = [x] [v^{-1}] = L V^{-1}\)
    \end{description}
    }

    A próxima seção discorrerá com maiores detalhes sobre os conceitos de unidade, dimensão e medida.


\section{Unidades, dimensões e medidas}

\subsection{A medida do tempo: O Pulso do Tempo ou o Tempo do Pulso}

\begin{quote}
``Eu estava convicto de que sabia perfeitamente o que era o ’tempo’, até quando alguém me perguntou seriamente sobre o seu significado'.

\rightline{Santo Agostinho (Teólogo)}

``O tempo é a maneira pela qual a Natureza se organiza e impede que tudo aconteça simulaneamente''.

\rightline{\tiny Filosofia Medieval popularizada por Woody Allen)}
\end{quote}

A matematização do fenômeno mecânico representado pela oscilação de um pêndulo se inicia com a escolha de uma maneira de representar o tempo de oscilação, isto é, o tempo utilizado no seu percurso entre suas posições de maior amplitude.

Como o modelo mental (ingênuo) que a espécie humana desenvolveu sobre o conceito de tempo é representável por um contínuo, linear e ordenado, então, torna-se quase imediato que o objeto matemático natural para a sua representação abstrata é a Reta Real, ou seja, os Números Reais. A representação discreta do tempo tem sido argumentado ultimamente por alguns físicos como sendo a mais fidedigna, e por questões meramente práticas e computacionais a representação de tempo pelos números inteiros é um artificio frequente.

Para concretizar a representação contínua é necessário dispôr de uma \textbf{unidade de tempo}, que deve ser \textit{acessível} (ou seja, ``\textit{regularmente repetitivo}''), para que se possa utilizá-lo comparativamente e, além disso, experimentalmente fracionável, para que possamos utilizá-lo na medida de qualquer período. (Nenhum destes requisitos é óbvio e isento de dificuldades conceituais ou práticas. Por exemplo, como é que se pode saber quando um padrão é ``regularmente repetitivo'' , sem aceleração, se não existe um padrão absoluto para confrontá-lo? E, como fracionar unidades de tempo, como o período de oscilação do átomo de césio-133?).

Qualquer unidade de tempo deve ser necessariamente caracterizada por um processo físico; não existe unidade absoluta ou abstrata de tempo obtida como resultado de uma pura construção mental! A unidade de tempo mais familiar é, claro, o ``dia'', representada pelo ``tempo'' necessário para que a Terra realize uma volta completa em torno de seu eixo de rotação mas, diversas outras unidades podem ser igualmente definidas.

A sub-unidade de tempo 'segundo' é o período de tempo correspondente a \(1/86.400\) do dia solar médio ou, \(9.192.621.770\) períodos da transição do átomo de césio-133. Galileu, que foi um dos iniciadores da ciência moderna, ao estudar o movimento do enorme pêndulo preso ao teto da catedral de Pisa, utilizou o seu próprio pulso como padrão de tempo. É claro que o pulso de Galileu, (logo ele, que passou por tantos sobressaltos e chegou tão perto da fogueira da Inquisição), certamente não deve ter sido o mais estável (uniforme) dos padrões, mas era o que ele tinha naquela ocasião! Além disso, é difícil saber como Galileu poderia ``fracionar'' a unidade de tempo (pulso) e como medir múltiplos muito grandes desta unidade sem parar para dormir?

Os relógios mecânicos construídos posteriormente a Galileu, por ironia, fazem uso exatamente da periodicidade oscilatória do pêndulo, e surgiram como consequência de seus trabalhos e os de Christiaan Huygens (Haia, 14 de abril de 1629 — Haia, 8 de julho de 1695). A história ``moderna'' do desenvolvimento de métodos para a medida do tempo teve seu início com as navegações do século XV que exigiam um conhecimento preciso e consistente da medida de ``tempo'' nos diversos navios de uma frota em alto mar que permaneciam incomunicáveis entre si e sem contato com um ``relógio'' central em terra firme. O estudo aprofundado do conceito de medida do tempo, acabou por levar o matemático Henri Poincaré (1854-1912) e o físico Albert Einstein (1879-1957) a desenvolverem a Teoria de Relatividade no princípio do século XX, o que, convenhamos, não foi pouca coisa! (Sobre este assunto, e muito mais, que ainda é tema de pesquisas fundamentais, um bom começo pode ser a referência: Peter L. Galison: - ``Einstein’s clocks, Poincarés maps: Empires of Time”, New York: W. W. Norton, 2003).

A primeira observação sobre o procedimento de mensuração do tempo se refere à liberdade (experimental) de escolha da unidade, que pode ser qualquer uma desde que satisfaça às condições de acessibilidade (repetitividade) e fracionamento. É importante observar que a unidade de tempo escolhida não é um número, mas um conceito totalmente dependente de sua definição experimental e, portanto, exterior à Matemática. A representação quantitativa (isto é, numérica) desta observação, surgirá somente quando for possível comparar, experimentalmente, qualquer período de tempo com a unidade padrão e suas frações; nisto consiste em suma a Mensuração do tempo e, consequentemente, da sua matematização.

Designando uma \textbf{unidade de tempo} pela letra \(T\), a \textbf{medida de um período de tempo} nesta unidade (\(T\)) será representada por expressões algébricas do tipo \(xT , x \in \mathbb{R}\). Se, por exemplo, a unidade de tempo escolhida for o segundo, representada pelo símbolo \(T = s\), então \(3,141516 s\) representa três segundos inteiros, mais a fração \(141516 \cdot 10^{-6} s\) desta unidade.

Uma vez escolhida uma unidade de tempo \(T_0\), qualquer outra unidade de tempo \(T_1\) pode ser definida como uma medida em relação à primeira, \(T_1 = a_1T_0\), e vice-versa, \(T_0 = \dfrac{1}{a_1} T_1\), onde \(a_1\) é um número real positivo. Portanto, uma mesma medida pode ser representada, em unidades diferentes da seguinte maneira:
\(x_0T_0 = x_1T_1\), onde a ``conversão de unidades'' é realizada por uma simples e natural operação algébrica:
\[x_0T_0 = x_1(a_1T_0)=(x_1a_1)T_0,\]
ou seja, \(x_0 = a_1x_1 \mbox{ ou } x_1 = \dfrac{1}{a_1}x_0.\)

Observe que o sinal de igualdade na expressão ``\(x_0T_0 = x_1T_1\)'' \textbf{não é}, em princípio, uma igualdade matemática mas se refere ao fato de que ambos os termos correspondem a um mesmo período de tempo. Entretanto, como vimos acima, podemos tratar esta igualdade no sentido algébrico que facilitará os cálculos da modificação de medidas com respeito à mudança de unidades.



\subsection{A medida de comprimento: A crise histórica da Unidade de Comprimento}

    A unidade de comprimento que parece ser um conceito tão simples e corriqueiro, também apresenta dificuldades surpreendentes se analisamos melhor a sua mensuração como já vimos nos exemplos geométricos tratados anteriormente. A origem do conceito de medida de comprimento confunde-se com a da Geometria Euclideana.

    A Geometria Euclideana deve a sua origem a um Modelo Matemático cujo objetivo é representar a nossa concepção mental mais imediata do espaço ecológico ambiente (isto é, o ``espaço como percepção'') e, por isto mesmo, depende intimamente da cognição espacial da espécie humana. Como a concepção mental do espaço até o século XIX era considerada imutável e absoluta, assim também era considerada a Teoria matemática que a representava. A representação mental que os gregos clássicos registravam do espaço ambiente baseava-se nos conceitos de Ponto e de Reta que, segundo Euclides, tinha a sua manifestação concreta mais exata nas trajetórias de raios luminosos. Não por acaso, os \textbf{Elementos} de Euclides contem a Ótica como um de seus capítulos. E, para a representação quantitativa destas ideias, era necessário estabelecer o conceito de medida de comprimento de um segmento de reta e, portanto, de unidade de comprimento (Geo Terra, Metros-Medida).

    Apesar da sua origem nitidamente ecológica (isto é, relativa à percepção cognitiva do espaço pelo homo sapiens), a Geometria Euclideana, vem sendo apresentada em textos ao longo de séculos, apenas como uma Teoria Matemática axiomática como se ela precedesse à noção de espaço e, suas relações com o espaço físico são meramente citadas como aplicações inevitáveis, o que inverte completamente a história do assunto e a sequencia mais natural de seu aprendizado! Raramente, ou quase nunca, a estreita dependência da Teoria Matemática e a percepção cognitiva do espaço é analisada nestes textos. Como veremos em outro tópico, esta antiquíssima negligência foi uma das razões para que as chamadas Geometrias não-Euclideanas levassem tanto tempo para serem desenvolvidas. Quando entendemos que o conceito euclideano de reta e suas propriedades axiomáticas são de fato, histórica e psicologicamente, baseadas na percepção experimental da trajetória de um raio de luz, imediatamente verificamos que outras Geometrias são igualmente possíveis, e até mesmo necessárias, e não meros exercícios de abstração. Esta observação foi feita por Poincaré no final do século XIX e utilizada por ele mesmo para desenvolver um modelo de Geometria Não-Euclideana (Hiperbólica) de que trataremos em outro capítulo.

    Dentre vários resultados notáveis da Geometria Euclideana, um deles é estreitamente ligado ao conceito de medida: A conhecida expressão algébrica que relaciona as medidas dos comprimentos da hipotenusa e os comprimentos dos seus catetos em um triângulo retângulo (Teorema de Pitágoras). A impossibilidade de ``medir'' (isto é, representar univocamente) o comprimento da hipotenusa de um triângulo retângulo com catetos de medida unitária por intermédio de números racionais (que os gregos supunham serem os únicos números ``reais'') foi facilmente demonstrada por eles mesmos, e desencadeou uma das maiores crises da Matemática que atravessou séculos e somente foi resolvida no final do século XIX! O impasse poderia ser descrito como ``\textit{A incapacidade da estrutura de números (racionais) de representar a Medida de alguns segmentos de reta do Modelo Euclideano}'', de onde vem o termo ``\textbf{incomensurável}'' utilizado em textos antigos para designar o comprimento da referida hipotenusa.

    Um outro notável resultado que todos os textos de Geometria Elementar supõem ser da Teoria Matemática Euclideana, mas que, na verdade, é do Modelo Matemático Euclideano, pode ser enunciado da seguinte maneira: ``\textit{O valor numérico para a razão entre as medidas da circunferência e do diâmetro de um círculo plano é invariante, ou seja, independe da unidade de comprimento e do círculo considerado}''. Muito mais tarde (séc. XVIII) mostrou-se que a circunferência era também ``incomensurável'' com a medida do diâmetro, ou seja, não representável por um número racional para um diâmetro unitário. Os matemáticos da antiguidade, que, suspeitavam, mas não conheciam este resultado, calcularam esta razão invariante com vários graus de aproximação, que mais tarde foi designado por Euler com o símbolo ``\(\pi\)''.

    A invariância da relação entre estas duas medidas foi detectada experimentalmente pelos babilônicos e, certamente, por todas as civilizações que conheciam a roda. Em Geometria Euclideana o seu enunciado surge como um ``teorema'' representado pela fórmula \(C = \pi d\).

    Veremos logo abaixo que este resultado é decorrente da Análise Dimensional do Modelo Matemático Euclideano construído para a representação do espaço plano. O que raramente se ressalta é o fato de que este ``teorema'' depende completamente da ``planitude'' do disco no qual se mede o círculo. É fácil ver intuitivamente que em superfícies não planas (isto é, com curvatura não nula, como uma esfera) a razão medida da circunferência/diâmetro do círculo é sempre menor e atinge seu máximo na superfície plana. Mais uma vez, se a definição geométrica de \(\pi\) tivesse sido estudada segundo a sua interpretação ``ecológica'', isto é, como resultante de um Modelo e não de propriedades intrínsecas do espaço, talvez o conceito de curvatura como medida da discrepância local da ``planitude'' tivesse entrado para a Matemática antes de Gauss. A demonstração que o número \(\pi\) também é irracional, teve que esperar o desenvolvimento da Análise, no século XVIII, e foi obtida, não de forma trivial, por J. H. Lambert (1728-1777). Esta ``ignorância'' poupou os gregos do escândalo que o número \(\pi\) acrescentaria ao assombro da irracionalidade do \(\sqrt{2}\), embora Aristóteles também já desconfiasse desta `anomalia’ (ref. R. Remmert - ``What is \(\pi\)?'', pág. 123-153, in H.- D. Ebbinghaus \& al.-Numbers, Springer, 1991.

    A importância da percepção cognitiva do espaço ecológico e suas enormes variações dependentes da espécie em estudo e do ambiente em que se processa foi ressaltada de maneira apropriada somente na metade do século XX pelos estudos do importante biólogo Jakob von Uexküll (1864-1944). Contemporaneamente este tema é uma ampla área de pesquisa (ref. J. von Uexküll, R. Wehner, K. von Frisch, C. Gallistel, ...). Segundo von Uexküll, a Geometria ecológica de um organismo depende de seus interesses e, portanto, da forma como ele percebe cognitivamente o espaço ambiente. Cada espécie desenvolve sua ``Geometria Clássica''.

    De maneira lúdica e simples o assunto foi ilustrado por um despretensioso livro infantil escrito pelo Reverendo Edwin A. Abbott-Flatland, em 1884 e hoje considerado um texto ícone da Geometria por alguns geômetras proeminentes da atualidade.

    A percepção do conceito de reta como um objeto linearmente ordenado também nos indica a razoabilidade de que sua representação quantitativa se faça por intermédio da estrutura de Números Reais.

    Um procedimento análogo ao utilizado para a introdução da unidade de tempo se repete agora para se estabelecer a Mensuração da haste do pêndulo, inciando-se com a escolha de uma (arbitrária) Unidade de comprimento. Assumimos tacitamente (o que parece ser óbvio!) que a unidade de tempo não pode ser utilizada para a medida de comprimento, ou, como se diz, as ``dimensões'' de tempo e comprimento são independentes.

    Digamos que, genericamente, uma unidade de comprimento seja designada por \(L\); neste caso, a haste terá, respectivamente, medida \(lL\), onde \(l \in \mathbb{R}\).

    Uma vez escolhida a unidade padrão de comprimento \(L\) o deslocamento do peso do pêndulo com respeito à posição de equilíbrio pode também ser quantificado por uma medida de comprimento ao longo do círculo, que designaremos por \(AL\), utilizando a mesma unidade de comprimento para esta mensuração. Em outras palavras, a mensuração de todos os comprimentos de segmentos de reta em um Modelo deve sempre se referir a uma única (seja qual for) unidade. Observe que \(L\) é simplesmente um símbolo (não um número) que denota uma determinada unidade que deve ser experimentalmente especificada.

    O conceito de linha/curva como resultado da deformação não elástica de um segmento de reta nos leva ao conceito de medida de comprimento das mesmas por um processo limite, inaugurado por Eudoxus e aperfeiçoado por Arquimedes. Uma curva/linha que admite a possibilidade de mensuração de seu comprimento é denominada ``\textit{Retificável}'', um tema que foi objeto de extenso estudo em Matemática.

    Consideremos agora a questão de quantificação do conceito de velocidade linear de movimentação da partícula suspensa ao longo da curva circular. O conceito Físico de velocidade é definido pelo comprimento percorrido uniformemente em uma unidade de tempo. Utilizando as unidades de tempo e comprimento já introduzidas, digamos \(T\) e \(L\), definimos uma ``\textit{unidade composta de velocidade}'' a ser designada por: \(V = LT^{-1}\), que representa (simbolicamente) a velocidade unitária com que em uma unidade de tempo \(1T\) o espaço percorrido será : \(1T\dfrac{L}{T} = 1L\), ou seja, a velocidade \(yLT^{-1}\) significará o percurso do comprimento \(yL\) em um tempo de medida \(1T\).

    Analogamente, é possível, em seguida, atribuir uma unidade ao conceito Físico de Aceleração, (variação de velocidade com o tempo), que será representada na forma simbólica: \(A = (LT^{-1}T^{-1}=LT^{-2}\) (e representa a variação de uma unidade de velocidade \(LT^{-1}\) em uma unidade de tempo \(T\)). Assim, a aceleração com medida \(zLT^{-2}\) significará a variação da velocidade \(z(LT^{-1})\) em uma unidade de tempo \(T\). Assim, as unidades de velocidade e de aceleração serão denominadas compostas pois dependem das unidades básicas de tempo e comprimento.

    Digamos, agora, que novas unidades de tempo e comprimento sejam consideradas: \(T_1 = aT\) e \(L_1 = bL\). Neste caso, a nova unidade (composta) de velocidade será, respectivamente, \(V_1 = L_1T_1^{-1}\) que, medida com relação à anterior é dada por:
    \[
    V_1
    = (bL)(aT^{-1})
    = \dfrac{b}{a}LT^{-1}
    = \dfrac{b}{a} V,\]
    e a nova unidade de aceleração será
    \[A_1 = L_1T_1^{-2}=(aL)(bT)^{-2}=\dfrac{b}{a^2} A.\]

    Este exemplo mostra claramente que a relação entre unidades compostas pode ser facilmente obtida por simples manipulações algébricas e dependem diretamente da definição de seu significado experimental.

\begin{exercise}
    A medida da aceleração da gravidade \(g\) (isto é, a aceleração experimentada por um corpo submetido à atração da Terra na sua superfície) tem dimensão \([g] = LT^{-2}\) e, na unidade \(A = cm/ s^2\), mede \(g = 980 A\). Utilizando a representação algébrica, obtenha esta medida nas seguintes unidades compostas de aceleração: (a) \(A_1=L_1T_1^{-2}\), onde \(L_1 = 13 cm\), \(T_1 = 10^{-5} s\) e; (b) genericamente, na unidade composta \(A_{\ast} = L_{\ast} T_{\ast}^{-2}\), onde \(L_{\ast} = \lambda\ cm\) e \(T_{\ast} = \theta\ s\).
\end{exercise}

    \textbf{Solução}:
    {\color{orange}
    De acordo com o que foi solicitado, temos:
    \[\begin{array}{rcl}
    (a)\ \ g &=& 980\ cm\ s^{-2}\ \Rightarrow \\
    \ [g]
    &=& 980\ [cm] [s^{-2}] \\
    &=& 980 L_1\ 13^{-1} (T_1\ 10^{5})^{-2} \\
    &=& 980\ 13^{-1}  10^{-10} L_1 T_1^{-2} \\
    &\approx& 7,538\ 10^{-9} A_1 \mbox{ \quad e;}
    \end{array}\]
    
    (b) genericamente:
    \[\begin{array}{rcl}
    g &=& 980\ cm s^{-2} \Rightarrow \\
    \ [g] &=& 980 [cm] [s^{-2}] \\
    &=& 980 (L_\ast \lambda^{-1}) (T_\ast^{-2} \theta^2) \\
    &=& 9,8\ 10^2\ \theta^2 \lambda^{-1} A_\ast.
    \end{array}\]
    }


\subsection{As medidas de massa e de força: Os Fundamentos da Mecânica de Newton}

    A próxima medida a ser introduzida no Modelo Matemático do pêndulo se refere ao conceito de massa cuja unidade é independente das unidades de tempo e comprimento. Na Mecânica Clássica, a massa de uma partícula é definida pela ``Segunda lei de Newton'' como o fator de proporcionalidade (inércia) entre a força aplicada a uma partícula e a aceleração decorrente desta influência.

    Se \(M\) for uma unidade de massa, \(mM\) for a medida de massa de uma partícula e a aceleração desenvolvida por ela tiver a medida \(aLT^{-2}\), então a segunda lei de Newton afirma que a medida da força \(F\) responsável por esta aceleração deve ser \(F = (mM)(aLT^{-2}) = ma MLT^{-2}\), ou seja, a unidade composta de Força neste sistema de unidades ``básicas'' \(\{M, L, T\}\) é \(MLT^{-2}\). Este símbolo mostra como a medida de uma Força varia com a variação das unidades básicas \(\{M, L, T\}\). Assim, em um novo sistema com unidades \(\{M_1 = \delta M, L_1 = \beta L, T_1 = \gamma T\}\) a mesma força que no sistema \(\{M, L, T\}\) tem medida \(ma\), isto é, \(F = maMLT^{-2}\), no novo sistema será medida por
    \[\begin{array}{rcl}
    F &=& ma \dfrac{1}{\delta} M_1 \dfrac{1}{\beta}L_1 \left(\dfrac{1}{\gamma}T_1\right)^{-2} \\[0.4cm]
    &=& \left(\dfrac{ma\gamma^2}{\delta\beta}\right) M_1L_1T^{-2}.
    \end{array}\]

    Um outro sistema de unidades básicas e independentes para a Mecânica, pode ser constituída das unidades de Comprimento, Tempo e de Força, dito sistema \(\{F, T, L\}\). Alguns sistemas de unidades estabelecem uma base constituída pelas dimensões de força, comprimento e tempo \(\{F_1, L_1, T_1\}\), o que nos levaria a definir (ainda com a segunda lei de Newton) uma unidade (agora composta) de massa \(M_1 = F_1L_1^{-1}T_1^{-2}\). Fica claro portanto, que o conceito de sistema de unidades básicas é arbitrário desde que elas sejam independentes. É claro que um sistema \(\{L_1, T_1, V_1\}\) não é nem independente e nem suficiente para descrever um Modelo Mecânico.

    A Mecânica faz uso de diversos outros conceitos como, por exemplo, Pressão (razão da força aplicada em uma superfície e sua área), Trabalho/Energia (produto de uma força e o seu deslocamento), Potência (trabalho por unidade de tempo) que são mensurados por unidades compostas. Na verdade, poderíamos definir a Mecânica como a Ciência que trata apenas de medidas em unidades compostas das unidades básicas \(M, L, T\).

    \begin{exercise}
    Determinar as unidades compostas de Pressão (\(P\)), Energia (\(E\)) e Potência (\(W\)) (a) a partir do conjunto (genérico) de unidades básicas: \(\{M, L, T\}\) e; (b) de um outro conjunto \(\{M_{1} = a\ M, L_1 = b\ L, T_1 = c\ T\}\).
    \end{exercise}

    \textbf{Solução}:
    {\color{orange}
    Tomando as unidades de Massa \(M\), de comprimento \(L\) e de tempo \(T\), temos:

    (a) quanto à \(\{M, L , T\}\):
    
    \begin{itemize}
    \item Pressão: \([P] = [F A^{-1}] = [F][A]^{-1} = (MLT^{-2}) (L^2)^{-1} = M L^{-1} T^{-2}\);
    \item Energia: \([E] = [F x] = [F][x] = (M LT^{-2}) (L) = ML^{2}T^{-2}\);
    \item Potência: \([W] = [Et^{-1}] = [E]([t])^{-1} = (ML^{2}T^{-2})(T)^{-1} = ML^{2}T^{-3}\).
    \end{itemize}
    
    (b) quanto à \(\{M_1, L_1, T_1\}\), temos:
    
    \begin{itemize}
    \item Pressão: \([P] = \dfrac{M_1}{a} \left(\dfrac{L_1}{b}\right)^{-1} \left(\dfrac{T_1}{c}\right)^{-2} = a^{-1}bc^2 M_1 L_1^{-1} T_1^{-2}\);
    \item Energia: \([E] = \dfrac{M_1}{a}\ \left(\dfrac{L_1}{b}\right)^{2} \left(\dfrac{T_1}{c}\right)^{-2} = a^{-1}b^{-2}c^2 M_1L_1^{2}T_1^{-2}\);
    \item Potência: \([W] = \dfrac{M_1}{a}\ \left(\dfrac{L_1}{b}\right)^{2}\ \left(\dfrac{T_1}{c}\right)^{-3} = a^{-1} b^{-2} c^3 M_1L_1^{2}T_1^{-3}\).
    \end{itemize}
    }
    
    \begin{exercise}
    Obtenha as unidades derivadas das unidades básicas para as seguintes medidas: Área, Volume, Pressão, Densidade de Massa, Trabalho, Potência.
    \end{exercise}
    
    \textbf{Solução}:
    {\color{orange}
     Seja:

    \begin{itemize}
    \item \(A\) a área de uma região limitada do plano;
    \item \(V\) o volume de um sólido limitado;
    \item \(\rho\) a densidade de um sólido;
    \item \(\mathcal{T}\) o trabalho que a força \(F\) exerce ao deslocar de \(x\) unidades de comprimento um corpo de massa \(m\).
    \end{itemize}
    
    Tomando como unidades padrão de massa, comprimento e tempo, as do conjunto \(\{M, L, T\}\), temos:
    
    \begin{itemize}
    \item \([A] = [x] [x] = (L) (L) = L^{2}\);
    \item \([V] = [A][x] = L^{2} (L) = L^{3}\);
    \item \([\rho] = [m/V] = [m]/[V] = M L^{-3}\) 
    \item \([\mathcal{T}] = [F\ x\ \cos(\theta)] = [F] [x] [\cos(\theta)]= (MLT^{-2})L = ML^{2}T^{-2}\).
    \end{itemize}
    
    \textbf{Observação}: As medidas de Pressão e Potência já foram obtidas no exercício anterior.
    }


\textbf{OBSERVAÇÕES}:

\begin{enumerate}
\item Estabelecido um sistemas básico de unidades, digamos, \(\{M_1, L_1, T_1\}\), todas as medidas no sistema devem ser feitas em unidades da forma \(C_1 = M_1^{\alpha} L_1^{\beta} T_1^{\gamma}\), onde \((\alpha, \beta, \gamma) \in \mathbb{Z}^3\).
\item A medida de um parâmetro obtido da soma de duas medidas somente é definida se forem referentes à mesma unidade, básica ou composta. Isto é, apenas medidas de mesmas unidades são somadas.
\item A medida de um parâmetro obtido do produto (divisão) de duas medidas referentes a unidades \(U_1\) e \(U_2\) é medido pela unidade \(U_3 = U_1 U_2\) (respectivamente, \(U_3 = U_1 U_2^{-1}\)).
\item Diz-se que a unidade \(C_1 = M^{\alpha} L^{\beta} T^{\gamma}\), ou uma medida \(z_1C_1\), tem dimensão \((\alpha, \beta, \gamma)\), ou \(M^{\alpha}, L^{\beta}, T^{\gamma}\) e denota-se o fato por (notação de Maxwell) \([C_1] = [z_1C_1] = M^{\alpha} L^{\beta} T^{\gamma}\). O conjunto de unidades compostas da forma \(\{M^{\alpha} L^{\beta} T^{\gamma}, (\alpha, \beta, \gamma) \in \mathbb{Z}^3\}\) é chamado sistema gerado pela base \(M_1, L_1, T_1\).
\item Dizemos que o número real \(z_1\) é o valor da medida \(z_1C_1\). Quando desejamos especificar a ``dimensão'' de uma maneira genérica sem que as unidades básicas, designamos os símbolos das dimensões na forma \(\{M, L, T\}\) para denotarmos apenas aquelas independentes. Assim, o conjunto de dimensões compostas da forma \(\{M^{\alpha} L^{\beta} T^{\gamma}, (\alpha, \beta, \gamma) \in \mathbb{Z}^3\}\) é chamado sistema gerado pela base de dimensões \(\{M, L, T\}\).
\item Dizemos que a base é completa para o modelo se todas as unidades para medidas necessárias na sua descrição puderem ser representadas por suas dimensões compostas na forma \(\{M_{1}^{\alpha} L_{1}^{\beta} T_{1}^{\gamma}, (\alpha, \beta, \gamma) \in \mathbb{Z}^3\}\).

Por exemplo, \([F] = MLT^{-2} = (1, 1, -2)\).
\end{enumerate}






\subsection{Transformação dos valores das medidas com mudança de unidades}

    Como as unidades básicas são arbitrárias é comum que as transformemos homoteticamente (isto é, substituído por fatores delas) mesmo que não modifiquemos as dimensões básicas, por exemplo, em vez de \(\{kg, cm, s\}\) podemos tomar \(\{g, km, h\}\) ou na verdade, qualquer tripla do tipo \(\{a~kg, b~cm, c~s\}\), onde \(a, b, c\) são números reais não nulos e positivos.

    Analisemos, agora, a forma como se transformam as unidades compostas e os valores de suas respectivas medidas quando modificamos os valores das unidades básicas, sem alterar o conjunto de dimensões.

    Tomemos, então, um novo conjunto de unidades básicas \(\{aM_1 = M_2, bL_1 = L_2, cT_1 = T_2\}\), onde \(a, b, c\) são números reais não nulos, positivos. Consideremos, agora, uma medida \(z_{1}C_{1}\), com valor \(z_{1}\) na dimensão composta \(C = \{M^{\alpha} L^{\beta} T^{\gamma}\), gerada pela base \(\{M_1, L_1, T_1\}\). Então, a mesma (!) medida \(z_{2}C_{2}\) assumirá o valor \(z_{2}\) na base \(\{M_{2},L_{2},T_{2}\}\), e devemos ter \(z_{1}C_{1} = z_{2}C_{2}\). Desta igualdade vem:
    \[\begin{array}{rcl}
    z_{1}C_{1}
    &=& z_{1} (M_{1}^{\alpha} L_{1}^{\beta} T_{1}^{\gamma}) \\[0.2cm]
    &=& z_{2} (M_{2}^{\alpha} L_{2}^{\beta} T_{2}^{\gamma}) \\[0.2cm]
    &=& z_{2} (aM_{1})^{\alpha} (bL_{1})^{\beta} (cT_{1})^{\gamma}) \\[0.2cm]
    &=& z_{2} a^{\alpha} b^{\beta} c^{\gamma} M_{1}^{\alpha} L_{1}^{\beta} T_{1}^{\gamma},
    \end{array}\]
    de onde concluímos que \(z_{1} = z_{2} a^{\alpha} b^{\beta} c^{\gamma}\). Esta é a forma de transformações do valor da medida de uma dimensão composta entre duas bases de mesmas dimensões relacionadas pelas unidades básicas na forma: \(aM_{1} = M_{2}\), \(b L_{1} = L_{2}\) e \(cT_{1} = T_{2}\).

    Na prática, conhecendo-se os princípios, o processo de mudança de unidades se reduz a uma simples aritmética/álgebra. Consideremos, por exemplo, uma força no sistema \(\{M_{1} = g, L_{1} = cm, T_{1} = s\}\) (chamado CGS em mecânica) que tem valor \(15\), ou seja, sua medida é \(15(g)^{1} (cm)^{1} (s)^{-2}, f_1 = 15\). A mesma força no sistema \(\{M_{2} = kg = 10^{3} g, L_{2} = m = 10^{2} cm, T_{2} = min = 60 s\}\) terá uma medida \(f_2 = (kg) (m) (min)^{-2}\). Assim, \(15 (10^{-3}kg)^{1} (10^{-2}m)^{1} (\frac{1}{60} min)^{-2} = 15 \cdot 36 \cdot 10^{-1} (kg)(m)(min)^{-2}\), de onde vem que \(f_2 = 0,54\).

\textbf{RESULTADO BÁSICO}:

    Em uma mudança de unidades básicas como acima, uma dimensão composta \(C\) tem a sua unidade modificada da seguinte maneira
    \[\begin{array}{rcl}
    C_{2}
    &=& (M_{2}^{\alpha} L_{2}^{\beta} T_{2}^{\gamma}) \\[0.2cm]
    &=& (aM_{1})^{\alpha} (bL_{1})^{\beta} (cT_{1})^{\gamma} \\[0.2cm]
    &=& (a)^{\alpha} (b)^{\beta} (c)^{\gamma} (M_{1}^{\alpha} L_{1}^{\beta} T_{1}^{\gamma}) \\[0.2cm]
    &=&  (a)^{\alpha} (b)^{\beta} (c)^{\gamma} C_{1},
    \end{array}\]
    ou seja, o fator \((a)^{\alpha} (b)^{\beta} (c)^{\gamma}\) ocorre, de certa forma, do lado ``oposto'' ao da transformação da medida, \(z_{1} = z_{2} (a)^{\alpha} (b)^{\beta} (c)^{\gamma}\), o que é natural, pois se uma unidade é maior o valor da mesma medida deve ser proporcionalmente menor, e vice-versa. (Pense nisto!).



\subsection{Dimensões de parâmetros}

    Consideremos agora um segundo Modelo Mecânico Clássico e também pedagogicamente útil pela sua simplicidade. O fenômeno a ser observado consiste em um sistema Mecânico formado por uma massa pontual que desliza linearmente sobre uma superfície sem atrito de roçamento, mas que sofre a ação de um atrito viscoso (como que mergulhado em líquido) proporcional em valor, e na direção oposta ao movimento. Suponhamos ainda que esta massa esteja presa a uma mola que na outra extremidade está afixada em um ponto fixo (parede) de tal forma que considerando a origem como sendo o ponto em que esta mola não tenha qualquer deformação. Assim, um deslocamento desta posição provocará uma força de restauração da mola sobre a massa proporcional e na direção contrária a este deslocamento. Se estas são todas as forças que atuam no dispositivo, podemos utilizar a Mecânica de Newton e escrever:
    \[m \dfrac{d^2x}{dt^{2}} = -c \dfrac{dx}{dt} - kx,\] onde \(x(t)\) é o deslocamento da massa a partir da posição de equilíbrio no instante \(t\).

    A equação acima determina uma condição sobre o movimento, mas não o determina completamente pois, para isto, é necessário que sejam especificadas a posição e a velocidade inicial da partícula, \(x(0) = x_0\) e \(\dfrac{dx}{dt}(0) = v_0\). (A necessidade e a suficiência da especificação destes dois parâmetros para a descrição completa da trajetória \(x(t)\) pode ser argumentada fisicamente, mas também matematicamente. Para isto, suponha que a trajetória seja descrita por uma fórmula de Taylor:
    \[x(t) = \sum_{k = 0}^{\infty} \dfrac{1}{k!} x^{(k)}(0) t^{k}.\]
    Portanto, para descrevê-la é necessário e suficiente conhecer todas as derivadas \(x^{(k)}(0)\). Mas, utilizando a equação e os dois valores iniciais, \(x^{(0)}(0) = x_0\) e \(x^{(1)}(0) = v_0\) e derivando sucessivamente a equação e, posteriormente fazendo \(t = 0\), obtemos, sucessivamente, todos os valores \(x^{(k)}(0)\), para \(k \ge 2\).)

    Com este modelo completo, observamos que a posição \(x\) da massa depende necessariamente dos seguintes parâmetros: \(x = (t, m, c, k, x_0, v_0)\).

    A dimensão de um termo em uma equação é facilmente obtida considerando-se o princípio de que igualdade e soma de valores de medidas somente são possíveis quando se referem à mesma unidade. Isto é, não se igualam e nem se somam valores de medidas de referentes a unidades distintas.

    Assim, por exemplo, a derivada \(\dfrac{dx}{dt}\), sendo o limite de uma razão entre diferenças de comprimento (portanto, um comprimento), e um período de tempo, \[\dfrac{dx}{dt} = \displaystyle\lim_{\delta \to 0} \dfrac{x(t+\delta)-x(t)}{\delta},\] é natural que a sua dimensão seja \(LT^{-1}\). Por outro lado, a segunda derivada pode ser pensada tanto como uma razão da primeira com relação ao tempo quanto a razão entre uma variação de comprimento e o quadrado de uma variação de tempo: \[\dfrac{d^2x}{dt^{2}} = \lim_{\delta \to 0} \dfrac{x(t+2\delta) - 2x(t+\delta)+x(t)}{\delta^2}\] o que nos dá a dimensão \(LT^{-2}\).
    
    Portanto,
    \[\left[m \dfrac{d^2x}{dt^{2}} \right] = MLT^{-2}\]
    que é a dimensão da força, que deve ser a mesma dimensão de todos os outros termos da equação 
    \[\begin{array}{rcl}
    %m \dfrac{d^2x}{dt^{2}} \right]
    MLT^{-2}
    &=& \left[c \dfrac{dx}{dt}\right] = [c] \left[\dfrac{dx}{dt}\right] = [k] [x].
    \end{array}\]
    Daí, tiramos
    \[[c] = MT^{-1} \mbox{  e  }\ [k] = MT^{-2}.\]




\subsection{Unidades intrínsecas e parâmetros adimensionais}

\begin{quote}
    ``My friend Johnny used to say: ’With four parameters I can fit an elephant, and with five I can make him wiggle his trunk’ ''.
    
    Enrico Fermi (Físico, dirigindo-se a um candidato a doutorado cujo Modelo tinha uma infinidade de parâmetros, e referindo-se ao matemático John von Neumann)
\end{quote}

    Observamos, no item anterior, que fazendo uso dos parâmetros originais do modelo \((m, c, k, x_0, v_0)\), podemos obter outros parâmetros com as dimensões básicas, de massa,
    \[M = [m] = \left[\dfrac{c^{2}}{k}\right],\]
    de comprimento,
    \[
    L
    = [x_{0}]
    = \left[v_{0} \left(\dfrac{m}{k}\right)^{\frac{1}{2}}\right]
    = \left[v_{0} \dfrac{c}{k}\right]
    = \left[v_{0} \dfrac{m}{c}\right]\]
    e do tempo 
    \[T
    = \left[\dfrac{c}{k}\right]
    = \left[\left(\dfrac{m}{k}\right)^{\frac{1}{2}}\right]
    = \left[\dfrac{m}{c}\right].\]

    As medidas destes parâmetros podem ser interpretadas e tomadas como \textbf{UNIDADES INTRÍNSECAS} do modelo. Assim, por exemplo, \(m_{1} = \dfrac{c^{2}}{k}\) pode ser interpretado e tomado como uma unidade de massa, e tem tudo a ver com o modelo matemático, ao contrário das unidades das dimensões básicas utilizadas inicialmente para a formulação do modelo, que são arbitrárias e podem ser totalmente inconvenientes para o modelo uma vez que não têm, em princípio, nada a ver com o fenômeno a ser tratado.

    Observe-se, por exemplo, que as unidades de comprimento poderiam ser tomadas tanto da ordem de anos luz (comprimento percorrido pela luz durante um ano!), como da ordem de angstroms, (raio de um átomo de hidrogênio). Não há qualquer impedimento teórico nestas escolhas, mas é fácil ver que unidades excessivamente grandes ou pequenas de comprimento (comparadas com as medidas do fenômeno específico estudado) acarretarão a necessidade de empregar números reais muito grandes ou muito pequenos para representar a medidas do modelo, o que é obviamente um inconveniente.

    Estes parâmetros cujas medidas dependem apenas das unidades básicas, por outro lado, são de natureza intrínseca do modelo e é natural considera-los como alternativa razoável como unidades para a descrição do modelo. Na verdade, a sua escolha como unidades básicas não tem apenas uma vantagem notacional, mas, como veremos mais adiante, os seus valores têm também um significado intrínseco no que diz respeito a aspectos qualitativos do modelo. As Unidades Intrínsecas de um modelo são também denominadas de suas ESCALAS INTRÍNSECAS.

    A utilização de unidades intrínsecas na formulação de um modelo matemático resultará no aparecimento dos chamados \textbf{parâmetros adimensionais} como, por exemplo,
    \[\epsilon = \dfrac{mk}{c^{2}} = \dfrac{m}{\dfrac{c^2}{k}}\]
    obtidos simplesmente da razão entre dois parâmetros com mesma dimensão, no caso, de massa (\(M\)). Neste caso, temos \([\epsilon] = M^{0} L^{0} T^{0}\), o que significa em particular que qualquer que seja o conjunto de unidades das dimensões básicas, o valor da medida de \(\epsilon\) não se modificará, ou seja \(\epsilon\) é invariante com as unidades.

    Esta propriedade dos parâmetros adimensionais é de grande importância pois, como já observamos, a dependência das medidas com relação às unidades arbitrárias as torna também arbitrários e sem um significado intrínseco. (O valor numérico da medida \(m\) de massa, que na unidade \(M\) vale \(m\), pode assumir qualquer valor muito grande, ou muito pequeno, \(m\lambda\) desde que tomemos uma unidade \(M_0 = \dfrac{1}{\lambda} M\)).

    Portanto, é obviamente interessante escrever um modelo matemático em termos de parâmetros e variáveis adimensionais de tal maneira que qualquer escolha ( mesmo inconveniente) de unidades básicas não afetaria o resultado final.

    Para determinarmos os parâmetros adimensionais independentes de um modelo a partir dos parâmetros originais \((m, c, k, x_{0}, v_{0})\) basta obtermos expoentes \(\alpha, \beta, \gamma, \delta, \lambda\), para os quais
    \[\begin{array}{rcl}
    & & m^{\alpha} c^{\beta} k^{\gamma} x_0^{\delta} v_0^{\lambda} \\
    &=& M^{\alpha} (MT^{-1})^{\beta} (MT^{-2})^{\gamma} (L)^{\delta} (LT^{-1})^{\lambda} \\
    &=& M^{\alpha+\beta+\gamma} L^{\delta+\lambda} T^{-\beta-2\gamma-\lambda} \\
    &=& M^{0}L^{0}T^{0},
    \end{array}\]
    ou seja, que satisfaçam o sistema de equações lineares homogêneas:
    \[\begin{array}{rcl} \alpha+\beta+\gamma &=& 0 \\ \delta+\lambda &=& 0 \\ -\beta-2\gamma-\lambda &=& 0.\end{array}\]

    Este sistema é constituído de \(3\) equações (número de dimensões da base) com \(5\) incógnitas (número de parâmetros do modelo) o que nos fornecerá duas soluções independentes. Neste exemplo, portanto, teremos exatamente dois parâmetros adimensionais independentes. Observe que qualquer múltiplo \(h(\alpha, \beta, \gamma, \delta, \lambda)\) de uma solução \((\alpha, \beta, \gamma, \delta, \lambda)\) do sistema significará meramente um outro parâmetro adimensional \((m^{\alpha} c^{\beta} k^{\gamma} x_0^{\delta} v_0^{\lambda})^{h}\) que é uma simples potencia do anterior.

    O Princípio de Similaridade seguinte nos mostra como este número de parâmetros adimensionais é uma informação importante sobre o Modelo Matemático e representa um conceito de complexidade do Modelo.


\subsection{Princípio de similaridade dimensional}

    Como vimos no parágrafo anterior, \textbf{todo modelo matemático pode ser descrito equivalentemente por parâmetros adimensionais em número igual à diferença entre o número de parâmetros dimensionais e o número de dimensões da base de unidades}. E mais, \textbf{isto pode ser obtido simplesmente pela escolha de unidades intrínsecas da base construídas com os parâmetros dimensionais do modelo}.

    Uma demonstração formal deste princípio somente é possível com a hipótese de invariância do modelo com o sistema de unidades, o que é fisicamente argumentável, mas nem sempre mais convincente do que a própria asserção do princípio. Por este motivo, tomaremos a propriedade de Similaridade como um ``Princípio'' e não como um ``Teorema'' deduzido do Princípio de Invariância. O(A) leitor(a) interessada deve consultar os excelentes livros de Lin-Segel e Barenblatt no item ``\textbf{Teorema Pi de Buckinham}''.

    Para esclarecer melhor este Princípio, analisaremos o modelo acima onde os procedimentos são claros e representam exatamente os argumentos da demonstração.

    Observe que a constituição deste modelo é representado em \(\mathbb{R}^7\), isto é, por \(7\) ``variáveis'' (reais) \((x, t, m, c, k, x_0, v_0)\). (A distinção entre variável dependente \((x)\), variável independente \((t)\) e parâmetros \(m, c, k, x_0, v_0)\) é meramente convencional e não tem base intrínseca. É comum também distinguir variáveis independentes \(t\), variáveis dependentes \(x\), parâmetros constitutivos \(m, c, k\) e parâmetros ``eventuais'' \(x_0, v_0\)).

    A formulação de um Modelo Matemático consiste essencialmente de três etapas:

    \begin{enumerate}
    \item Estabelecimento de um sistema básico de dimensões/unidades;
    \item Discriminação de todas as variáveis descritivas que determinam o Modelo e suas dimensões;
    \item Descrição da Função Matemática que determine quais os valores das variáveis descritivas são simultaneamente admissíveis (correlação funcional entre suas medidas), em geral, o que se considera como sendo o próprio modelo.
    \end{enumerate}

    As duas primeiras etapas são em geral executadas implicitamente e usualmente destaca-se apenas a terceira etapa que consiste na maior parte das vezes de equações diferenciais/integrais e condições iniciais e de fronteira. De qualquer forma, seja por que método for, a terceira etapa consiste em determinar uma função \(\Phi\) que estabelece os estados admissíveis do sistema por meio de uma equação implícita \(\Psi(x, t, m, c, k, x_0, v_0) = 0\). Esta formulação geral tem a vantagem de não distinguir um papel especial \textit{a priori} para nenhuma variável que ocorre quando ``resolvemos'' a equação implícita e escrevemos, por exemplo \(x = x(t, m, c, k, x_0, v_0)\).

    Entretanto, é importante frisar que as duas primeiras etapas envolvem hipóteses fundamentais sobre o problema a ser tratado: é nesta etapa que se decide, à priori, quais influencias serão consideradas e, portanto, o sistema de unidades básicas e os variáveis descritivas suficientes para caracterizar o sistema! Nesta etapa é possível retirar conclusões fundamentais e às vezes surpreendentemente específicas sobre a Função \(\Psi\) (ou \(\psi\)) que descreverá o Modelo Analítico. Exemplos destas circunstâncias já foram apresentadas e serão apresentados mais abaixo.

    Voltemos ao modelo mecânico. Este modelo pode ser completamente descrito por uma Função matemática que relaciona as medidas entre todas as variáveis descritivas do Modelo \(x = \psi(t, m, c, k, x_0, v_0)\), isto é, \(6\) variáveis dimensionais cujas medidas dependem das unidades básicas escolhidas.

    Como a base de unidades tem três dimensões independentes, \(\{M, L, T\}\), concluímos, pelo exposto acima, que poderemos escrever o modelo adimensional na forma \(\eta = \varphi(\tau, \epsilon, \mu)\), onde \(\eta\) será variável adimensional dependente, \(\tau\) a variável adimensional independente e \(\epsilon, \mu\) dois parâmetros adimensionais.

    Tomemos, por exemplo, como \textbf{unidades intrínsecas} de comprimento \(x_0 = L_{1}\), de tempo \(\dfrac{m}{c} = T_{1}\) e de massa \(m\). Portanto, a função incógnita (variável dependente) passará a ter seus valores medidos adimensionalmente por
    \[\eta = \dfrac{x}{L_{1}} = \dfrac{x}{x_0}\]
    e a variável (independente) tempo por
    \[\tau = \dfrac{t}{T_{1}} = \dfrac{t}{\dfrac{m}{c}} = \dfrac{ct}{m}.\]
    
    Para escrevermos a equação nestas novas variáveis basta reescrevermos o problema original da seguinte maneira autoexplicativa e conceitualmente simples, desde que observada com cuidado: (não há necessidade e nem é recomendável, usar o teorema de derivação composta - regra da cadeia)
    
    {\tiny
    \[\begin{array}{rcl}
    m \dfrac{L_{1}d^2\left(\dfrac{x}{L_{1}}\right)}{(T_{1})^{2}d\left(\dfrac{t}{T_{1}}\right)^{2}} +
    c \dfrac{L_{1}d\left(\dfrac{x}{L_{1}}\right)}{T_{1}d\left(\dfrac{t}{T_{1}}\right)}
    +kL_{1} \left(\dfrac{x}{L_{1}} \right) &=& 0, \\
    L_{1} \left(\dfrac{x(0)}{L_{1}}\right) &=& x_{0} \\
    \dfrac{L_{1}d\left(\dfrac{x}{L_{1}}\right)}{T_{1}d\left(\dfrac{t}{T_{1}}\right)}\Bigg|_{t=0} &=& v_{0}
    \end{array}\]}

    Recolhendo as novas variáveis e fazendo-se as simplificações óbvias, temos:
    \[\begin{array}{rcl}
    \dfrac{d^2\eta}{d\tau^2} + \dfrac{d\eta}{d\tau} + \epsilon \eta &=& 0 \\
    \eta(0) &=& 1 \\
    \dfrac{d\eta}{d\tau}(0) &=& \mu
    \end{array}\]
    onde, \(\epsilon = \left(\dfrac{km}{c^2}\right)\) e \(\mu = \dfrac{mv_0}{cx_0}\).

    Portanto (como já sabíamos), este modelo, que foi modificado com a mera utilização (apropriada) de operações algébricas elementares, tem a sua descrição completa reduzida a uma Função Matemática \(\eta(\tau, \epsilon, \mu)\) de (apenas) três variáveis (dois parâmetros adimensionais).

    É claro que o Modelo Matemático representado por uma função de três variáveis \(\eta(\tau, \epsilon, \mu)\) é muito mais simples do que uma função \(x = \psi(t, m, c, k, x_0, v_0)\) de \(6\) variáveis! A Complexidade deste Modelo portanto, pode ser considerada como de nível \(3\).

\textbf{OBSERVAÇÕES}

    \begin{enumerate}
    \item A representação reduzida do Modelo Matemático tem um significado prático crucial para o experimentalista (físico e numérico). Nestes casos, o estudo computacional ou experimental de um Modelo Matemático exige a consideração de pelo menos três valores (pequeno, médio e grande) para cada variável descritiva. Para o Modelo Adimensional isto significa \(3^2 = 9\) simulações (ou experimentos), o que se compara extraordinariamente bem com os \(3^5 = 243\) necessários para o Modelo dimensional original.

    A explicação deste fato é simples: em todos os experimentos (físicos e numéricos) em que os cinco parâmetros se agrupam com valores \(\epsilon = \dfrac{km}{c^2}\) e \(\mu = \dfrac{mv_0}{cx_0}\) iguais, obtém-se, essencialmente, o mesmo resultado qualitativo, já que as funções que descrevem os modelos dimensional e adimensional diferem apenas por fatores numéricos.

    \item Dizemos que dois modelos dimensionais com parâmetros \[(m, c, k, x_0, v_0) \mbox{ e } (m', c', k', x'_0, v'_0)\] são \textbf{similares} se os parâmetros adimensionais \(\epsilon = \dfrac{km}{c^2}\) e \(\epsilon' = \dfrac{k'm'}{c'^2}\) são iguais (\(\epsilon = \epsilon'\)) e; \(\mu = \dfrac{mv_0}{cx_0}\) e \(\mu' = \dfrac{m'v'_0}{c'x'_0}\) também (\(\mu = \mu'\)).

    Observe que a possibilidade de variação dos parâmetros dimensionais é ainda enorme sob a restrição de que mantenham os mesmos valores dos parâmetros adimensionais.Para todos estes modelos dimensionais, o mesmo modelo adimensional é exatamente o mesmo.

    \item O acréscimo de mais variáveis (parâmetros) na descrição do modelo (mantendo as dimensões básicas de unidade, \(L, M, T\)) significa um aumento da complexidade do modelo. Este princípio é muito importante na construção de modelos físicos miniaturizados (os protótipos) de sistemas de grande porte. Por exemplo, se o modelo original tem um \(x_0\) da ordem de quilômetros, podemos construir um pequeno protótipo com \(x_0\) da ordem de centímetros que terá o mesmo comportamento qualitativo \textbf{desde que} modifiquemos os outros parâmetros dimensionais de tal maneira que \(\epsilon\) e \(\mu\) permaneçam com o mesmos valores.

    Ao contrário do que usualmente se crê, a construção de um pequeno protótipo que não modifique os outros parâmetros apropriadamente, não se comportará da mesma forma que o sistema original. Esta foi a causa da queda de muitos aviões, rompimento de barragens e desabamentos de pontes construídos com base em experimentos com pequenos modelos que não foram adequadamente adimensionalizados.
\end{enumerate}



\subsection{Métodos de similaridade}

    Um Modelo Matemático quantitativo é representado por uma Função Matemática de valores reais que impões uma relação entre todas as medidas numéricas constantes do Modelo. Por exemplo, no caso do sistema Mecânico Massa-Mola-Viscosidade, \(\Phi(x, t, m, c, k, x_0, v_0) = 0\), em que a Função Matemática \(\Phi(\xi_{1}, \xi_{2}, \xi_{3}, \xi_{4}, \xi_{5}, \xi_{6}, \xi_{7}) = \Phi\), tem \(7\) variáveis, ou seja, \(\Phi: U \subset \mathbb{R}^{7} \to \mathbb{R}\).

    O Método de Similaridade tem por objetivo representar a Função Matemática \(\Phi\) como composição de Funções Matemáticas definidas com menor número de variáveis; uma questão muito relacionada ao problema de Hilbert-Kolmogorov-Arnold que por sua vez foi sugerido pelo Método Nomográfico de representação de modelos matemáticos desenvolvido pelo engenheiro Maurice d’Ocagne no princípio do século XX. Hoje, o Método Nomográfico caiu em desuso substituído pelo desenvolvimento dos computadores digitais.

    O argumento do Princípio de Similaridade acima exposto, mostra que é possível reduzir a representação do Modelo Matemático, originalmente descrito por
    \[\begin{array}{rcl}
    & & \Phi(x, t, m, c, k, x_0, v_0) \\
    &=& \psi(\eta, \tau, \epsilon, \mu) \\
    &=& \psi\left(\dfrac{x}{x_0}, \dfrac{ct}{m}, \dfrac{km}{c^2}, \dfrac{mv_0}{cx_0}\right),
    \end{array}\]
    onde {\red \(\psi: V \subset \mathbb{R}^4 \to \mathbb{R}\)} é uma Função Matemática de APENAS quatro variáveis que, uma vez conhecida, produz imediatamente a função original \(\Phi(x, t, m, c, k, x_0, v_0)\) de sete variáveis, pela composição com \(4\) funções Matemáticas elementares conhecidas:
    \[\begin{array}{rcl}
    \psi_{1}(\zeta_{1}, \zeta_{2})
    &=& \dfrac{\zeta_{1}}{\zeta_{2}} \\[0.3cm]
    \psi_{2}(\zeta_{1}, \zeta_{2}, \zeta_{3})
    &=& \dfrac{\zeta_{1}\zeta_{2}}{\zeta_{3}} \\[0.3cm]
    \psi_{3}(\zeta_{1}, \zeta_{2}, \zeta_{3})
    &=& \dfrac{\zeta_{1}\zeta_{2}}{\zeta_{3}^{2}} \\[0.3cm]
    \psi_{4}(\zeta_{1}, \zeta_{2}, \zeta_{3}, \zeta_{4})
    &=& \dfrac{\zeta_{1}\zeta_{2}}{\zeta_{3}\zeta_{4}}.
    \end{array}\]

    É interessante comparar a extrema complexidade de um problema Matemático puro desta natureza quando formulado sem qualquer referência à sua origem e a simplicidade dos argumentos e manipulações aritméticas que caracterizam os Métodos de Similaridade.

    Voltemos, agora, ao Modelo Matemático \(T = \psi(m, g, l, A)\) do pêndulo de massa \(m\) suspensa por uma haste de comprimento \(l\) sob ação da gravidade \(g\) que realiza um movimento oscilatório de amplitude \(A\) sem resistência e com período \(T\).

    As dimensões dos parâmetros relacionados são:
    \[[A] = [l] = L, [m] = M, [T_{0}] = T, [g] = LT^{-2},\]
    {\red e os parâmetros \([A/l], [T_0/\sqrt{l/g}]\) são os dois únicos parâmetros adimensionais independentes}.
    
    Utilizando as unidades intrínsecas de massa \(m\), de comprimento \(l\) e de tempo \(T_{0} = \sqrt{\dfrac{l}{g}}\), o modelo matemático \(T = \psi(m, g, l, A)\) passa a ser escrito na seguinte forma:
    \[\dfrac{T}{\sqrt{\dfrac{l}{g}}} = \psi\left(1, 1, 1, \dfrac{A}{l}\right).\]

    Portanto, a representação do Modelo passa de uma Função Matemática de \(4\) variáveis a um Modelo Matemático reduzido que utiliza uma função de uma única variável \(\varphi(\zeta)\). Isto significa que para pêndulos com a mesma razão \(\dfrac{A}{l} = {\red \zeta_{1}}\), o período de oscilação será dado pela função
    \[T_{0} = \varphi\left(\dfrac{A}{l}\right) \sqrt{\dfrac{l}{g}}.\]

    Portanto, o Modelo Matemático geral deste fenômeno pode ser \textit{experimentalmente} obtido, simplesmente, pelo registro dos períodos de oscilação de um mesmo pêndulo para diversos valores de \(\dfrac{A}{l}\), isto é, tomando-se seguidos valores de \(A_{k}\) o que produz uma série de dados pontuais da função de uma variável
    \[\varphi(\zeta_{k}) = \dfrac{T_{k}}{\sqrt{\dfrac{l}{g}}},\]
    para \(\zeta_{k} =\dfrac{A_{k}}{l}\). (Compare com o ``experimentalista força-bruta'' que teria que fazer \(3^5 = 243\) experimentos e obteria apenas alguns pontos esparsos e nenhuma relação funcional).

    Se admitirmos que a função \(\varphi\) é contínua, os movimentos pendulares de pequena amplitude, isto é, tais que \(\dfrac{A}{l} << 1\), serão todos bem descritos por:
    \[T_{0} \simeq \varphi(0) \sqrt{\dfrac{l}{g}}.\]
    É interessante observar que \(\varphi(0) = \sqrt{\pi}\), resultado este que a Análise Dimensional \textbf{não} produz, mas que pode ser facilmente obtido resolvendo-se o Modelo linearizado (que aproxima o Modelo completo para oscilações de pequena amplitude, isto é, \(\dfrac{A}{l} << 1\)) e descrito por uma simples equação diferencial ordinária linear de segunda ordem.

\begin{exercise}
    Supondo que a função matemática \(\varphi\) é continuamente diferenciável, obtenha uma segunda aproximação para o período:
    \[T_{0} \simeq \left(\varphi(0) + \varphi'(0) \dfrac{A}{l}\right) \sqrt{\dfrac{l}{g}},\] calculando \(\varphi'(0)\) usando uma expansão no parâmetro \(\dfrac{A}{l} = \epsilon_{1}\) após adimensionalizar adequadamente o modelo diferencial da dinâmica do pêndulo:
    \[\begin{array}{rcl}
    m \dfrac{d^2(l\theta)}{dt^{2}} &=& -mg \sin(\theta) \mbox{\tiny (2\textordfeminine\ lei de Newton tangencial)} \\
    l\theta(0) &=& A \\
    \dfrac{d\theta}{dt}(0) &=& 0.
    \end{array}\]
\end{exercise}

    \textbf{Solução}:
    {\color{orange}
    Processo de adimensionalização:
    
    Temos que
    \[\begin{array}{rcl}
    [T_0] &=& [g]^{\alpha}\ [l]^\beta\ [m]^\gamma\ [\theta] \\[0.2cm]
    T^{1} &=& (L\ T^{-2})^{\alpha}\ L^\beta\ M^\gamma \\[0.2cm]
    T^{1} &=& L^{\alpha+\beta}\ T^{-2\alpha}\ M^\gamma
    \end{array}\]
    
    Segue que
    \[
    \left\{\begin{array}{rcl} \alpha+\beta &=& 0 \\ -2\alpha &=& 1 \\ \gamma &=& 0 \end{array}\right.
    \Rightarrow
    \left\{\begin{array}{rcl} \alpha &=& -\dfrac{1}{2} \\[0.3cm] \beta &=& \dfrac{1}{2} \\[0.3cm] \gamma &=& 0  \end{array}\right.
    \]
    
    Dessa forma, pelo princípio da similaridade, temos:
    \begin{equation}\label{eq:Tps}
    T_0 = g^{-\frac{1}{2}} l^{\frac{1}{2}} \varphi(\theta) \Rightarrow T_0 = \sqrt{\dfrac{l}{g}}\ \varphi(\theta)
    \end{equation}
    
    
    Por outro lado, temos:
    \[m \dfrac{d^2(l\theta)}{dt^2} = - m g \sin(\theta).\]
    
    Como \(m \ne 0\) é fator comum  e \(l\theta = A\), podemos reescrever a equação anterior da seguinte maneira:
    \[l \dfrac{d^2(A/l)}{dt^2} = - g \sin(A/l).\]
    
    O processo de adimensionalização é feito a seguir:
    \[\dfrac{[l]}{[g]} \dfrac{d^2(A/l)}{[T_0]^2\ d(t/[T_0])^2} = -\sin(A/l),\]
    ou ainda,
    \begin{equation}\label{eq:admensionalizada}
    \dfrac{d^2\epsilon}{d\tau^2} = -\sin(\epsilon),\quad \epsilon = A/l.
    \end{equation}
    
    Uma vez que a equação \eqref{eq:admensionalizada} possui como coeficientes, funções analíticas, consideremos a solução \(\epsilon\) como uma série de Taylor em torno do zero, ou seja:
    \begin{equation}\label{eq:episilontaylor}
    \epsilon(\tau) = \displaystyle \sum_{k=0}^{\infty} \dfrac{1}{k!} \epsilon^{(k)}(0) \tau^k
    \end{equation}
    
    Os três primeiros termos da série \eqref{eq:episilontaylor} aproximam \(\epsilon\). Assim,
    \[
    \epsilon(\tau) \simeq \epsilon(0) + \dfrac{d\epsilon(0)}{d\tau} \tau + \dfrac{1}{2} \dfrac{d^2\epsilon(0)}{d\tau^2} \tau^2.
    \]
    
    Uma vez que 
    \[0 = \dfrac{d\theta}{dt}(0) = [T_0]^{-1} \dfrac{d(A/l)}{d(t/[T_0])} \Rightarrow \dfrac{d\epsilon}{d\tau}(0) = 0\]
    e pela EDO \eqref{eq:admensionalizada},
    \[\dfrac{d^2\epsilon(0)}{d\tau^2} = -\sin(\epsilon(0)) = - \sin(A/l).\]
    
    Portanto, temos
    \[
    \epsilon(\tau) \simeq \epsilon_1 - \dfrac{1}{2} \sin(\epsilon_1)\tau^2.
    \]
    
    Assim, com base nesta aproximação, consideremos:
    \begin{equation}\label{eq:desadimensionalizacao}
    \tau = \sqrt{2 \csc(\epsilon_1)(\epsilon_1 - \epsilon)}
    \end{equation}
    
    
    Desadimensionalizando a equação \eqref{eq:desadimensionalizacao}, temos:
    \begin{equation}\label{eq:desarrumada1}
    \dfrac{T_0}{T} = \sqrt{2 \csc(A/l)(A/l-\theta)}
    \end{equation}
    ou ainda
    \begin{equation}\label{eq:desarrumada2}
    T_0 = \sqrt{\dfrac{l}{g}} \sqrt{2 \csc(A/l) (A/l-\theta)}.
    \end{equation}
    
    Comparando com a equação \eqref{eq:Tps}, temos:
    \[\begin{array}{rcl}
    \varphi(\theta)
    &=& \sqrt{2 \csc(A/l)} \sqrt{(A/l-\theta)}
    \Rightarrow \\
    \varphi(0) &=& \sqrt{2 \csc(A/l)} \sqrt{(A/l)}
    \end{array}\]
    
    Derivando-se \(\varphi\), obtemos:
    \[\begin{array}{rcl}
    \dfrac{d}{d\theta}\varphi(\theta)
    &=& -\dfrac{1}{2}\sqrt{\dfrac{2 \csc(A/l)}{A/l-\theta}}
    \Rightarrow \\
    \varphi'(0) &=& 
    -\dfrac{1}{2}\sqrt{\dfrac{2 \csc(A/l)}{A/l}}
    \end{array}\]
    
    Substituindo-se estes dois últimos resultados na equação \eqref{eq:aprox_freq}, obtemos:
    \[\small
    \begin{array}{l}
    T_{0} \\
    \simeq \left[\sqrt{2 \csc\left(\dfrac{A}{l}\right)} \sqrt{\dfrac{A}{l}} - \dfrac{1}{2}\sqrt{\dfrac{2 \csc\left(\dfrac{A}{l}\right)}{\left(\dfrac{A}{l}\right)}} \left(\dfrac{A}{l}\right)\right] \sqrt{\dfrac{l}{g}} \\
    = \dfrac{\sqrt{2}}{2} \sqrt{\dfrac{A}{g} \csc\left(\dfrac{A}{l}\right)}
    \end{array}
    \]
    
    Assim, a aproximação requerida é:
    \[
    T_{0} \simeq
    \dfrac{\sqrt{2}}{2} \sqrt{\dfrac{A}{g} \csc\left(\dfrac{A}{l}\right)}
    \]
}

    Passaremos, agora, a apresentar exemplos didáticos fundamentais para a Biomatemática e cujos estudos mais detalhados serão temas de capítulos seguintes.



\section{A dinâmica de populações}

\subsection{O modelo malthusiano para dinâmica de populações}

    O conceito de População Biológica se refere, em geral, a uma grande quantidade de indivíduos sendo que o primeiro, e mais importante, aspecto que se deseja mensurar sobre ela é, naturalmente, o seu ``\textit{tamanho}''. À primeira vista, e ingenuamente, não parece haver dúvidas de que esta informação pode ser imediatamente obtida pela contagem (censo) ``\textit{cabeça por cabeça}'' (o que exige a identificação de cada um dos seus indivíduos) e a representação do resultado final por um Número Natural \(\mathbb{N}\). Entretanto, para grandes populações (como \(10^6\) (insetos sociais), \(10^8\) (demografia-epidemiologia), \(10^{10}\) (células-imunologia-neurobiologia) e etc.) fica também imediatamente claro que a especificação do número exato de indivíduos da população a cada instante é certamente um excesso de informação, não apenas indisponível na prática mas, felizmente, desnecessária para a maioria dos estudos.

    O procedimento que logo de início descarta este excesso de informação, pois evita a necessidade de identificação individual de seus componentes, consiste em medir o ``tamanho'' de uma grande população fazendo uso de unidades formadas por grandes ``\textbf{lotes}'' de \(N_{0}\) indivíduos por exemplo, \(N_{0} = 10^8\) indivíduos no caso de demografia. Desta forma, o tamanho da população brasileira, por exemplo, pode ser representada em um intervalo de tempo razoável (séculos) na forma \(xN\), onde \(0 \le x \le 10\) é um número racional.

    Para realizar este procedimento, basta separar a população total em subpopulações, aproximadamente, do mesmo tamanho (que pode ser feito por amostras sem necessidade de uma ``individualização'') pois, pequenos erros relativos são toleráveis. (Por exemplo, erros de \(1.000\) indivíduos resultarão em erros da ordem de \(10^{5}\) em uma medida com unidade \(N_{0} = 10^{8}\). Contagens de populações de micro-organismos ou celulares são feitas com amostras em gotas sob escrutínio de um microscópio manejado por estudantes de pós-graduação!).

    A variação temporal desta população em unidades de tempo da ordem de dias, que é uma unidade de tempo pequena para estudos demográficos e epidemiológicos, será ``infinitesimal''. Assim, o segundo (e não menos importante) bônus desta representação é o gráfico da sua evolução em um período de tempo \(0 \le t \le 1.000\) (por exemplo) é, facilmente, visualizada como uma curva contínua e não um aglomerado de pontos. Na verdade, o gráfico de qualquer função matemática é visualmente representada por um número equivalente de pontos discretos, nunca por uma curva ``contínua''.

    Portanto, as unidades a serem utilizadas para a descrição do tamanho de uma grande população serão sempre tomadas como grandes ``lotes'' de indivíduos de tal forma que numericamente ela possa ser descrita por números racionais com variações temporais pequenas em unidades de tempo adequadas. Esta abordagem é que torna plausível a representação de um Modelo Matemático para a dinâmica temporal de uma grande população por intermédio de funções diferenciáveis definidas na reta e com valores reais e, com isto, permite a utilização do vasto arsenal de Métodos Matemáticos da Análise Matemática em seu estudo, especialmente das Equações Diferenciais.

    O Modelo Mathusiano para descrever a dinâmica de uma população sob o ponto de vista de mortalidade (e que será analisado com maiores detalhes no capitulo 2) dispõe, inicialmente, de dois parâmetros de observação de interesse: \(n\) (tamanho da população) e \(t\) (tempo), o que exige a escolha de um sistema de unidades \(\{N,T\}\).

    Entretanto, verifica-se por observação, que é impossível estabelecer um Modelo Matemático que relacione as medidas destes dois parâmetros na forma \(\Phi(n,t) = 0\), pois isto não levaria em conta, nenhuma informação biológica que representasse de alguma forma uma medida da ``mortalidade`` característica da população estudada, assim como a sua óbvia dependência com relação ao seu valor inicial. Para remediar esta deficiência conceitual do modelo proposto, o Modelo de Malthus acrescenta, então, um Parâmetro Descritivo \(k\) definido como a taxa de mortalidade per capita que se supõe constante com o tempo e tem a sua determinação experimental definida como
    \[k = -\dfrac{1}{n(t)} \dfrac{dn}{dt}.\]
    Com isto, verifica-se que a medida do parâmetro \(k\) tem que necessariamente assumir a dimensão \(T^{-1}\) para que a dimensão seja mantida dos dois lados da equação. Portanto, o Modelo Malthusiano explícito pode ser representado matematicamente na forma \(n = \varphi(t, n_0, k)\) e a função \(\varphi\) é então obtida da equação diferencial
    {\red \[-\dfrac{1}{\varphi(t)} \dfrac{d\varphi}{dt} = k,\ \ \varphi(0) = n_0.\]}
    É claro que este Modelo, matematicamente trivial, toma a forma:
    \[n = \varphi(t, n_{0}, k) = n_{0}\ e^{-k t}.\]

    Como a medida do parâmetro \(k^{-1}\) tem a dimensão de tempo (isto é, varia de forma diretamente proporcional com a unidade do tempo) e n0 obviamente tem dimensão de população, adotaremos as unidades intrínsecas para a medida de população como sendo \(n_0\) e de tempo como sendo \(k^{-1}\). Com isto, verificamos, imediatamente, que \(\eta = \varphi(\tau, 1, 1) = \varphi^\ast(\tau) = e^{-\tau}\) é o \textbf{Modelo Malthusiano Reduzido/Adimensional} para \(\eta = \dfrac{n}{n_0}\) e \(\tau = \dfrac{t}{k^{-1}} = kt\).

    Um experimentalista ou analista numérico ``\textit{puro}'' que desejasse ``\textit{plotar}'' um gráfico da Função Matemática \(n = \varphi(t, n_0, k)\) que representa o Modelo Malthusiano, teria que registrar/calcular o valor numérico de \(\varphi\) para, digamos, \(p\) valores de \(0 = t_{1} < \ldots < t_{p} = T\) e vários valores de \(n_0\) e \(k\). Se \(n_0\) e \(k\) fossem, conservativamente escolhidos apenas com valores (pequeno, médio e grande) o resultado experimental (trabalhosamente registrado ou, calculado) constaria de \(3 \cdot 3 = 9\) gráficos obtidos com \(9\) experiências/computações. Por outro lado, um experimentalista/analista numérico versado em Análise Dimensional realizaria apenas \textbf{uma única} experiência/computação para descrever (discretamente) a Função Matemática Reduzida (Adimensional) \(\varphi^\ast\) e representaria muito melhor a Função dimensional na forma: \(\varphi(t, n_{0}, k) = n_{0} \varphi^\ast(kt)\).

    Uma leitura panorâmica dos itens seguintes é indicada como uma primeira abordagem. A apresentação detalhada destes importantes temas será parte fundamental da matéria tratada no capítulo ``Principio de Difusão''.



\subsection{Primeiro problema fundamental de difusão}

    Os Modelos Matemáticos de Difusão estão dentre os mais fundamentais da Matemática Aplicada e são particularmente importantes para a descrição de inúmeros fenômenos da Biologia de Populações continuamente distribuídas no espaço. Além disso, a estrutura matemática deste Modelo é representada por uma Equação Diferencial Parcial cujas soluções desempenham papéis fundamentais também como Método em toda a Análise Matemática.

    Consideremos o Modelo clássico de difusão em uma dimensão espacial sem fronteiras finitas partindo de uma condição inicial pontual, o que pode ser visualizado como a dinâmica de difusão molecular de \(N_0\) moles de corante químico, ou como a dispersão difusiva (``\textit{aleatória}'') de \(N_0\) indivíduos, colocados na origem no instante \(t = 0\). Este Modelo Matemático pode ser representado pelo seguinte problema diferencial com respeito à função densidade \(\rho(x, t)\):
    \[\dfrac{\partial \rho}{\partial t} = D \dfrac{\partial^2 \rho}{\partial x^2},\] em que
    \(\rho(x, 0) = N_0 \delta(x)\), para \(x \in \mathbb{R} \mbox{ e } t > 0.\)

    A fronteira infinita significará neste caso que não há fluxo nem matéria em distâncias ``muito grandes'', ou seja,
    \[\lim_{|x| \to \infty} -D \dfrac{\partial \rho}{\partial x} (x, t) = 0 \mbox{ e } \lim_{|x| \to \infty} = 0.\]

    A condição inicial \(\rho(x, 0) = N_{0}\delta(x)\) é simbólica e não determina de fato a função densidade no instante \(t = 0\). Aqui o delta de Dirac \(\delta\) não é considerado uma função no sentido clássico, mas no sentido generalizado (distribuição) que pode ser também interpretada com o seguinte significado: No limite, para \(t \downarrow 0\) a densidade \(\rho(x, t)\) (definida apenas para \(t>0\)) se comporta como uma sequência de Dirac: {\red
    \begin{enumerate}
    \item \(\rho(x, t) \ge 0\);
    \item \(\displaystyle \int_{-\infty}^{\infty} \rho(x, t)\ dx = 1\);
    \item \(\displaystyle \lim_{t \downarrow 0} \int_{-a}^{a} \rho(x, t)\ dx = 1, \forall\ a > 0\) (Visualize esta descrição).
    \end{enumerate}
}


    As unidades básicas e de dimensões independentes para a quantificação deste problema serão dadas por \(\{N, L, T\}\), onde \(N\) é a dimensão da medida de populações (moléculas ou indivíduos), \(L\) é comprimento e \(T\) é o tempo.
    
    A solução do problema é uma função \(\rho(x, t, D, N_{0})\) e as dimensões dos parâmetros são:
    \[
    \begin{array}{rcl}
    \ [N_{0}] &=& N, \\
    \ [x] &=& L, \\
    \ [t] &=& T, \\
    \ [\rho] &=& NL^{-1}, \\
    \ [D] &=& L^{2}T^{-1}.
    \end{array}\]
    Observamos, assim, que há apenas dois parâmetros adimensionais independentes, já que temos \(5\) medidas \((\rho, x, t, D, N_{0})\) e três dimensões de base \(\{N, L, T\}\). Podemos, facilmente, obter os dois representantes adimensionais que, neste caso, tomaremos como:
    \[\dfrac{\rho \sqrt{Dt}}{N_{0}} \mbox{ e } \dfrac{x}{\sqrt{Dt}}. \mbox{ (Verifique!)}\]

    Portanto, pelo ``Princípio de Similaridade'', um parâmetro deve ser função do outro, ou seja, devemos ter:
    \[\dfrac{\rho \sqrt{Dt}}{N_{0}} = \varphi\left(\dfrac{x}{\sqrt{D t}}\right),\]
    para alguma função real de uma variável real \(\varphi(\zeta)\). Esta conclusão simples, além de facilitar enormemente a tarefa de construir o modelo matemático (pois agora devemos obter uma função de uma variável \(\varphi(\zeta)\) e não \(\rho(x,t)\), de duas, já nos fornece um resultado de grande importância:
    \[\rho(x, t) = \dfrac{N_{0}}{\sqrt{Dt}} \ \varphi\left(\dfrac{x}{\sqrt{D t}}\right).\]
    Por exemplo, com base neste resultado podemos concluir o seguinte:

\begin{enumerate}
    \item A densidade na origem cai com o tempo na forma \(\rho(x, t) = \dfrac{N_{0}}{\sqrt{Dt}} c\), onde \(c = \varphi(0)\).

    \item A observação da densidade em um único ponto qualquer (não na origem), digamos, \(x_0\) nos fornecerá a função \(\rho(x, t)\) ao longo de todo o espaço, em todo instante, para quaisquer valores dos parâmetros \(N_{0}\) e \(D\). Claro, pois isto nos daria a função
    \[\varphi(\zeta) = \rho\left(x_0, \dfrac{x_0^2}{D\zeta^{2}}\right) \dfrac{\zeta}{N_{0}x_0}.\] Portanto, em princípio, basta uma única experimentação/simulação e a observação em apenas um ponto ao longo do tempo para determinarmos completamente o modelo matemático.

    \item Pelas condições de fronteira concluímos que \(\varphi(\zeta) \to 0\) e \(\varphi'(\zeta) \to 0\), quando \(|\zeta| \to \infty\).

    \item Digamos que \(\varphi(\epsilon) = \delta\) seja a menor densidade detectável. Para os pontos que ``viajam'' como \(x = \epsilon \sqrt{Dt}\) a densidade será dada por \(\rho(x, t) = \dfrac{N_{0}}{\sqrt{Dt}} \varphi(\epsilon) = \dfrac{N_{0} \epsilon}{x} \varphi(\epsilon)\) e, portanto, além destes pontos e, posteriormente ao instante \(t_0 = \dfrac{N_{0}^{2}}{D}\), não haverá quantidade detectável de indivíduos desta população.

    \item Para um problema em duas (três) dimensões, observamos que a segunda variável adimensional permanece idêntica \(\dfrac{x}{\sqrt{Dt}}\), mas a primeira variável adimensional deve ser modificada para \(\dfrac{\rho(Dt)}{N_{0}}\), \(\left(\mbox{respectivamente}, \dfrac{\rho(Dt)^{\frac{3}{2}}}{N_{0}}\right)\), uma vez que a densidade agora tem dimensão \([\rho] = NL^{-2}\), (respect. \([\rho] = NL^{-3}\)).

    Generalizando, observamos que em dimensão \(n\) qualquer, a solução do problema fundamental, sendo isotrópica e, portanto, com simetria esférica, terá a seguinte forma funcional \(\rho(x, t) = \dfrac{N_{0}}{(Dt)^{\frac{n}{2}}} \varphi_{n}\left(\dfrac{r}{\sqrt{Dt}}\right)\).

    Veremos, mais abaixo, que a função \(\varphi_{n}\) é a mesma para qualquer dimensão.

    \item E, enfim, como a função \(\rho(x, t)\) é solução da equação diferencial, calculando as derivadas necessárias segundo a expressão
    \[\rho(x, t) = \dfrac{N_{0}}{\sqrt{Dt}} \varphi\left(\dfrac{x}{\sqrt{D t}}\right)\]
    e substituindo-as mecanicamente na equação
    \[\begin{array}{rcl}
    \dfrac{\partial \rho}{\partial t} &=& D \dfrac{\partial^2 \rho}{\partial x^2}
    \end{array}\]
    temos,
    \[\small
    \begin{array}{l}
    -\dfrac{1}{2} \dfrac{N_{0}}{\sqrt{D} t^{\frac{3}{2}}} \varphi\left(\dfrac{x}{\sqrt{Dt}}\right) %\\[0.3cm]
    -\dfrac{1}{2} \dfrac{N_{0}x}{\sqrt{Dt} \sqrt{D} t^{\frac{3}{2}}} \varphi'\left(\dfrac{x}{\sqrt{Dt}}\right) \\[0.3cm]
    = D \dfrac{N_{0}}{\sqrt{Dt}} \dfrac{1}{Dt} \varphi''\left(\dfrac{x}{\sqrt{Dt}}\right),
    \end{array}\]
    após cancelamentos de termos e lembrando que \(\zeta = \dfrac{x}{\sqrt{Dt}}\), obtemos a seguinte equação diferencial ordinária para \(\varphi(\zeta)\):
    \[2\varphi''(\zeta)+\zeta\varphi'(\zeta)+\varphi(\zeta)= 0,\]
    que integrada uma vez nos dá:
    {\red \[\varphi'(\zeta) + \dfrac{\zeta}{2}\varphi(\zeta) + c_{0} = 0,\]}
    pela conclusão 3. acima. Multiplicando pelo fator integrante \(\exp\left(\dfrac{\zeta^2}{4}\right)\), reescrevemos:
    \[\left[\varphi\ \exp\left(\dfrac{\zeta^2}{4}\right)\right]' = 0\] e, finalmente,
    \[\varphi(\zeta) = c\ \exp\left(\dfrac{\zeta^2}{4}\right),\]
    ou, como queríamos:
    \[\rho(x, t) = c \dfrac{N_{0}}{\sqrt{Dt}} \exp\left(\dfrac{-x^2}{4Dt}\right).\]

    Para determinarmos a constante \(c\), ou lembramos (como um matemático aplicado deveria fazer) que o modelo não admite morte nem nascimento e não apresenta fluxos no infinito e, portanto,
    \[\displaystyle \int_{-\infty}^{\infty} \rho(x,t)\ dx = N_{0}\]
    constante, para todo \(t\). Argumentando como um matemático ``puro'' calculamos a variação da população total com a expressão:
    \[\begin{array}{rcl}
    & & \dfrac{\partial}{\partial t} \displaystyle\int_{-\infty}^{\infty} \rho(x, t) \ dx \\[0.4cm]
    &=& \displaystyle\int_{-\infty}^{\infty} \dfrac{\partial}{\partial t} \rho(x, t) \ dx \\[0.4cm]
    &=& \displaystyle\int_{-\infty}^{\infty} \dfrac{\partial^2}{\partial x^2} \rho(x, t) \ dx \\[0.4cm]
    &=& \dfrac{\partial \rho}{\partial x} (\infty, t) - \dfrac{\partial \rho}{\partial x} (-\infty, t) \\[0.4cm]
    &=& 0
    \end{array}\]
    e concluímos o mesmo.

    Portanto, integrando a expressão acima, temos:
    \[\begin{array}{rcl}
    & & N_{0} \\
    &=& \displaystyle\int_{-\infty}^{\infty} c \dfrac{N_{0}}{\sqrt{Dt}} \exp\left(\dfrac{-x^2}{4Dt}\right) \\[0.4cm]
    &=& 2N_{0} c \displaystyle\int_{-\infty}^{\infty} x \exp\left(\dfrac{-x^2}{4Dt}\right) d\left(\dfrac{x}{2\sqrt{Dt}}\right) \\[0.4cm]
    &=& 2N_{0} \ c \sqrt{\pi},
    \end{array}\]
    e, finalmente,
    \[\rho(x, t) = \dfrac{N_{0}}{\sqrt{4\pi Dt}} \exp\left(\dfrac{-x^2}{4Dt}\right).\]

    Consideremos, agora, o problema \(n\) dimensional com \(n \ge 1\). Como já vimos a forma funcional da solução neste caso será
    \[\rho(x, t) = \dfrac{N_{0}}{(Dt)^{\frac{n}{2}}} \varphi_n\left(\dfrac{r}{\sqrt{Dt}}\right).\]

    Como a equação de difusão com simetria esférica é dada por
    \[\dfrac{\partial \rho}{\partial t} = D \left(\dfrac{n-1}{r} \dfrac{\partial \rho}{\partial r} + \dfrac{\partial^2 \rho}{\partial r^2}\right),\]
    podemos, facilmente, repetir o argumento acima e obteremos o seguinte importante resultado:
\end{enumerate}

\begin{exercise}
    Mostre que a solução fundamental do problema de difusão em dimensão \(n\) é dada por:
    \[\begin{array}{rcl}
    \rho(x, t)
    &=& \dfrac{N_{0}}{(Dt)^{\frac{n}{2}}} \exp\left(\dfrac{-||x||^2}{4Dt}\right) \\[0.4cm]
    &=& \dfrac{N_{0}}{(Dt)^{\frac{n}{2}}} \exp\left(\dfrac{-r^2}{4Dt}\right).
    \end{array}\]
\end{exercise}

    \textbf{Solução}:



\subsection{Segundo problema fundamental de difusão}

    Consideremos, agora, o seguinte problema unidimensional de difusão: Um tubo longo de secção circular estreita é conectado em uma de suas extremidades ( a ``finita'') a um reservatório suficientemente grande para que a difusão de moléculas (ou indivíduos) no tubo não afete a concentração constante \(\rho_0\) mantida internamente nele. Inicialmente, o tubo não contém a substância a ser difundida e a conexão com o reservatório, em \(x = 0\), somente é liberada no instante \(t \ne 0\).
    
    Suponhamos que o tubo é um meio propício a um processo de difusão, cujo coeficiente tomaremos como \(D\). Como o tubo é longo e estreito, consideraremos um problema unidimensional semi-infinito, em \([0,\infty]\). Portanto, o Modelo Matemático para este processo é escrito da seguinte forma:
    \begin{eqnarray}
    \dfrac{\partial \rho}{\partial t} &=& D \dfrac{\partial^2 \rho}{\partial x^2},\ x > 0 \\
    \rho(x,0) &=& 0 %\mbox{ (condições iniciais)}
    \\
    \rho(0,t) &=& \rho_0,\\
    \rho(\infty, 0) &=& 0 \\
    \dfrac{\partial \rho}{\partial x}(\infty,t) &=& 0 %\mbox{ (cond. de fronteira, finita e infinita)}
    \end{eqnarray}

    A base de dimensões para a quantificação do problema é \(\{N, L, T\}\) e as variáveis descritivas dimensionais são 5: \(\{\rho, x, t, \rho_0, D\}\). Portanto, como no caso anterior, teremos apenas duas variáveis adimensionais independentes, que tomaremos agora como \(\dfrac{\rho}{\rho_0}\) e \(\dfrac{x}{\sqrt{Dt}}\) de onde tiramos que:
    \[\rho = \rho_0 \phi\left(\dfrac{x}{\sqrt{Dt}}\right).\]
    Repetindo o argumento acima, temos:
    {\red
    \[
    \dfrac{\partial \rho}{\partial t}
    = D \dfrac{\partial^2 \rho}{\partial x^2}
    \]
    ou seja,
    \[-\dfrac{1}{2} \dfrac{\rho_0 x}{\sqrt{D} t^{\frac{3}{2}}} \phi'\left(\dfrac{x}{\sqrt{Dt}}\right)
    = D \rho_0 \dfrac{1}{Dt} \phi''\left(\dfrac{x}{\sqrt{Dt}}\right)\]}
    
    de onde tiramos a EDO para \(\phi(\zeta)\):
    \[\phi''(\zeta) + \dfrac{\zeta}{2}\phi'(\zeta) = 0.\]
    
    Multiplicando pelo fator integrante e integrando, temos:
    \[\phi(\zeta) = c_0+c\int_{0}^{\zeta} \exp\left(-\dfrac{s^2}{4}\right)\ ds.\]
    
    Portanto,
    \[\rho(x,t) = \rho_0 \left[c_0+c\int_{0}^{\frac{x}{\sqrt{Dt}}} \exp\left(-\dfrac{s^2}{4}\right)\ ds\right],\]
    mas como \(\rho(x,0) = 0\), temos:
    \[0 = c_0 + c \dfrac{1}{2} \sqrt{\pi}\]
    e como \(\rho(0,t) = 0\), concluímos, finalmente, que:
    \[\rho(x,t) = \rho_0 \left[1-\dfrac{2}{\sqrt{\pi}} \int_{0}^{\frac{x}{\sqrt{Dt}}} \exp\left(-\dfrac{s^2}{4}\right)\ ds\right].\]

    A importância destas soluções, especialmente a primeira, extravasa para inúmeras outras questões de Análise de Métodos Matemáticos que serão abordados ao longo de todo este texto.

    É importante ressaltar como a argumentação dimensional empregada nos itens acima não apenas esclarece a estrutura do Modelo Matemático em questão, mas também o simplifica consideravelmente, possibilitando a obtenção de suas soluções explícitas de maneira elementar.

    Convido aos/às leitores/as compararem a exposição acima com o tratamento que usualmente se dá a esta questão em textos de Análise e de Métodos Matemáticos em que se emprega a elegante, poderosa, mas sofisticada e por vezes obscura, Teoria de Fourier. O Método de Fourier será também abordado neste texto em capítulo posterior, mas sob uma perspectiva distinta daquela empregada pela Análise Matemática usual.


\subsection{``Lei'' de ação de massas - Smoluchowski}

    Tal como acontece com diversas teorias da Física e Química, a Lei de Ação de Massas não é ``Lei'' no sentido exato e inexorável do termo, mas sim um Modelo Matemático para a descrição de diversos fenômenos, e que pode ser muito útil, mas nunca acima de qualquer suspeita. Proposto pelos noruegueses Guldberg (químico) e Waage (matemático) no século XIX, esta ``Lei'' ganhou jurisprudência e tornou-se matéria de fé em Química e, posteriormente, em Biologia Matemática com sua aplicação por A. Lotka (químico-matemático) e V. Volterra (matemático) à Dinâmica de Populações.

    O objetivo do Modelo de Ação de Massas é estabelecer uma descrição para a taxa de colisões entre \textit{pequenos} objetos, que consideraremos todos de uma mesma de forma esférica, homogeneamente distribuídos no espaço e realizando movimentos microscópicos aleatórios que, macroscopicamente, serão descritos como um processo de \textit{Difusão}. Um processo de difusão é genericamente definido como uma dinâmica na classe de dinâmicas de distribuições populacionais, \(\rho(x, t) = \rho_{t}(x)\), caracterizadas por serem monotonicamente homogenizadoras, independente das suas causas. O movimento aleatório microscópico é uma das causas mais frequentemente alegadas para isso.

    Em Físico-Química de reações moleculares supõe-se que a agitação térmica seja a origem física deste movimento aleatório, o Movimento Browniano (Berg, 1984). O químico-matemático polonês Marian von Smoluchowski (...-1912) foi o primeiro a dar um tratamento matemático à fundamentação da ``Lei de Ação de Massas'' (Ulam, 1958), Levich [...-]).).

    Em Biologia, o movimento aleatório microscópico que resulta em um processo macroscópico de difusão é resultante de um comportamento de busca incessante que os organismos executam quando eles não dispõem de fortes indícios da localização de seu objeto de desejo (ou, caso dispuserem destes indícios, não saibam ou não queiram utiliza-los). O processo de busca por movimentos aleatórios é o de execução mais simples pois não exige nenhuma das múltiplas e sofisticadas etapas necessárias em uma busca ``inteligente``: 1) Coleta de informações, 2) Arquivamento, 3) Recuperação destas, 4) Análise e 5) Decisão. Mesmo em buscas ``inteligentes'' que leva em conta informações locais, a coleta preliminar destas informações, é necessariamente executada sob ignorância da localização dos objetos procurados, e, portanto, ela somente tem sentido se for feita aleatoriamente. Ou seja, a pesquisa aleatória é sempre executada em todas as circunstâncias, como um fim em si mesma ou como um meio. Além disso, como a busca completamente aleatória pode ser rapidamente executada, sem vacilações, ela permite vasculhar uma grande área em muito pouco tempo, e, portanto, pode ser altamente eficiente quando comparada sob certas circunstâncias com a própria busca inteligente que é naturalmente mais demorada. E este tempo de busca é essencial tratando-se de nutrientes. (Ou, como diz o ditado popular:``De tanto pensar, morreu um burro'').

    Movimentos aleatórios não tão microscópicos, tais como os ``Caminhos de Lévy'' em que o organismo realiza alguns raros mas longos ``vôos'' , não são exatamente o cenário previsto para a validade do Modelo de Ação de Massas e tem sido pouco estudado sob este aspecto. (Viswanathan [...]).

    O modelo da Ação de Massas tem por objetivo determinar a taxa de colisões que deve ocorrer em uma população de partículas/organismos (indivíduos) que realizam um movimento microscópico aleatório. Uma hipótese implícita deste modelo é de que a dimensão espacial dos indivíduos da população seja suficientemente pequena para que não ocorram interações à distância (sejam de atração, repulsão ou percepção) que possam interferir e influenciar no movimento aleatório de dois indivíduos próximos. Uma maneira de contornar esta questão é supor que cada indivíduo seja considerado como envolto por uma ``auréola'' que passa assim a ser a sua dimensão efetiva; quando duas ``auréloas'' se superpõem, consideramos então que uma colisão de fato já ocorreu. Sendo a difusão um processo dinâmico homogenizador que é, espera-se que uma população sem interferência externa atinja um estado limite em pouco tempo. Neste estado de equilíbrio macroscópico (isto é, com densidade invariante no espaço), o movimento microscópico continua ativo, inclusive para manter a estacionalidade macroscópica.

    Preparado o cenário a ser discutido, consideremos uma população de (``pequenos'') indivíduos de dimensão \(r\) homogeneamente distribuídos segundo uma densidade constante \(\rho\) e, microscopicamente executando movimentos completamente aleatórios e independentes. Assim, discriminando um indivíduo qualquer dentre eles e fixando as coordenadas espaciais neste elemento, os movimentos relativos dos outros indivíduos serão também aleatórios e de intensidade \(D\). Consideremos agora a ``taxa de colisão'' que este indivíduo estacionário recebe, ou seja, o número de indivíduos que colidem com o mesmo por unidade de tempo, e que será designada por \(C\), cuja dimensão deverá ser \([C] = NT^{-1}\), onde \(N\) é o símbolo da dimensão de população.

    A hipótese Física fundamental é que para um cenário semelhante, \(C\) dependerá somente dos valores dos seguintes parâmetros: a densidade, \(\rho\), o raio de interação \(r\) e a intensidade do processo de difusão, que é medido por \(D\), ou seja, \(C = \varphi(\rho, r, D)\) em que \(\varphi\) é uma função matemática ``pura'', i.e., que independe de quaisquer outras dimensões físicas. Sendo \([D] = L^{2}T^{-1}\) e \([\rho] = NL^{-3}\), utilizaremos as unidades intrínsecas: \(L_0 = r\), \(T_0 = r^{2}D^{-1}\) e \(N_0 = \rho r^{3}\), de onde a adimensionalização da relação suposta nos dá:
    \[\dfrac{C}{(\rho r^3)(r^2 D^{-1})^{-1}} = \varphi(1,1,1)\]
    e de onde vem que
    \[C = c_0 \rho r D,\]
    sendo \(c_0 =\varphi(1, 1, 1)\) uma constante matemática, adimensional. Esta fórmula singela mas carregada de significado foi obtida por Marjan Smoluchowski no início do século XX fazendo uso da equação de difusão e sua solução radial, assim como argumentos físico-químicos sutis (V. Levich. Physico-Chemical Hydrodynamics, 1962). A taxa total de colisões \(c\) em toda a população é obtida multiplicando-se \(C\) pelo número total de indivíduos. Em uma região grande, mas finita de dimensão \(l\), teremos \(N_\ast = \rho l^{3}\) indivíduos, o que nos dará
    \[c = (c_0 l^{3}r)D\rho^{2} = c_\ast D \rho^{2},\]
    ou então o número de colisões por unidade de volume: \[c^\ast = (c_0^\ast r) D \rho^2.\]

    Consideremos, agora, duas populações \(A\) e \(B\), de indivíduos perfeitamente misturados entre si, que se movimentam igualmente na escala microscópica na forma aleatória com coeficiente de Difusão \(D\). De acordo com o que foi visto, um individuo \(a \in A\) estacionário, sofre uma taxa de colisões de indivíduos da população \(B\) que pode ser descrita pela seguinte fórmula \[c_A = (c_0 r D)\rho_B.\]

    Considerando, agora, as duas populações ao mesmo tempo em um recipiente finito de dimensão \(l_0\), com \(\rho_A l^{3}\) indivíduos da espécie \(A\), a taxa de colisão total entre elementos de \(A\) e elementos de \(B\) poderá ser expressa como:
    \[\rho_A l^3 c = (c_0 l^3 r D) \rho_A \rho_B = \kappa D\rho_A \rho_B.\]

    Finalmente, esta última forma representa de maneira até mais detalhada o que é conhecido como a ``Lei'' de ação de massas geralmente expressa da forma:

    \begin{quotation}
    ``A taxa de colisões entre duas populações submetidas a um movimento aleatório
    microscópico de mesma intensidade \(D\), e uniformemente distribuídas em uma região, é
    proporcional ao produto das suas respectivas concentrações \(\rho_A \rho_B\) e proporcional também ao coeficiente de difusão \(D\)''.
    \end{quotation}

    O Modelo de Ação de Massas tem vários aspectos ainda pouco estudados. Por exemplo,

\begin{enumerate}
    \item Tratando-se de duas populações biológicas distintas, \(A\) e \(B\) que realizam realizam movimentos Brownianos independentes e de intensidades diferentes, \(D_a\) e \(D_b\), é claro que fixada a atenção em um elemento de uma delas podemos considerar que a outra população realiza uma Movimento Browniano relativo. Entretanto, não é imediatamente claro qual seria o parâmetro efetivo \(D_{eff} = \varphi(D_a, D_b)\) deste movimento relativo.
    \item O Modelo de Ação de Massas é utilizado sob a hipótese implícita de que os indivíduos das duas populações realizam movimentos microscópicos aleatórios independentes e que a distribuição de cada espécie é, pelo menos localmente, estabilizada pelo mesmo processo de difusão. Se a distribuição (densidade) dos indivíduos for localmente variável em uma escala quase microscópica, ou se estiverem submetidos a um processo de transporte determinístico (convecção) não está claro que este Modelo seja igualmente válido. Um estudo clássico da influência de transporte \(v\) em movimentos de difusão é o chamado ``Problema de Taylor'' da dinâmica de fluidos que tem por objetivo obter um coeficiente efetivo de difusão \(D_{eff} = \varphi(D, v)\) (Levich, 1962), Caflisch [...], Evans/Neu[...]). Tais variações são também analisadas em textos de Termodinâmica Irreversível (de Groot, 1962).

    \item Havendo comportamentos ``inteligentes'' da presa e/ou do predador, os seus movimentos microscópicos não podem ser considerados independentes. O Modelo de Ação de Massas terá que, necessariamente, levar este fato em conta, pois um comportamento inteligente é ``caro'' e evolui somente se houver vantagens de adaptação. E, neste caso, comportamento inteligente significa (respectivamente) uma maior ou menor taxa de ``colisões''.
    \item Movimentos Aleatórios não Brownianos, por exemplo, ``Caminhos de Lévy'', caracterizados pela ocorrência de ``vôos'' raros mas de longo alcance, também não fazem parte do cenário típico do Modelo Clássico de Ação de Massas. (Viswanathan [..], Nathan [...]).
\end{enumerate}




\subsection{Tempo de busca por movimento aleatório}

    Consideremos um indivíduo que assume uma estrategia de busca executando apenas um movimento aleatório de intensidade \(D\), independente de qualquer informação, em um meio constituído de ``presas'' fixas a uma densidade \(\rho\). Se a cada encontro uma presa é retirada, é razoável indagar sobre o tempo médio que ele levará para consumi-las todas. Para analisarmos este processo consideremos a dinâmica de retirada das presas que pelo modelo de ação de massas será a equação Malthusiana:
    \[\dfrac{dn}{dt} = -\kappa Dn.\]
    O tempo médio de ``sobrevivência'' de uma presa será \(\dfrac{1}{\kappa D}\) e a probabilidade de uma em particular ser capturada em um intervalo de temo \(t\) é \(p(t) = 1 - e{-\kappa Dt}\).

    Este argumento utiliza a teoria que será desenvolvida no capítulo destinado ao Princípio de Malthus.

%Exercícios:

\begin{exercise}
    Obtenha o tempo médio de encontro da primeira presa por este processo de busca.
    \end{exercise}

    \begin{exercise}
    Analise a dependência deste tempo médio considerando a dimensão do espaço como uma variável.
    \end{exercise}

    Sabe-se, depois de tediosos cálculos matemáticos que um movimento aleatório torna-se um método de busca cada vez menos eficiente com o aumento da dimensão e este fato é de grande importância na busca de informações ``encriptadas'' em espaços de alta dimensão. Processos biológicos que dependem de encontros ``furtivos'' (isto é, decorrentes de movimentos aleatórios) utilizam a estrategia de apenas recorrer a eles em dimensão reduzida. Por exemplo, uma molécula de feromônio que deve encontrar um receptor ``escorrega'' por um filamento unidimensional até encontrar um poro. J. Murray-Nonlinear Differential equations in Biology, Oxford UP 1974. M.Delbruck - Dimension Reduction in Diffusion....





BIBLIOGRAFIA:

Textos:

G.I.Barenblatt-Scaling Phenomena in Fluid Dynamics, Cambridge Univ. Press 1994

T.McMahon\_J.T.Bonner-On Size and Life, W.H.Freeman-Sci. American 1983

S.Mahajan-Street-Fighting Mathematics: The art of educated guessing and opportunistic problem solving, MIT Press, 2012

Geoffrey West-Scale-The Universal Laws of Life, Growth, and Death in Organisms, Cities and Companies, Penguin, 2017

Leitura Auxiliar:

James D. Murray- Simple Models explain 95\% of the phnomena.....

Referências:

R.McN.Alexander-Estimation of speeds of dinosaurs, Nature (1976), 129-130.

R.McN.Alexander-Optima for Animals, Princeton Univ.Press 1996.

Aristoteles- ``Sobre o Movimento``

R.Banks-Towing Icebergs, Falling Dominoes and other Adventures in Applied Mahtematics, PUP2013

G.I.Barenblatt-Similarity, Self-Similarity and Intermediate Asymptotics, Plenum 1979

G.I.Barenblatt-Scaling Phenomena in Fluid Dynamics, Cambridge Univ. Press 1994

G.I.Barenblatt-Dimensional Analysis, G. Breach 1987

G.I.Barenblatt-Self-Similar Intermediate Asymptotics for Nonlinear Degenerate Parabolic Free-Boundary Problems that Occur in Image Processing, Proc.Nat.Acad.Sci. USA 98(23), 12878-81, 2001

G.I.Barenblatt-V.M.Entov-V.M.Ryzhik-Theory of Fluid Flows Through Natural Rocks, Kluwer1990

M. Batty-The New Science of Cities, MIT Press 2014

G.D.Birkhoff-Hydrodynamics: A Study in Logic, Fact and Similarity, Princeton Univ. Press/Dover

G.Bluman-S.Kumei-Symmetries and Differential Equations, Springer 1989

J.T.Bonner-Why Size Matters-From Bacteria to orange Whales,Princeton Univ. Press 2006

J.L.Borges-Conto- O Cartografo que fez uma mapa perfeito-Ref- E.S. de Decca-Jornal da Unicamp : 02-08 junho
2008, pg.08.

D.Brockmann-L.Hufnagel-T.Geisel-The Scaling Laws of Human Travel, Nature 439(26Jan2006), 462-465.

J.H.Brown-G.B.West-editors-Scaling in Biology, Oxford U.Press 2000.

L.Carroll-Euclid and his Modern Rivals, BNoble

J.Case-Book Review-SIAM News 2005-7-11- ~Poincaré \& Einstein-J. Rigden- Two Theories of Relativity-All but Identical in Substance,

Paul Colinvaux-Why big fierce animals are rare, PUP 1970.

Stanislas Dehaene-The Number Sense-How the Mind Creates Mathematics, Oxford UP 1996- ultima ed. 2011

T.Deisboeck-Morphological Instability and Tumour Invasion-arXiv2006 (Harvard University-Complex Systems in Biology)

Max DelbrückM. Denny-Limits to running speed in dogs, horses and humans, J.Exp.Biol. 211 (2008), 3836-3848

S.R.de Groot-P.Mazur-Non-Equilibrium Thermodynamics, North-Holland 1962

Maurice d’Ocagne- Traité de nomographie, Gauthier-Villars. Paris, 1899; 2ª edição, 1921.

F.Dyson-A meeting with Enrico Fermi, Nature 427 (2004)- doi:10.1038/427297a

F.Frenkel-A. de Pace-Visual Strategies-A Practical Guide for Scientists and Engineers, Yale UP 2012

Harald Fritzsch-The Creation of Matter, BB 1984

Galileu Galilei-Dialogues Concerning Two New Sciences, -1637-Trad.

J.Gillooly-J.HBrown-G.B.West-V.M.Savage-E.L.Charnov-Effects of Size and Temperature on Metabolic Rate, Science, vol. 293, (2001), 2248-2251

P.Goldreich-S.Mahajan-S.Phinney-Order-of-Magnitude Physics: Understanding the World with Dimensional Analysis, Educated Guesswork and White Lies,122pg.- MIT Online 2010.

N.Goldenfeld-O.Martin-Y.Oono-Intermediate asymptotics and renormalization group theory, J. Scient. Comp. 4 1989 355-372

J.Gordon-Strutures: Why Things dont Fall Dawn,

J.Harte-Consider a Spherical Cow, 1988

J.Harte-Consider a Cylindrical Cow,

W.Jetz-C.Carbone-J.Fulford-J.H.Brown-The Scaling of Animal Space Use, Science 306(08OcT\_{2}004), 266-268

J.Jun-J.W.Pepper-V.M.Savage-J.F.Gillooly-J.H.Brown-Allometric scaling of ant foraging trail networks, Evol.Ecol 5 (2003): 297-303

J.B.Keller-A Theory of Competitive Running, Physics Today 26 1973, 42-46.

A.N.Kolmogorov-Selected Works....

A.Kossovsky-Benford‘s Law in Forensic Fraud, WSP 2014

B.Kuipers-Qualitative Reasoning-Modeling and Simulating with Incomplete Knowledge, MIT 2004

B.Lautrup-Tsunami Physics, pp.Feb.2005

B.Lautrup-Physics of Continuous Matter, IoP Press 2005.

M.Levy-Why a Cat always land on his feet, Princeton UP

Rogerio C. de Cerqueira Leite-O etanol e a solidão das vaquinhas brasileiras, Folha de São Paulo (quando RCCL tiha 76 anos).

C.C.Lin-L.A.Segel-Mathematics Applied to Natural Sciences, SIAM 1990

P.Lockhart-Measurements, HarvUP2012

J.D.Logan-Similarity Solution to a Heat Exchange Problem, SIAM Rev. 40(4), (1998), 918-921.

L.Mahadevan- artigos diversos- v. HomePage-Harvard Univ.-

L.Mahadevan \& M.Bandi-A pendulum in a flowing soap film, Phys.Fl. 25 (2013), 041702

L.Mahadevan \& J.Kim-The Hydrodynamics of Writing with Ink, Pr.Roy.Soc. 107, 2011, 264501

S.Mahajan-Street-Fighting Mathematics: The art of educated guessing and opportunistic problem solving, MIT Press 2012

T.McMahon-Rowing:A Similarity Analysis, Science Mag. 173 july 1971, 349-351

T.McMahon\_J.T.Bonner-On Size and Life, W.H.Freeman-Sci. American 1983.

T.M.McMahon-Scaling Physiological Time, pg. 131-163, Lect on Appl.Math.-AMS-vol 13, 1980.

T.M.McMahon-Muscles Reflex and Locomotion, Princeton Univ.Press 1984.

B.Mandelbrojt-The Fractal Geometry of Nature, WHFreeman 1982

J.Mazur-The Motion Paradox: The 2500 years old puzzle behind the Mysteries of Time and Space, Dutton 2007

J.C.Maxwell-A Treatise on Electricity and Magnetism, 1873

Andreas Nieder-A Brain for Numbers-The Biology of the Number Instinct, MIT Press 2019

P.J.Nahin-In praise of simple Physics-The Science and Mathematics of Everyday Questions, PUP2016

K.J.Niklas-Plant Allometry-The Scaling of Form and Process, Univ. of Chicago Press 1994.

L.V.Ovsiannikov-Group Analysis and Differential Equations, Academic Press 1982

T.J.Pedley-ed.-Scale Effects in animal locomotion, Academic Press 1977.

R.Phillips-R.Milo-A feeling for the numbers in biology, PNAS 106(51), (2009), 21465-21471

Platão-Parmenides-Zeno

B.E.Pobedrya-D.V.Georgievskii-On the Proof of the Pi-Theorem in Dimension Theory, Russ.J.Math.Phys. 13(4), 2006, 431-37.

H.Poincaré-La Science et l’Hypothese, 1902-Science and Hypothesis, Dover 1952- Ciencia e Hipótese, UnB 1987

James F. Price-Lectures on Dimensional Analysis of Models and Data Sets-Similarity Solutions and Scalling Analysis-online- Woods

Hole Ocean.Inst. 2006.

E.Purcell-Life at Low Reynolds Number, Am.Journal of Physics 45, 1977, 3-11

L.Raleigh-The Principle of Similitude, Nature 95 (1915) 66-68

V.M.Savage-J.Gillooly-J.HBrown-G.B.West-E.L.Charnov-Effects of Body Size and Temperature on Population Growth, Am.Natur. 163(3), (2004), e–

V.M.Savage-J.Gillooly-W.H.Woodruff-G.B.West-A.P.Allen-B.J.Enquist-J.HBrown-The predominance of quarter-power scaling in biology, Funct.Ecol. 18 (2004), 257-282.
K. Schmidt-Nielsen-Scaling:Why is Animal Size so Important?, Cambridge UP 1984.

L.Sedov-Similarity and Dimensional Methods in Mechanics, J.Wiley 1971.

R.Strichartz-Evaluating Integrals by Self-Similarity, AmMathMonthly 104(4), 2000, 316-326.

T.P.Smith-How Big is Big and How Small is Small-The Size of everything and Why, OUP2014

C.Swarz-Back of the Envelope Physics, J.Hopkins Univ Press 2003

A.A. Sonin-The Physical Basis of Dimensional Analysis- MIT Open Course Ware-OCW-Fluid Dynamics-Lectures online

I.Stewart-Counting the Pyramid Builders, Sci.Am. Sept. 1998, 76-78.

G.I.Taylor-The formation of blast wave by very intense explosion, Proc.Roy.Soc. A 201 (1950), 159-186.

J.C.Taylor-F.Kersen-Huygen’s Legacy:The Golden Age of the Pendulum Clock, Fromanteel 2004

D’Arcy W.Thompson-On Growth and Form, Cambridge UP 1917.

E. Van den Eijnden-Lectures online: Introduction to Math Modelling-Dimensional Analysis and Scaling, New York Univ.-online

S.Vogel-Comparative Biomechanics, Princeton UPress 2003

K. von Tritz-The discovery of incomensurability, Ann Math 46(1954), 242-264.

Jakob von Uexküll-Dos Animais e dos Homens: Digressões pelos seus mundos próprios. Doutrina do Significado.(trad.) Lisboa, 1982.

R.WehnerL.Weinstein\_J.Adam-Guesstimation: Solving world’s problems in the back of a cocktail napkin, Princeton UP 2008.

L.Weinstein-Guesstimation 2.0, Princeton UP 2008.

D.Weintraub-How old is the Universe? PUP2010

Weisberg, M. - Simulations and Similarity: Using Models to Understand theWorld, Oxford UP 2013

V.Weiskopf-Search for Simplicity: Mountains, waterwaves and leaking ceilings, Am.J.Phys. 54(2) 1986, 110-111.

B.J..West-Beyond the Principle of Similitude, J.Appl.Phys. 60 (1981) 1989-97.

Geoffrey West-Scale-The Universal Laws of Life, Growth, and Death in Organisms, Cities and Companies, Penguin 2017

G.B.West-J.H.Brown-B.J.Enquist-A General Model for the Origin of Allometric Scaling Laws in Biology, Science 276 (04 April 1997), 122-126.

A.B.Herman-V.M.Savage-G.B.West-A Quantitative Theory of Solid Tumor Grwoth, Metabolic Rate and Vascularization, PLoS One 6 (2011)

G.B.West-J.H.Brown-B.J.Enquist-Growth models based on first principles or phenomenology?, Funct.Ecol. 18 (2004), 188-196.

G.B.West-J.H.Brown-B.J.Enquist-A general model for structure and allometry of plant vascular systems, Nature, (1999), (400), 664-667.

G.B.West-J.H.Brown-B.J.Enquist-The Fourth Dimension of Life:Fractal Geometry and Allometric Scaling of Organisms, Science 284 (1999), 1677-1679.

H.Whitney-The Mathematics of Physical Quantities: I-Am.Math.Monthly(1968), 113-, II- AMM(1968), 227-

H.S.Wu-Estimation- chap. 10-Numbers in Elementary School, AMS 2011

Ya.B.Zeldovich-Yu.P.Raizer-Physics of Shock Waves and High Temperature Hydrodynamics, A.Press 1966

Ya.Zeldovich \& al. The Almighty Diffusion, WSP1989.

G.K.Zipf-Human Behavior and the Principle of Least Effort, Addison-Wesley 1949

Anotações:

Harald Fritzsch- The Creation of Matter BB 1984 pg.165:'' Decay of a Proton: A proton lives for at least 1030 years...How this figure was arrived at since the earth is only about 5 billion years old? The thing is...we do not observe just one, but many protons...a block of iron is composed of about 1030 protons and 1030 neutrons....The reason of the existence of human life is proof that proton enjoy a long
life..Our body contais about 1028 protons...If...the human body could not resist.''.

Richard Feynman -( ~Physics World -Book Review - V.Vedral-Decoding Reality-25Jan2011)



\section{Apêndice}


    \href{https://pt.wikipedia.org/wiki/Alan_Turing}{Alan M. Turing}: ``This model will be a simplification and an idealization, and consequently a falsification''. Primeira linha de um dos mais importantes artigos de Biomatemática (e de Matemática Aplicada) que trata de um modelo de Morfogênese, 1950.)

    \href{https://pt.wikipedia.org/wiki/Giordano_Bruno}{Giordano Bruno}: ``Magia consiste no conhecimento do Universo e na representação deste conhecimento por intermédio de símbolos'' (Modelos Matemáticos)

    \href{https://pt.wikipedia.org/wiki/Jakob_von_Uexk\%C3\%BCll}{Jakob von Uexküll} - ``Umwelt'' - Thure von Uexküll: ``Frase aproximadamente correta, mas inteligível (artigo revista Galaxia-PUC-SP-2001) Anonimo: Frase Gramaticalmente correta e completamente sem sentido.

    A Magia de \href{https://pt.wikipedia.org/wiki/Ren\%C3\%A9_Descartes}{René Descartes}: A Representação/Descrição do Modelo Mental de Espaço (Umwelt) por intermédio de símbolos e da estrutura algébrica (\href{https://pt.wikipedia.org/wiki/Fran\%C3\%A7ois_Vi\%C3\%A8te}{François Viète})

    A Síntese Cognitiva de Gaetano Kanizsa: O Conteúdo Encapsulado na Forma-Completamento visual triângulos - Curva vs Conjunto de Pontos Esparsos (Nem todo

    A Síntese Funcional de \href{https://pt.wikipedia.org/wiki/Galileu_Galilei}{Galileu Galilei}: A descrição de uma tabela de dados experimentais por intermédio de uma função elementar (Completamento ``visual'') - Estrutura Matemática utilizada para a representação do Fenômeno: Funções Elementares - (Extensão da Estrutura Funcional: Newton, \href{https://pt.wikipedia.org/wiki/Paul_Dirac}{Dirac}-\href{https://pt.wikipedia.org/wiki/Sergei_Sobolev}{Sobolev}.

    A Síntese Diferencial de \href{https://pt.wikipedia.org/wiki/Leonhard_Euler}{Leonhard Euler} - \href{https://pt.wikipedia.org/wiki/John_Graunt}{Graunt} - \href{https://pt.wikipedia.org/wiki/Christiaan_Huygens}{Huygens} e o Modelo de ``Malthus'' - Síntese Funcional de Dados e a Caracterização Diferencial da Função - Redescoberta de \href{https://pt.wikipedia.org/wiki/Ernest_Rutherford}{Rutherford} (Decaimento Radioativo)

    \href{https://pt.wikipedia.org/wiki/Isaac_Newton}{Isaac Newton}: A Síntese Diferencial (Eq. Difer.) como Método de construção de Modelos Matemáticos Funcionais- Aritmética Infinita-Analogia com a construção decimal-(Tycho Brae-Kepler-Newton)

    Stanislau Ulam - Representação digital (algoritmo) de Fenômenos- Autômatos Celulares- Stephen Wolfram - The New Science

    Galileu e Ganizsa redivivos: Construção de Modelos Contínuos Minimais e Consistentes com Dados Massivos-DDM Equation Free Models
    
    Princípio de Ockham-Gombrich-Segel: Representação Minimal (Caricatural) de Fenômenos Naturais.

    E. Schumacher- Reconhecimento da Incompletude Circunstancial \& Essencial da Representação do conhecimento- Bayes \& Jaynes vs. Amaldi-Bohr.

    P. Davis-R.Hersh - ``Para os matemáticos o número Pi é mais concreto do que pedra de Gibraltar, pois, ao contrario desta, a sua existência pode ser demonstrada'' (~Mario Saad)

    John von Neumann: ``Science does not explain, it hardly even interprets. Science places models above all''.

    Richard Feynman: ``Mathematics is a language that carries a logic ([structure]) within''.

    Crises de Consistência de Estruturas Numéricas: 1) Incomensurabilidade, 2) Produto entre números negativos, 3)Números complexos

\subsection{Introdução}

    O tema a ser tratado nesta seção é o conceito de Modelo Matemático que é, acertadamente, considerado exterior ao corpo da Matemática e, por este motivo, solenemente ignorada em sua literatura. Todavia, os Modelos Matemáticos constituem uma área subjacente à fronteira da Matemática, e é através destes que ela recebe a sua mais profícua influência e também demonstra boa parte de sua relevância.
    
    Pelo outro lado desta mesma fronteira, o tratamento deste tema também ocorre com infrequência na literatura de Matemática Aplicada, mas, neste caso, inexplicavelmente, considerando-se que a existência desta está totalmente fundamentada na formulação aceitável de Modelos Matemáticos.
    
    Portanto, não é de se admirar que a compreensão deficiente do conceito de Modelo Matemático tenha sido uma das fontes mais notórias da dificuldade encontrada no entendimento tanto da natureza da própria Matemática quanto de seu caráter instrumental, o que é comprovado de forma notável em diversas instâncias na longa história de ambas.
    
    O presente capítulo introduz alguns aspectos elementares do conceito de Modelos Matemáticos acompanhados de exemplificações históricas que são fundamentais para um apanhado preliminar do tema.

\subsection{A Gênese comum dos Modelos Matemáticos e da própria Matemática: O Processo de Mensuração de Fenômenos}

\begin{quote}
    ``Em Matemática não se sabe sobre o que e nem do que está se tratando, a Matemática trata apenas de relações entre símbolos mudos''.
    
    \rightline{\href{https://pt.wikipedia.org/wiki/Bertrand_Russell}{Bertrand Russell}}
\end{quote}

    Uma Estrutura Matemática é constituída de um conjunto de Elementos Simbólicos dotado de uma estrutura definida por operações e relações entre eles que obedecem a um sistema de axiomas \textbf{consistentes}, (isto é, afirmações que não admitem deduções contraditórias a partir delas), e que, segundo Russell, em princípio, não se referem a objetos concretos específicos e nem a fenômenos exteriores.
    
    A Estrutura Matemática com origem mais antiga, encontrada desde tempos imemoriais e utilizada nas mais variadas situações, é a Estrutura Aritmética dos Números Naturais (\(\mathbb{N}\), em que está definida a simples operação de adição \(+\)) desenvolvida a partir da necessidade de mensuração do tamanho de ``Populações'', ou seja, a cardinalidade de conjuntos.
    
    O conceito matemático de Números Naturais também se constituiu na pedra angular a partir da qual todas as Estruturas da Matemática foram construídas, a começar pela Estrutura de Números Reais \((\mathbb{R}, +, \cdot)\) que consiste em uma parte algébrica relacionada às operações de soma e produto, acrescentada da Topologia de convergência.
    
    A história da concepção e do desenvolvimento das primeiras Estruturas Numéricas (\(\mathbb{N}\), \(\mathbb{Q}\) e \(\mathbb{R}\)) se confunde com a própria emergência das civilizações humanas (Egito, Babilônia e etc.) e está intimamente relacionada à formulação de Modelos Matemáticos de Mensuração de Populações/Cardinalidade (\((\mathbb{N}, (\mathbb{Q}\)) e de Extensões Lineares/Comprimentos (\(\mathbb{Q}, \mathbb{R}\)).

\subsection{Modelo Matemático: a visão esotérica de Giordano Bru\-no (1548-1600) a sua aplicação concreta na Geometria Analítica de René Descartes (1596-1650) e a sua explicação cognitiva por Jakob von Uexküll (1864-1944)}

\begin{quote}
    ``Magia consiste no conhecimento do Universo e na representação deste conhecimento por intermédio de símbolos''.
    
    \rightline{\href{https://pt.wikipedia.org/wiki/Giordano_Bruno}{Giordano Bruno}}
\end{quote}

    Uma introdução ao conceito de Modelo Matemático que abrange a maioria dos casos estudados no presente texto será apresentada a seguir na forma de uma definição sintética e geral que se tornará mais inteligível quando exemplificada por situações especificas e elementares.

\begin{definition}
Um Modelo Matemático é uma \textit{Estrutura Matemática associada a uma Interpretação} que a identifica com um Conjunto de Aspectos (``traços'') considerados relevantes e suficientes para a descrição de um \textit{Fenômeno Natural} (isto é, exterior à Matemática).
\end{definition}

    O primeiro aspecto notável desta definição tem um sentido negativo quando ressalta que um Modelo Matemático não é constituído apenas de uma Estrutura Matemática (pois isto o tornaria uma parte interna da Matemática), mas inclui de forma indispensável uma Interpretação que a relaciona exteriormente e de forma explicita a um respectivo Fenômeno Natural. Na verdade, uma mesma Estrutura Matemática pode ser parte de inúmeros e distintos Modelos Matemáticos, a depender da Interpretação que lhe é conferida.
    
    A essência de um Modelo Matemático é uma representação Matemática (simbólica, conforme Giordano Bruno) para a descrição (na verdade, bem restrita) de um Fenômeno Natural segundo alguns poucos aspectos que o caracterizam suficientemente para o proposito em questão. O objetivo desta representação reduzida é estudar o referido Fenômeno em sua representação dentro de uma Estrutura Matemática.

    Uma analogia interessante deste procedimento pode ser feita com a técnica de microcirurgia remota com a qual um médico especialista manipula um sistema (Estrutura) simbólico (tela do monitor) que representa, segundo uma interpretação bem estabelecida, aspectos da anatomia sob intervenção (o Fenômeno Natural). O aparelho mediador faz o papel do Modelo Matemático.

    Ressalte-se, (em acordo com Richard Feynman), que o Modelo Matemático, tal como a instrumentalização cirúrgica, é mais do que uma linguagem descritiva porque as manipulações efetuadas nele são possíveis apenas dentro da estrutura do aparelho. Ainda (agora de acordo com Segel) os aspectos anatômicos apresentados na tela do monitor são reduzidos a um mínimo necessário. A incompletude de informações é, por um lado resultado de uma restrição natural, mas, por outro, é também uma necessidade prática para evitar um desvio de atenção para detalhes sem importância.

    A Geometria Analítica desenvolvida por René Descartes foi o primeiro exemplo moderno de Modelo Matemático a representar o ``Espaço Físico Euclideano''. A Geometria Analítica não apenas permite representar, pontos, retas e planos em termos de uma linguagem simbólica, mas também manipular ``remotamente'' estruturas do Espaço Físico (no caso, constituída por pontos, retas, planos, curvas, superfícies e sólidos) com a utilização de símbolos da Estrutura de Espaços Vetoriais \(\mathbb{R}^n\) e da Álgebra elementar. (Segundo a importante teoria cognitiva do biólogo teórico Jakob von Uexküll (1864-1944), e ao contrário do que normalmente se afirma, o ``Fenômeno'' representado pela Estrutura Matemática da Geometria Analítica é um ``Espaço Ecológico'' determinado pela cognição humana, ou seja, um ``Modelo Mental'' (Umwelt, segundo von Uexküll) que foi definido por Euclides. O matemático K.F. Gauss já suspeitava ao final do século XVIII que o Umwelt Euclideano era apenas um Modelo útil para o espaço físico, (como diria George Box), mas poderia ser diferente, o que ele tentou, sem sucesso, demonstrar. A Geometria Analítica desenvolvida segundo o Umwelt de um caranguejo ou de um peixe seria bem diferente. Observe que para um peixe que vive nas imediações da superfície d’água, uma ``Reta'' definida como o trajeto de um raio de luz que conecta dois pontos (que é exatamente a ideia utilizada por Euclides) é uma ``linha quebrada'' por refração nesta superfície. Uma espécie de peixe que adotasse (por teimosia ideológica) o Modelo Euclideano de Espaço, teria a sua sobrevivência em sério risco diante dos erros de percepção para a aproximação de um pássaro predador. Se a teoria de von Uexküll fosse conhecida à época de Gauss, certamente a introdução da chamada Geometria não-Euclideana na Matemática dita Pura, teria se antecipado por muitas décadas.)

    A Biologia Matemática, ou Biomatemática, consiste em uma ``Matematização da Biologia'' , ou seja, é uma metodologia para estudar um Fenômeno biológico segundo as propriedades da Estrutura Matemática que o espelha por intermédio do respectivo Modelo escolhido para representá-lo.

    É importante ressaltar que a ponte que conecta ``interpretativamente'' uma Estrutura Matemática e um Fenômeno Natural tem duas vias e o caminho inverso é também possível de ser percorrido, isto é, a ``Biologização da Matemática'' (ou, Matemática Biológica) que consiste em uma metodologia para estudar uma Estrutura Matemática segundo o conhecimento da Biologia do Fenômeno que é representado por ela. Exemplos da aplicação desta Metodologia que tem produzido resultados notáveis e já amplamente utilizados na tecnologia de informática são os Algoritmos de Otimização de Caminhos (``AntColony'') e de Reconhecimento de padrões (Redes Neurais).

    A teoria de von Uexküll sobre a representação mental que os organismos desenvolvem internamente para a sua interação com o ambiente (Umwelt) permite uma melhor compreensão do conceito de Modelo Matemático.

\subsection{O Elogio da simplicidade inevitável e sua Utilidade: Ockham vs. Michelângelo}

    Um Modelo Matemático de Malthus para a demografia simples de uma População consiste na observação da Cardinalidade do conjunto de seus indivíduos que, neste caso, é o único ``traço descritivo'' que nos interessa conhecer (e representar) sobre ela. (A idade, sexo, peso, nacionalidade, grau de infecciosidade, etc. , etc., etc., dos indivíduos desta população não fazem parte dos ``traços descritivos'' de interesse para o Modelo de Malthus).

    O ``Conjunto de aspectos/traços considerados relevantes para a descrição de um Fenômeno Natural" pode parecer extremamente radical na maioria dos exemplos, mas de qualquer forma é importante ressaltar o inevitável: Em nenhuma circunstancia este conjunto de aspectos descritivos poderá ser exaustivo e completo. (Obviamente, pode-se descrever muita coisa sobre uma população, mas não ``tudo'', e esta limitação não é uma causada por uma deficiência da linguagem utilizada na descrição (seja ela linguística ou matemática), mas, inescapavelmente, decorrente da limitação intelectual humana.

    Parafraseando \textbf{Lee Segel}, (que parafraseou Ernst Gombrich com respeito à arte) ``Um Modelo é apenas uma caricatura constituída de poucos traços que enfatizam aspectos de interesse do objeto em observação'' ou ainda, ``Um Modelo Matemático é uma mentira (por omissão) que nos ajuda a entender parte da verdade''.

    Uma das tarefas mais difíceis na construção de um Modelo Matemático (como de uma caricatura) é estabelecer o conjunto de informações (traços) que são suficientes para produzir uma descrição ``útil'' do Fenômeno, desprezando assim, explicita ou implicitamente toda a infinidade de outras informações, disponíveis ou não. (Segundo o matemático Mark Kac: ``É importante saber descartar corretamente as informações, ou seja, é importante saber como não lançar fora o bebê junto com a água do banho'').

    Há duas maneiras metodológicas para proceder neste empreendimento que, caracteristicamente, é processado em várias etapas.

    O primeiro método encara a tarefa de desenvolvimento de um Modelo Matemático para a descrição de um determinado Fenômeno Natural com um procedimento ``de Cima para Baixo'' (Metodologia ``\textit{Top-Down}'', segundo Segel). Neste caso, incorpora-se deliberadamente e de saída todas as informações disponíveis e imagináveis do Fenômeno estudado em um super Modelo para, posteriormente, descartar as informações supérfluas até que encontremos um Modelo minimalista e suficiente para os nossos propósitos. Este procedimento é, obviamente, impossível de ser estritamente implementado, se não por falta de conhecimento completo das informações sobre o Fenômeno, como também pelo enorme trabalho que envolveria a simplificação do mostrengo resultante. (A metodologia \textit{Top-Down} poderia ser associado ao nome de Michelangelo a quem é atribuída a frase sobre sua escultura de David: ``\textit{A imagem já estava no bloco rustico de mármore que me foi entregue, a minha função foi apenas aquela de retirar as partes supérfluas}''). Em Matemática, Aplicada ou não, os \textit{Michelangelos} são ainda mais raros do que na escultura!

    É importante ressaltar, todavia, que os Métodos de Redução de Modelos são técnicas matemáticas indispensáveis na simplificação e ajuste de Modelos que, mesmo incompletos podem ser excessivamente detalhistas para o objetivo em vista. Os Métodos de Redução serão abordados em um capítulo à parte e fazem parte essencial dos instrumentos matemáticos necessários para a construção de Modelos Matemáticos.

    Resta-nos, portanto, a segunda alternativa que encara a tarefa de construção de um Modelo Matemático de ``Baixo para Cima'' (ou, ``\textit{Bottom-Up}'', como dizia Segel). Neste caso, partimos de um Modelo drástica e deliberadamente simplificado (às vezes chamados de ``\textit{Toy Models}'') em que se preserva apenas uma semelhança fundamental com o Fenômeno estudado, para, posteriormente, acrescentarmos, passo a passo, se necessário, informações ao Modelo original tornando-o sucessivamente mais representativo, até o ponto que seja satisfatório para os fins a que se destina. Esta atitude é construtiva e permite um maior controle sobre cada passo uma vez que se inicia com Modelos que, em tese, são matematicamente analisáveis e cujas discrepâncias com a ``Realidade'' podem ser diagnosticadas, e resolvidas, com maior facilidade. O termo ``\textit{Toy Model}'' é apropriado para este procedimento lembrando-se que muitas questões científicas tem, de fato, o seu início em curiosidades meramente lúdicas. (J. Huisinga - ``Homo Ludens'').

    Modelos extremamente simples, tanto pela sua estrutura matemática elementar quanto pela descrição deliberadamente caricatural do Fenômeno estudado, podem ser de grande utilidade para a sua compreensão se elaborados com perspicácia e conhecimento do tema em questão. \textbf{James Murray}, um dos iniciadores da moderna Biomatemática (e autor de um de seus textos mais influentes - Mathematical Biology, Springer - 1a. ed. 1989) argumentou de maneira convincente e com autoridade sobre este fato em uma memorável conferência plenária da European Soc. for Theor. and Math. Biology-ESMTB, em 2005 (Dresden, Alemanha). O conteúdo desta conferencia foi posteriormente publicada na forma de artigos. (Refer. de Leitura). Segundo ele: ``\textit{95\% or more of a phenomenon can be explained with very elementary mathematics, while adding more explanation of it usually requires an exponentially increasing amount of Mathematics which may not even be available}''.

    Neste ponto, é válido citar o famoso e antigo \textbf{Princípio de Parcimônia} do monge inglês \textbf{William Ockham} (séc. XIII) expresso pela imponente frase em latim ``\textit{Pluralitas non est ponenda sine neccesitate}'' que nos adverte sobre a importância de evitar a tentação em acrescentar complicações além do necessário. Em outras palavras, poderíamos dizer que a Natureza já é suficientemente complicada e não necessita de nossa ajuda voluntária neste sentido; o nosso papel é atuar exatamente no sentido oposto, ou então iremos nos deparar rapidamente com uma barreira intransponível, bem antes de obter qualquer informação útil sobre o Fenômeno.


\subsection{As estruturas numéricas: a gênese de modelos matemáticos em procedimentos de mensuração de cardinalidade e comprimento}

\begin{quote}
    ``What is man that can understand the concept of number, and what is number that man can understand it''.

    \rightline{W. McCulloch}.
\end{quote}


    As Estruturas mais antigas e importantes da Matemática são, sem dúvida, as Estruturas Numéricas (\(\mathbb{N}, \mathbb{Q}\)) que tiveram a sua origem em problemas práticos do cotidiano relacionados ao conceito de mensuração. Dentre todas as estruturas matemáticas utilizadas como Modelos Matemáticos, a Aritmética dos números naturais é a mais fundamental e presente em todas as civilizações humanas.

    A avaliação do tamanho de uma população, ou de um conjunto de objetos, é uma habilidade cognitiva fundamental para que um organismo adquira conhecimento do seu ambiente (\textbf{Jakob von Uexkull}) e possa avaliar rapidamente os perigos e oportunidades que se lhe apresentam. Esta habilidade é observada até mesmo no comportamento de alguns animais e aparentemente está impressa na fisiologia inata do homo sapiens (S. Dehaene, A. Nieder).
    
    (Bactérias talvez não disponham de Estruturas Matemáticas de contagem, mas certamente possuem um mecanismo de avaliação de populações de seus co-específicos que a circundam. Um comportamento denominado de ``\textit{quorum sensing}'' foi recentemente observado em microrganismos (pseudomonas aerosas") que somente atacam quando há um número mínimo de indivíduos que possa garantir o sucesso da empreitada. O mesmo ocorre com certos predadores como em alcatéias de lobos diante de presas maiores.)

    A Aritmética dos números naturais é a mais antiga e útil dentre todas as estruturas matemáticas e a primeira aprendida nas escolas. (Matematicamente, a estrutura de \(\mathbb{N}\) é o fundamento de todas as outras estruturas numéricas e de grande parte da Matemática. Segundo L. Kronecker: ``Os números naturais foram dádivas divinas e perfeitas, o resto são construções, imperfeitas, dos homens'').

    A concepção e o aprendizado da Aritmética na escola elementar é um fenômeno cognitivo notável que se faz por intermédio de uma abstração inconsciente de processos concretos de contagem, digamos, de laranjas. A operação de soma é interpretada facilmente como uma manipulação especifica com ``populações de laranjas'' e se tornam de tal forma entranhadas que mal distinguimos o conceito de Modelo Matemático que se forma inconscientemente.

    De alguma maneira, ainda misteriosa e intensamente estudada nos últimos anos (Dehaene, Nieder), a criança percebe que o procedimento de contagem não se restringe aos objetos concretos utilizados para representá-la, (laranjas), mas pode ser abstraído na forma de uma estrutura matemática abstrata subjacente ao processo. Este salto de conceptualização é, certamente, um dos mais significativos na história da civilização e na Psicologia do aprendizado infantil, pois nos permite utilizar uma mesma estrutura abstrata para enumerar e ordenar objetos e situações que nunca foram, nem poderiam ser, mencionadas ou previstas durante o seu aprendizado. Em outros termos, com esta abstração, felizmente não precisamos aprender uma Aritmética para laranjas, outra para contar caixas ou caminhões de laranjas, outra para bolinhas de gude, outra para dinheiro,..., e outra para contar quantas Aritméticas já aprendemos. Além disso, também não necessitamos de um manual de aplicação da Aritmética, pois imediatamente percebemos as circunstâncias em que podemos aplicá-la.

    A compreensão da operação de multiplicação entre números naturais exige do aprendiz um entendimento (pelo menos inconsciente) desta abstração pois, a representação ``fenomenológica'' desta operação envolve a mesma representação numérica (Modelo Matemático) para objetos matemáticos distintos dentro de uma mesma manipulação. Por exemplo, ``\(3\) \textbf{caixas} (de Laranjas), cada uma com \(5\) \textbf{laranjas}'' utiliza os números \(3\) e \(5\) em referência a classes distintas de objetos para introduzir o conceito do produto \(3 \cdot 5\). Portanto a operação produto somente é, de fato, apreendida após a abstração do conceito de numero.

    A representação decimal (que tem origem meramente anatômica) faz uso desta abstração repetidas vezes: Inicialmente trata de conjuntos de \textbf{Laranjas} individuais até dez, depois de conjuntos de \textbf{Caixas} de dez laranjas, depois de conjuntos de ``\textit{\textbf{Containers}}'' de dez caixas de laranjas e daí por diante.

    Estes fatos conceituais, obviamente, não devem ser ``ensinados'' ao aprendiz infantil (como tentou a defunta ``Matemática Moderna'' dos anos 1960-1980), mas o professor deve ser completamente consciente deles para agir de forma adequada em sua missão, o que nem sempre ocorre.

\subsection{A análise dimensional: a formalização dos modelos matemáticos de mensuração baseados em estruturas numéricas}

    A construção de um Modelo Matemático \textbf{quantitativo} (isto é, baseado na estrutura de números) é um procedimento que depende de uma escolha arbitrária, mas indispensável, que fica a cargo do observador: a Unidade \textbf{Padrão} de tamanho.

    Por exemplo, Lotes (caixas) para a medida de Conjuntos, um Segmento Padrão de referência para a medida de Comprimento linear, um Quadrado padrão para a medida de área, um Peso padrão para a medida de massa, o Período de oscilação de um pendulo (batida do pulso, o movimento do Sol, ou uma oscilação de um átomo de césio) para a medida de tempo e etc..

    Além desta escolha arbitrária inicial da Unidade padrão, o processo de mensuração exige a descrição de um procedimento "pratico" para a comparação entre o objeto a ser medido e o objeto padrão. Estes dois fatos demonstram que a mesuração não é um processo estritamente matemático, mas é intimamente ligada ao fenômeno estudado e se encontra na fronteira entre as duas ciências envolvidas, no caso da Biomatemática, entre a Matemática e a Biologia.

    Explica-se assim, pelo menos parcialmente, porque este tema nunca é sequer mencionado em textos de Matemática. Entretanto, apesar de sua importância óbvia para a Matemática Aplicada, é notável que nem sempre ele é se quer mencionado em alguns textos do assunto e, frequentemente, tratado com menos ênfase e detalhes do que o necessário para a sua compreensão. (Para uma notável exceção consulte C. C. Lin - L. A. Segel - \textit{Mathematics Applied to Natural Sciences}, SIAM 1990).

    A \textbf{Análise Dimensional} é o estudo do processo de mensuração e, especialmente, do efeito que a escolha arbitrária de unidades tem na constituição de um Modelo Matemático quantitativo. O seu estudo é um pre-requisito indispensável para o entendimento de qualquer ciência quantitativa e será o tema central do presente capítulo.

    A Matemática utilizada na exposição da Análise Dimensional (Aritmética e elementos de Álgebra Linear) é considerada elementar e isto a torna, no julgamento de alguns, indigna de ser apresentada em textos mais sisudos. Entretanto, como veremos mais adiante, há uma série de resultados surpreendentes e profundos decorrentes de sua aplicação que por vezes se bastam para a compreensão de um Modelo Matemático. (Barenblatt, West, MacMahon).

    Estruturas Matemáticas baseadas em conjuntos de objetos matemáticos distintos mensurados com Números Reais, tais como Espaços Vetoriais \(\mathbb{R}^n\), Matrizes, Funções de variável real e valores reais e etc., são amplamente utilizadas na construção de Modelos Matemáticos para a representação dos mais diversos fenômenos naturais e sociais. Por exemplo, a população humana segundo o modelo etário de Euler é representada matematicamente por um vetor \(p = (p_{1}, p_{2}, \ldots, p_{n}) \in \mathbb{R}^n\) em que a coordenada \(p_k\) representa o ``tamanho'' da subpopulação com idade entre \(k\) e \(k + 1\), para, digamos, \(n = 100\). Se, por outro lado o conjunto de aspectos descritivos da população envolve também a medida do grau de infecção dos indivíduos (digamos, medida pela população de anticorpos em uma amostra padrão de volume de sangue), então esta população é matematicamente representável por uma matriz \(p_{kj} \in \mathbb{R}^{nm}\), onde o elemento \(p_{kj}\) representa o ``tamanho'' da população que tem idade entre \(k\) e \(k + 1\) e apresenta um grau de infecção entre \(j\) e \(j + 1\).

\subsection{A análise dimensional com unidades não uniformes: ``lei'' de Miller e ``Lei'' Weber-Fechner}

\begin{quote}
``Ladies and gentleman. I have been persecuted by the number 7, plus or minus two all my life!'' (Introdução a uma histórica conferencia plenária da Am. Psychol. Soc. em 1956 pelo seu presidente).

\rightline{George A. Miller (1920-2012).}
\end{quote}


    A Análise Dimensional tem por \textbf{objetivo formal} atribuir uma medida a determinados fenômenos naturais e tem por utilidade estabelecer uma percepção de ``tamanho'' que é importante ecologicamente (cognitivamente). O procedimento ingênuo de ``contagem'' atribui um número natural a uma população determinado por uma bijeção entre ela e um conjunto padrão que o representa. Portanto, para representar o ``tamanho'' de uma população bastaria, em princípio, atribuir um símbolo distinto a cada um destes conjuntos padrões. Entretanto, a memória humana é curta e, de acordo com a Psicologia, uma pessoa normal tem grandes dificuldades de lembrar mais do que sete itens (lembram-se do jogo dos sete erros?) que lhes são apresentados e, portanto, o procedimento ``ingênuo'' não teria qualquer utilidade cognitiva. Por outro lado, segundo experiencias de Ernst H. Weber (1795-1878) e Gustav Th. Fechner (1801-1887) a percepção sensorial humana (seja ela a visão, tato, audição, olfato, paladar e a cognição de cardinalidade) não percebe (isto é, reage a) variações absolutas de medida de intensidade de um estímulo (que usualmente é proporcional a sua energia), mas apenas a variações logarítmicas acima de um limiar. (Por exemplo, uma andorinha só, a mais ou a menos, no meio de um bando de \(100\) andorinhas não faz diferença ao observador, pois \(1/100\) é uma fração muito pequena de variação. Todavia, uma andorinha em um bando de \(7\) andorinhas é perceptível à mente humana, pois \(1/7\) está (ligeiramente) acima do respectivo ``limiar'' de percepção e cardinalidade. Analogamente, o sentimento de cegueira experimentado ao se adentrar uma sala de um cinema em penumbra proveniente de uma rua ensolarada, é um outro exemplo do mesmo efeito com relação o estímulo visual. A conhecida receita para cozinhar vivo um sapo em um tacho de água, aquecendo-a vagarosamente de um em um grau a partir da temperatura fria, também se baseia neste mesmo princípio).

    Formalmente, a ``Lei de Weber-Fechner'' afirma que a percepção de um estímulo somente ocorre se a relação entre a variação da medida do estímulo \(\Delta S\) dividido pela intensidade ``de fundo'' \(S\) do estímulo for maior do que uma constante limiar \(c\) (específica para cada organismo e cada estímulo), ou seja, se \(\Delta S/S > c\), o que em termos infinitesimais, pode ser descrito na forma logarítmica \(d(\log S) > c\).

    Esta é uma importante razão da eficiência da representação decimal para números inteiros, pois além de determinar um procedimento claramente algorítmico para construí-lo, as suas variações são da forma \(d10^m/10m > 1\) onde o simbolo \(10^m\) posicional, determina a ordem de grandeza de fundo. Da mesma forma, os gráficos em escala logarítmica também tem a mesma utilidade, como a "Escala Richter" para a designação da potência de um terremoto. Se a potencia de um terremoto fosse uma medida da (enorme) energia que lhes caracteriza em valores absolutos, a nossa percepção cognitiva sobre a seriedade do fenômeno estaria completamente obscurecida.


    (A propósito, a tabuada decimal está mesmo um pouco acima do limiar de nossa capacidade de memorização, o que nos permite justificar biologicamente os frequentes enganos que cometemos nestes cálculos. Imaginem a tabuadas em um sistema sexagesimal utilizado pelos maias! Por outro lado, o sistema binário tem tabuada pequena, mas, por outro, exige uma quantidade enorme de símbolos para representar números naturais relativamente pequenos. Enfim, não se pode ganhar de todos os lados).

    Embora a escala logarítmica seja a medida não uniforme mais frequentemente utilizada, ela não é a única, como se pode facilmente imaginar.Na verdade, ela é importante quando uma media uniforme de um parâmetro tem a forma analítica de uma potência.(G. West-Scales, Pen 2018).

    A ``Lei de Weber-Fechner'' permanece como um importante conceito sobre o processo sensorial necessário no estudo de várias situações da fisiologia de muitos organismos, de bactérias a homo sapiens. Aperfeiçoamentos mais recentes da expressão que determina o limiar de percepção para diversas circunstancias e estímulos foram apresentados na literatura, sendo importante citar o trabalho de S. S. Stevens, da década de 1970.

\subsection{Funções como Modelos Matemáticos: A Síntese de Galileu (Funções) e a Síntese da Síntese de Newton (Equações Diferenciais)}

    Dentre os objetos matemáticos utilizados para a representação de Fenômenos Naturais, certamente as \textbf{Funções} de variável e valores numéricos, constituem a mais importante classe de objetos matemáticos para a construção de Modelos Matemáticos, depois das estruturas numéricas.

    As funções são objetos matemáticos cruciais na descrição sintética de relações de causalidade e variação temporal cuja origem pode ser detectada nos trabalhos de Galileu. Após Descartes, as funções também são objetos matemáticos apropriados para a representação de \textbf{Formas Geométricas} no Espaço.

    Consideremos, com Galileu, a descrição da queda livre de um objeto solto do alto da torre de Pisa, utilizando para isso uma tabela de duas colunas sendo a primeira constituída do registro dos tempos \(t_k\) de observação, (digamos, a cada batida de pulso) e a segunda constituída da respectiva altura \(x_k\) do objeto no respectivo instante. Esta tabela representa uma descrição do fato consumado, tal como a História humana que frequentemente é apresentada na forma de um grande tabela de ``efemérides'' impossível de ser memorizada, difícil de ser visualizada em contexto e, portanto, com uma utilidade muito reduzida.

    A grande ideia de Galileu foi buscar a representação desta enorme tabela de observações na forma de um Algoritmo que fosse capaz de produzir, ``à pedido'', qualquer valor da posição \(x(t)\) do objeto uma vez apresentado o respectivo instante \(t\) de queda, e não apenas para os arbitrários intervalos de tempo de uma tabela. Assim, em lugar de uma tabela explícita, mas inadministrável e sem uma estrutura matemática subjacente, teríamos uma sintética ``máquina analítica'' representada pela sua famosa \textbf{Fórmula} \(x = x_0 {\red v_0t}- \frac{1}{2} gt^2\). (O conceito de Algoritmo é mais antigo do que o conceito de Função e tem suas origens no século IX com o matemático iraquiano M. Al-Kwarismi de onde provem o termo (Knuth, Ershov). O italiano Fibonacci utilizou exatamente esta ideia de Algoritmo no século XIII para representar uma tabela de números naturais que descreveriam o tamanho da população de uma criação ``virtual'' de coelhos. Com isto, Fibonacci inaugurou, com um ``\textit{Toy Model}'', uma das áreas mais importantes da Biomatemática que será desenvolvida em outro capítulo. O conceito de Formula e Algoritmo foi mais tarde extraordinariamente expandido com o advento de operações limite do Cálculo, especialmente as somas (séries).

    A mais notável característica da representação funcional introduzida por Galileu para uma tabela de dados foi a sua capacidade de \textbf{Síntese}. Uma tabela com \(N\) instantes de observação da queda de um objeto envolve 2N números. Uma simples (e ingênua) representação "exata" desta tabela com a interpolação polinomial destes dados exigiria um polinômio de grau N , cujos coeficientes, todavia, não tem, em principio, qualquer significado cinemático.

    A barganha entre uma excesso de precisão obtida com a interpolação polinomial (precisão que é suspeita em qualquer conjunto de dados), Galileu optou por representar este conjunto de dados com parcimônia utilizando a expressão mais simples, desde que fosse uma boa aproximação. A conclusão de que uma função polinomial na forma \(x = x-0 - \frac{1}{2} gt^2\) descrevia (e previa) a queda do objeto com uma aproximação aceitável (mas, não exata), também permitia uma interpretação cinemática dos únicos parâmetros necessários: \(x_0\) e \(g\).

    Entretanto, apesar da extraordinária \textbf{síntese} e simplificação que a descrição funcional de Galileu representa com relação às tabelas de dados, este fato não esgotaria as vantagens deste procedimento.

\subsection{A síntese da síntese: a metodologia diferencial de newton e euler}

    A grande vantagem oriunda da estratégia de Galileu é a possibilidade de definir e utilizar uma rica \textbf{estrutura matemática} no conjunto de funções o que foi proporcionado pela espetacular invenção do Cálculo Diferencial e Integral no século seguinte. Ou seja, as funções não seriam apenas ``máquinas de produzir tabelas, ou números'', mas \textbf{objetos} com identidade matemática pertencentes a uma estrutura matemática, na qual é possível definir operações (adição, multiplicação, composição, derivação, integração e limites) que permitem aumentar enormemente a sua capacidade representativa. Esta é uma situação de todo análoga àquela que ocorreu com a invenção da Aritmética que transforma os números de símbolos inertes (``mudos'') em elementos de uma estrutura matemática, com a qual é possível representar diversos procedimentos. (Em especial, com esta estrutura é possível representá-los e construí-los a partir de pequenos blocos; a expansão decimal).

    É importante lembrar que invenção do Cálculo foi, na verdade, motivada pelo objetivo de Newton em representar o exaustivo conjunto de observações astronômicas de Tycho Brahe (já drasticamente sintetizadas pelas ``leis'' de Johannes Kepler) por intermédio de funções que seriam caracterizadas, não apenas por Fórmulas algébricas elementares (``explícitas'') à moda de Galileu, mas como soluções de equações diferenciais expressas com as operações do Cálculo que expandiram consideravelmente a classe de funções disponíveis.

    Funções Aritméticas finitas (isto é, aquelas obtidas de sequencias finitas de operações de adição, multiplicação e composição aplicadas a um conjunto básico de funções: constantes, identidade, recíproca, \(\frac{1}{x}\)) se mostraram rapidamente incapazes de representar a variedade de Fenômenos Naturais que se propunha estudar, especialmente na Mecânica Celeste e na Geometria do Espaço. A invenção do Cálculo por Newton e Leibniz na mesma época introduziu uma nova operação funcional (Soma infinita) o que aumentou consideravelmente a Estrutura Matemática Funcional e, com isso, expandia a Síntese de Galileu para horizontes extraordinariamente amplos. A caracterização das funções que tinham por finalidade representar Modelos Matemáticos de agora em diante era feita tanto como de maneira explicita (como uma Fórmula Aritmética, segundo Newton na forma:
    \(f(x) = a_0 + a_1x + a_2x^2 + \ldots + a_n x^n + \ldots\)) quanto como solução de uma equação diferencial.

    O Modelo de Malthus para a Dinâmica de mortalidade de uma População Homogênea desenvolvida por Leonhard Euler, no século XVIII, a partir das famosas Tabelas de mortalidade de John Graunt (séc. XVII) e da análise de Christiaan Huygens (séc. XVIII) será analisado com detalhes no próximo capítulo. O Modelo diferencial de Euler-Malthus se constitui em um exemplo simples, mas extraordinário da Síntese funcional Newtoniana que exerceu e tem um papel fundamental na Biomatemática, e também na Física.

    A caracterização diferencial tornou-se a Metodologia Newtoniana prevalente para a definição de um Modelo Matemático funcional, e de tal maneira, que o conceito de Modelo Matemático (erradamente) tem sido frequentemente tomado como equivalente a uma Equação Diferencial. De fato, a maioria dos Modelos Matemáticos da Biomatemática contemporânea são funcionais e caracterizados por Equações Diferenciais, Ordinárias ou Parciais. Os argumentos e as técnicas mais úteis para a obtenção destas Equações Diferenciais a partir de conceitos biológicos, especialmente relacionados à Dinâmica de Populações, são o tema da primeira parte deste texto denominada de: ``\textbf{Principios}''.

    Apesar da importância que a Metodologia Newtoniana tem para a formulação de Modelos Matemáticos, deve-se ressaltar que novos Métodos tem sido desenvolvidos com sucesso nas últimas décadas para este mesmo proposito que não utilizam explicitamente a caracterização diferencial das funções, e por isto são denominadas ``Modelos Sem-Equações''. (Em inglês, ``\textit{Equation-Free Models}''. refer. I. Kevrekidis)

\subsection{A síntese de Galileu na era dos dados massivos}

    As grandes massas de dados (``Big Data'') amplamente disponibilizadas pela moderna tecnologia informática para as mais diversas áreas do conhecimento instigaram a invenção de novos procedimentos que visam sintetiza-las compactamente e estruturalmente na forma de Funções. A Metodologia moderna não repete, todavia, exatamente o procedimento histórico Brahe-Newton, pois lança mão de diversas teorias e técnicas matemáticas surgidas apenas no ultimo século. Este será o tema de um dos próximos capítulos destinado ao Princípio de Reconstrução de Modelos, denominado pelo descritivo acrônimo DDM (``\textit{Data Driven Models}'') (J. N. Kutz).

\subsection{Operadores lineares e outros objetos Matemáticos}

    A teoria Quântica segundo a versão inventada pelos físicos W. Heisenberg, P. Dirac e formalizada pelo matemático J. von Neumann nas primeiras décadas do século XX, dá um passo adiante no conceito de Modelos Matemáticos, quando introduz a estrutura de operadores lineares em espaços de Hilbert ("matrizes infinitas") para a representação de fenômenos físicos.

    O conceito clássico e Newtoniano de funções também foi ampliado durante a primeira metade do século XX segundo a teoria de funções generalizadas, ou distribuições, sugerido por Paul Dirac dentre outros, desenvolvido por K.-O. Friedrichs, L. S. Sobolev e formalizado pelo russo I. M. Gelfand (``Funções Generalizadas'', ed. MIR e pelo francês Laurent Schwartz (``\textit{Théorie des Distribuitions}", Hermann, 1950).

    No presente texto não abordaremos modelos matemáticos fundamentados na estrutura de operadores lineares e o conceito de função generalizada será mencionado de passagem no estudo de processos difusivos.

\subsection{A síntese de Galileu ``in silico'': o mundo discreto de Stanislau Ulam}

    A construção de computadores digitais de grande capacidade de memória e velocidade de processamento permitiu (e instigou) o desenvolvimento de Modelos Matemáticos Discretos na forma de Algoritmos finitos, o que, de certa maneira, retorna à Síntese de Galileu. A distinção atual ente esta Síntese e a Funcional é a inexistência ainda de uma Estrutura Matemática para os Algoritmos Discretos. Os Autômatos Celulares são o que mais próximo se tem deste cenário. A Metodologia DDM (\textit{Data Driven Model}) tem conexões promissoras com esta visão. Para que a analogia do procedimento de Galileu seja levado adiante, é necessário desenvolver uma Estrutura Matemática para o Conjunto de Algoritmos (ou de regras definidoras de Automatos Celulares) uma questão totalmente em aberto hoje. (Refer. Trabalhos de Goldenfeld \& Israel- citado em Simon Dedeo-Santa Fe Inst.-Lecture on Renormalization)




% PARTE A

\chapter{PRINCÍPIO DE MALTHUS}
\addt


\section*{Parte 1: Modelo Minimalista de Malthus-Homogeneidade, Independência e Espontaneidade}

\begin{citacao}
``The aim of a mathematical lecture should not be the logical derivation of some incomprehensible assertions from others (equally incomprehensible); it is necessary to explain to the audience what the discussion is about and to teach them to use not only the results presented but (and this is major) the methods and the ideas''.

\rightline{Vladimir I. Arnold \footnote{Matemático, Professor, Escritor, e Polemista Extraordinários, e um patriota russo.} (1937-2010).}
\end{citacao}

\section{O Princípio de Comenius, a Metodologia Funcional de Galileo e a Representação Analítica de Newton}

\begin{citacao}
```A sabedoria é construída no percurso entre as coisas simples e as mais complexas, entre as visíveis e as invisíveis e entre as coisas terrenas e as coisas divinas''.

\rightline{Jan Komensky\footnote{``Comenius'' - Pai da Didática}  (1592-1670).}

    ``\textit{La filosofia 'e scritta in questo grandissimo libro che continuamente ci sta aperto innanzi a gli occhi (io dico l'universo) ma non si pu'o intender se prima non s'impara a intender la lingua e conoscere i caratteri ne' quali 'e scritto. Egli 'e scritto in lingua matematica e i caratteri sono triangoli, cerchi, ed altre figure geometriche senza i quali mezi 'e impossibile a intenderne umanamente parola; senza questi 'e un aggirarsi vanamente per un oscuro laberinto}''.

    \rightline{Galileo Galilei (Dialogo, 1632)}
\end{citacao}

    A Ciência antiga, até à época Medieval, esteve empenhada na busca de uma chave única de todos os mistérios, da Pedra Filosofal (a partir da qual toda a matéria, especialmente o ouro, procederia) à Palavra Mágica Criadora (``Abracadabra'', i.e., ``\textit{avrah kahdabra}'' que em aramaico significa: ``\textit{Eu crio com a minha palavra}''), à procura do Santo Graal e muitos outros empreendimentos fantasiosos, todos eles sem qualquer sucesso, a não ser por alguns dividendos colaterais inesperados e pela literatura associada. (Aparentemente, esta quimera ainda continua vivíssima a julgar pela prometida ``Teoria do Tudo'' anunciada com frequência e alarde e renunciada com a mesma frequência mas sem tanto alarde).

    Aristóteles (385-322 a.C.), por sua vez, que dominou todo o pensamento europeu até então, ensinava que o conhecimento provinha essencialmente da especulação abstrata que menospreza a influência da realidade experimental, talvez por que esta necessariamente envolveria algum esforço físico e manual e isso seria assunto reservado a escravos.

    A Ciência Moderna foi inaugurada com uma mudança radical de paradigmas introduzidos com os trabalhos de Galileo Galilei (1564-1642) nos quais foi estabelecida uma nova Metodologia Científica que rompia com a ilusão da Idade Média e com a passividade de Aristóteles no que consistia de três fundamentos:

\begin{enumerate}
    \item O \textbf{procedimento Analítico} (\(\alpha\nu\alpha\lambda\nu\sigma\iota\zeta\)), isto é, o estudo de aspectos deliberadamente isolados (artificialmente, mas com critério) de um determinado fenômeno;
    \item A \textbf{observação ativa do fenômeno}, com a realização de experimentos exploradores e o registro quantitativo (Tabelas) e qualitativo (Regras) dos resultados e, finalmente;
    \item A \textbf{Síntese} (\(\sigma Ú\nu\theta\epsilon\sigma\iota\zeta\)) \textbf{Funcional} dos Dados registrados para representar o fenômeno na forma de um \textbf{Modelo Matemático} descritivo e preditivo.
\end{enumerate}

    (É justo citar que Arquimedes (288-212 a.C.), talvez a mente mais criativa da Antiguidade, ou de todos os tempos, já havia indicado que a observação ativa da natureza, até quando se tomava banhos, era indispensável à sua compreensão).

    A ideia de construir Modelos Matemáticos Funcionais que sintetizam e generalizam a informação contida em extensas \textbf{Tabelas} numéricas de dados experimentais e qualitativos foi a grande contribuição de Galileo (séc. XVII) iniciada com o seu estudo da queda livre de corpos e da trajetória de balas de canhão. Apesar dessa origem tão prosaica (que causava horror aos aristotélicos da época), foi com esse trabalho que ele estabeleceu as bases para a Matemática Aplicada e para todas as Ciências Modernas quantificadas. A expressão quadrática \(x = \frac{1}{2}gt^2\) para descrever a posição da altura \(x\) de uma partícula em queda livre em qualquer instante t consistiu em uma extraordinária \textbf{síntese funcional} de dados experimentais discretos que também audaciosamente interpola instantes contínuos \(t\) e extrapola preditivamente o resultado para qualquer instante futuro.

    É importante ressaltar que a proposição da ``Fórmula'' para a descrição sintética de suas Tabelas de Dados não foi uma epifania meditativa de misteriosa origem, mas resultado de uma análise laboriosa de dados com a qual Galileo identificou as propriedades geométricas características de uma parábola, segundo um sólido conhecimento da Geometria Euclideana previamente adquirida por ele nos textos clássicos e, possivelmente, da Geometria Analítica aprendida nos trabalhos de seu famoso contemporâneo René Descartes (1596-1650). Às origens românticas comumente relatadas como explicação de episódios revolucionários na Ciência é possível citar inúmeros e variados contra-exemplos que, a começar por Galileo, valorizam o trabalho e não a sorte (genética ou histórica).

    Theodore von Kármán (1881-1963): ``A Intuição é resultado de muito estudo e cálculos''; Louis Pasteur (1822-1895): ``As ideias e as descobertas, ou a sorte em uma pesquisa, somente ocorrem para aqueles que já trabalharam muito no assunto''; Henri Poincaré em ``Ciência e Hipótese'', UnB 1985.

    Isaac Newton (1643-1727), entendendo a importância da Metodologia de Galileo, empregou a mesma ``estratégia funcional'' para descrever sinteticamente a enorme tabela de dados astronômicos colecionados por Tycho Brahe, cujas propriedades geométricas haviam sido sublimadas por uma análise exaustiva e previamente realizada por Johann Kepler. Observando esta conjuntura histórica, é possível entender melhor o significado da conhecida frase atribuída a Newton: ``Se eu consegui ver mais adiante que outros foi porque eu pude subir em ombros de gigantes".

    O grande e revolucionário aspecto diferencial introduzido pela versão Newtoniana da Metodologia de Galileo consistiu em ampliar o universo de funções disponíveis para a descrição de Dados Experimentais caracterizando-as como soluções analíticas de equações diferenciais e não apenas como Fórmulas algébricas elementares ``\textit{prêt à porter}''.

    De fato, a invenção do Cálculo e os trabalhos de Isaac Newton (1643-1727) sobre a Mecânica Celeste expandiram consideravelmente o alcance das ideias de Galileo resultando na contemporânea e prevalente Metodologia Galileo-Newtoniana em que a síntese funcional de dados observados é ``encriptada'' na forma de equações diferenciais, cujas soluções são, em geral, funções analíticas que compõem um universo muito maior do que as funções elementares. Isto é, enquanto Galileo tinha à sua disposição apenas uma reduzida Biblioteca de funções algébricas elementares, a Metodologia Newtoniana disponibilizou o universo de todas as funções analíticas transcendentais do Cálculo para a representação de fenômenos naturais.

    A propósito, é interessante citar a passagem da peça ``Galileo'' de Bertolt Brecht em que o personagem se dirige aos seus contendores com uma pergunta retórica, mas premonitória:

\begin{citacao}
```Vossas eminências se expressam em termos de círculos, elipses e velocidades regulares; movimentos simples que a mente pode imaginar. Muito conveniente, de fato. Mas, suponha agora que o Criador decidisse fazer com que as estrelas se movimentassem assim... (E, no ato ele descreve com seu indicador uma trajetória irregular no espaço)... Nesse caso, como ficaremos?''
\end{citacao}

\begin{exercise}
Determinar se está registrado em alguma biografia ou em algum relato de Galileo uma frase equivalente à citação acima.
\end{exercise}

    A única extensão notável da bem sucedida Metodologia de Galileo após Newton somente se deu em meados do século XX com os trabalhos dos matemáticos Israel M. Gelfand, Laurent Schwartz e outros com a introdução do conceito de Funções Generalizadas, ou, Distribuições. Embora usualmente se atribua a origem concreta do conceito de funções generalizadas à versão de Paul Dirac para a Teoria Quântica desenvolvida nas primeiras décadas do século XX (ca. 1920), é importante notar que a semente destas mesmas ideias comparecem de forma natural e quase explicita no Modelo Matemático para a Dinâmica de Fluidos desenvolvida \textit{ab ovo} por Leonhard Euler em meados do século XVIII. (Voltaremos a abordar esta questão em capítulos seguintes.)

    A obtenção de uma descrição funcional de fenômenos naturais (isto é, os Modelos Matemáticos) nos primórdios das Ciências Quantitativas foi resultado direto do trabalho de gigantes e resultado de circunstâncias históricas que não surgem com frequência. Portanto, era necessário que procedimentos padrões fossem desenvolvidos para tornar, na medida do possível, o cumprimento desta tarefa de uma maneira mais sistemática, ainda que sempre dependente da imaginação criativa. Estes Métodos são os temas desenvolvidos especificamente nos capítulos denominados ``\textbf{Princípios}'' deste texto.

    A \textit{Linearização Global} e o Método de \textit{Koopman} são dois exemplos recentes de métodos matemáticos destinados a implementar a Metodologia funcional de Galileo-Newton que foram desenvolvidos somente nas últimas décadas, e serão abordadas em capítulos separados.

    As Tabelas de Dados demográficos das colônias britânicas, especialmente na América do século XVIII, foram a principal motivação e fundamentação fenomenológica para os estudos que resultaram nos dois primeiros trabalhos especificamente destinados à formulação de uma Teoria Populacional de Reprodução devidos a Euler e Malthus. O Modelo de Leonhard Euler, de 1748, foi o primeiro deles e tinha um caráter matemático exemplar, como tudo o que ele abordava, mas não despertou qualquer interesse científico por mais de um século em que permaneceu praticamente incógnito. (O formato matemático escolhido por Euler na exposição deste modelo populacional de 1748 tem grandes semelhanças conceituais, mas é incomparavelmente menos sofisticado do que aquele empregado no seu trabalho em que ele estabelece os fundamentos da dinâmica de fluidos e publicado na mesma época. É razoável supor que Euler tenha apresentado o seu modelo demográfico deliberadamente na forma elementar de uma equação recursiva sem o uso do Cálculo para benefício dos menos letrados em Matemática. Apesar disso, e mesmo assim, os demógrafos provavelmente se intimidaram com a Matemática elementar de Euler, enquanto que os Matemáticos não se deram conta da importância de um Modelo Populacional. Talvez, por este mesmo motivo, o modelo de Euler tenha sido redescoberto várias vezes nos séculos que se seguiram, o último em 1958 por H. von Foerster).

    Uma trajetória histórica completamente distinta daquelas descritas pelo modelo de Euler esteve associada ao trabalho de Malthus. De fato, a questão populacional se tornou uma ``febre'' cultural no início do século XIX com a publicação, em 1798, do ``\textit{Essays on Population}'' por Thomas R. Malthus, em que ele expõe a sua rumorosa Doutrina, coincidentemente, motivada também pelas Tabelas demográficas da América colonial. Embora o trabalho de Malthus não tenha um formato matemático explicito, ele foi o ``motor cultural'' que incentivou o início de um estudo mais intenso da dinâmica populacional.

    Aproximadamente, nesta mesma época, o pastor protestante Jan Amos Komensky (1592-1670), considerado pela UNESCO como o pai da Didática moderna, publicou a sua obra fundamental em que ele expressa a sua máxima pedagógica em um notável paralelismo conceitual com a Metodologia cientifica de Galileo. Neste texto, ele enfatiza o aprendizado como um processo de síntese progressiva (e sinergética) de partes mais simples que são indispensáveis na construção de um todo mais complexo cuja essência representa muito mais do que um mero empilhamento de fatos desconexos.

    Assim, em deferência aos ``pais'' da Ciência e da Pedagogia Modernas, iniciaremos o estudo da Dinâmica de Populações com a construção de um Modelo Populacional Minimalista a partir do qual serão construídos Modelos Populacionais progressivamente mais inclusivos.

    Analogamente, o presente texto adota deliberadamente a Metodologia geral de Galileo-Newton como formato expositivo, que se apresenta em duas partes distintas: os \textbf{Princípios}, que sistematizam procedimentos matemáticos para a obtenção das representações funcionais de fenômenos biológicos a partir de de Regras Qualitativas e Dados Numéricos experimentais, e os \textbf{Métodos}, que são instrumentos Matemáticos destinados a analisar propriedades relevantes das Funções que representam um Modelo Matemático na sua síntese diferencial.

\section{Em busca do modelo populacional minimalista (Minimum Minimorum)}

Ingredientes Fundamentais da Biologia de Populações:

\begin{enumerate}
\item Reprodução;
\item Morte;
\item Mutação;
\item Interação.
\end{enumerate}

Mutação: Espontânea-Sem Causalidade Intrínseca-Quebra do Paradigma Clássico Origem e Manutenção da Diversidade

Modelo Minimalista:

\begin{enumerate}
\item Homogeneidade (Sem diversidade),
\item Independência (Sem Interação)-Descrição: Tamanho e Traço Biológico uniforme.
\end{enumerate}

Restrições Dimensionais para Modelo Minimalista:
\[N = \varphi(t, N_0, r)\]
onde
\(N_0 = \varphi(0, N_0, k)\) e \([k] = T \gg n = \dfrac{N}{N_0} = \varphi\left(\frac{t}{k}, 1, 1\right) = \psi\left(\frac{t}{k}\right)\).

\textbf{Metodologias} de \textbf{Síntese Funcional} (\(\psi(\zeta)\)) de Tabela de Dados \& Princípios

\begin{enumerate}
\item Metodologia de Galileo: Geometria Cartesiana - Biblioteca de Funções Algébricas Elementares
\item Metodologia de Newton: Equações Diferenciais Ordinárias - Biblioteca de Funções Transcendentais- Extensão do Mundo Linear Micro/Instantaneo
\item Metodologia de Linearização Global: Linearização Não Uniforme (Log)-Nomogr/d’Ocagne/Arnold \& Kolmogorov - Linear. Assint.(West) - NNW \& AI
\item Metodologia de Koopman \& al. - DDM - (``Data Driven Models'') ``Life is a self-sustaining chemical system capable of Darwinian evolution'' NASA-Definition of Life-1994.
\end{enumerate}

O tema central da Biologia consiste, essencialmente, no estudo de Populações cujos indivíduos estão sujeitos a quatro fenômenos distintos:

\begin{enumerate}
\item Reprodução;
\item Morte;
\item Interação; e
\item Mutação.
\end{enumerate}

    Dentre estes quatro fenômenos básicos, o mais sutil deles é, sem dúvida, a ``Mutação'', o que explica a sua tardia incorporação à Biologia Teórica e a vigorosa rejeição que este conceito enfrentou à época de sua introdução por Charles Darwin e Alfred Wallace (ca. 1858). Para a sua representação, a Mutação requer, naturalmente, que o modelo matemático descreva explicitamente uma gama de diversidade para os indivíduos que constituem a População em estudo e, com isto, levanta, de saída a difícil questão de estabelecer e representar características (estados biológicos) que distinguem uns indivíduos dos outros e, portanto, suas respectivas subpopulações.

    O fenômeno de Mutação é, na verdade, a fonte inicial da diversidade biológica de uma População e, desde Darwin, a sua ocorrência é assumida como resultado de eventos aleatórios, ou seja, não se supõe que ela seja efeito de qualquer ``causa'' primária explícita.

    Esta hipótese coloca o fenômeno de Mutação em confronto direto com a visão determinística que prevaleceu na ciência clássica e, por este motivo tornou-se um dos pontos frágeis que Darwin e Wallace não podiam justificar melhor, já que desconheciam os experimentos genéticos do seu contemporâneo, o monge tcheco Gregor Mendel (1822-1884) e, obviamente, não podiam conectá-la à origem química e molecular do fenômeno, o que somente veio a ser descoberta, em 1957, por Francis Crick e James Watson. Consequentemente, uma discussão apropriada deste tema, mesmo que elementar, exige:
    
    \begin{enumerate}
    \item A descrição do conceito de heterogeneidade no modelo de uma população (o que somente será abordada no próximo capítulo) e,
    \item Uma familiaridade com processos probabilísticos que, coincidentemente, tem a sua expressão mais elementar derivada do Modelo Mathusiano a ser apresentado neste capítulo. Por estes motivos, o fenômeno de Mutação deverá ser um tema para um capítulo posterior, exclusivamente dedicado aos Modelos Evolutivos. (Para uma interessante discussão a respeito deste importante tema e referências posteriores, consulte P. Bühlmann-Toward Causality and External Validity, Proc. Nat. Acad. Sci., USA, 2020, e J. Pearl - D. Mackenzie - The Book of Why: The New Science of Causation and Effect, Basic 2018, cujos argumentos sobre ``causalidade'' serão importantes na fundamentação de Metodologias para a construção de Modelos Matemáticos a partir de dados registrados, um tema atual a ser especificamente abordado em outros capítulos, e que tem um papel central na Metodologia de Galileo.)
    \end{enumerate}

\begin{exercise}
    Leia o artigo de \href{https://www.pnas.org/content/pnas/117/42/25963.full.pdf}{P. Buhlmann} (mencionado acima) e a resenha de \href{https://www.ams.org/journals/notices/201907/rnoti-p1093.pdf}{L. Goldberg} sobre o livro de Pearl-Mackenzie e faça sua resenha de 20 linhas sobre o tema. 
\end{exercise}

{\color{red}
%
\section*{Rumo à causalidade e melhoria da validade externa}


    ``\textit{Felix, qui potuit rerum cognoscere causas}'', do poeta latino Virgílio (1), traduzido literalmente como ``Afortunado, que era capaz de conhecer as causas das coisas'', sugere a importância da causalidade desde muito tempo atrás. Em PNAS, Bates et al. (2) começa sua contribuição com a frase ``O objetivo final dos estudos de associação do genoma (GWAS) é identificar regiões do genoma contendo variantes que afetam causalmente um fenótipo de interesse'' e fornecem uma metodologia estatística original e altamente inovadora para fornecer respostas sólidas a este objetivo. Como argumentaremos, o problema de inferência causal é ambicioso e deve-se confiar em suposições. Os pressupostos na ref. 2 são fáceis de comunicar; a capacidade de comunicar suposições subjacentes torna sua abordagem transparente e, em nossa própria avaliação, suas suposições são muito plausíveis.

    Quando observamos correlação ou dependência entre algumas variáveis de interesse, uma questão central é sobre a direcionalidade: se uma variável é a causa ou o efeito de outra. Claro, pode acontecer que nenhum dos dois seja verdade, por causa de confusão oculta. Veja a Fig. 1 para uma visão esquemática onde todas as variáveis observadas estão exibindo dependência de associação entre si, mas estas são, em parte, decorrentes de fatores ocultos invisíveis. Se pudéssemos obter conhecimento da direcionalidade causal, obviamente, isso levaria a muitas melhorias na compreensão e interpretabilidade de um sistema subjacente. Na Fig. 1, isso significa inferir as relações causais direcionadas entre as variáveis observadas.


    Medidas de associação sozinhas, como correlação ou regressão (multivariada potencialmente não linear), com base nos chamados dados observacionais (dados do ``estado estacionário''), não podem fornecer respostas para a direcionalidade e, portanto, para a causalidade em geral; são necessárias suposições ou dados adicionais de outras configurações de projeto experimental. Um ensaio de controle randomizado (RCT) é um poderoso padrão ouro para inferir causalidade, graças ao seu desenho experimental muito especial (cf. ref. 3 e também Dados de perturbação como entrada). No entanto, infelizmente, esse método padrão-ouro costuma ser inviável ou antiético. Na ausência de RCTs, outra metodologia deve ser usada, sempre dependendo crucialmente de algumas suposições. Bates et al. (2) fornece uma abordagem altamente interessante com suposições plausíveis para inferência causal no campo particular de GWAS; ver abaixo. Antes de discutir isso, elaboramos brevemente de forma mais geral o propósito da causalidade.

\noindent
\begin{minipage}[!ht]{\columnwidth}
\epsfig{figure=figs/fig01bullman.png,width=\columnwidth}
\captionof{figure}{\scriptsize Sistema observado e verdadeiro em duas configurações diferentes (configuração A e B e configuração C e D). Variável de resposta Y (fenótipo) e covariáveis \(X_j (j = 1,2)\) (por exemplo, SNPs). (A e C) Variáveis observadas \(X_1, X_2, Y\) em azul. Uma borda não direcionada representa a associação entre as variáveis correspondentes, por exemplo, em termos de correlação ou de dependência de regressão (não linear) (correlação parcial) dadas todas as outras variáveis observadas. (B e D) Sistemas subjacentes verdadeiros, com variáveis observadas em azul e variável latente H oculta em vermelho. Uma aresta direcionada representa uma relação causal direta entre as variáveis correspondentes, com a cauda sendo a causa e a cabeça sendo o efeito (ou seja, a variável que é diretamente influenciada pela variável causadora). (A e B) Configuração onde todas as setas entre Xj a Y em B devem apontar para Y, como em (a maioria) GWAS. (C e D) A direção da seta em D entre \(X_j\) e Y pode ir para qualquer lado, como em situações gerais. Os verdadeiros sistemas subjacentes em B e D geram a dependência de associação em A e C, em termos de correlação ou dependência de regressão (não linear). A observação de tais associações leva a descobertas espúrias, ou seja, falsos positivos com relação à causalidade.}
\label{fig:01}
\end{minipage}

\section*{Escopo Principal de Causalidade}

     Além de ter melhorado a compreensão de um mecanismo, graças ao conhecimento causal, destacamos dois objetivos principais (adicionais) da inferência causal. Eles são frequentemente menos ambiciosos e mais realistas do que inferir toda a rede ou gráfico com pesos de aresta funcionais correspondentes, como na Fig. 1.

\section*{Prevendo intervenções específicas: efeito do tratamento}

     Um objetivo clássico de causalidade é a previsão de uma intervenção ou manipulação que não foi observada antes. A causalidade dá respostas quantitativas a perguntas como: O que aconteceria se tratássemos um paciente com um determinado medicamento (e a intervenção do tratamento ainda não tivesse sido feita)? O que aconteceria se eliminássemos um determinado gene (e a intervenção genética ainda não tenha sido realizada)? Assim, a causalidade dá uma resposta a uma pergunta do tipo ``e se eu fizer'' (4, 5). Em muitas aplicações, é altamente desejável ter previsões precisas para essas questões.

\section*{Robustez contra Perturbações Inespecíficas: Validade Externa}

    O problema abordado na ref. 2 talvez não esteja tão diretamente relacionado a intervenções específicas, uma vez que lida com polimorfismos de nucleotídeo único (SNPs) em GWAS, onde intervenções em SNPs não podem ser feitas. Como um experimento de pensamento, entretanto, ainda se pode pensar sobre o que aconteceria com o estado de uma doença se um certo SNP interviesse. Nossa mensagem é que, mesmo na ausência da possibilidade de fazer intervenções diretas, a inferência causal é altamente interessante (além da questão da interpretação mencionada acima). A principal razão é que a estrutura causal leva a certas invariâncias e robustez, como explicamos brevemente a seguir.

    A maioria dos estudos científicos afirmam que as descobertas e resultados se generalizam para outros indivíduos ou populações e objetivam a validade externa. Em outras palavras, o objetivo é a replicabilidade das descobertas: queremos inferir resultados estáveis em diferentes subpopulações, em que cada uma delas pode ser uma versão perturbada de uma referência. Curiosamente, essa estabilidade em diferentes subpopulações ou diferentes perturbações tem uma relação muito intrínseca com a causalidade: a regressão nas variáveis causais, a solução causal, exibe (alguma) robustez ou estabilidade contra perturbações decorrentes de diferentes subpopulações (6-8) e, portanto, uma solução causal com sua robustez leva a uma melhor replicabilidade e melhor validade externa (em novos estudos, para novos pacientes, etc.). Em nossa opinião, esta é uma grande vantagem da abordagem e das conclusões da ref. 2: A metodologia deles, por visar relações causais, melhora a validade externa!

\section*{Métodos de inferência causal}

    Inferir causalidade a partir dos dados é uma tarefa ambiciosa e depende crucialmente do planejamento de experimentos ou de suposições adicionais, muitas vezes não testáveis.

\section*{Dados de perturbação como entrada}

    Aprender a estrutura causal e os efeitos é mais fácil com o acesso a dados de diferentes perturbações do sistema de interesse. Como já mencionado, o padrão ouro é uma perturbação na forma de um RCT. Lá, o experimentador tem a capacidade de fazer uma intervenção em uma variável (sendo um candidato a ser causal) ou de atribuir um tratamento: A randomização quebra todas as dependências entre a variável interveniente e qualquer possível confusão oculta. A conclusão poderosa é que, após a randomização, se sobrar um efeito entre a variável intervencionada ou de tratamento e uma resposta de interesse, deve ser um efeito causal (total). Um RCT leva à estabilidade e validade externa dos efeitos (regressão ou comparação de grupo) para uma grande classe de perturbações. Este é exatamente o objetivo, digamos, do desenvolvimento de uma farmacoterapia robusta: o medicamento ou os efeitos do tratamento ativo devem ser ``sempre'' externamente válidos. Se um RCT for inviável, os dados de perturbação de intervenções específicas (não randomizadas) ou de mudanças inespecíficas de ambiente ainda são muito mais informativos do que ter apenas acesso a dados observacionais. A informação dos dados de perturbação leva a invariâncias e estabilidade de efeitos (regressão) que são induzidos pelos diferentes ambientes, mas onde não se tem realmente controle sobre a ``natureza'' das perturbações que são inofensivas ou prejudiciais para inferir efeitos (regressão). Porém, grosso modo, ao observar mais perturbações, pode-se identificar mais invariância, estabilidade e robustez e, eventualmente, a estrutura causal e os efeitos (8). Assim, o cenário mais desafiador para inferir efeitos causais acontece quando apenas dados observacionais do ``estado estacionário'' estão disponíveis.


\section*{A Abordagem de Bates et al. (2) usando apenas dados observacionais}

    O método na ref. (2) usa apenas dados observacionais como entrada. No entanto, duas premissas principais são exploradas. Primeiro, a direcionalidade é postulada naturalmente apontando de SNPs genéticos para o fenótipo; ou seja, se houver associação de regressão infundada entre um fenótipo \(Y\) e uma variável SNP \(X_j\), ela deve ser direcionada \(X_j \to Y\). Esta é a situação na Fig. 1 A e B. A mesma direcionalidade é assumida de haplótipos parentais para SNPs descendentes . Em segundo lugar, para inferir a associação de regressão não-fundada, isto é, a força da regressão que resta após o ajuste para confusão oculta em potencial, um assim chamado estudo de design de trio especial leva, de maneira elegante, a tais efeitos de regressão não-fundados. A suposição é que o mecanismo estocástico de SNPs condicional aos haplótipos parentais, ou seja, a distribuição condicional correspondente, é independente de outros potenciais confundidores ocultos, e isso, por sua vez, permite a conclusão de que uma associação de regressão (potencialmente não linear) entre um SNP e um fenótipo, dados todos os outros SNPs e os haplótipos parentais, devem implicar uma dependência causal. Isso é uma analogia exata com um RCT: o condicionamento dos haplótipos serve como um substituto para a randomização! Bates et al. (2) referem-se a isso como ``variação na herança como um experimento aleatório''. Ambas as suposições podem ser comunicadas com clareza e são muito plausíveis, o que torna as alegadas descobertas causais muito convincentes. Claro, ainda pode haver violações de suposições, e os autores mencionam SNPs não medidos ou viés de seleção, para citar dois exemplos proeminentes. No entanto, em geral, a metodologia na ref. (2) é um grande passo em frente para chegar mais perto da ``verdadeira causalidade subjacente''.

    Além da maneira como a metodologia lida com suposições fundamentais para causalidade, ela fornece garantias estatísticas de amostra finita sobre a descoberta falsa ou a taxa de erro familiar. A principal suposição aqui é que o modelo de Haldane (9) é considerado ``verdadeiro'' (ou seja, uma aproximação muito boa), e as técnicas de inferência construídas em belos trabalhos anteriores de simulação de falsos recursos sintéticos que servem para contar falsos positivos (10, 11).

    Particularmente fascinante é a possibilidade de incluir dados GWAS externos (projeto não tri) para melhorar a energia; estudos de desenho de trio são raros e de tamanho de amostra muito menor do que estudos GWAS padrão, que podem vir em grande escala. Conforme ilustrado na ref. (2), pode-se usar qualquer algoritmo de aprendizado de máquina em dados GWAS externos para melhorar potencialmente a potência, enquanto a garantia de amostra finita na detecção de falso positivo ainda é válida.


\section*{Pensamentos Adicionais}

    Bates et al. (2) demonstra bem o uso de dados externos para aumentar potencialmente o poder de detecção de SNPs causais em estudos de projeto de trio. Invertendo o papel de usar dados externos, pode-se, e talvez deva, também usar parte deles para validar os resultados (e não usá-los na fase de descoberta); veja também ref. (12). Conforme mencionado em Robustness against Inspecific Perturbations: External Validity, se a estrutura inferida for causal, ela deve exibir alguma validade externa em novos dados, idealmente, em alguns conjuntos de dados de diferentes ambientes ou subpopulações. Como proposta, pode-se inspecionar a estabilidade da distribuição condicional do fenótipo, dados os SNPs causais encontrados, por exemplo, testando a independência condicional do fenótipo e dos ambientes dados os fenótipos causais (13, 14). Em particular, isso poderia ser feito com conjuntos de dados externos GWAS de design não triplo padrão que estão disponíveis em várias plataformas.

    Na ausência de estudos de design de trio e na ausência de direcionalidade postulada (como em GWAS de SNPs para fenótipos), o problema de inferência causal é muito mais difícil. As Fig. 1 C e D indicam esta configuração, que inclui, por exemplo, transcriptômica ou proteômica em biologia, onde postular a direcionalidade é frequentemente difícil ou sujeito a erros. Os dados de perturbação desempenharão um papel crucial para fazer um progresso confiável no sentido de inferir estruturas causais e efeitos. Mesmo quando não é possível ter experimentos randomizados, as perturbações não randomizadas ajudam enormemente. Para campos como biologia molecular e muitos outros, priorizar bons candidatos com respeito a ser causal é muito valioso, mesmo quando declarações de confiança estatística estritas parecem fora do escopo (15). Claramente, tal priorização causal deve ser realizada por métodos de inferência causal, em vez de técnicas de associação pura, onde as últimas variam de correlação simples a regressão não linear avançada ou aprendizado de máquina de classificação.
    
\section*{Agradecimentos}

     A pesquisa foi apoiada pelo Conselho Europeu de Pesquisa sob o Acordo de Subvenção 786461 (CausalStats - ERC-2017-ADG).

\section*{Referências}

1 Virgil, Georgica (vers 490, Book II, 29 BC).

2 S. Bates, M. Sesia, C. Sabatti, E. Candès, Causal inference in genetic trio studies. Proc. Nat. Acad. Sci. U.S.A. 117, 24117–24126 (2020).

3 G. Imbens, D. Rubin, Causal Inference for Statistics, Social, and Biomedical Sciences (Cambridge University Press, 2015).

4 J. Pearl, Causality: Models, Reasoning and Inference (Cambridge University Press, ed. 2, 2009).

5 J. Pearl, D. Mackenzie, The Book of Why: The New Science of Cause and Effect (Basic, 2018).

6 T. Haavelmo, The statistical implications of a system of simultaneous equations. Econometrica 11, 1–12 (1943).

7 A. P. Dawid, V. Didelez, Identifying the consequences of dynamic treatment strategies: A decision-theoretic overview. Stat. Surv. 4, 184–231 (2010).

8 J. Peters, P. Bühlmann, N. Meinshausen, Causal inference using invariant prediction: Identification and confidence interval (with discussion). J. R. Stat. Soc. Ser. B
Stat. Methodol. 78, 947–1012 (2016).

9 J. B. S. Haldane, The combination of linkage values and the calculation of distances between the loci of linked factors. J. Genet. 8, 299–309 (1919).

10 R. F. Barber, E. Candès, Controlling the false discovery rate via knockoffs. Ann. Stat. 43, 2055–2085 (2015).

11 E. Candès, Y. Fan, L. Janson, J. Lv, Panning for gold: Model-X knockoffs for high dimensional controlled variable selection. J. R. Stat. Soc. Ser. B Stat. Methodol.
80, 551–577 (2018).

12 B. Yu, K. Kumbier, Veridical data science. Proc. Natl. Acad. Sci. U.S.A. 117, 3920–3929 (2020).

13 R. Shah, J. Peters, The hardness of conditional independence testing and the generalised covariance measure. Ann. Stat. 48, 1514–1538 (2020).

14 M. Azadkia, S. Chatterjee, A simple measure of conditional dependence. arXiv:1910.12327 (27 October 2019).

15 N. Meinshausen et al., Methods for causal inference from gene perturbation experiments and validation. Proc. Nat. Acad. Sci. U.S.A. 113, 7361–7368 (2016).

%\section*{O livro do porquê: uma crítica de Lisa R. Goldberg}



\noindent
\begin{minipage}[!ht]{0.25\columnwidth}
\epsfig{figure=figs/fig00goldberg.png,width=0.9\columnwidth}
%\captionof{figure}{\scriptsize }
%\label{fig:00}
\end{minipage}
\begin{minipage}[!ht]{0.75\columnwidth}\scriptsize
The Book of Why \\
The New Science of Cause and Effect \\
Judea Pearl and Dana Mackenzie \\
Basic Books, 2018 \\
432 pages \\
ISBN-13: 978-0465097609
\end{minipage}


Judea Pearl tem a missão de mudar a maneira como interpretamos os dados. Um eminente professor de ciência da computação, Pearl documentou suas pesquisas e opiniões em livros e artigos acadêmicos. Agora, ele tornou suas ideias acessíveis a um amplo público em O Livro do Porquê: A Nova Ciência de Causa e Efeito, em coautoria com a escritora científica Dana Mackenzie. Com o lançamento deste livro historicamente fundamentado e instigante, Pearl salta da torre de marfim para o mundo real.

    O Livro do Porquê visa as limitações percebidas dos estudos observacionais, cujos dados subjacentes são encontrados na natureza e não controlados por pesquisadores. Muitos acreditam que um estudo observacional pode elucidar a associação, mas não causa e efeito. Não pode te dizer por quê.

    Talvez o exemplo mais famoso diga respeito ao impacto do tabagismo na saúde. Em meados da década de 1950, os pesquisadores estabeleceram uma forte associação entre tabagismo e câncer de pulmão. Somente em 1984, no entanto, o governo dos Estados Unidos determinou a frase "fumar causa câncer de pulmão".

    O atraso era o espectro de um fator latente, talvez algo genético, que poderia causar câncer de pulmão e desejo por tabaco. Se o fator latente fosse responsável pelo câncer de pulmão, limitar o tabagismo não impediria a doença. Naturalmente, as empresas de tabaco gostavam dessa explicação, mas também foi defendida pelo proeminente estatístico Ronald A. Fisher, co-inventor do chamado padrão ouro de experimentação, o Randomized Controlled Trial (RCT).

    Os participantes de um ECR sobre tabagismo e câncer de pulmão teriam sido designados a fumar ou não no cara ou coroa. O estudo tinha o potencial de desqualificar um fator latente como a principal causa de câncer de pulmão e elevar os cigarros à posição de principal suspeito. Uma vez que um ECR de tabagismo seria antiético, no entanto, os pesquisadores se contentaram com estudos observacionais que mostravam associação e objetaram sobre a questão de causa e efeito por décadas.

    O problema era simplesmente que as ferramentas disponíveis nas décadas de 1950 e 1960 eram muito limitadas em escopo? Pearl aborda essa questão em sua escada de causalidade de três etapas, que organiza métodos inferenciais em termos dos problemas que podem resolver. O degrau inferior é para métodos estatísticos sem modelo que dependem estritamente de associação ou correlação. O degrau do meio é para intervenções que permitem a medição de causa e efeito. O degrau mais alto é para a análise contrafactual, a exploração de realidades alternativas.

    As primeiras investigações científicas sobre a relação entre tabagismo e câncer de pulmão baseavam-se em métodos estatísticos sem modelos, degraus inferiores, cujos análogos modernos dominam a análise dos estudos observacionais hoje. Em uma das muitas anedotas históricas maravilhosas do The Book of Why, a predominância desses métodos é atribuída ao trabalho de Francis Galton, que descobriu o princípio da regressão à média em uma tentativa de compreender o processo que impulsiona a hereditariedade das características humanas. A regressão à média envolve associação, e isso levou Galton e seu discípulo, Karl Pearson, a concluir que a associação era mais central para a ciência do que a causalidade.

    Pearl coloca o aprendizado profundo e outras ferramentas modernas de mineração de dados no degrau inferior da Escada da Causalidade. Os métodos inferiores incluem AlphaGo, o programa de aprendizado profundo que derrotou os melhores jogadores humanos de Go do mundo em 2015 e 2016 [1]. Para o benefício daqueles que se lembram dos tempos antigos antes da mineração de dados mudar tudo, ele explica,

\begin{quotation}
    Os sucessos do aprendizado profundo foram realmente notáveis e pegaram muitos de nós de surpresa. No entanto, o aprendizado profundo foi bem-sucedido principalmente ao mostrar que certas questões ou tarefas que pensávamos serem difíceis, na verdade não o são.
\end{quotation}

    A questão é que algoritmos, ao contrário de crianças de três anos, fazem o que mandam, mas para criar um algoritmo capaz de raciocínio causal, ... temos que ensinar o computador como quebrar seletivamente as regras da lógica. Os computadores não são bons em quebrar regras, uma habilidade na qual as crianças se destacam.



\noindent
\begin{minipage}[!ht]{\columnwidth}\centering
\epsfig{figure=figs/fig01goldberg.png,width=0.9\columnwidth}
\captionof{figure}{\scriptsize Modelo causal de relações presumidas entre fumo, câncer de pulmão e um gene do fumo.}
\label{fig:01}
\end{minipage}



    Métodos para extrair conclusões causais de estudos observacionais estão no degrau intermediário da Escada de Causalidade de Pearl e podem ser expressos em uma linguagem matemática que estende a estatística clássica e enfatiza os modelos gráficos.

\begin{quotation}
    Existem várias opções para modelos causais: diagramas causais, equações estruturais, afirmações lógicas e assim por diante. Estou fortemente convencido de diagramas causais para quase todas as aplicações, principalmente devido à sua transparência, mas também devido às respostas explícitas que fornecem a muitas das perguntas que desejamos fazer.
\end{quotation}

    O uso de modelos gráficos para determinar causa e efeito em estudos observacionais foi iniciado por Sewall Wright, cujo trabalho sobre os efeitos do peso ao nascer, tamanho da ninhada, duração do período de gestação e outras variáveis sobre o peso de uma cobaia de porco de 33 dias de idade está em [2]. Pearl relata a persistência de Wright em resposta à recepção fria que seu trabalho recebeu da comunidade científica.

\begin{quotation}
    Minha admiração pela precisão de Wright só perde para a minha admiração por sua coragem e determinação. Imagine a situação em 1921. Um matemático autodidata enfrenta sozinho a hegemonia do sistema estatístico. Eles dizem a ele ``Seu método é baseado em uma compreensão totalmente equivocada da natureza da causalidade no sentido científico.'' E ele retruca: ``Não é assim! Meu método é importante e vai além de qualquer coisa que você possa gerar.''
\end{quotation}

    Pearl define um \textit{modelo causal} como um gráfico acíclico direcionado que pode ser emparelhado com dados para produzir estimativas causais quantitativas. O gráfico incorpora as relações estruturais que um pesquisador assume que estão gerando resultados empíricos. A estrutura do modelo gráfico, incluindo a identificação de vértices como mediadores, confundidores ou aceleradores, pode guiar o projeto experimental por meio da identificação de conjuntos mínimos de variáveis de controle. As exposições modernas sobre modelos gráficos de causa e efeito são [3] e [4].

\noindent
\begin{minipage}[!ht]{\columnwidth}\centering
\epsfig{figure=figs/fig02goldberg.png,width=0.9\columnwidth}
\captionof{figure}{\scriptsize Modelo causal mutado que facilita o cálculo do efeito do tabagismo no câncer de pulmão. A seta do gene que confunde o fumo para o ato de fumar foi deletada.}
\label{fig:02}
\end{minipage}


    Dentro dessa estrutura, Pearl define o operador \(\operatorname{do}\), que isola o impacto de uma única variável de outros efeitos. A probabilidade de \(Y \operatorname{do} X\), \(P[Y|\operatorname{do}(X)]\), não é a mesma coisa que a probabilidade condicional de \(Y\) dada \(X\). Em vez disso, \(P[Y|\operatorname{do}(X)] \) é estimado em um modelo causal mutado, do qual as setas apontando para a causa assumida são removidas. \textit{Confundir} é a diferença entre \(P[Y|\operatorname{do}(X)]\) e \(P[Y|X]\). Na década de 1950, os pesquisadores estavam atrás do primeiro, mas só podiam estimar o último em estudos observacionais. Esse foi o ponto de Ronald A. Fisher.

    A Figura 1 mostra uma relação simplificada entre tabagismo e câncer de pulmão. As bordas direcionadas representam relações causais assumidas, e o gene do fumo é representado por um círculo vazio, indicando que a variável não era observável quando a conexão entre fumo e câncer estava em questão. Círculos preenchidos representam quantidades que podem ser medidas, como taxas de tabagismo e câncer de pulmão em uma população. A Figura 2 mostra o modelo causal mutado que isola o impacto do tabagismo no câncer de pulmão.

    A conclusão de que fumar causa câncer de pulmão foi finalmente alcançada sem recorrer a um modelo causal. Uma quantidade enorme de evidências, incluindo a poderosa análise de sensibilidade desenvolvida em [5], acabou influenciando a opinião. Pearl argumenta que seus métodos, se estivessem disponíveis, poderiam ter resolvido o problema mais cedo. Pearl ilustra seu ponto em um cenário hipotético em que fumar causa câncer apenas por depositar alcatrão nos pulmões. O diagrama causal correspondente é mostrado na Figura 3. Sua \textit{fórmula da porta da frente} corrige a confusão do gene do fumo não observável, sem nunca mencioná-lo. O impacto corrigido do preconceito do tabagismo, \(X\) no câncer de pulmão, \(Y\) pode ser expresso
    \[P[Y|\operatorname{do}(X)] = \sum_ {Z} P[Z|X] \sum_ {X'} P[Y|X',Z] P[X'].\]


\noindent
\begin{minipage}[!ht]{\columnwidth}\centering
\epsfig{figure=figs/fig03goldberg.png,width=0.9\columnwidth}
\captionof{figure}{\scriptsize A fórmula da porta da frente de Pearl corrige o viés devido a variáveis latentes em certos exemplos.}
\label{fig:03}
\end{minipage}

    \textit{O Livro do Porquê} extrai um corpo substancial da literatura acadêmica, que explorei a fim de obter um quadro mais completo do trabalho de Pearl. De uma perspectiva matemática, uma aplicação importante é o estudo de 2007 de Nicholas Christakis e James Fowler descrito em [6] argumentando que a obesidade é contagiosa. A alegação que chamou a atenção foi controversa porque o mecanismo de contágio social é difícil de definir e porque o estudo foi observacional. Em seu artigo, Christakis e Fowler atualizaram uma associação observada, grupos de indivíduos obesos em uma rede social, para a afirmação de que indivíduos obesos fazem com que seus amigos e amigos de seus amigos se tornem obesos. É difícil compreender a complexa rede de suposições, argumentos e dados que compõem este estudo. Também é difícil compreender suas refutações matizadas por Russell Lyons [7] e por Cosma Shalizi e Andrew Thomas [8], que surgiram em 2011. Há um momento de clareza, no entanto, no comentário de Shalizi e Thomas, quando eles citam o teorema de Pearl sobre a não identificabilidade em modelos gráficos particulares. Usando os resultados de Pearl, Shalizi e Thomas mostram que na rede social que Christakis e Fowler estudaram, é impossível separar o contágio, a propagação da obesidade por meio da amizade, das inclinações compartilhadas que levaram a amizade a se formar em primeiro lugar.

    O degrau mais alto da Escada da Causação diz respeito aos contrafatuais, que Michael Lewis chamou a atenção do mundo com seu livro mais vendido, The Undoing Project [9]. Lewis conta a história dos psicólogos israelenses Daniel Kahneman e Amos Tversky, especialistas em erro humano, que mudaram fundamentalmente nossa compreensão de como tomamos decisões. Pearl se baseia no trabalho de Kahneman e Tversky em The Book of Why, e a abordagem de Pearl para analisar contrafactuais pode ser melhor explicada em termos de uma questão que Kahneman e Tversky colocaram em seu estudo [10] de como exploramos realidades alternativas.

\begin{quotation}
    Quão perto os cientistas de Hitler chegaram de desenvolver a bomba atômica na Segunda Guerra Mundial? Se eles o tivessem desenvolvido em fevereiro de 1945, o resultado da guerra teria sido diferente?
    
        —A Simulação Heurística
\end{quotation}

    A resposta de Pearl a esta pergunta inclui a \textit{probabilidade de necessidade} para a Alemanha e seus aliados terem ganho o Mundo II se tivessem desenvolvido a bomba atômica em 1945, dado nosso conhecimento histórico de que eles não tinham uma bomba atômica em fevereiro de 1945 e perderam a guerra. Se \(Y\) denota a Alemanha ganhando ou perdendo a guerra (0 ou 1) e \(X\) denota a Alemanha tendo a bomba em 1945 ou não a tendo (0 ou 1), a probabilidade de necessidade pode ser expressa no linguagem dos resultados potenciais,
     \[P[Y_{X=0} = 0|X = 1, Y = 1].\]
    
    Dual a \textit{probabilidade de suficiência}, a probabilidade de necessidade reflete a noção legal de causalidade ``\textit{but-for}'' como em: se não fosse por seu fracasso em construir uma bomba atômica em fevereiro de 1945, a Alemanha provavelmente teria vencido a guerra. Pearl aplica o mesmo tipo de raciocínio para gerar declarações transparentes sobre as mudanças climáticas. O aquecimento global antropogênico foi responsável pela onda de calor de 2003 na Europa? Todos nós já ouvimos que, embora o aquecimento global devido à atividade humana tenda a aumentar a probabilidade de ondas de calor extremas, não é possível atribuir nenhum evento específico a essa atividade. De acordo com Pearl e uma equipe de cientistas do clima, a resposta pode ser enquadrada de forma diferente: há 90\% de chance de que a onda de calor de 2003 na Europa não teria ocorrido na ausência do aquecimento global antropogênico [11].

    Essa formulação do impacto do aquecimento global antropogênico na Terra é forte e clara, mas está correta? O princípio do \textit{``lixo-dentro e do lixo-fora''} nos diz que os resultados baseados em um modelo causal não são melhores do que suas suposições subjacentes. Essas suposições podem representar o conhecimento e a experiência de um pesquisador. No entanto, muitos estudiosos estão preocupados com o fato de que os pressupostos do modelo representam o viés do pesquisador ou simplesmente não são examinados. David Freedman enfatiza isso em [12], e como ele escreveu mais recentemente em [13],

\begin{quotation}
        As suposições por trás dos modelos raramente são articuladas, muito menos defendidas. O problema é exacerbado porque os periódicos tendem a favorecer um grau moderado de novidade nos procedimentos estatísticos. Modelagem, a busca por significância, a preferência por novidades e a falta de interesse em suposições - essas normas provavelmente geram uma enxurrada de resultados não reproduzíveis.
        
        —Oasis ou Mirage?
\end{quotation}

    Os modelos causais podem ser usados para retroceder a partir das conclusões que preferimos para as suposições de suporte. Nossa tendência de raciocinar a serviço de nossas crenças anteriores é um tópico favorito do psicólogo moral Jonathan Haidt, autor de The Righteous Mind [14], que escreveu sobre ``o cão emocional e sua cauda racional''. Ou como Udny Yule explicou em [15],

\begin{quotation}
        Agora, suponho que seja possível, com um pouco de engenhosidade e boa vontade, racionalizar quase tudo.
        
        — Discurso presidencial de 1926 na Royal Statistical Society
\end{quotation}

    A preocupação com o impacto de preconceitos e preconceitos em estudos empíricos está crescendo, e vem de fontes tão diversas como o Professor de Medicina John Ioannides, que explicou por que a maioria das descobertas de pesquisas publicadas são falsas [16]; o comediante John Oliver, que nos alertou para sermos céticos ao ouvirmos a frase ``estudos mostram'' [17]; e o ex-escritor nova-iorquino Jonah Lehrer, que escreveu sobre os problemas com a ciência empírica em [18], mas mais tarde foi desacreditado por representar coisas que inventou como fatos.

    A abordagem gráfica da inferência causal que Pearl favorece tem sido influente, mas não é a única abordagem. Muitos pesquisadores contam com o modelo de resultados potenciais de Neyman (ou Neyman-Rubin), que é discutido em [19], [20], [21] e [22]. Na linguagem dos ensaios clínicos randomizados, um pesquisador que usa esse modelo tenta quantificar a diferença de impacto entre o tratamento e o não tratamento em indivíduos de um estudo observacional. Os escores de propensão são combinados em uma tentativa de equilibrar as desigualdades entre assuntos tratados e não tratados. Uma vez que nenhum assunto pode ser tratado e não tratado, no entanto, a estimativa necessária do impacto às vezes é formulada como um problema de valor ausente, uma perspectiva que Pearl contesta veementemente.

    Em outra direção, o conceito de fixação, desenvolvido por Heckman em [23] e Heckman e Pinto em [24], se assemelha, pelo menos superficialmente, ao operador do que Pearl utiliza. Aqueles que gostam de disputas acadêmicas podem olhar para o blog de Andrew Gelman, [25] e [26], para troca de ideias entre os discípulos de Pearl e Rubin (o próprio Rubin não parece participar - naquele fórum, pelo menos) ou para o tributos escritos por Pearl [27] e Heckman e Pinto [24] ao solitário Prêmio Nobel, Trygve Haavelmo, que foi o pioneiro da inferência causal em economia na década de 1940 em [28] e [29]. Esses diálogos têm sido controversos às vezes e trazem à mente a lei de Sayre, que diz que a política acadêmica é a forma mais cruel e amarga de política porque os riscos são muito baixos. É a opinião deste revisor que as diferenças entre essas abordagens para inferência causal são muito menos importantes do que suas semelhanças. Suporte para isso inclui construções por Pearl em [3] e por Thomas Richardson e James Robins em [30] incorporando contrafactuais em modelos gráficos de causa e efeito, unificando assim vários tópicos da literatura de inferência causal.

    
\noindent
\begin{minipage}[!ht]{\columnwidth}\centering
\epsfig{figure=figs/fig04goldberg.png,width=0.9\columnwidth}
\captionof{figure}{\scriptsize Inspetores do National Transportation Safety Board examinando o Uber sem motorista que matou um pedestre em Tempe, Arizona, em 18 de março de 2018.}
\label{fig:04}
\end{minipage}



    No final de uma tarde de julho de 2018, a coautora de Pearl, Dana Mackenzie, falou sobre inferência causal no Simons Institute da UC Berkeley. Sua apresentação foi na primeira pessoa do singular da perspectiva de Pearl, a mesma voz usada em The Book of Why, e concluiu com uma imagem do primeiro carro que dirige sozinho a matar um pedestre. De acordo com um relatório [31] do National Transportation Safety Board (NTSB), o carro reconheceu um objeto em seu caminho seis segundos antes da colisão fatal. Com um tempo de avanço de um segundo e meio, o carro identificou o objeto como um pedestre. Quando o carro tentou engatar o sistema de frenagem de emergência, nada aconteceu. O relatório do NTSB afirma que os engenheiros desativaram o sistema em resposta a uma preponderância de falsos positivos nos testes.

    Os engenheiros estavam certos, é claro, que paradas frequentes e abruptas tornam um carro que dirige sozinho inútil. Mackenzie gentil e otimista sugeriu que dotar o carro com um modelo causal que pode fazer julgamentos matizados sobre a intenção do pedestre pode ajudar. Se isso levasse a carros autônomos mais seguros e inteligentes, não seria a primeira vez que as ideias de Pearl levariam a uma tecnologia melhor. Seu trabalho fundamental em redes bayesianas foi incorporado à tecnologia de telefone celular, filtros de spam, biomonitoramento e muitas outras aplicações de importância prática.

    A professora Judea Pearl nos deu uma teoria da causalidade elegante, poderosa e controversa. Como ele pode dar a sua teoria a melhor chance de mudar a maneira como interpretamos os dados? Não existe uma receita para fazer isso, mas formar uma parceria com a escritora de ciências e professora Dana Mackenzie, um estudioso por direito próprio, foi uma ideia muito boa.


ACKNOWLEDGMENT.

    Esta revisão se beneficiou de diálogos com David Aldous, Bob Anderson, Wachi Bandera, Jeff Bohn, Brad DeLong, Michael Dempster, Peng Ding, Tingyue Gan, Nate Jensen, Barry Mazur, Liz Michaels, LaDene Otsuki, Caroline Ribet, Ken Ribet, Stephanie Ribet, Cosma Shalizi, Alex Shkolnik, Philip Stark, Lee Wilkinson e os participantes do grupo de almoço social do Departamento de Estatística de Berkeley da Universidade da Califórnia. Agradeço a Nick Jewell por me informar sobre os estudos científicos sobre a relação entre exercícios e colesterol, o que aumentou minha apreciação do Livro do Porquê.

References

    [1] Silver D, Simonyan JSK, Antonoglou I, Huang A, Guez A, Hubert T, Baker L, Lai M, Bolton A, Chen Y, Lillicrap T, Hui F, Sifre L, van den Driessche G, Graepel T, Hassabis D. Mastering the game of Go without human knowledge, Nature, vol. 550, pp. 354–359, 2017.

    [2] Wright S. Correlation and causation, Journal of Agricultural Research, vol. 20, no. 7, pp. 557–585, 1921.
    
    [3] Pearl J. Causality: Models, Reasoning, and Inference. Cambridge University Press, second ed., 2009. MR2548166
    
    [4] Spirtes P, Glymour C, Scheines R. Causation, Prediction and Search. The MIT Press, 2000. MR1815675
    
    [5] Cornfield J, Haenszel W, Hammond EC, Shimkin MB, Wynder EL. Smoking and lung cancer: recent evidence and a discussion of some questions, Journal of the National Cancer Institute, vol. 22, no. 1, pp. 173–203, 1959.
    
    [6] Christakis NA, Fowler JH. The spread of obesity in a large social network over 32 years, The New England Journal of Medicine, vol. 357, no. 4, pp. 370–379, 2007.
    
    [7] Lyons R. The spread of evidence-poor medicine via flawed social-network analysis, Statistics, Politics, and Policy, vol. 2, no. 1, pp. DOI: 10.2202/2151–7509.1024, 2011.
    
    [8] Shalizi CR, Thomas AC. Homophily and contagion are generically confounded in observational social network studies, Sociological Methods \& Research, vol. 40, no. 2, pp. 211–239, 2011. MR2767833

    [9] Lewis M. The Undoing Project: A Friendship That Changed Our Minds. W.W. Norton and Company, 2016.
    
    [10] Kahneman D, Tversky A. The simulation heuristic, in Judgment under Uncertainty: Heurisitics and Biases (D. Kahneman, P. Slovic, and A. Tversky, eds.), pp. 201–208, Cambridge University Press, 1982.
    
    [11] Hannart A, Pearl J, Otto F, Naveu P, Ghil M. Causal counterfactural theory for the attribution of weather and climate-related events, Bulletin of the American Meterological Society, vol. 97, pp. 99–110, 2016.
    
    [12] Freedman DA. Statistical models and shoe leather, Sociological Methodology, vol. 21, pp. 291–313, 1991.
    
    [13] Freedman DA. Oasis or mirage? Chance, vol. 21, no. 1, pp. 59–61, 2009. MR2422783
    
    [14] Haidt J. The Righteous Mind: Why Good People Are Divided by Politics and Religion. Vintage, 2013.
    
    [15] Yule U. Why do we sometimes get nonsense-correlations between time-series?–a study insampling and the nature of time-series, Royal Statistical Society, vol. 89, no. 1, 1926.
    
    [16] Ionnidis JPA. Why most published research findings are false, PLoS Med, vol. 2, no. 8, p. https://doi.org/10.1371/journal.pmed.0020124, 2005. MR2216666

    [17] Oliver J. Scientific studies: Last week tonight with John Oliver (HBO), May 2016.
    
    [18] Lehrer J. The truth wears off, The New Yorker, December 2010.
    
    [19] Neyman J. Sur les applications de la theorie des probabilities aux experiences agricoles: Essaies des principes., Statistical Science, vol. 5, pp. 463–472, 1923, 1990. 1923 manuscript translated by D.M. Dabrowska and T.P. Speed. MR1092985
    
    [20] Rubin DB. Estimating causal effects of treatments in randomized and non-randomized studies, Journal of Educational Psychology, vol. 66, no. 5, pp. 688–701, 1974.
    
    [21] Rubin DB. Causal inference using potential outcomes, Journal of the American Statistical Association, vol. 100, no. 469, pp. 322–331, 2005. MR2166071
    
    [22] Sekhon J. The Neyman-Rubin model of causal inference and estimation via matching methods, in The Oxford Handbook of Political Methodology (J. M. Box-Steffensmeier, H. E. Brady, and D. Collier, eds.), Oxford Handbooks Online, Oxford University Press, 2008.
    
    [23] Heckman J. The scientific model of causality, Sociological Methodology, vol. 35, pp. 1–97, 2005.
    
    [24] Heckman J, Pinto R. Causal analysis after Haavelmo, Econometric Theory, vol. 31, no. 1, pp. 115–151, 2015. MR3303188
    
    [25] Gelman A. Resolving disputes between J. Pearl and D. Rubin on causal inference, July 2009.
    
    [26] Gelman A. Judea Pearl overview on causal inference, and more general thoughts on the reexpression of existing methods by considering their implicit assumptions, 2014.
    
    [27] Pearl J. Trygve Haavelmo and the emergence of causal calculus, Econometric Theory, vol. 31, no. 1, pp. 152–179, 2015. MR3303189
    
    [28] Haavelmo T. The statistical implications of a system of simultaneous equations, Econometrica, vol. 11, no. 1, pp. 1– 12, 1943. MR0007954
    
    [29] Haavelmo T. The probability approach in econometrics, Econometrica, vol. 12, no. Supplement, pp. iii–iv+1–115, 1944. MR0010953
    
    [30] Richardson TS, Robins JM. Single world intervention graphs (SWIGS): A unification of the counterfactual and graphical approaches to causality, April 2013.
    
    [31] NTSB, Preliminary report released for crash involving pedestrian, uber technologies, inc., test vehicle, May 2018.



}

    Analogamente, o estudo do fenômeno de Interação, exige o conhecimento e a representação matemática de conceitos não de todos elementares relacionados aos estados biológicos e à dinâmica de transferência de informações entre indivíduos.

    Considerando que a interação ocorre e é descrita pela modificação do ``estado biológico'' específico de indivíduos devido a um intercambio mútuo de informações e influências entre eles, a sua descrição e análise também torna necessária a representação de uma diversidade de possíveis estados em uma população biologicamente heterogênea. A descrição do fenômeno geral de Interação (que participa subsidiariamente na descrição de fenômenos de Reprodução e de Mortalidade e como consequência também do processo de Seleção Natural) exige também a representação da heterogeneidade espacial ou estrutural da população, que são temas a serem tratados somente em um próximo capítulo.

    Uma situação simplificada e importante de interação é aquela que assume um caráter simplesmente populacional, isto é, quando se supõe que a homogeneidade biológica da população é preservada quando os estados biológicos individuais são sempre uniformemente iguais, ainda que variáveis simultaneamente. Neste caso, o estado biológico comum é descrito como uma propriedade populacional e não individual. (Por exemplo, o tempo médio de sobrevivência é uma característica ``populacional'' e não individual. Pode-se tratar da ``expectativa'' (probabilística) de sobrevivência de um individuo de onde se obtém o tempo de sobrevivência médio característico da população, mas não o contrário).

    Portanto, seguindo o Princípio da Simplicidade de Comenius e a Metodologia de Galileo, o presente capítulo iniciará o tratamento dos Modelos Matemáticos de Dinâmica Populacional restritos a um cenário minimalista definido pelas seguintes peculiaridades:

    \begin{enumerate}
    \item \textbf{Homogeneidade Fixa}: a População é formada por indivíduos em um mesmo ``estado'' biológico comum e invariável com o tempo, e para todos os efeitos completamente indistinguíveis entre si, excluindo assim a descrição da mutação e reduzindo a descrição biológica da população apenas uma característica biológica comum e constante de seus indivíduos, além do tamanho da população.

    \item \textbf{Independência entre os seus indivíduos}: não há troca de informações e influências que possam modificar o estado biológico comum dos indivíduos, o que significa a ausência de interações individuais e de diversidade biológica, consequentemente) restando considerar tão apenas os fenômenos básicos de Reprodução e Morte nestas condições.
    \end{enumerate}

    O Modelo minimalista desenvolvido neste cenário será denominado \textbf{Modelo Malthusiano} por razões históricas que apresentaremos oportunamente.
    
    Observemos, de saída, que um Modelo de Dinâmica Populacional não descreve qualquer distinção entre os seus indivíduos e, portanto, todos eles são considerados igualmente suscetíveis à reprodução e/ou à morte. Isto significa também que as únicas informações disponíveis para a representação populacional destes processos são o tamanho da própria população e o ``estado biológico comum'' que caracteriza uniformemente uma susceptibilidade à morte e a reprodução de seus indivíduos. A ausência de intercambio de informações (e de influencias externas), por sua vez, significa que os estados biológicos dos indivíduos serão constantes, inclusive os relativos à reprodução e mortalidade.

    Antes de passarmos à determinação específica do Modelo Malthusiano minimalista, consideremos a sua formulação dimensional (mensuração) que decorre das restrições acima.

    O conjunto mínimo desejável e indispensável de parâmetros descritores para um Modelo de Dinâmica Populacional deve necessariamente incluir \(n\) - o tamanho observado da população e \(t\) - o instante da respectiva observação. Considerações de Análise Dimensional desenvolvidos no capítulo anterior nos compelem a incluir também o parâmetro \(n_0\) - tamanho da população inicial e um parâmetro \(k\), com dimensão do tempo, que caracteriza uniformemente a população (ainda sem interpretação biológica).
    
    O Modelo Minimal de Dinâmica Populacional teria então a sua expressão dimensional completa da forma
    \[n = \varphi(t, n_0, k)\]
    e sua expressão adimensional da forma
    \[N =\psi(\tau),\]
    em que foram tomadas unidades intrínsecas de população e tempo
    \[N = \dfrac{n}{n_0} \mbox{ e } \tau = \dfrac{t}{k},\]
    e onde \(\psi\) é uma função matemática, isto é, independente de unidades básicas de população e tempo.
    
    Para a determinação completa de um Modelo Malthusiano específico será, portanto, necessário discriminar a função matemática de uma variável \(\psi(\tau)\), o que será feito nas próximas seções quando abordaremos alguns exemplos históricos que originaram o Modelo Minimalista.

\section{Origens históricas do Modelo Malthusiano e das Metodologias de Síntese Funcional de Galileo-Newton}

\begin{enumerate}
\item Homo Ludens - Modelos Brinquedos - ``Xadrez'' (Malba Tahan) e os ``Coelhos'' de Fibonacci - Metodologia Especulativa
\item Tabelas Demográficas das Colônias Britânicas - Malthus - Euler - Gauss - Regra da Proporção - Linearização Logarítmica
\item Tabelas de Mortalidade de Graunt-Neumann-Huygens-Euler - Regra Multiplicadora
\item Tabelas de Decaimento Radioativo - Rutherford - Metodologia de Galileo-Newton
\end{enumerate}

\subsection{Homo Ludens - Brinquedos e Jogos como Motivação: ``O Inventor do Xadrez'' sec. Malba Tahan, e os ``Coelhos'' Especulativos de Fibonacci}

\begin{citacao}
```O estulto quando se depara com um fato corriqueiro, prossegue o seu caminho sem se perturbar. O sábio, ao contrário, detém-se, observa o fato com cuidado e medita
longamente sobre ele, aumentando assim a sua sabedoria''.

\rightline{Confúcio}
\end{citacao}

    O modelo de Malthus, certamente não foi o mais antigo modelo matemático que previa um crescimento exponencial (geométrico) de populações. Na verdade, esta precedência é devida com inteira justiça ao matemático veneziano Leonardo di Pisa (1170-1250), conhecido como Fibonacci quem, por volta de 1202 (século XIII), escreveu um influente livro \textit{Liber Abaci}, talvez, o primeiro texto relevante de matemática na Europa pós clássica. (Kline[], Devlin[..]). Neste texto, Fibonacci apresenta a ciência aprendida por ele em suas andanças pelas terras islâmicas desde o califado de Bagdad passando pelo norte da África até a península Ibérica. (Este fato deve nos alertar para a possibilidade de que o seu modelo talvez já fizesse parte de um antigo conto das Mil e uma Noites e isso pode ser pesquisado pelo/a leitor/a interessado/a).

    O livro de Fibonacci não se dedica à teoria matemática segundo a tradição grega mas, é prático, segundo a tradição oriental (Babilônica e Egípcia) e apresenta suas ideias por intermédio da proposição de questões (desafios!) seguidas de sua resolução, dentre as quais se encontra um conhecido problema sobre uma improvável população de coelhos. A formulação deste problema pode ser modificada para o contexto de dinâmica populacional da seguinte maneira:

    Considere uma população de ``casais de coelhos'' representada pelo número de fêmeas, supondo tacitamente que há um equilíbrio populacional entre os sexos.

\begin{enumerate}
\item No período de observação (digamos, um ano) não há mortalidade de coelhos,
\item Os coelhos se tornam maduros (férteis), a partir do mês seguinte ao seu nascimento.
\item Os coelhos, uma vez maduros, produzem um novo coelho (i.e., um casal) por mês.
\end{enumerate}

\begin{exercise}
Se, no início do primeiro mês, dispomos de um coelho imaturo, quantos coelhos teremos em 12 meses?
\end{exercise}

{\color{red} \textbf{Solução}:
{\scriptsize
\[\begin{array}{cccccccccccccc}
t & 0 & 1 & 2 & 3 & 4 & 5 & 6 & 7 & 8 & 9 & 10 & 11 & 12 \\
M & 0 & 1 & 1 & 2 & 3 & 5 & 8 & 13 & 21 & 34 & 55 & 89 & 144 \\
N & 1 & 0 & 1 & 1 & 2 & 3 & 5 & 8 & 13 & 21 & 34 & 55 & 89 \\
\mbox{Total} & 1 & 1 & 2 & 3 & 5 & 8 & 13 & 21 & 34 & 55 & 89 & 144 & 233
\end{array}\]
}
onde \(t\) é o tempo em meses, \(M\) representa a quantidade de coelhos maduros, \(N\), de coelhos novos. Claramente, a linha de totais é uma sequência de Fibonacci e é definida pela fórmula recursiva:
\[F(n+2)=F(n+1)+F(n),\]
com \(n \ge 1\)  e \(F(1) = F(2) = 1.\)
}


\begin{exercise}
Represente a população descrita ``biologicamente'' por Fibonacci, agora por intermédio de um grafo (``árvore'').
\end{exercise}


    Uma terceira forma mais matemática e contemporânea de descrever a população de ``coelhos de Fibonacci'' utiliza um argumento recursivo (algorítmico) para caracterizar a função \(N(k)\) que representa a sua medida no \(k\)-ésimo mês. Registrando
    \begin{description}
    \item \(M(k) = \) ``Número total de coelhos maduros no \(k\)-ésimo mês'';
    \item \(I(k) = \) ``Número de coelhos imaturos'',
    \end{description}
    é fácil estabelecer a veracidade das recorrências:
    \[M(k+1) = M(k) + I(k) \mbox{ e } I(k+1) = M(k),\]
    de onde se obtém a recursão de Fibonacci (``Número total de coelhos no \(k+2\)-ésimo mês''):
    \[N(k+2) = M(k+2) + I(k+2) = N(k+1) + N(k).\]

    Assim, por exemplo, com valores iniciais \(N(0) = 1\) e \(N(1) = 2\), todos os valores seguintes são obtidos recursivamente.

    Surpreendentemente, a função \(N(k)\) pode ser também representada de uma quarta forma, agora por intermédio de uma \textit{Fórmula Elementar} (isto é, Algébrica):
    \[N(k) = c_1 \left(\dfrac{1+\sqrt{5}}{2}\right)^k + c_2 \left(\dfrac{1-\sqrt{5}}{2}\right)^k.\]
    Ver (Bassanezi-Ferreira, 1988).

    Portanto, para \(k \gg 1\) (grande) , temos
    \[N(k) \simeq c_1 \left(\dfrac{1+\sqrt{5}}{2}\right)^k = c_1 e^{\gamma k}\]
    onde \(\gamma = \ln\left(\dfrac{1+\sqrt{5}}{2}\right)\), o que indica um crescimento geométrico, ou exponencial, já que \(\dfrac{1+\sqrt{5}}{2} > 1\).

\begin{exercise}
Verifique as três afirmações acima, i.e., sobre a veracidade
\begin{enumerate}
\item da Recursão como representativa do Fenômeno;
\item da Fórmula para representar a População e;
\item do comportamento assintótico exponencial/geométrico da Fórmula.
\end{enumerate}
\end{exercise}


{\color{red}
\textbf{Solução}:
1. Utilizando a fórmula recursiva
\[N(k+2) = N(k+1) + N(k),\]
encontramos a tabela a seguir:
\[\begin{array}{|c|c|c|c|} \hline
k & N(k+1) & N(k) & N(k+2) \\ \hline
0 & 2 & 1 & 3 \\ \hline
1 & 3 & 2 & 5 \\ \hline
2 & 5 & 3 & 8 \\ \hline
3 & 8 & 5 & 13 \\ \hline
4 & 13 & 8 & 21 \\ \hline
5 & 21 & 13 & 34 \\ \hline
6 & 34 & 21 & 55 \\ \hline
7 & 55 & 34 & 89 \\ \hline
8 & 89 & 55 & 144 \\ \hline
9 & 144 & 89 & 233 \\ \hline
10 & 233 & 144 & 377 \\ \hline
11 & 377 & 233 & 610 \\ \hline
12 & 610 & 377 & 987 \\ \hline
\end{array}\]
que verifica o obtido no exercício anterior.

2. Considere que a sequência de Fibonacci \(N(n+2) = N(n+1) + N(n)\), com \(n\ge 0\) é uma sequência geométrica de razão \(0 < q \ne 1\), com \(N(1) = 1\).

Claramente, \(N(n) = \{1, q, q^2, \ldots, q^{n-1}\}\)

Portanto,
\[N(n+2) = N(n+1) + N(n)\]
equivale a
\[q^{n+1} = q^{n} + q^{n-1},\]
ou seja,
\[q^2-q-1 =0,\]
cujas raízes são \(q_1 = \dfrac{1-\sqrt{5}}{2}\) e \(q_2 = \dfrac{1+\sqrt{5}}{2}\).

Dessa forma, podemos escrever:
\[N(n+2) = q_1^{n} + q_1^{n-1} = (1+q_1^{-1}) q_1^{n}\]
ou
\[N(n+2) = q_2^{n} + q_2^{n-1} = (1+q_2^{-1}) q_2^{n}.\]

Seque que
\[N(n+2) = c_1 \left(\dfrac{1-\sqrt{5}}{2}\right)^{n} + c_2 \left(\dfrac{1+\sqrt{5}}{2}\right)^{n},\]
com \(c_1 = \dfrac{1+q_1^{-1}}{2}\) e \(c_2 = \dfrac{1+q_2^{-1}}{2}\).


}

\begin{exercise}
Mostre que a função \(N(k)\) pode ser representada por intermédio de uma quinta forma, isto é, como solução de uma equação de diferenças finitas da forma \(P(E) \varphi = 0\) ou \(Q(\Delta) \varphi = 0\) onde \(P, Q\) são polinômios e \(E\) é um operador de deslocamento e \(\Delta\) um operador de diferença, ambos aplicáveis no conjunto de funções discretas \(\mathcal{F} = \{\varphi: \mathbb{N} \to \mathbb{R}\}\). (Consulte Bassanezi-Ferreira (1988) a respeito).
\end{exercise}


    A ideia fundamental do modelo de Fibonacci pode ser interpretada hoje como a representação de dois estados biológicos para os indivíduos na forma de duas sub-populações (imaturos e maduros), e a atribuição de uma fertilidade distinta a cada uma delas (\(0\) e \(1\)). Observe que Fibonacci não atribui uma mortalidade para os indivíduos de sua população, e podemos justificá-lo (a revelia dele) supondo que o seu problema trata de determinar o número de coelhos sadios e bem tratados após um curto período de tempo, 12 meses.

    O conceito de crescimento exponencial já fora tratado de várias formas na antiguidade. Talvez a mais conhecida e pitoresca seja a estória sobre a invenção do jogo de xadrez que um servo hindu ofereceu como presente ao Rei de seu país. Instado a apresentar um pedido de recompensa a seu critério por essa extraordinária invenção, ele maliciosamente, sugeriu um pagamento em trigo na quantidade que resultasse da seguinte disposição: um grão seria colocado na primeira casa de um tabuleiro de xadrez e o dobro na casa seguinte a cada avanço de uma das suas \(64\) casas. Aceito o valor, até com certo desdém pela sua aparente humildade, o Rei foi logo surpreendido pelos seus conselheiros mais letrados com a notícia da impossibilidade de resgatar tamanha dívida, da ordem de \(2^{63} \simeq 10^{19}\) grãos, mesmo com todas as colheitas de trigo do planeta. (Ref. Malba Tahan - O Homem que Calculava, )

\begin{exercise}
    Avalie o número de sacas de 20kg de trigo e o número de caminhões que seriam necessários para transportar esta quantidade de trigo. Compare esta quantidade com a produção mundial anual deste cereal. (Ref. Maharajan, Weinstein).
\end{exercise}

{
\def\atrigo{0.30}%cm
\def\btrigo{0.25}%cm
\def\ctrigo{0.24}%cm

\color{red}
\textbf{Solução}: Considere que:

(a) grãos de trigo se aproximem do formato de um elipsoide de equação
\[\dfrac{x^2}{a^2} + \dfrac{y^2}{b^2} + \dfrac{z^2}{c^2} = 1\]
e que, em média, temos os valores de \(a = \atrigo\ cm\), \(b = \btrigo\ cm\) e \(c = \ctrigo\ cm\).

Portanto, o volume médio de um grão de trigo é dado por:
\[V = \dfrac{4}{3}\pi abc = \ca{4/3*3.14*\atrigo*\btrigo*\ctrigo*10^{-6}}\ m^3\]
\def\vtrigo{.000000075359999999} % volume (m^3) de um grão de trigo
\def\dtrigo{770} % densidade (kg/m^3) de um grão de trigo - 770 não é o valor correto!

(b) a densidade média de um grão de trigo seja \(\varrho = 770\ kg/m^3\) \href{https://ainfo.cnptia.embrapa.br/digital/bitstream/item/161751/1/246.pdf}{(BENASSI, 2017. p. 247)}.%Essa referência está errada!

Portanto, a massa de um grão de trigo é
\[M = \ca{\vtrigo*\dtrigo}\ kg\]
\def\mtrigo{0.00005802719999923}

(c) a quantidade \(N\) de grãos de trigo que deveria ser paga ao servo hindu é:
\[N = 1+2+4+\ldots+2^{63},\]
ou seja, a soma dos \(n = 64\) termos de uma progressão geométrica de razão \(q = 2\). Dessa forma:
\[
N
= a_1 \dfrac{q^n-1}{q-1}
= 2^{64}-1 \approx 1,84\ \cdot 10^{19}
%= \ca{2^{64}-1}
\]

\def\ntrigo{18446744073709600000}
(d) a quantidade \(S\) de sacos de \(20\ kg\) é:
\[S = \dfrac{M \cdot N}{20} = \ca{\mtrigo/20*\ntrigo}\]
\def\strigo{535206453849.878854091621804}

\def\kcaminhao{500}
(e) um caminhão consiga transportar \(\kcaminhao\ kg\) de mercadoria. Então, seria necessário:
\[C = S/500 = \ca{\strigo/\kcaminhao}\]
caminhões para transportar todo o trigo.
}

    O problema de Fibonacci quase nunca é ``levado à sério'' ou citado como um modelo populacional na literatura contemporânea, provavelmente devido à sua apresentação original que toma a forma de um ingênuo quebra-cabeças. Por este motivo, apesar de amplamente conhecido na Europa desde a sua publicação, o caráter fundamental dos argumentos contidos no texto de Fibonacci passaram desapercebidos por gerações de importantes cientistas e matemáticos aplicados como se fossem meros truques infantis irrelevantes para a Matemática Aplicada. Até que quinhentos anos mais tarde, Euler, um discípulo não presencial de Confucio, os empregou e generalizou na formulação de seu modelo de 1760 que ainda hoje é a base para a demografia. (N. Bacaer, H. Caswell, J. Cohen, N. Keyfitz)

    Votaremos a tratar destes e de outros assim chamados ``Modelos Brinquedos'' (``Toy Models'') ao longo do texto, assim como especificamente da lição histórica que este episódio encerra quando sinaliza que as origens de vários Modelos Matemáticos, alguns até muito sisudos, são encontráveis em atividades lúdicas enraizadas na cultura popular. (J. Huisinga - Homo Ludens: Ensaio sobre a função social do jogo, 1938,
    \href{http://www.math.pitt.edu/~bard/bardware/toys2.pdf}{G. Bard Ermentrout}, \href{https://mathematics.stanford.edu/people/tadashi-tokieda}{Tadashi Tokieda}, \href{https://www.youtube.com/playlist?listPLt5AfwLFPxWI9eDSJREzp1wvOJsjt23H}{YouTube}).

\begin{exercise}
Variações sobre o tema de Fibonacci.
\end{exercise}

{\color{red}
Resposta: A sequência de Fibonacci (Leonardo de Pisa (1175-1240)) apresenta vários fatos curiosos:

(a) através do quociente de um número com o seu antecessor, obtém-se uma sequência cujos termos tendem a constante 1,6180339887... (número áureo);

(b) foi utilizada por Da Vinci, que chamou a sequência de Divina Proporção, para fazer desenhos perfeitos;

(c) A partir dessa sequência, pode ser construído um retângulo, que é chamado de Retângulo de Ouro e ao desenhar um arco dentro desse retângulo, obtemos, por sua vez, a Espiral de Fibonacci;

(d) Pode ser encontrada em alguns elementos da natureza como:
nas folhas das cabeças das alfaces;
na couve-flor;
nas camadas das cebolas;
nos padrões de saliências dos ananases;
nas sementes das pinhas (oito irradiando no sentido horário e 13 no anti-horário);
na concha de Nautilus;
na folha de uma Bromélia;
na cauda de um camaleão;
nas presas de marfim de um elefante se crescessem infinitamente;
nas sementes de um girassol (Neste caso são dois conjuntos de espirais, 21 no sentido horário e 34 no sentido anti-horário).

(e) Na espirradeira ou na cevadilha, mostra os números da sucessão de Fibonacci nos seus pontos de crescimento. Quando tem um novo rebento, leva dois meses a crescer até que as ramificações fiquem suficientemente fortes. Se a planta ramifica todos os meses, depois disso, no ponto de ramificação, obtemos ramificações que correspondem aos números de Fibonacci.

(f) no número de abelhas em cada geração da árvore genealógica de um zângão. Um zângão tem apenas um dos pais (pois provém de um ovo não fertilizado), ao passo que a fêmea exige ambos os pais (pois provem de um ovo fertilizado).
}


\begin{exercise}
    Consideremos uma população de organismos que se reproduzem individualmente. Suponhamos, como Fibonacci, que o tempo é medido discretamente e registrado em períodos uniformes, de tal forma que os indivíduos desta população tornam-se reprodutíveis após uma unidade de tempo e que, após esta maturação, cada um deles, enquanto vivo, produz \(\nu\) novos indivíduos em cada unidade de tempo. Suponhamos também que a cada unidade de tempo uma fração \(\mu\) de indivíduos desta população perece. (Este modelo de reprodução e de mortalidade é dito proporcional já que são diretamente proporcionais ao tempo).

\begin{enumerate}
    \item Mostre que se \(\varphi: \mathbb{N} \to \mathbb{C},\ \varphi(k)\) representa o número de indivíduos desta população no instante \(k\), então a população acima descrita lhe induz a seguinte restrição (Equação Recursiva de segunda Ordem) \(P(E) \varphi = 0\), onde \(P(z) = a_0 + a_1 z + a_2 z^2\) é uma polinômio algébrico de segundo grau, isto é, \(a_0f(k) + a_1f(\color{red}{n}+k) + a_2 f(k+2) = 0\). (\textbf{Sugestão}: Segundo Fibonacci, a população no instante \((k+2)\) é constituída da população sobrevivente do ano passado, mais os novos integrantes nascidos no período anterior, que são filhos de quem já estava presente na população dois períodos atrás e não morreu.)
    \item Mostre, portanto, que a recursão é de ordem \(2\) e completamente determinada quando são conhecidos \(F(0)\) e \(F(1)\).
    \item Obtenha condições nos parâmetros \(\mu\) e \(\nu\) para que a população apresente um crescimento exponencial
    \item Obtenha condições nos parâmetros \(\mu\) e \(\nu\) para que a população apresente uma extinção em decrescimento exponencial.
    \item Obtenha condições nos parâmetros \(\mu\) e \(\nu\) para que a população apresente uma oscilação e crescimento exponencial
    \item Mostre que descrevendo esta população pela função \(\Phi: \mathbb{N} \to \mathbb{C}^2\), \(\Phi_1(k) =\) ``População de Férteis no instante \(k\)'', \(\Phi_2(k) =\) ``População de Imaturos no instante \(k\)'' o argumento de Fibonacci pode ser representado na forma recursiva de primeira ordem:
    \[\Phi(k+1) = A \Phi(k),\] onde \(A\) é uma matriz \(2 \times 2\).
    \item \textbf{Método de Fourier - Transformação}: Obtenha uma matriz \(P\) invertível tal que \(P^{-1}A P = D\) é diagonal, \(D = \operatorname{diag}\{\lambda_1, \lambda_2\}\). Obtenha a função \(\Phi(k) = P \operatorname{diag}\{\lambda_1^k, \lambda_2^k\} P^{-1}\).
    \item \textbf{Método de Fourier - Espectral}: Se \(A v_j = \lambda_j v_j\) (onde \(v_j\) é autovetor e \(\lambda_j\) é autovalor), mostre que \(\Phi^{(j)}(k) = (\lambda_j)^{k} v_j\) são soluções da recorrência vetorial. Mostre como obter a solução da equação com condição inicial \(\Phi(0)\) dada.
    \item* \textbf{Método das Funções Geradoras}: Represente os valores da função discreta \(\varphi(k)\) como coeficientes da expansão de uma função analítica \(f\), isto é, \(f(z) = \displaystyle \sum_{k=0}^{\infty} \varphi(k) z^k\), obtenha uma equação funcional para \(f\) com base nas propriedades de \(\varphi\) e obtenha uma representação elementar para a função \(f(z)\). Com base nesta sintetização dos valores da função discreta em uma ``Função Geradora'', mostre como recuperar os valores de seus coeficientes utilizando a Análise Complexa.
    \item Generalize o Modelo de Fibonacci considerando fertilidade variável, \(\nu_j =\) ``Numero de descendentes produzidos por indivíduo com idade \(j\) em uma unidade de tempo'' e também mortalidade variável, \(\mu_j =\) ``Fração de indivíduos de uma população de idade \(j\) que falecem em uma unidade de tempo''.
\end{enumerate}
\end{exercise}


\section{Modelos de reprodução exponencial de Malthus-Euler: Tabelas Demográficas - Regra da Proporção Simples - Retificação Logarítmica dos Dados - Método de Representação Funcional (linear) de Gauss}
    
    \begin{citacao}
    ```Para a existência da morte, a vida é necessária e suficiente, mas, para procriar, é necessário, mas não suficiente, a vida e a morte''.\footnote{Ditado Popular parafraseado com passagem bíblica.}
    \end{citacao}

    Historicamente há duas vertentes distintas que levaram á formulação do Modelo Malthusiano minimalista. Um deles, que considera separadamente o ``fenômeno'' de Mortalidade, enquanto que o outro, focaliza apenas o ``fenômeno'' de Reprodução. Ambos representam os dois lados da mesma moeda, surgiram de forma independente no século XVIII e devem muito credito a Leonhard Euler, direta ou indiretamente.

    Tanto a doutrina Malthusiana quanto o Modelo matemático de Euler a serem apresentados, são claramente baseados em fatos expressos nas Tabelas demográficas sobre as colônias britânicas da América no século XVIII o que os coloca, aparentemente, de acordo com um dos requisitos básicos da Metodologia de Galileo que prevê um fundamento experimental para a formulação de um modelo matemático.

    Apresentaremos, a seguir, duas classes de argumentos que poderiam ter sido empregados, implícita ou explicitamente, na formulação original dos Modelos Reprodutivos de Malthus-Euler segundo a Metodologia de Galileo. Estes Métodos (denominados DDM - ``De Dados a Modelos'', ou ``\textit{Data Driven Models}'') tem antigas e nobres origens como poderemos verificar em seguida e continuam fundamentais e em pleno desenvolvimento na Matemática Aplicada contemporânea. (J. Nathan Kutz).

    \subsection{O Dilema entre acuracidade e Interpretação - Método de Gauss para a Representação Linear Ótima-Retificação Logarítmica}

    A identificação de uma função analítica cujo gráfico descreva geometricamente a ``nuvem de pontos'' resultante da representação cartesiana de uma Tabela de correlação numérica é uma tarefa difícil de ser visualmente implementada em geral, a não ser no caso em que estes pontos se acumulem claramente no entorno de uma reta.

    De qualquer forma, mesmo que visualmente, a nuvem de pontos definida pela Tabela de dados não se apresente claramente acumulada em torno de alguma reta, é possível determinar, dentre todas elas, aquela que ``melhor se ajusta'' a nuvem de pontos segundo um critério bem determinado.

    A definição de um critério de ``ajuste'' com fundamentos geométricos e determinar matematicamente a reta que melhor representaria uma distribuição de pontos no plano, Carl Friedrich Gauss desenvolveu o que hoje é denominado o ``\textbf{Método de Minimização de Quadrados}''. Assim, com este Método, é possível descrever a melhor expressão de primeiro grau que descreve funcionalmente a referida Tabela.
    
    Entretanto, é claro que nem sempre a restrição a funções lineares produz uma representação suficientemente acurada dos dados experimentais, embora a sua simplicidade (que é carácter {\color{red} ??????}

    Estamos, portanto, diante de um dilema, que confronta a desejável acuracidade da representação funcional e a indispensável possibilidade de interpretá-la biologicamente.

    A interpolação polinomial de Lagrange para uma Tabela de dados, por exemplo, é um extremo em que uma acuracidade máxima acompanha uma quase certa impossibilidade de interpretação biológica com suas centenas de parâmetros.

    Com a sua característica perspicácia e conhecimento de Geometria Euclideana, Galileo conseguiu o feito notável de ``visualizar'' uma parábola (função quadrática com três parâmetros) para a Tabela da trajetória de balas de canhão e fornecer uma interpretação física completa para ela. O mesmo se pode dizer de Kepler que estabeleceu o caráter elíptico (função quadrática) das órbitas de cometas a partir das Tabelas astronômicas registradas por Tycho Brahe.

    Todavia, é razoavelmente claro que a ``\textit{detecção à ôlho nu}'' de uma representação funcional acurada e ao mesmo tempo interpretável para uma enorme Tabela de dados experimentais, se constitui em notável exceção e uma tarefa quase impossível para os menos eruditos e perspicazes do que aqueles gigantes.

    Portanto, para fazer uso da brilhante Metodologia de Galileo (sem a ajuda de um Galileo ou de um Newton) é necessário desenvolver Métodos mais algorítmicos e com fundamentos mais matemáticos para implementá-la.
    
    Assumindo, por enquanto, que o ajuste {\red por} retas {\red a uma} nuvens de pontos seja um problema adequadamente resolvido, a representação funcional dos pontos de uma Tabela geral de dados passa necessariamente pela possibilidade de uma \textit{retificação} da Tabela, ou seja, o desenvolvimento de procedimentos que ``\textit{retifiquem}'' a Tabela inicial de dados, no sentido de torná-la bem representável pelo Método de Gauss. E, para este propósito, lançaremos mão de uma das ideias mais férteis e comumente utilizadas em Matemática: A ``\textit{mudança de variáveis}".

    Para esclarecer as bases da ideia de ``\textit{Retificação}'' de uma Tabela de dados, consideremos o exemplo abaixo.

\begin{example}[Nuvens de Pontos Retificáveis por Mudança de Variáveis]

    Consideremos a representação cartesiana de uma ``enorme'' Tabela de dados \(T = \{(x_k, y_k)\}\) como uma ``nuvem de pontos'' gerada pelo cálculo da função
    \[y = \varphi(x) = \dfrac{x+a}{x+b},\]
    com \(a \ne b\) positivos, para vários pontos \(x_k > -b\).
    
    Conhecendo, de antemão, o gráfico da função \(\varphi\), sabemos que, no caso de uma razoável proximidade dos pontos obtidos, o \textbf{Efeito de completamento de Kanizsa} nos levará, indubitavelmente, a associar esta nuvem de pontos a uma curva contínua descrita pelo gráfico da função \(\varphi\) que, sem maiores informações, seria muito difícil identificar apesar de sua simplicidade algébrica. Entretanto, ``massageando'' a relação na forma:
    \[y = \varphi(x) = \dfrac{x+a}{x+b} = 1+\dfrac{a-b}{x+b},\]
    ou seja,
    \[y = 1+(a-b)\dfrac{1}{x+b},\]
    e fazendo as mudanças de variáveis, \(v = y\) e \(u = (x+b)^{-1}\), a ``nuvem'' de pontos \((u_k, v_k)\) é, certamente, ``retificada'' (na verdade poderá ser retificada com a reta \(v = 1+(a-b)u\)).
\end{example}

    Afora a circunferência, a reta no plano é a mais simples das curvas, pois exige apenas dois parâmetros para caracterizá-la biunivocamente e, talvez, seja a única curva que podemos identificar visualmente com razoável fidelidade, assim como identificar matematicamente sua natureza com métodos elementares. (Essencialmente, isto se faz utilizando apenas a semelhança de triângulos, segundo a Geometria Euclideana plana). Por esta razão, estamos dispostos a sacrificar muita coisa e lançar mão de muitos argumentos para reduzirmos uma questão à determinação de uma reta.
    
    Entretanto, embora a representação funcional linear de uma nuvem de pontos \((x_k, y_k)\) seja muito conveniente e mais simples, ela nem sempre é suficientemente acurada, como é o caso do exemplo acima. A estratégia mais comum da Matemática para a resolução de um problema (que consiste em transformá-lo para a forma que podemos solucioná-lo por mudanças de variáveis \(u = \alpha(x)\) e \(v = \beta(y)\) de tal forma que a nova nuvem de pontos \(S_k = (u_k, v_k) = (\alpha(x_k), \beta(y_k))\) seja acuradamente representável por uma reta. Assim, a obtenção das funções retificadoras \(\alpha(x)\) e \(\beta(y)\), torna-se o novo problema.

    O exemplo explícito apresentado, embora obviamente artificial (porque já conhecemos a representação funcional para a Tabela original e este é exatamente o problema a ser resolvido!), exemplifica bem como o mecanismo interno do procedimento de linearização pode funcionar. É interessante observar que este simples exemplo comparece na derivação do \textbf{método de Lineweaver-Burke} para determinação gráfica de constantes de reações enzimáticas e que será apresentado mais adiante. É importante também observar que nem sempre uma nuvem de pontos pode ser ``retificada'' por mudança de variáveis e a geometria de uma ``nuvem circular'' é o exemplo mais simples desse fato.

    O cerne da questão (``\textit{The crux of the matter}'') é que o estabelecimento de um procedimento geral para a obtenção de funções retificadoras é tão difícil quanto a solução dos problemas originais, o que parece não favorecer muito a utilidade desta ideia. O que a salva, entretanto, são duas situações específicas, mas importantes: A primeira relacionada ao fato de que a \textbf{função logarítmica} é uma função retificadora que desempenha um papel frequentemente apropriado para muitos casos, especialmente se uma das variáveis tem uma faixa de variação muito maior do que a outra. A segunda decorre do fato de que, em muitos casos, não é necessária uma retificação ``global'' para todo o domínio da função, mas apenas nas vizinhanças de um limite, em geral infinito. Trataremos destas duas alternativas logo abaixo.

%Observações:

\begin{remark}
    A ``escala logarítmica'' já deve ser familiar para o/a leitor/a por intermédio da utilização do conhecido ``papel com escala logaritmo'' (ou a régua de cálculo) que tem a vantagem de ``compactar'' uma grande variação de pontos em um ``pequeno'' intervalo, de maneira não uniforme, claro.
\end{remark}

\begin{remark}
    Se alguém considera que uma representação linear reduzida a uma restrita vizinhança é ``pouca coisa", lembre-se que todo o Cálculo Diferencial é totalmente baseado na aproximação linear ótima em vizinhanças ``infinitesimais'' (i.e., locais) de pontos do domínio de uma função, ou seja, a tangente representada pela derivada. No presente caso, espera-se que a representação linear tenha um caráter um pouco mais dilatado no sentido de prover uma boa aproximação para pontos de um intervalo inteiro, e não apenas ``infinitesimal".
\end{remark}

    Retornando à questão populacional, suponhamos que \(P(k)\) represente os casais de uma grande população humana (com equilíbrio de sexos) para cada ano \(k\). Como esta população é muito grande quando medida por número de cabeças e varia enormemente, é plausível que tenha ocorrido a Euler que melhor seria ``compactar'' a sua representação em uma escala não uniforme, no caso tomando \(p(k) = \ln(P(k))\) (Lembre-se de que Euler foi praticamente o responsável pela introdução sistemática na literatura da representação analítica para as funções exponencial e logarítmica).
    
    Uma vez compactada, a sua representação devido a uma compressão vertical das coordenadas, tornou-se razoavelmente claro para Euler (e esta é uma observação experimental que faz uso do Efeito de completamento visual de Kanizsa) que a ``nuvem de pontos'' \((\ln(P(k)), k)\) se aglutinam na proximidade de uma mesma reta, ou seja, a transformação logarítmica ``retifica'' suficientemente a Tabela inicial.

    Assim, se for possível representar esta Tabela \textit{logaritmizada} por uma reta, então será possível representá-la funcionalmente (utilizando o Método de Gauss) na forma \(\ln(P(k)) = \gamma k+b\). Com isto, obtém-se a representação funcional para a Tabela original de dados demográficos em suas próprias variáveis na forma exponencial, \(P(k) = e^{b+\gamma k} = A e^{\gamma k}\), ou recursivamente, \(P(k+1) = r P(k)\) que representa exatamente o modelo de reprodução proporcional (\(r = 1+\alpha\)) de Euler-Malthus.

    É difícil saber se este foi, de fato, o argumento de Euler, mas se ``\textit{Non è vero, è molto ben trovato}'' e o argumento, de qualquer maneira, é de grande utilidade geral e deve ser conhecido.

    Voltemos, agora, à questão da representação linear ótima desenvolvida por Gauss.
    
    Em muitos casos, a representação linear é apenas uma ``sugestão'' intuitiva, que exige uma abordagem mais matemática para determinar, com exatidão matemática, o sentido da expressão ``\textit{a reta que melhor se ajusta à nuvem de pontos dados}''.
    
    Um Método matemático de ``\textit{ajuste ótimo}'' de uma reta a uma ``\textit{nuvem de pontos no plano}'' foi introduzido pelo influente matemático Carl F. Gauss (1777-1855) e denominado \textbf{Método de Minimização de ``Êrros'' Quadráticos}. Este Método determina univocamente a reta que minimiza a soma dos quadrados das distâncias entre os pontos da nuvem e a reta e, com isso, apresenta uma medida da máxima acuracidade possível para o ``\textit{ajuste}'' linear.

    Lembremos que a insistência em buscar uma (simples) reta para representar uma nuvem de pontos (Tabela) decorre exatamente da sua simplicidade, por ser completamente definida com apenas \textbf{dois} parâmetros, o que aumenta a chance de que os mesmos tenham interpretações associadas ao problema biológico. (Uma interpolação de Lagrange com 100 parâmetros não tem qualquer chance de ser biologicamente interpretada).

    Embora se atribua a Gauss a introdução formal deste método na literatura matemática (que ele inventou para analisar dados astronômicos sobre o planeta Mércurio estudados por ele) as ideias pertinentes certamente já eram conhecidas pelo grande calculador Leonhard Euler (1707-1787) e é razoável assumir que ele as utilizava com frequência.

%Exercício:

\begin{exercise}
Verifique, historicamente, se Euler utilizou ideias semelhantes em algum contexto de sua volumosa Opera Omnia. (Ref. Biografias de Euler, R. Graham \& D. Knuth).
\end{exercise}

\begin{exercise}
Consulte uma referência \textit{Baliza} sobre o Método de Quadrados Mínimos, sua origem e generalizações e faça um resenha de 20 linhas a respeito. (A generalização das ideias da
representação linear de Gauss para funções de variáveis vetoriais com o ajuste de hiperplanos a uma nuvem de pontos \(P^k = x_1^{(k)}, \ldots, x_n^{(k)} \in \mathbb{R}^n\) é um dos Métodos contemporâneos mais importantes na busca de uma representação funcional para Tabelas de dados discretos multivariados e se constitui em um instrumento computacional indispensável na computação científica contemporânea (G. Strang - Introduction to Linear Algebra, Wellesley, J. N. Kutz -...).
\end{exercise}

{\red 
\textbf{Resposta}: Segundo ..., Gauss chegou independentemente à curva dos erros (a função \(f(x)\) de frequência dos erros admite máximo para \(x=0\), simétrica e admitir valor zero fora do limite dos erros possíveis) através do seu estudo, feito de forma empírica e adotando como axioma o princípio de que o valor mais provável de uma quantidade desconhecida, observada com igual precisão várias vezes sob as mesmas circunstâncias é a média aritmética das observações. Esse resultado foi publicado em \textit{Theoria Motus Corporum Coelestium in Sectionibus Conicis Solum Ambientum}, de 1809. Esses estudos, levaram-no a enunciar o \textit{Princípio dos Mínimos Quadrados}.
}


(Publicado no livro em 1809)


Ver https://www.alice.cnptia.embrapa.br/alice/bitstream/doc/110361/1/sgetexto21.pdf



\subsection{O Teorema Fundamental da Distribuição de Números Primos}

    É interessante observar que o famoso Teorema Fundamental sobre a distribuição de Números primos foi sugerido por Gauss com base nas Tabelas que ele obsessivamente calculava sobre este tema desde a sua infância (W. Buhler, T. Hald, M. Abramowitz \& I. Stegun).
    
    O seu objetivo era representar funcionalmente (segundo uma Metodologia de Galileo) as suas Tabelas que, para cada inteiro \(n\) relaciona a densidade média
    \[\rho_{n} = \frac{\pi(n)}{n}\]
    dos números primos no intervalo \([0, n]\), onde \(\pi(n)\) é a ``Quantidade de números primos existentes no intervalo \([0, n]\).
    
    Analisando a sua Tabela ao longo dos anos, Gauss não conseguiu inferir qualquer função elementar então conhecida para a representação funcional da Tabela de números primos. Entretanto, observando que a quantidade \(n\) varia enormemente comparativamente a \(\pi(n)\) e que \(\frac{\pi(n)}{n} \to 0\), ou seja, que há proporcionalmente ``muito mais'' números inteiros não-primos, Gauss decidiu utilizar a escala logarítmica para construir uma nova Tabela em que a ``variável \(n\)'' era substituída pela variável ``compactada'' \(l_n = \ln(n)\).
    
    Embora esta nova representação gráfica não sugerisse ainda nenhuma representação funcional, por outro lado, ela sugeria, claramente, que a ``nuvem de pontos \(\left(\frac{n}{\pi(n)}, \ln(n)\right)\) se aproximava da reta bissetriz. Com base nesta observação, Gauss foi motivado a suspeitar (em 1792, com quinze anos de idade) de uma das hipóteses mais famosas da Matemática na forma da existência do limite:
    \[\displaystyle\lim_{n \to \infty} \dfrac{\rho(n)}{\dfrac{1}{\ln(n)}} = \lim_{n \to \infty} \dfrac{\pi(n)}{n} \ln(n) = 1.\]
    
    Esta hipótese foi demonstrada, \textit{a duras penas}, por Jacques Hadamard e Ch. de la Vallée Poussin somente um século depois (1896), utilizando a teoria de funções analíticas, reformulando a antiga hipótese de Gauss como o famoso \textbf{Teorema de distribuição dos Números Primos}. Observe que, neste caso, ao contrário de Galileo, Gauss desistiu de uma representação elementar ``\textbf{global}'' de toda a Tabela de números primos e se conformou com uma representação apenas no limite \(n \to \infty\), o que denominamos hoje como ``assintótica", uma vez que a descrição funcional da Tabela se torna progressivamente melhor nas imediações do infinito.

    Este é apenas um dentre notáveis episódios que demonstram a eficiência da Metodologia de Galileo, mesmo quando aplicado a questões puramente matemáticas. De fato, com o advento dos computadores, a Matemática Experimental se tornou um dos procedimentos mais importantes para a obtenção de hipóteses matemáticas que, se demonstradas, se tornam Teoremas Matemáticos (J. Borwein).

    A retificação de Tabelas nem sempre é possível com a aplicação da transformação logarítmica mas, como mostra o exemplo no Exemplo do início desta seção, isto pode ser conseguido com outras classes de mudanças de variáveis.
    
    Uma metodologia de fundo prático que tem por objetivo construir estas representações gráficas foi desenvolvida pelo engenheiro Maurice d'Ocagne (1862-1938) e estabelecida como uma disciplina que se tornou conhecida como \textit{Nomografia} (até pouco tempo muito importante em Engenharia, mas hoje substituída pela Matemática Computacional). Todavia, as ideias de d'Ocagne são importantes e foram levadas à sério por um dos mais influentes matemáticos do século XIX, David Hilbert (1862-1943), na forma do problema de número 13, dentre os 23 que ele formulou como programa de pesquisa para a Matemática do século XX. Esta questão foi resolvida por dois dentre os maiores matemáticos e professores do século XX, Andrei Kolmogorov (1903-1987) e seu aluno Vladimir I. Arnold (1937-2010), o que demonstra a linhagem nobre destas questões. Métodos de Linearização global tem sido utilizados em vários outros contextos. (ref. Kowalski-Ferreira-Ferreira, 1995). A construção de curvas que se ajustam a ``nuvem de pontos'' não lineares em geral é uma técnica recente (\(\sim\)1990) denominada ``\textit{Principal Curves}'', a ser abordada no capítulo sobre Métodos de DMKD mais adiante).

\subsection{Linearização logarítmica assintótica}

    Em várias situações, tais como no caso do Teorema dos Números Primos de Gauss, não há interesse (ou esperança) de que uma ``logaritmização'' da Tabela permita uma representação linear ``global", isto é, em todo o domínio das variáveis correlacionadas. Entretanto, a linearização aproximada em uma região especifica do domínio, em particular no limite infinito, pode ser de grande interesse e o bastante para muitos objetivos, e isto amplia grandemente a utilidade da aplicação da escala logarítmica.

    A linearização (aproximada) de relações entre variáveis no limite infinito de uma delas ou das duas (em geral quando todas as outras variáveis se esvaem) é particularmente útil no estudo de tabelas experimentais de Análise Dimensional (G. West-Scales, T. MacMahon - ...) e se baseia no seguinte conceito matemático:
    
    \textbf{Funções Assintoticamente Equivalentes} - Assintoticamente exponenciais - Linearização assintótica e Escala logarítmica (Logaritmização).

    Considere funções \(\mathcal{F} = \{f: (a, \infty) \to \mathbb{C}\) que são definidas em alguma vizinhança do infinito (de variável real \((a, \infty) \subset \mathbb{R} \to \mathbb{C}\) ou, inteira \((a, \infty) \subset \mathbb{N}\) e que tem limite infinito, isto é, \(\displaystyle\lim_{x \to \infty} f(x) = \infty\).

    Define-se uma relação de ordem (de ``grandeza'') \(\preceq\) entre duas funções \(f, g \in \mathcal{F}\), da seguinte forma:

\begin{definition}
(a) Diz-se que \(f \preceq g\) (ou, ``A função \(f\) tem \textbf{ordem menor ou igual a ordem} de \(g\) no infinito'') se existe o limite real
\[\displaystyle\lim_{x \to \infty} \dfrac{f(x)}{g(x)}\]

(b) Diz-se que \(f \prec g\) (ou, ``A função \(f\) tem \textbf{ordem menor do que a ordem de} \(g\) no infinito'') se
\[\displaystyle\lim_{x \to \infty} \dfrac{f(x)}{g(x)} = 0\]

(c) Diz-se que \(f \sim g\) (ou, ``As funções \(f\) e \(g\) são de \textbf{mesma ordem} na vizinhança do infinito") se existe o limite
\[\displaystyle\lim_{x \to \infty} \dfrac{f(x)}{g(x)} = A \neq 0\]

(d) Diz-se que as funções \(f\) e \(g\) são \textbf{assintoticamente equivalentes} na vizinhança do infinito se
\[\displaystyle\lim_{x \to \infty} \dfrac{f(x)}{g(x)} = 1\]

(e) Uma função \(f\) é dita assintoticamente exponencial (respectivamente logarítmica, polinomial) se
\[\displaystyle\lim_{k \to \infty} \dfrac{f(k)}{Ae^{\gamma k}} = 1 \left(\lim_{k \to \infty} \dfrac{f(k)}{A\ln(k)} = 1, \lim_{k \to \infty} \dfrac{f(k)}{Ak^{n}} = 1\right).\]
\end{definition}

\begin{remark}
    O objetivo deste conceito é caracterizar (aproximadamente) o comportamento assintótico (isto é, no limite) de funções gerais complicadas em termos de funções assintoticamente equivalentes Elementares (logaritmos, polinômios, exponenciais e funções obtidas como produtos e composições entre elas). Uma estratégia perfeitamente condizente com o Princípio de Comenius.
\end{remark}

\begin{exercise}
(a) Mostre que duas funções assintoticamente equivalentes no infinito não implica, necessariamente, que a diferença entre elas tende a zero, e que pode até mesmo se tornar ilimitada.

(b) Mostre que essa equivalência é uma aproximação no sentido relativo (ou seja, um erro da ordem de 1km na distância entre a Terra e a Lua não é ``a mesma coisa'' que o mesmo erro na distância entre a Unicamp e Campinas).
\end{exercise}

\begin{exercise}
Mostre, na verdade, que duas funções são assintoticamente equivalentes no infinito se na escala logarítmica os seus valores se aproximam.
\end{exercise}

\begin{exercise}
Em particular, mostre que se \(f(k) \sim e^{\gamma k}\), para \(k \sim \), então há uma ``aproximação linear na escala logarítmica da função f'' no infinito, ou seja,
\[\displaystyle\lim_{k \to \infty} \{\ln(f(k)) - (\ln(A)+\gamma k)\} = 0.\]
\end{exercise}

\begin{exercise}*
Analise uma tabela de números primos (v. Abramowitz \& Stegun) e repita o argumento de Gauss.
\end{exercise}

\begin{exercise}[Fórmula de Stirling]*
Verifique, experimentalmente, utilizando as Tabelas de Abramowitz \& Stegun que a importante Função Gama de Euler \(\Gamma: \mathbb{N} \to \mathbb{C}\), definida recursivamente por \(\Gamma(k+1) = (k+1) \Gamma(k),\ \Gamma(1) = 1\) é assintoticamente equivalente à função elementar (transcendental) \(g(k) = \sqrt{2\pi} k^{k+\frac{1}{2}} e^{-k}\) logaritmizando as duas variáveis, \(u = \ln(\Gamma(k))\) e \(v = \ln(k)\) e comparando os pontos com o gráfico de {\color{red} ?????}

\textbf{Observação}: Ao contrário da Função de Fibonacci e tal como a função densidade de números primos de Gauss, a função Gama não é representável por funções Elementares em todo o seu domínio, mas é assintoticamente equivalente a uma Função Elementar, um importantíssimo resultado que é fundamental para a Análise Combinatória (Flajolet \& Sedgewick), Teoria de Computação (Graham \& Knuth), Teoria de Probabilidade (Gnedenko), Física Estatística (van Kampen) e etc. Nenhum estudante sério de Matemática (Aplicada ou não) pode desconhecer este resultado que será abordado com maiores detalhes no capítulo sobre Métodos Assintóticos).
\end{exercise}

\begin{exercise}
Dada uma ``grande'' Tabela de dados \(P^{(k)} = (x_{1}^{(k)}, x_{2}^{(k)})\) representada cartesianamente na forma de uma ``nuvem de pontos'' no plano imagine uma forma de representá-la sinteticamente (e aproximadamente) da melhor maneira possível com apenas \textbf{duas} informações numéricas independentes (dois parâmetros reais). Convença-se de que esta questão pode ser geometricamente resolvida, obtendo-se uma reta \(r(a, b)\) (\(r = \{(x, y); y = ax+b\}\)) que minimiza a soma dos quadrados das distâncias entre \(P^{(k)}\) e \(r\) medidas das seguintes maneiras:
\begin{description}
\item (a) \(d(P^{(k)}, r) = \min\{(x_{1}^{(k)}, y), r\}\) (distância vertical) e
\item (b) \(d^\ast(P^{(k)}, r) = \min\{P^{(k)}, (x, y) \in r\}\) (distância ortogonal).
\end{description}

Resolva matematicamente este problema.
\end{exercise}

\begin{exercise}
Mostre que a População produzida por um Modelo de Fibonacci \(F(k)\) é assintoticamente exponencial e determine esta função exponencial \(A e^{\gamma k}\).
\end{exercise}

\begin{exercise}
Consulte a Linearização assintótica de inúmeros dados biológicos, físicos e geométricos apresentados em G. West e, particularmente, a de Suzana Herculano-Houzel, fazendo uma resenha a respeito desta última..
\end{exercise}

\begin{exercise}
Obtenha uma tabela de censo demográfico do Brasil em sua época de maior crescimento populacional (excetuando imigração) e utilize o Método de Linearização para determinar um modelo proporcional de população de Euler para a sua representação.
\end{exercise}

\begin{exercise}
Faça o mesmo para o início de crescimento de mortalidade pela COVID-19 em 2020 no Brasil e alguns países da Europa e EUA.
\end{exercise}


\subsection{Representação Funcional resultante de Equações Funcionais e o Princípio de Parcimônia de Ockham}

    A Tabela demográfica da dinâmica populacional nas colônias britânicas no século XVIII despertou um grande interesse devido à sua razoável precisão para a época e ao fato de que se tratava de uma situação inédita, considerando-se que a região esteve livre de grandes guerras e pestes durante um largo período de tempo e havia uma disponibilidade contínua de alimentos.

    Assim, não escapou aos olhos matematicamente treinados de Thomas Malthus a seguinte observação que ele extraiu após uma análise exploratória da representação gráfica desta Tabela demográfica:

\begin{citacao}
    ``In the United States of America, where the means of subsistence have been more ample, the manners of the people more pure, and consequently the checks on early marriages fewer than in any of the modern states of Europe, the population has been found to double itself in twenty-five years''.
    
\rightline{Th. R. Malthus\footnote{``Population: The First Essay'' 1798, pg. 8-9}}
\end{citacao}

    Em termos mais matemáticos esta observação pode ser descrita na forma \(P(k+1) = 2P(k)\), onde a unidade de tempo consiste de vinte e cinco anos e \(P(k)\) é a população no \(k\)-ésimo período (\(25k\) anos).

    A Dinâmica recursiva de uma População na forma \(P(k+1) = \lambda P(k)\) representa uma \textbf{hipótese de reprodução proporcional}, ou seja, em que a população se reproduz segundo um determinado fator (\(\lambda\)) para cada período unitário de tempo. No caso acima, a observação consistia na reprodução proporcional com \(\lambda = 2\) para a unidade de tempo de \(25\) anos para a população dos EUA, em seus primórdios.

    Resta mostrar que esta observação é suficiente para a obtenção de uma representação funcional (aproximada) da referida Tabela conforme prescreve a Metodologia de Galileo.

    Os exercícios abaixo mostram que há inúmeras funções contínuas e diferenciáveis \(P: \mathbb{R}_+ \to \mathbb{R}\) que satisfazem esta equação recursiva discreta, \(P(k+1) = 2P(k)\). Entretanto, dentre todas estas soluções, o Principio de Parcimônia de Ockham nos sugere a escolha de uma função analítica (no sentido de Newton-Weierstrass, isto é, caracterizável por uma série de potências), que seria única se existir. Como a função exponencial (que é analítica) resolve a equação recursiva, concluímos que a representação funcional desta Tabela deve ser exponencial.

\begin{remark}
    As funções analíticas podem ser consideradas ``mais simples'' porque formam um subconjunto extremamente reduzido das funções contínuas, ou diferenciáveis e dispõe de um grau de liberdade menor. Por exemplo, basta conhecer os valores de uma função analítica em uma sequência convergente no domínio para determiná-la completamente, enquanto que uma função, mesmo infinitamente diferenciável tem muito mais liberdade. Esta é uma questão interessante em Análise, mas abordaremos o tema aqui apenas sob o ponto de vista intuitivo).
\end{remark}

    Uma outra forma de caracterizar a solução exponencial para representar esta tabela é verificar se a propriedade multiplicativa vale em geral, ou seja, se para qualquer período \(T\), \(P(t+T) = \lambda(T) P(t)\) para um respectivo fator \(\lambda(T)\) e qualquer instante \(t\). Na verdade, basta que a propriedade seja consistentemente válida para duplicações e metades sucessivas da unidade de tempo,
    \[P\left(t+2^{-k}\right) = 2^{-\frac{1}{k}} P(t) \mbox{ e } P(t+2^{k}) = 2^{k} P(t).\]


%Exercícios:

\begin{exercise}
    Discuta as afirmações acima e apresentando argumentos que as justifiquem.
\end{exercise}

    Portanto, a observação de Malthus e o Princípio de Parcimônia (e um conhecimento da função exponencial) nos levam à representação funcional da Tabela demográfica segundo a Metodologia de Galileo na forma \(P(t) = \lambda^t P(0)\).

\section{Modelos de mortalidade: Tabelas Demográficas de Mortalidade de Graunt \& Neumann: Regra de Huygens e Função de Euler \\ A pertinácia de John Graunt, a perspicácia de Christiaan Huygens e astúcia do grande calculador Euler}


\begin{citacao}
``A Vida é incerta e muito rara, enquanto a Morte é certa, ainda que muito incerta, em sua data e hora''.

\rightline{Ditado Popular}

``Cuidemos da vida porque a morte é certa'', ou, ``Enterremos os mortos e cuidemos dos vivos''.

\rightline{Marquês de Pombal\footnote{após o terremoto/maremoto que devastou Lisboa, 1755.}.}
\end{citacao}

    Os matemáticos do século XVIII eram mais conhecedores e hábeis em argumentos Geométricos (o que incluía especialmente o estudo de propriedades das curvas quadráticas e cúbicas) do que em procedimentos simbólicos analíticos que haviam sido introduzidos apenas recentemente por François Viéte e René Descartes. A Geometria Analítica de Descartes tinha por objetivo conectar as duas linguagens matemáticas, a geométrica e a simbólica. O exemplo mais contundente desta mudança de paradigma expositivo da Matemática é representado pelos trabalhos de Newton, cujo histórico ``\textit{Principia}'' utilizava apenas métodos geométricos para expor a sua Nova Mecânica Celeste, algo impensável e difícil de ser compreendida hoje em dia. (ref. S. Chandrasekhar). A exposição posterior da Mecânica Newtoniana em termos simbólicos (analíticos) utiliza o Cálculo inventado pelo mesmo Newton e representou uma revolução na Matemática.

    Os já citados exemplos da parábola detectada por Galileo para trajetória de balas de canhão e as elipses detectadas por Kepler para trajetórias de cometas são exemplos notáveis e de grande coincidência, considerando que as soluções eram funções (curvas) já conhecidas por geômetras desta época. A história da Matemática Aplicada certamente seria completamente distinta se as funções apropriadas para descrever estas duas classes de trajetórias não fossem elementares e, portanto, não identificáveis por Galileo de Kepler?

    No exemplo a seguir, veremos como Christiaan Huygens, um matemático de enorme habilidade do século XVII ao examinar as tabelas de mortalidade de Graunt e Neumann, foi incapaz de identificar a função que as representaria por desconhecer a função exponencial, cujo estudo foi levado a efeito somente em anos mais tarde por Euler. (Euler-Introductio in Analysin Inifinitorum, 1748 (trad. Introduction to Analysis of the Infinite - Springer). Entretanto, não escapou à Huygens a propriedade geométrica fundamental que caracteriza esta função e mais tarde foi utilizada pelo próprio Euler para representar analiticamente a curva de mortalidade.

\begin{exercise}
Comentar a frase acima.
\end{exercise}

    Curiosamente, o Modelo Malthusiano tem a sua origem Matemática mais precisa na observação de dados sobre a Mortalidade em populações humanas, e não de Reprodução como tanto enfatizou seu propositor e de onde vem a sua fama.

    A construção do Modelo de Mortalidade é um exemplo histórico de aplicação da Metodologia Cientifica de Galileo e, portanto, é de grande interesse pedagógico, pois expõe de maneira simples alguns dos princípios mais fundamentais da difícil arte de construção de Modelos Matemáticos.

    Um dos conjuntos de dados demográficos mais antigos e influentes foi coletado por John Graunt no século XVII (1662) sobre a mortalidade em Londres. Nesta Tabela, (``\textit{Bill of Mortality}''), Graunt registra as causas de morte (segundo as variadas ``\textit{causa mortis}'' conhecidas da época) assim como o número de respectivos óbitos. O objetivo deste trabalho era tentar detectar preventivamente um surto de ``peste'' e evitar a sua avassaladora propagação, tal como ocorreu em diversas ocasiões na Idade Média. Um outro conjunto de dados demográficos de grande importância nesta época foi registrado por Caspar Neumann, na cidade de Breslau, Alemanha, e estudado por Edmund Halley, o mesmo que nomeou um famoso cometa. (Halley, Bacaer, Stigler, Wainer).

    É claro que uma representação funcional da Tabela de dados de Graunt poderia, facilmente, ser obtida, por exemplo, com uma interpolação polinomial obtida com o Método de Lagrange, dentre várias outras formas. Embora uma coleção de 200 pontos de um gráfico possa ser interpolado exatamente com um polinômio de grau 200, esta seria uma vitória de Pirro, pois não reduz nem organiza conceitualmente os dados discretos, uma vez que é obviamente impossível interpretá-los todos biologicamente. Enfim, a construção de um ``Modelo Interpolador'', ou uma Regressão paramétrica não acrescenta ``\textbf{Conhecimento}'' ao fenômeno, apenas modifica o seu arquivamento e na verdade, introduz uma grande quantidade de informações espúrias.

    A curva contínua decrescente obtida de uma interpolação manual da representação cartesiana da Tabela de Gaunt foi analisada pelo matemático holandês Christiaan Huygens (1629-1695) um pouco mais tarde. Em seu estudo, não fugiu à sua conhecida perspicácia e erudição a seguinte propriedade geométrica (aproximadamente) satisfeita pelo gráfico da curva.

\begin{remark}[Regra de Huygens:]
``Uma mesma fração de mortalidade da população ocorre para intervalos de mesmo comprimento de tempo''.
\end{remark}

    O fato de que esta propriedade quando expressa funcionalmente de fato caracteriza uma bem determinada classe de funções, somente foi verificado por Euler no século seguinte com a invenção da função exponencial. Na verdade, Euler apresenta esta questão como exemplo de aplicação da função exponencial, tal como definida em seu histórico texto de Cálculo (\textit{Introducito in Analysin Infinitorum}, 1748).

\begin{exercise}
Mostre que a regra de Huygens, aplicada a funções diferenciáveis positivas, determina unicamente a função exponencial.
\end{exercise}

    A grande revolução matemática que ofereceu uma explicação de extrema síntese para o fenômeno de mortalidade descrito pela tabela (gráfico) de Graunt foi resultado da invenção da função exponencial. Euler observou que, considerando o parâmetro \(N_0\) (de interpretação óbvia), se admitirmos um outro parâmetro positivo \(\mu > 0\) era possível interpolar razoavelmente (embora não tão exatamente quanto uma interpolação polinomial!) o gráfico discreto da tabela de Mortalidade com a função exponencial \(N_0 e^{-\mu t}\).

    Uma vez obtida a função síntese \(N(t) = N_0 e^{-\mu t}\) da tabela de mortalidade de Graunt por via da análise de Huygens, Euler imediatamente a ``encapsulou/codificou'' como solução de uma Equação Diferencial,
    \[\dfrac{1}{N} \dfrac{dN}{dt} = -\mu N,\ N(0)=N_0,\]
    o que registra o Modelo Malthusiano de Mortalidade perfeitamente no figurino da Metodologia Newtoniana.

    A redução de dados que esta representação funcional exibe enquanto mantém a informação essencial sobre o fenômeno é simplesmente extraordinária, pois, com isto, uma enorme tabela com \textbf{centenas} (talvez milhares) de entradas é resumida a apenas \textbf{dois} parâmetros reais, \(N_0\) e \(\mu\). Assim, se o parâmetro \(\mu\) for relacionável a alguma medida de caráter biológico da população, o modelo poderá ser aplicado a outras populações, mesmo que não se disponha de tabelas de mortalidade para as mesmas. É claro que há, neste procedimento, uma ``barganha'' entre a ``exatidão pontual'' de uma interpolação e uma representação funcional aproximada, mas sintética e biologicamente interpretável da Tabela de dados.

    Portanto, para que o modelo exponencial assuma esta vantagem, é indispensável interpretar biologicamente o parâmetro \(\mu\), pois caso contrário, ele não passaria de uma outra interpolação ``muda'' dos dados. A ``biologização'' do parâmetro \(\mu\) será descrita na seção III deste capítulo.
    
    (Observe que Newton escreveu todo seu \textit{Principia} com apenas construções geométricas e que Huygens descobriu a curva isócrona, que lhe possibilitou a construção do relógio mecânico, também fazendo uso exclusivo de construções geométricas. A representação e operação com funções transcendentais, e especialmente a função exponencial, talvez a mais importante da Matemática, são devidas aos trabalhos de Euler, em pleno século XVIII, baseados na técnica de expansão em séries de potências introduzida por Newton.)


\begin{exercise}
Considere uma população Malthusiana que inicia sua história com \(P_0 = P(0)\) indivíduos (vivos!) ``colonizadores'' submetida a uma taxa específica de mortalidade \(\mu\) e de natalidade \(\nu\). Determine o número total de nascimentos \(N(t)\) e o de óbitos \(M(t)\) durante o período \([0,t]\) nesta população.
\end{exercise}

\begin{exercise}
Mostre que a subpopulação de sobreviventes dentre os indivíduos colonizadores \(P_0\) no instante \(t\) é \(e^{-\mu t} = P(0)\) e que, em geral, os sobreviventes no futuro \(t+T\) da população \(P(t)\) existente no instante \(t\) é \(e^{-\mu T} P(t)\), e que os não sobreviventes são \((1-e^{-\mu T}) P(t)\).
\end{exercise}

\section{O decaimento radioativo: Um Modelo indedutível e a subversão Revolucionária da Metodologia Clássica Newtoniana}

\begin{citacao}
    ``..the proton lives for at least \(10^{30}\) years... . How this figure was arrived at since the earth is only about 5 billion years old? The thing is... we do not observe one, but many protons... a block of iron is composed of about \(10^{30}\) protons and \(10^{30}\) neutrons... The reason of the existence of human life is proof that proton enjoys a long life... Our body contains about \(10^{28}\) protons... If... the human body could not resist...''
    
    \rightline{H. Fritzsch\footnote{The Creation of Matter, BB1984- pg. 165.}.}
    
    ``The probability nature of Quantum Theory can be illustrated by a simple example. A radioactive atomic nucleus has what is called a ``half-life'' time during which it has \(50\%\) chance of desintegrating. For example the half-life time of Pu239, the usual isotope of Plutonium, is around 25000 years. If so much is unknowable about one atomic nucleus imagine how much is fundamentally unpredictable about the entire universe.''
    
    \rightline{Murray Gell’man\footnote{n.1929-Nobel de Física em 1969.}}
\end{citacao}

    A seção anterior descreveu como o Modelo Matemático Malthusiano participou de forma crucial na origem e desenvolvimento da Teoria da Seleção Natural de Darwin e Wallace que, hoje, se constitui na espinha dorsal da moderna Biologia teórica. (Segundo o importante biólogo Theodosius Dobzhansky (1900-1975) ``Em Biologia nada faz sentido fora da Teoria de Seleção Natural'').

    É curioso, portanto, embora não seja de todo misterioso, que este mesmo simples modelo matemático, ainda que com roupagens completamente distintas, também tenha sido o pivô da segunda grande revolução científica dos tempos modernos que, neste caso transformou completamente a Física a partir do princípio do século XX.

    Ao final do século XIX, a Física Clássica (Mecânica, Termodinâmica e Eletrodinâmica) era totalmente determinística e Newtoniana nos seus fundamentos e parecia ter concluído o que havia por descobrir na Natureza quanto aos seus princípios básicos (ref. Thompson...). Entretanto, a tinta desta afirmação de Thompson, ainda não havia secado quando a radioatividade descoberta acidentalmente por Henri Becquerel, em 1896, estudada com afinco pela polonesa Maria Skodlowska (mais conhecida como Madame Curie) e pelo neo-zelandês Ernest Rutherford se apresentaria como um enigma inexplicável dentro do contexto clássico.
    
    Experiências exaustivas no famoso Laboratório Cavendish levaram Rutherford e seus cooperadores a concluir, em 1900, que a transmutação de um átomo em outro átomo como resultado da emissão radioativa era um fenômeno com descrição essencialmente populacional e completamente incerto quando observado individualmente.
    
    A análise cuidadosa das criteriosas tabelas registradas em seu laboratório mostrou, claramente, que o decaimento radioativo de uma amostra (contendo uma quantidade da ordem de \(10^{23}\) átomos) poderia ser bem descrito pelo gráfico de uma curva que decrescia pela metade a cada período fixo de tempo, ou seja, se a \textbf{quantidade de átomos da substância original \(A(t)\) no instante \(t\)}, então determinava-se um intervalo de tempo \(T\) (denominado ``tempo de meia vida'', característico da substância), tal que
    \[A(t+T) = \dfrac{1}{2} A(t),\]
    para qualquer \(t\). Esta observação exibe uma evidente semelhança com aquela que Huygens descreveu com relação à tabelas de mortalidade de John Graunt e Kaspar Neumann há mais de dois séculos antes e com a observação de Malthus sobre a reprodução da população das colônias britânicas na América do Norte.

    Utilizando uma propriedade geométrica já amplamente conhecida das funções exponenciais e seguindo os passos de Galileo, Rutherford propôs que a dinâmica populacional de transmutação poderia ser bem representada por uma destas funções na forma \(N_0 e^{-\mu t}\), definida por apenas dois parâmetros: \(N_0\) e \(\mu\); cujas interpretações Físicas eram imediatas (\(\mu = T^{-1}\ln(2)\)). Em vista disso, e seguindo a Metodologia de Newton, Rutherford preferiu caracterizar sinteticamente a função que descrevia o processo de transmutação populacional na forma de uma solução da equação diferencial:
    \[\dfrac{dA}{dt} = -\mu A,\]
    onde, \(\mu = T^{-1}\ln(2)\) e \(A(t) = A_0 \exp(-T^{-1}\ln(2) t)\). Enfim, um argumento em tudo paralelo ao de Huygens-Malthus-Euler que produziu o Modelo Populacional Malthusiano para a Mortalidade.

    Naturalmente, pela forma como foi obtido, este modelo era considerado ``Fenomenológico'' e não Newtoniano, ou seja, não era deduzido de princípios básicos clássicos aplicados ao comportamento de átomos individuais, mas ``apenas'' procurava reproduzir as observações experimentais. (Em outras palavras, diríamos que este modelo matemático era ``suficiente'' mas, não ``necessário'').
    
    A preponderância do paradigma Newtoniano na Ciência da época não deixava dúvidas de que o modelo radioativo de Rutherford era tão somente uma representação provisória que eventualmente cederia lugar a um Modelo Newtoniano a ser (``\textit{Newtonianamente}'') deduzido da Física Clássica aplicada ao microcosmo, enfim, em tempo deveria se tornar um ``Modelo Necessário''. Esta perspectiva foi progressivamente sendo abandonada, não porque seria difícil de realizá-la, mas porque o desenvolvimento da Física e especialmente da Teoria Quântica apontava cada vez mais para o fato de que a descrição determinística Newtoniana seria inconsistente com novos resultados da própria Física microscópica e não por uma dificuldade eventual.
    
    Portanto, tornou-se inevitável concluir que o Modelo Fenomenológico para o decaimento radioativo seria a própria realidade, nua e crua, que assim descreveria a informação mais fundamental que se poderia obter para representar este processo. Ou seja, o fenômeno de decaimento radioativo observado era de fato populacional e resultante de comportamentos individuais independentes de átomos inteiramente iguais (homogeneidade), uma hipótese que, como vimos, caracteriza exatamente o Modelo Minimalista de Malthus.
    
    A primeira interpretação probabilística do decaimento radioativo foi apresentada por E. von Schweidler em 1905 e se baseia exatamente nestas condições. (Ref. von Plato, van Brakel, von Schweidler, Amaldi, Borel). Na verdade, a interpretação probabilística da Teoria Quântica que é considerada um dos pilares mais fundamentais da Física Contemporânea tem no decaimento radioativo o seu exemplo mais antigo e ``concreto". Considerando-se que a equação diferencial que representa o fenômeno radioativo é a mesma equação \(\dfrac{dA}{dt} = -\mu A\), a sua interpretação probabilística pode ser igualmente utilizada em qualquer outro contexto em que seja aplicável o modelo Malthusiano.

    A paradoxal conexão entre a descrição determinística populacional representada pela equação diferencial de Euler e a (necessária) descrição probabilística individual, foi resolvida na Física com a aceitação definitiva do caráter probabilístico intrínseco e fundamental, não somente do decaimento radioativo, mas de toda a Física microscópica. A Teoria Quântica é o resultado mais notável deste novo paradigma.

    É necessário ressaltar, entretanto, que o influente matemático Simeon Dennis Poisson (1781-1840), apresentou um modelo matemático demográfico na Academia de Ciências francesa, em 1828, que interpreta o nascimento e morte em uma população como um processo probabilístico individual. Para a justificação de seu modelo, Poisson não supõe que o fenômeno de morte biológica seja de fato ``intrinsecamente aleatório'' (o que é estritamente inaceitável, em geral), mas adota uma atitude pragmática ao verificar (à posteriori) que o Modelo resultante desta hipótese representa ``bem'' uma Dinâmica Populacional minimalista. Ou seja, o Modelo de Poisson é um modelo ``suficiente'' para este caso. A interpretação probabilística do Modelo Malthusiano que apresentaremos em seguida, confirmará o modelo especulativo de Poisson.

    A utilização de processos explicitamente probabilísticos para a representação de fenômenos não necessariamente aleatórios, e em alguns casos determinísticos mesmo, é exemplificado pelo procedimento de Poisson. Este procedimento foi ``reinventado'' em meados do século XX por sugestão do matemático Stanislau Ulam para a resolução numérica de equações diferenciais por intermédio de simulações de jogos aleatórios, que hoje é denominado \textbf{Método Monte Carlo}. Como se Poisson jogasse os dados para resolver a Equação de Malthus. (I. Sobol-The Monte Carlo Method, ed. MIR)

\chapter{Interpretações biológicas do modelo malthusiano}
\addt

\begin{enumerate}
\item Populacional: Tempo Médio de Sobrevivência \& Reprod \& Permanência-(Lewis \& BMB)
\item Individual: Expectativa de Vida-Poisson-Mutação e Decisão-DelbruckLuria-Kirman..
\end{enumerate}

    O Modelo Malthusiano minimalista pressupõe uma homogeneidade das características biológicas de seus indivíduos que a Análise Dimensional mostrou ser necessariamente representada por um parâmetro relacionado à dimensão do tempo. Resta determinar uma interpretação biológica razoável para este parâmetro.
    
    Nas seções seguintes, apresentaremos duas interpretações para este parâmetro e ambas são fundamentais para a conexão entre a estrutura matemática representada pela equação diferencial e as visões biológicas do problema.

\section{Interpretação biológica populacional do Modelo de Malthusiano}

    Consideremos o Modelo Malthusiano \(n = \varphi(t, n_0, \mu)\) em sua formulação diferencial Newtoniana para a Mortalidade:
    \[\dfrac{1}{n} \dfrac{dn}{dt} = -\mu,\]
    com \(\mu\) um parâmetro real positivo cuja dimensão \([\mu^{-1}] = T\) é, necessariamente, a mesma dimensão do tempo.
    
    Consideremos agora a função \(S(t)\) ``Quantidade de sobreviventes de uma população no instante \(t\) em que não há nascimentos nem migrações''.

    Se esta população tiver um valor inicial \(P_0\), então, \(S(t)\) decrescerá monotonicamente e a sub-população extraída durante um pequeno intervalo de tempo, \([t_{k},t_{k+1} = t_{k}+h]\), é dada por \(S(t_k) - S(t_{k}+h)\). Esta é a quantidade (\textit{aproximada}) da parte da população que sobreviveu até o instante \(t_{k}\), e não mais do que isso. Calculemos agora uma média aritmética (aproximada) do tempo de permanência destes indivíduos no censo dos ``vivos'', o que faremos por intermédio de uma média aritmética ponderada pelo tempo de sobrevivência:
    \[\dfrac{1}{P_0}\displaystyle\sum_{k=0}^{\infty} \{S(t_k)-S(t_k+h)\} t_k.\]
    Por conveniência matemática, reescrevemos esta expressão na forma:
    \[\begin{array}{rcl}
    & & \displaystyle\sum_{k=0}^{\infty}\dfrac{1}{P_0} \{S(t_k)-S(t_k+h)\} t_k \\
    &=& \dfrac{1}{P_0}\sum_{k=0}^{\infty} \dfrac{S(t_k)-S(t_k+h)}{h} t_k h\\
    &\simeq& \dfrac{1}{P_0}\sum_{k=0}^{\infty} \dfrac{-dS(t_k)}{dt} t_k h.
    \end{array}\]

    Esta última expressão \(\displaystyle\sum_{k=0}^{\infty} \dfrac{-dS(t_k)}{dt} t_k h\) representa a soma integral para a função \(f(t) = -\dfrac{dS(t)}{dt} t\) com repartições de comprimento \(h\).

    Portanto, tomando o limite ``integral'' para \(h \to 0\), temos:
    \[\begin{array}{rcl}
    & & \displaystyle\lim_{h \to 0} \dfrac{1}{P_0} \sum_{k=0}^{\infty} \{S(t_k)-S(t_k+h)\} t_k \\[0.4cm]
    &=& \dfrac{1}{P_0} \int_{0}^{\infty} \dfrac{-dS}{dt}\ t\ dt,
    \end{array}\]
    de onde obtemos:
    \[\begin{array}{rcl}
    & & \displaystyle\lim_{h \to 0} \dfrac{1}{P_0} \sum_{k=0}^{\infty} \{S(t_k)-S(t_k+h)\} t_k \\[0.4cm]
    &=& \dfrac{1}{P_0} \int_{0}^{\infty} \dfrac{-dS}{dt}\ t\ dt \\[0.4cm]
    &=& \dfrac{1}{P_0} \int_{0}^{\infty} S(t)\ dt \\[0.4cm]
    &=& \dfrac{1}{\mu}.
    \end{array}\]



    O resultado matemático acima obtido mostra que o parâmetro (de dimensão tempo) \(\frac{1}{\mu}\) tem significado biológico claro e significa o ``Tempo Médio de Sobrevivência'' para os indivíduos desta população.

    Este argumento, que foi sugerido pela análise que Christiaan Huygens realizou sobre Tabelas de Mortalidade, pode ser repetido de maneira totalmente semelhante para qualquer população que, por motivos quaisquer, decresçam.

    (Não nos deteremos para demonstrar dentro dos rigores usuais da Análise a existência e a representação do limite acima, o que pode ser feita por qualquer estudante desta disciplina. O fato da expressão final ser uma integral imprópria, isto é, com limite de integração infinito, exige uma atenção especial mas, a solução explicita da equação diferencial facilita a compreensão do resultado).

\section{Interpretação Biológica Populacional do Modelo Malthusiano de Reprodução}

    Analisemos agora o modelo Malthusiano de \textbf{reprodução},
    \[\dfrac{1}{P} \dfrac{dP}{dt} = \nu > 0,\]
    e interpretemos o significado do parâmetro \(\nu\), que nesta expressão matemática representa uma taxa de variação específica (``\textit{per capita}''). Mais uma vez, é importante observar que este modelo assume que todos os indivíduos, antigos ou recém incorporados à população, têm a mesma chance de se reproduzirem nos instantes seguintes, ou seja, não há influencia etária, isto é, de maturação, nem de outros indivíduos na sua capacidade reprodutiva.

    Consideremos a sub-população \(V\) da quantidade de indivíduos dentre os \(P_0\) iniciais, que ainda não se reproduziram até o instante \(t\) dentre uma população que se inicia no instante \(t=0\) com \(P_0\) indivíduos e descrita pelo modelo:
    \[\dfrac{dP}{dt} = (\nu-\mu) P = rP.\]


    Para analisar a dinâmica desta sub-população, observemos que para um pequeno período de tempo \([t, t+\delta t]\) o número de descendentes produzidos por esta população é:
    \[\nu V(t) \delta t + o(\delta t),\]
    de acordo com o próprio modelo Malthusiano. Portanto, a menos de um erro de ordem \(o(t)\), podemos afirmar que, no período \([t, t+\delta t]\), o decréscimo da população \(V(t)\) foi de \(\nu V(t) \delta t\), ou seja, podemos escrever a seguinte equação:
    \[\dfrac{dV}{dt} = -\nu V, V(0) = P_0,\]
    de onde \(V(t) = P e^{-\nu t}\).

    Utilizando, agora, o mesmo argumento desenvolvido quanto à dinâmica de mortalidade, concluímos que \(\nu^{-1}\) é o tempo médio de permanência de indivíduos na sub-população que ainda não procriou, ou, em termos mais biológicos, o ``Tempo médio esperado por um indivíduo para a sua primeira reprodução''.

    É importante notar que a dimensão temporal dos parâmetros \([\mu^{-1}] = [\nu^{-1}] = T\) permanecerá sempre a mesma, mas os significados das escalas de tempo \([\mu^{-1}]\) e \([\nu^{-1}]\) dependem, naturalmente, da interpretação do próprio modelo em que ele participa, embora não haja variações radicais daquela que acabamos de tratar.

    Voltando ao modelo diferencial de Malthus-Euler, observamos que a equação nos mostra que o termo à esquerda \(\left(\dfrac{1}{N} \dfrac{dN}{dt}\right)\) representa uma expressão com sentido biológico claro, ``Taxa de Mortalidade \textit{per Capita}'' e, por outro lado, a igualdade \(\left(\dfrac{1}{N} \dfrac{dN}{dt} = -\mu\right)\) nos afirma que esta taxa é constante e independe do números de indivíduos. (Claro, desde que forem em quantidade suficiente para que se justifique o emprego do modelo diferencial, como já foi discutido).

    Estas observações nos levam a concluir que o modelo de Malthus, necessariamente, trata de uma população completamente homogênea de indivíduos que não interagem entre si negativamente quanto à mortalidade. O fato de que o tempo médio de sobrevivência independe da quantidade inicial de indivíduos da população é também argumento para concluir que não há qualquer interação entre seus membros quanto ao aspecto de mortalidade. Argumentos análogos são aplicáveis à dinâmica de reprodução Malthusiana.
    
    A homogeneidade e a não interação entre indivíduos de uma ``População Malthusiana'' são fundamentais para o desenvolvimento do argumento da próxima seção.


\section{Interpretação biológica individual do modelo malthusiano: Processo de Poisson}

\begin{citacao}

    ''...on y reconnait la marque du hasard...'' H. Poincaré - in E. Amaldi, pg. 16.

    ``When the Lord created the World and people to live in it -an enterprise which, according to modern science, took a very long time- I could well imagine that He reasoned with Himself as follows: ``If I make everything predictable, these human beings, whom I have endowed with pretty good brains, will undoubtely learn to predict everything, and they will thereupon have no motive to do anything at all, because they will recognise that the future is totally determined and cannot be influenced by any human action. On the other hand, if I make everything unpredictable, they will gradually discover that there is no rational basis for any decision whatsoever and, as in the first case, they will thereupon have no motive to do anything at all. Neither scheme would make sense. I must therefore create a mixture of the two. Let some things be predictable and let others be unpredictable. They will then, amongst many other things, have the important task to find out which is which''. E. F. Schumacher: (in ``Small is Beautiful", Harper 1973, pg 211)
\end{citacao}

    A impossibilidade de determinar o momento exato da morte de um indivíduo específico de uma população completamente Homogênea, ou do momento de decaimento radioativo de um átomo \(C_{14}\) de uma amostra, não significa que estamos totalmente à mercê do processo, mas significa que dispomos apenas de uma quantidade parcial de informação sobre ele.
    
    Vejamos, portanto, qual é esta informação parcial e que previsão parcial podemos realizar com ela. Para isto, façamos uso do conceito de ``Probabilidade'' no seu sentido frequentista, para o seguinte fato: ``A probabilidade de um átomo de \(C_{14}\) decair em um intervalo de tempo \(t\) (fixo) é obtida (e definida experimentalmente) como a fração obtida pela divisão da quantidade de átomos que decaíram neste período com relação à quantidade total de átomos observados, em um ``grande número de observações'' . Esta definição é semelhante à determinação (frequentista, e `experimental' como sempre) da probabilidade de uma moeda produzir uma coroa em um lançamento, obtida após a observação de um ``grande'' número de lançamentos. Pois bem, quando temos uma amostra de átomos de \(C_{14}\), podemos realizar o experimento de ``decaimento'' simultaneamente em uma quantidade inimaginável de vezes na prática, (\(\sim 10^{23}\)), e a fração dos que decaíram após um período de tempo \(t\) deve nos dar ``quase exatamente'' a probabilidade frequentista deste processo. (ref. Mahadevan, Keller)


\begin{exercise}

\begin{description}
    \item (a) Realize as seguintes experiências REAIS: \(N\sim 20\) lançamentos sucessivos de uma moeda seguidos da anotação do resultado em uma Tabela concreta (computador ou papel) onde constam o número de experimentos, \(1 \le n \le N\), o tempo \(T(n)\) necessário para executá-los, o número de caras e o de coroas até a referida etapa \(n\), e obtenha uma representação linear (aproximada \(T(n) = \alpha n\)) entre estas variáveis. Extrapole o resultado para avaliar o tempo necessário para executar \(10^{23}\) experimentos. Compare com a idade da Terra, que segundo alguns físicos contemporâneos é da ordem de \(\simeq 6 \cdot 10^{9}\) anos (H. Fritzsch). (Sugestão: Método de Mínimos Quadrados de Gauss explicado).

{\color{red}
\textbf{Solução}:

Seja

\(N^t = \left[\begin{array}{cccc} n_1 & n_2 & \ldots & n_N \end{array}\right]^t\) (Pontos de entrada)

\(T^t = \left[\begin{array}{cccc} t_1 & t_2 & \ldots & t_N\end{array}\right]^t\) (Pontos de saída)

Queremos obter um \(Z \sim T\). Se \(Z\) é linear, temos:
\begin{eqnarray*}
Z
&=& \theta_1 N + \theta_2 \\
&=& \left[\begin{array}{c} n_1\theta_1+\theta_2 \\ n_2\theta_1+\theta_2 \\ \vdots \\ n_N\theta_1+\theta_2 \end{array}\right] \\
&=& \underbrace{\left[\begin{array}{cc} n_1 & 1 \\ n_2 & 1 \\ \vdots \\ n_N & 1 \end{array}\right]}_{\overline{N}}
\underbrace{\left[\begin{array}{c} \theta_1 \\ \theta_2 \end{array}\right]}_{\Theta}
\end{eqnarray*}

Vamos minimizar a função
\[E(\Theta)
= (T-Z)^2 = (T-\overline{N}\Theta)^2
\]
e, para tal, determinemos:
\[
\dfrac{\partial E}{\partial \Theta}
= \dfrac{\partial E}{\partial Z}\ \dfrac{\partial Z}{\partial \Theta}.\]

Mas
\[
\dfrac{\partial E}{\partial Z}
= \dfrac{\partial}{\partial Z} (T-Z)^2 = -2 \underbrace{(T-Z)}_{N \times 1}
\]
e
\[
\dfrac{\partial E}{\partial \Theta}
= \dfrac{\partial}{\partial \Theta} (\overline{N}\Theta) =  \underbrace{\overline{N}^t}_{2 \times N}.
\]

Segue que
\[0
= \dfrac{\partial E}{\partial \Theta}
= -2 \overline{N}^t (T-Z)
= -2 \overline{N}^t (T-\overline{N}\Theta),
\]
ou seja,
\[
\overline{N}^t T = \overline{N}^t \overline{N} \Theta
\]
implicando em
\[
\Theta = (\overline{N}^t\ \overline{N})^{-1} \overline{N}^t T.
\]
(ver \href{https://www.youtube.com/watch?v=txnrFZG7Ugs&ab_channel=LeonardoOlivi}{Youtube})


Os valores de \(\theta_1\) e \(\theta_2\), após algumas contas, são dados por:
\[\begin{array}{rcl}
\theta_1 &=& \dfrac{\displaystyle\sum_{i=1}^{N} n_i \sum_{i=1}^{N} t_i - N \sum_{i=1}^{N} n_i\ t_i}{\displaystyle\left(\sum_{i=1}^{N} n_i\right)^2 - N \sum_{i=1}^{N} n_i^2} \\
\theta_2 &=& \dfrac{\displaystyle\sum_{i=1}^{N} t_i - \theta_1 \sum_{i=1}^{N} n_i}{N}
\end{array}
\]

Os experimentos executados, bem como alguns valores necessários para a obtenção dos coeficientes da reta de regressão são apresentados na tabela a seguir e foram obtidos utilizando uma planilha eletrônica (Excel):
\[\tiny
\begin{array}{c|c|c|c|c|c|c|c|c}
n & 1 & f(n) & t_n & T(n) & K & C & n^2 & n\ t \\ \hline
1 & 1 & 0 & 24 & 24 & 1 & 0 & 1 & 24 \\ \hline
2 & 1 & 1 & 36 & 60 & 1 & 1 & 4 & 120 \\ \hline
3 & 1 & 1 & 27 & 87 & 1 & 2 & 9 & 261 \\ \hline
4 & 1 & 1 & 17 & 104 & 1 & 3 & 16 & 416 \\ \hline
5 & 1 & 0 & 44 & 148 & 2 & 3 & 25 & 740 \\ \hline
6 & 1 & 0 & 24 & 172 & 3 & 3 & 36 & 1032 \\ \hline
7 & 1 & 1 & 35 & 207 & 3 & 4 & 49 & 1449 \\ \hline
8 & 1 & 0 & 15 & 222 & 4 & 4 & 64 & 1776 \\ \hline
9 & 1 & 1 & 49 & 271 & 4 & 5 & 81 & 2439 \\ \hline
10 & 1 & 0 & 24 & 295 & 5 & 5 & 100 & 2950 \\ \hline
11 & 1 & 1 & 21 & 316 & 5 & 6 & 121 & 3476 \\ \hline
12 & 1 & 1 & 17 & 333 & 5 & 7 & 144 & 3996 \\ \hline
13 & 1 & 0 & 39 & 372 & 6 & 7 & 169 & 4836 \\ \hline
14 & 1 & 1 & 19 & 391 & 6 & 8 & 196 & 5474 \\ \hline
15 & 1 & 1 & 42 & 433 & 6 & 9 & 225 & 6495 \\ \hline
16 & 1 & 1 & 32 & 465 & 6 & 10 & 256 & 7440 \\ \hline
17 & 1 & 1 & 24 & 489 & 6 & 11 & 289 & 8313 \\ \hline
18 & 1 & 0 & 23 & 512 & 7 & 11 & 324 & 9216 \\ \hline
19 & 1 & 1 & 16 & 528 & 7 & 12 & 361 & 10032 \\ \hline
20 & 1 & 0 & 27 & 555 & 8 & 12 & 400 & 11100 \\ \hline
21 & 1 & 0 & 33 & 588 & 9 & 12 & 441 & 12348 \\ \hline
22 & 1 & 0 & 22 & 610 & 10 & 12 & 484 & 13420 \\ \hline
23 & 1 & 1 & 24 & 634 & 10 & 13 & 529 & 14582 \\ \hline
24 & 1 & 0 & 18 & 652 & 11 & 13 & 576 & 15648 \\ \hline
25 & 1 & 0 & 30 & 682 & 12 & 13 & 625 & 17050
\end{array}
\]
onde
\begin{description}
\item \(K\) representa o n\textordmasculine\ de caras obtidas até o \(n\)-ésimo lançamento;
\item \(C\) representa o n\textordmasculine\ de coroas obtidas até o \(n\)-ésimo lançamento.
\end{description}

Os valores obtidos a seguir, também com o auxílio do Excel, são utilizados para determinar os valores dos coeficientes da reta de regressão:
\[
\begin{array}{c|c}
S_n & 325 \\ \hline
S_T & 9150 \\ \hline
S_{nT} & 154633 \\ \hline
S_{n^2} & 5525 \\ \hline
N & 25 \\ \hline
\end{array}
\]
e, os coeficientes a determinar, são:
\[
\begin{array}{c|c}
\theta_1 & 27,44846154 \\ \hline
\theta_2 & 9,17 \\ \hline
\end{array}
\]

Para \(n = 10^{23}\), obtemos: \(T_n = 2,74485 \cdot 10^{24}\) segundos, ou seja, \(8,70385 \cdot 10^{16}\) anos. Comparando com a idade da Terra, temos que \(T_n\) é, aproximadamente, \(14.506.416\) vezes maior.
}



\item (b) No exercício anterior, você fez poucos lançamentos e não se cansou muito, o que resultou em uma reta bem representativa do resultado. Agora repita, sem descansar ou diversificar a atenção durante o experimento, \(100\) lançamentos sucessivos e verifique que a curva é ascendente. Interprete o resultado com relação ao exercício anterior. Utilize a escala logarítmica e argumente se a curva é assintoticamente exponencial ou polinomial. (Esta questão visa determinar o seu comportamento e é representativo de muitas experiências em Biologia. Este exercício pode ser feito em grupos em que apenas duas pessoas realizam a experiência para efeito de comparação).

    \item (c) Antes de se cansar, é possível que ocorra intermediariamente um ``aprendizado'' que tornará o procedimento mais rápido. Mas este aprendizado é saturado em pouco tempo. Em uma etapa posterior, ``vence o cansaço'' e o tempo necessário para cumprir a tarefa começa a se alongar. Analise estas fases do procedimento em termos da Tabela anotada e de um re-escalonamento logarítmico.

    \item (d) Suponha que você tenha vida quase-eterna. Utilizando o resultado anterior (com cansaço) calcule o tempo necessário para realizar os \(10^{23}\) experimentos, mas observe que o ``índice de cansaço", que pode ser medido pela curvatura do gráfico deve ser variável, ou seja, o cansaço é cumulativo, como sabemos). Supondo (especulativamente) que sua atenção seja exponencialmente decrescente, analise a questão acima.
    \end{description}
\end{exercise}

\begin{exercise}
    Aproveite a sua tabela e obtenha uma representação assintótica da função \(p_k: \mathbb{N} \to \mathbb{N}\), definida como, \(p_k(n) = p\), onde \(p\) representa o número de vezes que em \(n\) experiências produziu \(k\) sucessivos''. ``Logaritmize'' esta Tabela e conclua que ela sugere um comportamento assintótico da função \(p_k(n)\) exponencial e analise como variam os coeficientes exponenciais em dependência de \(k\) (Se crescem ou decrescem).
\end{exercise}

\begin{exercise}
    Registre (mentalmente) o resultado de cada sequência de \(n\) experimentos na forma de um ``sinal'' (ou ``palavra'') de comprimento \(n\) constituído de \(0's\) e \(1's\), onde \(0\) corresponde a coroa e \(1\) corresponde a cara da moeda. Seja, então, \(F_2(n) = f\), onde \(f\) é o número de possíveis ``sinais'' de comprimento \(n\) que não tem \(0's\) sucessivos''. Mostre que esta função satisfaz à recursão \(F(n+2) = F(n+1) + F(n)\) com dados iniciais \(F(1) = 2\), \(F(2) = 3\). Calcule uma Tabela para esta Função e mostre via linearização logarítmica que ela tem comportamento assintótico exponencial \(Ae^{\gamma n}\) e, aproximadamente, o seu coeficiente \(\gamma\).
\end{exercise}

    A interpretação probabilística do Modelo Malthusiano original depende, de forma crucial, de dois aspectos importantes: o processo de decaimento radioativo/morte individual é:
    \begin{itemize}
    \item \textbf{independente}, isto é, este comportamento individual não é influenciado por outros átomos/indivíduos da sua população e nem de fatores externos, e;
    \item \textbf{homogêneo}, ou seja, não varia de indivíduo para indivíduo. Assim, podemos sempre interpretar a dinâmica de um sistema na forma
    \[\dfrac{1}{N} \dfrac{dN}{dt} = -\mu\]
    como resultado populacional de uma grande quantidade de experimentos individuais (independentes) semelhantes, envolvendo toda a população \(N_0\).
    \end{itemize}
    
    Verificamos, então, que ao fim de um período de tempo \(t\), um total de \(N(t) = N_0e^{-\mu t}\) permaneceram sem decair/morrer e \(N_0 - N(t) = N_0(1-e^{-\mu t})\) decaíram/morreram.

    Interpretando, agora, a probabilidade de decaimento/morte de um átomo/indivíduo isolado durante o intervalo de tempo \(t\) pelo princípio frequentista, temos que a taxa entre o número de mortos durante um período de tempo \(t\) e o número total de indivíduos observados é:
    \[\dfrac{N_0-N_0 e^{-\mu t}}{N_0} = (1-e^{-\mu t}).\]
    Passando de uma afirmação de caráter coletivo para uma afirmação de caráter individual, o máximo que se pode garantir é uma probabilidade:
    \begin{itemize}
    \item ``A Probabilidade de que um átomo individual decaia durante o intervalo de tempo \(t\)'' é igual ao ``Número de átomos que decaíram durante este intervalo de tempo'' \textbf{DIVIDIDO PELO} ``Número de átomos que participaram da observação'' ou seja, ``A Probabilidade de que um átomo individual decaia durante o intervalo de tempo \(t\)'' é
    \[\dfrac{A_0(1-e^{-\mu t})}{A_0} = (1-e^{-\mu t}).\]
    
    Enquanto que
    \item ``A Probabilidade de que um átomo individual `sobreviva' durante o intervalo de tempo \(t\)'' é:
    \[\dfrac{A_0 e^{-\mu t}}{A_0} = e^{-\mu t}.\]
\end{itemize}

    Como no Modelo Malthusiano a reprodução futura de um indivíduo independe das reproduções anteriores, o tempo médio, \(\nu^{-1}\), pode ser contado a partir de qualquer instante.


    Refazendo o argumento acima para o Modelo populacional Malthusiano, podemos interpretar probabilisticamente a expressão
    \[1-e^{-\nu T} = \dfrac{V(0)-V(T)}{V(0)}\]
    como sendo a ``probabilidade de que um indivíduo se reproduza pela primeira vez no intervalo de tempo \(T\)'' ou,
    \[e^{-\nu T} = \dfrac{V(T)}{V(0)}\]
    como a ``Probabilidade de que um indivíduo não se reproduza em um intervalo de tempo \(T\)''.

    Observe que este fato independe do momento inicial, pois
    \[e^{-\nu T} = \dfrac{V(t+T)}{V(t)},\]
    ou seja, neste modelo, os indivíduos não envelhecem com respeito a fertilidade.

    Um \textbf{Processo de Poisson} é caracterizado como um processo temporal de espera por um acontecimento determinado (sinal do contador Geiger, chamada telefônica, ocorrência de acidentes, fisgada de um peixe no anzol do pescador e etc.) que satisfaz à seguinte condição infinitesimal: ``Probabilidade de que ocorra o evento durante um pequeno intervalo de tempo \(\delta t\)'' é igual a \(\mu \delta t + o(\delta t)\), onde o símbolo \(o(x)\) significa uma função de \(x\) que se aproxima de zero mais rápido do que \(x\), quando \(x \downarrow 0\), ou seja, \(\displaystyle\lim_{x \to 0} o(x) = 0\). Assim, a hipótese de Poisson diz que para pequenos intervalos de tempo a probabilidade de que ocorra um e, apenas um sinal, é proporcional ao comprimento do intervalo a menos de um pequeno erro de segunda ordem).

    Esta caracterização é necessariamente satisfeita pelo decaimento radioativo de Rutherford descrito acima pois,
    \[1-e^{-\mu \delta t} = \mu \delta t + \dfrac{1}{2} (\delta t)^2 + \ldots\]

    O que é interessante, é que esta propriedade é também suficiente, isto é, um processo de espera de Poisson pode também ser caracterizado na forma: ``Probabilidade de que um átomo específico decaia durante o intervalo de tempo \(t\)'' é igual a:
    A01 etA0 1  e t).
    \[\dfrac{A_0(1-e^{-\mu t})}{A_0} = 1-e^{-\mu t}.\]


    Assim, tanto o decaimento de átomos da teoria de Rutherford, quanto a Doutrina Malthusiana de morte e nascimento de indivíduos biológicos de uma grande população podem ser reinterpretados probabilisticamente o que enriquece conceitualmente o Modelo Malthusiano e abre novas possibilidades para as suas aplicações.

    Em particular, uma vez que temos disponível uma descrição (em termos probabilísticos),do comportamento especifico independente (quanto à mortalidade, reprodução, ou decaimento e etc.) de cada indivíduo, isto torna possível descrever (probabilisticamente) a Dinâmica Populacional de \textbf{Pequenas Populações}, isto é, aquelas que não são suficientemente grandes para que a suas dinâmicas possam ser representadas satisfatoriamente por uma curva contínua e suave. Neste caso, a descrição da população se faz por intermédio de uma família infinita de funções reais da forma \(\{p_n(t)\}_{0 \le n \le \infty}\) que representam a seguinte informação:
    \begin{itemize}
    \item \(P_n(t)\) ``Probabilidade de que a População tenha \(n\) indivíduos no instante \(t\)''.
    \end{itemize}

    Este Modelo, dentre outros temas, será apresentada em detalhes no capitulo referente a Princípios Probabilísticos.
    
    Processos de Poisson são Modelos Matemáticos importantes em diversas áreas da Física e Química (ref. van Kampen), da Biologia (ref. Ludwig, Koonin) da Economia (Cramer \& Lundberg) da Engenharia (Cramer \& Erlang) e, por conseguinte, amplamente estudados na Matemática Aplicada (Lange).



BIBLIOGRAFIA:

M. Abramowitz - I. Stegun-ed. - Handbook of Mathematical Functions - Formulas, Graphs and Mathematical Tables, NBS 1964-online;

W. C. Allee - Animal Aggregations, U. Chicago Press, 1939.

W. C. Allee - Cooperation Among Animals, H. Schumann, 1951.

E. Amaldi - Radioactivity: A Pragmatic pillar of probabilistic Conceptions, pp.1-28 in Proc.Int.School Phys. E.Fermi-Corso 72–Problemi dei Fondamenti della Fisica-ed. G. Toraldo di Francia, North-Holland 1979.

N. Arley - Introduction to Stochastic Processes and Cosmic Radiation, J. Wiley, 1948;

N.Bacaer-Short History of Mathematical Population Dynamics, Springer 2011 OL

A.L. Barabasi-Burst, 2010 - Nature 435, 2005

H.C. von Baeyer-Information-The new Language of Science, Harvard Univ.Press 2003.

Th. Bayes-An Essay towards solving a problem in the doctrine of chances,Phil.Trans.Royal Soc., 53, 370-418, (1763)

H.Bateman-The Probability Variations in the Distribution of Particles, Phil.Mag. 20 (1910), 698-704.

H.Berg-Random Walks in Biology, Princeton UP 1993.

E.Borel-Le Hasard, 1924

E.Borel-Radioactivité Probabilité et Determinism, Oeuvres vol. 4, 1972, pg. 2189-2196.

K.E.Boulding-Foreword to Malthus’ Essay, U.Michigan 1959

B.Brecht-Galileo, A Play, Grove 1966

J.Browne-Darwin’s Origin of Species, 2 vol.

S.G.Brush-Randomness and Irreversibility, AHES 5 1968, 1-36, 12, 1974, 1-80.

P.Buhlmann-Toward Causation and External Validity, Proc.Nat.Acad.Sci(pnas)2020

D.Calvetti-E.Somersalo-An Introduction to Scientific Bayesian Computing, Springer-Verlag 2008.

S.Chandrasekhar-Newton’s Principia for the Common Reader. Oxford UPress, 1995.

J.E.Cohen-How many People can Earth Support, Norton 1995

J.E.Cohen-Population Growth and Earth’s Human Carrying Capacity, Science (269), 1995, 341-

H.Caswell-Matrix Population Models, 3rd. Ed. Sinauer, 2005.

H.Cramer-Historical review of Filip Lundberg’s works on risk theory, pg. 1288-1294 in Collected Works of Harald Cramer.

Charles Darwin-The Origin of Species and the Descent of Man, 1859 (The Modern Library-Random House s/d)

L.Curtis-Concepts of Exponential Law prior to 1900, A.J.Ph. 46(9), 1978, 896-

O.Darrigol-World of Flows- History of Fuid Dynamics....

M.Delbrück-Statistical Fluctuations and Autocatalytic Reactions, J.Chem. Phys. 1940

J. DiamondN.Eldredge- Entrevista Revista Pesquisa da Fapesp- ``Sobre a capacidade de suporte da Terra"

L. Euler-``Recherches générales sur la mortalité et la multiplication du genre humain- Acad.Sci Berlin 1760- trad. Theor.Popul.Biol. 1:307-314, 1970, \& pg. 83-91 in Smith-Keyfitz(1977)-

Leonhardi Euleri-Opera Omnia -ser.I vol 7- pg. 345-352

L.Euler-Introductio Analysin Infinitorum, 1748- trad. inglês– Springer-Verlag

G.T.Fechner-KollektivemassLehre, 1897

W.Feller-An Introduction to Probability Theory and its Applications, 2 vol. J.Wiley 1966

W.Feller-On the logistic law of growth and its empirical verification in biology, Acta Biotheor. 5, (1940): 51-66

E.A.Fellmann-Leonhard Euler (Biography)-Birkhauser 2007

W.C.Ferreira Jr.-Dinâmica de Populações: De íons a sapiens, online- Revista ComCiência

W.C.Ferreira Jr.-The Multiple Faces of Diffusion, 2011

WCFerreira Jr.-O Silêncio dos Conformistas, Conf. Enc.Biomat I, 2017 e 2018-prelo.

R.P.Feynman-The Concept of Physical Law, MIT Press

L.Fibonacci-Liber abbaci di Leonardo Pisano, 1202 (online)

Ph.Flajolet-R.Sedgewick-Analytic Combinatorics, Cambridge UP, 2009

H. von Foerster-Some remarks on changing populations, The Kinetics of Cell Proliferation, pg. 382-407, 1959.

H.Fritzsch-The Creation of Matter, BB 1984 (pg. 165-The decay of the proton)

Galileo Galilei-Dialogo,Fiorenza 1632 (trad. Dialogue Concerning the Two Chief World Systems, UnivCalif.Press1967)

C.W.Gardiner-Handbook of Stochastic Methods for Physics,Chemistry and Natural Sciences, Springer 1985

M. Gellman-The Quark and the Jaguar, Norton 1985.pg.132

R.Graham-D.Knuth-O.Patashnik-Matemática Concreta: Fundamentos para a Teoria de Computação, Livr.Tecno-Cientifica 1988

John Graunt-Natural and Political Observations Mentioned in a Following Index and Made upon the Bills of Mortality, 1662-pg. 12-20 in Smith-Keyfitz(1977).

R.Gregory-Eye and the Brain-The Psychology of Seeing, Princeton UP 2015

E. Halley-An Estimation of the Degree of Mortality of Mankind, PhilTr RS,1693.

W.D.Hamilton-The Moulding of genes by Natural Selection, J.ThBiol 12 (1966), 12-45.

I.Hacking-The Emergence of Probability,

G.Hardin-The Tragedy of Commons, Science 1968

G.Hardin-Living within Limits, Oxford U.P.1993

S. Herbert-Darwin Malthus and Natural Selection, J.Hist.Biol. 4 (1971) 209-217

S. Herculano-Houzel S (2009) The human brain in numbers: a linearly scaled-up primate brain. Frontiers of Hum Neurosci 3:31

E.Hopf- On causality, statistics and probability, J. of Math. and Physics, vol 13, 1934.1763

F.Hoppensteadt-Mathematical Theories of Populations, SIAM 1972

David H. Hubel-Eye, Brain and Vision, Sci. Am. 1988

M.Kac-Lectures on Probabilistic Methods, 1958

D.Kahneman-D.Slovic-A.Tversky-Judgement under uncertainty: Heuristics and Biases, Cambridge Univ. Press 1982.

J.P.Keener-J.Sneyd-Mathematical Physiology, 2 vol. Springer 2008.

J.B.Keller-Mortality rate versus age, Th.Pop.Biol. 65 (2004) pg.113.

N.KEYFITZ - H.Caswell- Applied Mathematical Demography, 3rd Ed. SV2005

B.KEYFITZ-N.KEYFITZ-The McKendrick Partial Differential Equation and its Uses in Epidemiology and Population Study, Math.Comp.Mod. (1997):26, 1-9.

N.Keyfitz-Reconciliation of Population Models:Matrix, Integral and partial fraction, JRStat.Soc. A, 130, 1967, 61-83

N.Keyfitz-World Population and Ageing, UChicago Press 1990.

N.Keyfitz-J.Beekman-Demography through Examples, Springer 1984.

M.Kline-Mathematical Thought from Ancient to Modern Times, Oxford UP 1970

E.V.Koonin-A.Novozhilov-G.Karev-The Biological Applications of Birth \& Death Processes, Briefings in Bioinform. 7(1), 2010, 70-85.

H.Kragh-The Origin of Radioactivity: From solvable problem to unsolved Non Problem, Arch.Hist.ExactSci. 50(3-4), 1997, 331-358.

Peter Kropotkin-Mutual Aid:A Factor of Evolution, 1902

K.Lange-Applied Probability, Springer Verlag 2010

P.H.Leslie-On the use of matrices in certain population mathematics, Biometrika, 33 :183-212, 1945.

C.C.Lin-L.A.Segel-Mathematics Applied to Natural Sciences, SIAM 1990

A.J.Lotka-Elements of Physical Biology, 1924, Dover 1956.

D.LUDWIG-Stochastic Population Theories, Springer-Verlag Lect. Notes in Biomath. 3, 1974

D.LUDWIG-The Distribution of Population Survival Times, Am.Nat.147, (1996), 506-520.

S.Luria-M.Delbruck-Mutation of Bacteria, Genetics 28 (1943), 491-511

DJC McKay-Information Theory, Inference and Learning Algorithms, Cambride Univ. Press 2003.

Thomas Robert Malthus- Population: The First Essay, London 1798

R.M.May-When two and two do not make four: nonlinear phenomena in ecology, Proc.R.S.London B 228(1986), 241-66.

R.M.May-Stability and Complexity in Model Ecosystems, Princeton U.P. 1974.

E.Mayr-The Growth of Biological Thoght, Harvard U.Press 1982

A.G.McKendrick-Applications of mathematics to medical problems. Proc.Edinburgh Math.Soc., 44: 98-130, 1926

George A. Miller-The Magical Number Seven, Plus or minus Two: Some Limits on Our Capacity for Processing Information, Psych.Rev. 63(1956), 81-97: ``My problem, ladies

and gentleman is that I have been persecuted by an integer"

J.Monod-Le hasard et la necessité, Ed. du Seuil 1970-(trad. Chance and Necessity:An Essay on the Natural Philosopy of Modern Biology- Vintage 1971)

J.Monod-The Growth of Bacterial Culture, Ann.Rev. Microbiol. 1949, 3, 371-394.

J. Pearl-D.Mackenzie-The Book of Why: The New Science of Causation and Effect, Basic Books 2018

A.Perelson-P.W.Nelson-.... HIV Virus Dynamics.....SIAM Rev 1999

A.S.Perelson-P.W.Nelson-The Mathematics of HIV Infection, in J.Sneyd-ed-An Introduction to Mathematics of Biology, AMS 2001

Charles S.Peskin-Mathematical Aspects of Heart Physiology, Lect. Courant Inst.-NYU 1978- AMS2008

Physics Web-Bismuth break half-life.....online: http://physicsweb.org/article/news/7/716

PhysicsWeb- Carbon clock could show the wrong time, 10 May 2001-online: http://physicsweb.org/article/news/5/5/7/1J. von Plato-.......

David Pimentel-R.Hopfenberg-Human Population Numbers as a Function of Food Supply, -pp online 2001

S.D.Poisson-La proportion des Naissances des Filles et des Garçons, Memoire de l’Acad. des Science, 08 février 1829. https://babel.hathitrust.org/cgi/pt?idmdp.39015011958983;view1up;seq489

G.Polya-Patterns of Plausible Inference, Princeton Univ. Press 1968.

Th.Porter-A Statistical survey of Gases: Maxwell’s Social Physics, Hist.Studies in the Phys.Sci. 12(1) 1980, 77-116

L.Redniss-Radioactive–Marie and Pierre Curie; A Tale of Love and Fallout, 2010

M.Rose-The Evolution of Ageing since Darwin, J.Gen. 87(4) 2008

P.Samuelson-Resolving a historical confusion in population analysis, Human Biology, 48: 559-580, 1976 \& pg.109-129 in Smith \& Keyfitz(1977).

S.Schweber-The Origin of Origins, J.of the Hist Biol 10 (1977), 229-310.

Scientific American–The Mind’s Eye, Readings from Sci. Am. 1986

Scientific American- Image, Object Illusion, Readings from Sci. Am. 1974

A.Shapiro-ed.-The Oxford Handbook of Visual Illusion, Oxford UP 2017

D.Smith-N.Keyfitz-editors-Mathematical Demography-Selected Papers, Springer Verlag 1977.

J.Sneyd-ed-An Introduction to Mathematics of Biology, AMS 2001

J.Sung-J.Yu-Population System Control Springer 1988- rev. J.Cohen SIAM Review 1990

J.Sung-\& al-Population System Control Math.Comp.Mod. 11(1988) 11-16.

L.Szilard-Ageing Process, Proc.Nat.Acad,Sci., USA 1959

S.Ulam-Marian Smoluchowski and the Theory of Probability in Physics, Am.J.Phys. 25 (1957), 475-481.

J. van Brackel-Radioactivity as Probability , Arch.Hist.Exact Sci. 31 (1985), 369-385.

N. van Kampen-Stochastic Methods in Physics and Chemistry,North-Holland 1985.

J. von Plato-Creating Modern Probability-Mathematical Physics Perspective, Cambridge UP 1994

P. Vorzimmer-Darwin Malthus and Natural Selesction, J.Hist.Ideas 30 (1969), 527-542.

Howard Wainer-Picturing the Uncertain World-How to understand, Communicate and Control Uncertainty through Graphical Display, PrincetonUP 2009

Howard Wainer-Graphic Discovery, Princeton UP 2005

A.R.Wallace-Contribution to the Theory of Natural Selesction, 1870.

N.Wax-editor-Selected papers on Noise and Stochastic Processes, Dover 1954

E.Widiger-ed.-The Five Factor Model, Oxford Univ.Press 2014

R.M.Young-Malthus and the Evolutionist’-Common Context, pg. 23-55 in -Darwin’s Metaphor, R.M.Young-ed. CUP1985.

R.Zwanzig-....Verhulst logistic Equation......PNAS



\chapter{APÊNDICES: Leitura Opcional}
\addt

\section{O MODELO MALTHUSIANO DIFERENCIAL - Redução da Descrição: Uma Curva Suave em lugar de uma Enorme Tabela Discreta}

    Os Modelos Matemáticos da Mecânica Newtoniana formulados a partir do século XVIII são expressos na forma de equações diferenciais admitindo um conceito de tempo contínuo e sempre ``deduzidos'' como decorrência lógica e inexorável de ``leis fundamentais'' da Física. O Método Newtoniano para a construção de Modelos Matemáticos tornou-se o paradigma predominante da Matemática Aplicada. É natural, portanto, que a aplicação desta metodologia na formulação de um modelo demográfico fosse encarado não apenas como indispensável para estabelecer a sua validade a priori como também para tornar possível a utilização de todas as técnicas do Cálculo Diferencial e Integral para o seu tratamento.

    Mas, para que isto fosse possível, o modelo matemático adequado para representar ``o tamanho \(N\)'' da população em cada instante \(t\), \(N(t)\), teria que ser uma função contínua e diferenciável. Ora, mas diriam os apressados da objetividade, esta pretensão é um rematado contra-senso já que, notoriamente (!), a função \(N(t)\) varia aos saltos, ``de um em um'' e, portanto, é irremediavelmente descontínua.

    Na verdade, esta é uma objeção pertinente e bem fundamentada o que a faz merecer uma abordagem séria e cuidadosa muito embora na maioria das vezes os textos usuais de Matemática Aplicada apelem para um silêncio conformista da audiência. (W. C. Ferreira J - Conferencia/Artigo: O Silêncio dos Conformistas, 2018).

    A argumentação mais simples e direta para justificar a transição do discreto para o contínuo (e diferenciável) na representação de uma Dinâmica Populacional tem um fundamento essencialmente cognitivo e também é relevante para diversos outros contextos semelhantes na Matemática Aplicada.

    Iniciemos pela observação inequívoca de que em toda a Matemática, dispomos apenas de representações gráficas descontínuas de funções (quaisquer que sejam elas) e, somente por uma ilusão de ótica (efeito Kanizsa) é que mentalmente completamos algumas sequências pontilhadas por linhas (imaginárias) ``contínuas e suaves''. Ou seja, a continuidade é apenas uma construção mental que decorre de uma estratégia inata do cérebro humano para fazer sentido dos sinais que a retina lhe envia associando-os a memórias (imagens mentais) de elementos mais organizados, ou estruturados. O cérebro é patentemente incapaz de discriminar uma grande quantidade de informações não estruturadas (segundo George Miller [1956], \(7 \pm 2\)) e, por esta razão, a (enorme quantidade de) informação contida em um gráfico pontilhado tem que ser ``reduzida'' para ser ``entendida''. De fato, a representação gráfica-visual de uma tabela numérica foi uma das técnicas mais importantes inventadas pela ciência em geral, uma vez que, com isto, toda esta capacidade de percepção visual/mental torna-se disponível. É difícil imaginar que o extraordinário desenvolvimento da ciência nos últimos séculos, especialmente da Matemática, pudesse prescindir desta singular sinergia entre símbolos e representação visual. Uma outra maneira de organizar uma grande quantidade de dados ``desconexos'' como, por exemplo, uma série aleatória de dígitos (CPF, Telefone e etc.) é associá-los à uma simples melodia que tem estrutura e, portanto, é de mais fácil memorização, ainda que a associação seja completamente desprovida de sentido. A associação de um gráfico pontilhado a uma curva contínua é apenas uma das estratégias que a mente utiliza para ``reduzir'' uma grande massa de informações. (Sci. Am. [1974],[1986], Hubel [1988], Gregory [2015], Shapiro [2017], Widiger [2014]).

    Naturalmente, para que esta suavização de descontinuidades seja possível, é necessário que o conjunto de pontos não apresente ``grandes'' espaços interstícios e que, de fato, ele disponha de alguma estrutura propícia e não completamente aleatória. Para minimizar a dispersão dos pontos, ou seja, para promover uma ``aglutinação'' dos pontos do gráfico, usualmente lançamos mão da liberdade de escolha de unidades para as medidas das variáveis numéricas Tempo e População. Por exemplo, uma diminuição (aumento) da unidade Tempo resulta em uma compressão (respectivamente, dilatação) linear do gráfico no sentido horizontal. Efeito semelhante na ordenada vertical é obtido com a modificação da unidade de População. Deformações que utilizam o mesmo fator para as duas direções são homotetias e, portanto, não modificam a forma do grafico. Entretanto, as dimensões de Tempo e População são independentes e não há absolutamente nenhuma razão para que tenham unidades transformadas pelo mesmo fator.Enfim, a forma ``rígida'' da curva resultante não é essencial, apenas suas propriedades topológicas: monotonicidade, curvatura e etc.

    Um gráfico pontilhado pode, portanto, após compressão apropriada nas duas direções, fazer com que os pontos de gráficos discretos se aproximem o suficiente para que virtualmente descrevam uma curva contínua. Obviamente isto não a torna uma função matemática contínua, apenas faz com que seu gráfico, para efeito cognitivo, se apresente como um traço contínuo e assim favoreça psicologicamente esta interpretação.

    É bom deixar claro que a hipótese de que um gráfico discreto possa ser bem representado por uma curva contínua e suave é uma hipótese (ou, ``wishful thinking") fundamental para a construção do Modelo Matemático diferencial. A adequação do Modelo Matemático resultante desta hipótese somente poderá ser verificada a posteriori , nunca ``demonstrada'' a priori. Um conjunto de pontos completamente aleatórios dificilmente poderá ser associado a alguma estrutura simples que o represente razoavelmente.

    É interessante citar a possibilidade de generalizar estas ideias com a utilização de escalas não lineares cujas ``lentes de observação'' se modificam segundo as regiões e de acordo com a conveniência do objetivo descritivo do Modelo matemático. A escala (não linear) logarítmica é talvez o exemplo mais comum desta estratégia. Os Métodos Assintóticos também lançam mão desta estratégia com frequência. (Lin-Segel[1990], Segel [BullMathBiol1989]).

    A compressão vertical do gráfico de uma dinâmica populacional somente pode ser realizada se tratamos de grandes populações, por exemplo, da ordem de \(10^9\), como as populações do Brasil, de um grande formigueiro [Gordon [1999]), do número de células do sistema imunológico ou de neurônios [Herculano-Houzel-2009] e etc.. (O número de Avogadro é da ordem de \(10^{23}\). Assim, se a unidade empregada for \(P_0 = 10^{7}\) (isto é, um ``lote'' de \(10\) milhões de indivíduos) estas populações biológicas passam a ser descritas com valores \(0 \le n \le 100\).

    A compressão horizontal, por sua vez, pode, por exemplo, aglutinar 200 pontos para a representação discreta de uma dinâmica demográfica de 100 anos com censos semestrais. A dinâmica populacional de um formigueiro, por outro lado, se analisada em um período de 5 anos (Gordon [1999]) com dados semanais apresenta um total da ordem de 250 pontos. A dinâmica imunológica é mais rápida e um período de 1 mês, pode ser registrada discretamente em intervalos de trés horas o que também exige uma ordem de 250 pontos.

    A modificação linear de escalas, ou de unidades, funciona como uma lente regulável que pode, por um lado, focalizar pequenos detalhes locais, tal como um microscópio, (reduzindo o campo de observação), ou, por outro, possibilitar a percepção da estrutura geométrico-topológica do ``todo'' global com o afastamento do ponto de vista do observador, que aglutina pequenas variações (``borragem'') ,reduzindo, neste processo, a quantidade de informação. A construção de imagens contínuas pela iluminação de (invisíveis) pixels e a superposição de imagens em um filme cinematográfico são exemplos comuns e práticos desta técnica.

    O problema de redução de informações, todavia, não é apenas uma estrategia psicológica humana. Hoje em dia, qualquer área científica (Social ou da Natureza) é inundada por uma massa avassaladora de informações, de tal monta que se torna patentemente impossível enfrenta-la com os métodos computacionais ou analíticos tradicionais. Em vista disso, os chamados Métodos Matemáticos de Redução, que tem por finalidade exatamente reduzir e organizar apropriadamente enormes arquivos de informações, foram rapidamente desenvolvidos nas últimas décadas tornando-se uma classe especial de métodos da Matemática Aplicada Contemporânea. Um dos métodos mais efetivos e interessantes para este fim faz uso de um processo (artificial) de difusão (um operador de difusão) que de fato associa conjuntos discretos a figuras (gráficos) matematicamente suaves. (W. C. Ferreira Jr. [2018b]). Estes Métodos serão tratados em capítulo a parte. (Kutz [2015],...).



\section{O EFEITO KANIZSA, A METODOLOGIA ANALÍTICA DE GALILEO, E SUA EXTENSÃO NEWTONIANA:}

Redução e Síntese: Uma Curva Suave em lugar de uma Enorme Tabela Discreta (Kanizsa), sua Representação Cartesiana Funcional (Galileo) e sua Caracterização como Solução de uma Equação Diferencial (Newton)

\begin{citacao}
    ``Eu tenho uma maior admiração por aquele que, pela primeira vez, imaginou e construiu um instrumento musical que seria um tosco protótipo da harpa, do que pelos admiráveis artesãos que aperfeiçoaram este instrumento até a forma graciosa e a sonoridade perfeita que hoje o caracteriza''.
    
    \rightline{Galileo}
\end{citacao}

    A Metodologia de Galileo que busca sintetizar (e generalizar) funcionalmente a correspondência de uma Tabela de dados experimentais deu início a uma revolução científica quando a sua perspicácia (e conhecimentos de Geometria Elementar) detectou (aproximadamente) as propriedades de uma parábola nas imagens de trajetórias de balas de canhão que ele exaustivamente analisou. (G.Galilei-Dialogo). Daí, à uma formulação analítica para esta trajetória foi um pequeno passo em vista da recentemente desenvolvida Geometria Analítica de Renè Descartes. (Uma coincidência que não pode passar despercebida).

    Entretanto, se a ideia de Galileo era brilhante, por outro lado, a sua implementação era difícil, mesmo com a Geometria Analítica de Descartes, pois não havia um procedimento padrão para caracterizar a função que bem representasse uma Tabela de dados. Além disso não havia ainda uma ``biblioteca de funções'' suficientemente grande que permitisse encontrar esta representante, visto que as funções disponíveis à época eram apenas as Elementares Algébricas, i.e., obtidas como resultado de uma sequencia finita de operações de soma, produto, composição e potencias racionais aplicadas às funções básicas {funções constantes e função identidade}.

    Com a invenção do Cálculo Diferencial a ``biblioteca de funções'' aumentou consideravelmente pois incluiria a operação (transcendental) de soma infinita (além das operações algébricas) e, não menos importante, disponibilizou o emprego da Metodologia Newtoniana que caracteriza uma função de uma maneira extremamente sintética em termos da solução de uma equação diferencial. (Compare a expressão aritmética da função exponencial
    \[f(x) = \sum_{k=0}^{\infty} \dfrac{1}{k!} x^k,\]
    com a sua ultra-sintética descrição diferencial,
    \[\dfrac{df}{dx} = f,\ f(0) = 1).\]

    A Metodologia Newtoniana para a Matemática Aplicada se constitui, portanto, de duas partes:

    1) Os Princípios que permitem ``encriptar'' uma função representativa de um Modelo Matemático na forma de solução de uma Equação Diferencial e,

    2) Os Métodos analíticos que são instrumentos necessários para a ``abertura destes códigos'', ou seja, os métodos de resolução de equações diferenciais. (O Presente texto está organizado segundo esta Metodologia Newtoniana).

    {\color{red} Os Modelos Matemáticos da Mecânica Newtoniana formulados a partir do século XVIII são expressos na forma funcional, cujas funções são definidas como soluções de equações diferenciais. A Metodologia Newtoniana para a construção de Modelos Matemáticos tornou-se o paradigma predominante da Matemática Aplicada devido à sua capacidade de síntese que estava imersa na Teoria do Cálculo Diferencial e Integral inventado por Newton e Leibniz. É natural portanto que a aplicação desta metodologia na formulação de um modelo demográfico fosse encarado como indispensável.

     Entretanto, para que isto fosse possível o modelo matemático adequado para representar ``o tamanho \(N\)'' da população em cada instante \(t\), \(N(t)\), teria que ser uma função contínua e diferenciável. Ora, mas diriam os apressados da objetividade, esta pretensão é um rematado contra-senso já que, notoriamente (!), a função \(N(t)\) varia aos saltos, ``de um em um'' e, portanto, é irremediavelmente descontínua.

    Na verdade, esta é uma objeção pertinente e bem fundamentada o que a faz merecer uma abordagem séria e cuidadosa muito embora na maioria das vezes os textos usuais de Matemática Aplicada apelem para um silêncio conformista da audiência. (W. C. Ferreira J - Conferencia/Artigo: O Silêncio dos Conformistas, 2018).

    A argumentação mais simples e direta que sugere a transição do discreto para o contínuo (e para a suavidade do diferenciável) na representação de uma Dinâmica Populacional tem um fundamento essencialmente cognitivo e também é relevante para diversos outros contextos semelhantes na Matemática Aplicada.

    Iniciemos pela observação inequívoca de que em toda a Matemática, dispomos apenas de representações gráficas descontínuas de funções (quaisquer que sejam elas) e, somente por uma ilusão de ótica (efeito Kanizsa) é que mentalmente completamos algumas sequencias pontilhadas por linhas (imaginárias) ``contínuas e suaves". Ou seja, a continuidade e a suavidade é apenas uma construção mental que decorre de uma estratégia inata do cérebro humano para fazer sentido dos sinais que a retina lhe envia associando-os a memórias (imagens mentais) de elementos mais organizados, ou estruturados. O cérebro é patentemente incapaz de discriminar uma grande quantidade de informações não estruturadas (segundo George Miller[1956], 7  2) e, por esta razão, a (enorme quantidade de) informação contida em um gráfico pontilhado tem que ser ``reduzida'' para ser ``entendida", isto é, memorizada por intermedio de algumas de suas caracteristicas mais proeminentes. De fato, a representação gráfica-visual de uma tabela numérica foi uma das técnicas mais importantes inventadas pela ciência em geral, uma vez que, com isto, toda esta capacidade de percepção visual/mental torna-se disponível. É difícil imaginar que o extraordinário desenvolvimento da ciência nos últimos séculos, especialmente da Matemática, pudesse prescindir desta singular sinergia entre símbolos e representação visual.Uma outra maneira de organizar uma grande quantidade de dados ``desconexos'' como, por exemplo, uma série aleatória de dígitos (CPF,Telefone e etc.) é associa-los à uma simples melodia que tem estrutura e, portanto, é de mais fácil memorização, ainda que a associação seja completamente desprovida de sentido. A associação de um gráfico pontilhado a uma curva contínua é apenas uma das estratégias que a mente utiliza para ``reduzir'' uma grande massa de informações a um tamanho que pode ser ``arquivado'' e classificado segundo algumas poucas caracteristicas( Sci.Am. [1974],[1986], Hubel[1988],Gregory[2015],Shapiro[2017], Widiger[2014]).

    Naturalmente, para que esta suavização de descontinuidades seja possível, é necessário que o conjunto de pontos não apresente ``grandes'' espaços interstícios e que, de fato, ele disponha de alguma estrutura propícia e não completamente aleatória. Para minimizar a dispersão dos pontos, ou seja, para promover uma ``aglutinação'' dos pontos do gráfico, usualmente lançamos mão da liberdade de escolha de unidades para as medidas das variáveis numéricas Tempo e População. Por exemplo, um aumento da unidade Tempo resulta em uma compressão linear do gráfico no sentido horizontal. Efeito semelhante na ordenada vertical é obtido com a modificação da unidade de População. Estas deformações do grafico não modificam aspectos topologicos que representam informações essenciais como região de crescimento e de curvaturas (isto é, os sinais da primeira e segunda derivadas). Enfim, na representação gráfica, a forma ``rígida'' da curva resultante não é essencial, apenas suas propriedades topológicas: monotonicidade, curvatura e etc.

    Como as duas dimensões (Tempo e População), a principio, nada tem a ver uma com a outra, isto é, são independentes, não há absolutamente nenhuma razão para que tenham unidades transformadas pelo mesmo fator e este fato será utilizado em várias situações para enfatizar aspectos distintos do Modelo Matemático.

    Um gráfico pontilhado pode, portanto, após compressão apropriada nas duas direções, fazer com que os pontos de gráficos discretos se aproximem o suficiente para que virtualmente descrevam uma curva contínua. Obviamente isto não a torna uma função matemática contínua, apenas faz com que, para efeito cognitivo, seu gráfico se apresente como um traço contínuo e assim favoreça psicologicamente esta interpretação.

    De qualquer forma, é bom ressaltar que a hipótese de que um gráfico discreto possa ser bem representado por uma curva contínua e suave é uma hipótese (ou, ``wishful thinking") fundamental para a construção do Modelo Matemático diferencial. A adequação do Modelo Matemático resultante desta hipótese somente poderá ser verificada a posteriori, nunca ``demonstrada'' a priori. Um conjunto de pontos completamente aleatórios dificilmente poderá ser associado a alguma estrutura simples que o represente razoavelmente.

    Mesmo assim, veremos que há casos em que isto é possível.(....[..] ).

    É interessante citar a possibilidade de generalizar estas ideias com a utilização de escalas não lineares cujas ``lentes de observação'' se modificam segundo as regiões e de acordo com a conveniencia do objetivo descritivo do Modelo matemático. A escala (não linear) logaritma é talvez o exemplo mais comum desta estratégia.Os Métodos Assintóticos também lançam mão desta estatégia com frequencia.(Lin-Segel[1990], Segel[BullMathBiol1989]).

    A compressão vertical do gráfico de uma dinâmica populacional somente pode ser realizada se tratamos de grandes populações, por exemplo, da ordem de 109, como as populações do Brasil, de um grande formigueiro [Gordon[1999]), do número de células do sistema imunológico ou de neurônios [Herculano-Houzel-2009] e etc..(O numero de Avogadro é da ordem de 1023. Assim se a unidade empregada forP0  107 (isto é, um ``lote'' de 10 milhões de individuos) estas populações biológicas passam a ser descritas com valores 0  n  100.

    A compressão horizontal, por sua vez, pode, por exemplo, aglutinar 200 pontos para a representação discreta de uma dinâmica demográfica de 100 anos com censos semestrais. A dinâmica populacional de um formigueiro, por outro lado, se analisada em um período de 5 anos (Gordon[1999]) com dados semanais apresenta um total da ordem de 250 pontos. A dinâmica imunológica é mais rápida e um período de 1 mês, pode ser regstrada discretamente em intervalos de tres horas o que também exige uma ordem de 250 pontos.

    A modificação linear de escalas, ou de unidades, funciona como uma lente regulável que pode, por um lado, focalizar pequenos detalhes locais, tal como um microscópio,(reduzindo o campo de observação), ou, por outro, possibilitar a percepção da estrutura geométrico-topológica do ``todo'' global com o afastamento do ponto de vista do observador, que aglutina pequenas variações ("borragem") ,reduzindo, neste processo, a quantidade de informação. A construção de imagens contínuas pela iluminação de (invisíveis) pixels e a superposição de imagens em um filme cinematográfico são exemplos comuns e práticos desta técnica.

    O problema de redução de informações, todavia, não é apenas uma estrategia psicológica humana. Hoje em dia, qualquer área científica (Social ou da Natureza) é inundada por uma massa avassaladora de informações, de tal monta que se torna patentemente impossível enfrenta-la com os métodos computacionais ou analíticos tradicionais. Em vista disso, os chamados Métodos Matemáticos de Redução, que tem por finalidade exatamente reduzir e organizar apropriadamente enormes arquivos de informações, tem sido rapidamente desenvolvidos nas últimas décadas tornando-se uma classe especial de métodos da Matemática Aplicada Contemporânea. Um dos métodos mais efetivos e interessantes para este fim faz uso de um processo (artificial) de difusão (um operador de difusão) que de fato associa conjuntos discretos a figuras (gráficos) matematicamente suaves.(WCFerreira Jr.[2018b]).Estes Métodos serão tratados em capítulo a parte.(Kutz[2015],.....).


}


\section{A Metodologia de SÍNTESE FUNCIONAL DE DADOS EXPERIMENTAIS: Galileo-Newton-Kanizsa A Estética como Economia de Informação}

    ``The ability to describe underlying patterns from data has been called the fourth paradigm of scientific discovery. [However, according to Kanizsa, that is exactly what our cognitive senses authomatically do all the time since immemorial eras. Besides, people forget to point out that Kepler, Huygens, and Rutherford have done just that centuries ago]''.

    Primeira parte: J. G. Hey Anthony \& al; Report Microsoft Res., Redmond WA 2009. Segunda parte: paráfrase anônima.

    O trabalho de Galileo com a sua descoberta de que a tabela de dados observados para a queda livre de um objeto apresentava propriedades geométricas típicas de uma quádrica (uma parábola) e portanto, poderia ser descrita pela recém inventada Geometria Analítica de Descartes como uma função quadrática, inaugurou um Método revolucionário para a representação matemática da Natureza. Este procedimento foi logo abraçado por Isaac Newton que o empregou na descrição funcional das órbitas celestes, mas agora caracterizando as funções como solução de Equações Diferenciais, o que sintetizava ainda mais as informações e aumentava a ``biblioteca'' de funções disponíveis para a descrição.

    O enorme sucesso da Metodologia Newtoniana introduzida com a construção do Modelo Matemático da Mecânica no século XVIII, estabeleceu um paradigma que tem prevalecido sem contestação para a Matemática Aplicada desde então. Sob este ponto de vista, um Modelo Matemático para um fenômeno natural, em geral, faz uso da vasta ``biblioteca'' de Funções reais (diferenciáveis) do Cálculo para descrevê-lo e de Equações Diferenciais para caracterizar estas funções.(Frequentemente o Modelo é confundido com estas equações diferenciais, mas como o próprio Newton enfatizou, a escolha das funções que descreverão o ``estado'' do objeto de estudo é um passo preliminar e crucial para a construção do Modelo e antecede a sua caracterização especifica por intermédio de uma equação diferencial).

    É natural, portanto, que a formulação de um Modelo Matemático para a Dinâmica de grandes Populações também fosse buscada nestes termos que imediatamente coloca à disposição de seu tratamento todo o vasto e eficiente instrumental analítico do Cálculo Diferencial e Integral desenvolvido por Newton, e Leibniz. Entretanto, a descrição matemática de uma Dinâmica Populacional é necessariamente intermediada por uma função que representa a medida \(N\) do tamanho (cardinalidade) de um conjunto de organismos em cada instante t, que, a rigor, assume apenas valores inteiros não negativos. Sob o ponto de vista Newtoniano, a construção de um Modelo diferencial para este problema é, portanto, de saída um notório contra-senso já que esta função \(N(t)\) varia aos saltos ,"de um em um'' subitamente em instantes pontuais e, portanto, irremediavelmente descontínua e não diferenciável.

    De fato, esta é uma objeção pertinente e bem fundamentada o que a faz merecer uma abordagem séria e cuidadosa muito embora na maioria das vezes os textos usuais de Matemática Aplicada apelem para um silêncio conformista do/as leitores/as quanto a este aspecto.(Lin-Segel é um dos poucos textos didáticos que enfrentam honestamente esta questão. W. C. Ferreira Jr - Conferência/Artigo: O Silêncio dos Conformistas, 2018).

    A argumentação mais simples e direta para justificar a transição do discreto para o contínuo (e diferenciável) na representação de uma Dinâmica Populacional tem um fundamento essencialmente cognitivo e também é relevante para diversos outros contextos semelhantes na Matemática Aplicada.

    Iniciemos pela observação inequívoca de que em toda a Matemática, dispomos apenas de representações gráficas descontínuas de funções (quaisquer que sejam elas) e, somente por uma ilusão de ótica (efeito Kanizsa) é que mentalmente completamos algumas sequencias pontilhadas por linhas (imaginárias) ``contínuas e suaves". Ou seja, a continuidade é apenas uma construção mental ``estética'' que decorre de uma estratégia inata (evolutiva) que o cérebro humano dispõe para ``entender'' os sinais que a retina lhe envia associando-os a memórias (imagens mentais) de elementos mais organizados, ou estruturados. O cérebro é patentemente incapaz de discriminar uma grande quantidade de informações não estruturadas (segundo George Miller [1956], \(7 \pm 2\)) e, por esta razão, a (enorme quantidade de) informação contida em um gráfico pontilhado tem que ser ``reduzida'' para ser ``entendida", isto é, memorizada. De fato, a representação cartesiana gráfico-visual de uma tabela numérica foi uma das técnicas mais evolucionais e eficazes introduzidas pela ciência, que lança mão de toda esta capacidade de percepção visual/mental. É difícil imaginar que o extraordinário desenvolvimento da ciência nos últimos séculos, especialmente da Matemática, pudesse prescindir desta singular sinergia que associa formas geométricas familiares a tabelas numéricas ``caóticas". (v. citação de Galileo). (Uma outra maneira de organizar uma grande quantidade de dados ``desconexos'' como, por exemplo, uma série aleatória de dígitos (CPF,Senhas, Telefone e etc.) é associá-los à uma simples melodia que tem estrutura rítmica e, portanto, é de mais fácil memorização, ainda que a associação possa ser completamente desprovida de outro sentido.

    A associação de um gráfico pontilhado a uma curva contínua é apenas uma das estratégias que a mente utiliza para ``reduzir'' uma grande massa de informações a uma única e rememorável objeto. O reconhecimento de padrões faciais, por exemplo, assunto em que a cognição humana é especialista, se faz por intermédio de uma redução de detalhes a ponto de poder representa-las por intermédio de ``caricaturas'' com poucos traços notáveis.Este processo tem analogias óbvias com a técnica de Funções Geradoras utilizada para encapsular uma sequencia infinita de números por meio de uma função analítica. (Sci.Am. [1974],[1986], Hubel [1988], Gregory [2015], Shapiro [2017], Widiger [2014], Pólya, Wilf). Naturalmente, para que esta suavização de descontinuidades seja possível, é necessário que o conjunto de pontos não apresente ``grandes'' espaços interstícios e que, de fato, ele favoreça a associação à alguma estrutura subjacente e não seja completamente aleatória.

    Portanto, a primeira condição para que a representação de uma dinâmica populacional possa ser bem realizada por intermédio de funções continuas e suaves, é que a população seja formada por um grande números de indivíduos e que os dados observados ocorram em pequenos intervalos de tempo.Obviamente, os conceitos de ``grande'' e ``pequeno'' nesta afirmação são diretamente dependentes da representação geométrica da função e da cognição visual humana.

    Consideremos a representação geométrica cartesiana. Para minimizar a dispersão dos pontos, ou seja, para promover uma ``aglutinação'' dos pontos do gráfico, usualmente lançamos mão da liberdade de escolha de unidades para as medidas das variáveis numéricas Tempo e População. Por exemplo, uma diminuição (aumento) da unidade Tempo resulta em uma compressão (respectivamente, dilatação) linear do gráfico no sentido horizontal. Efeito semelhante na ordenada vertical é obtido com a modificação da unidade de População. Deformações que utilizam o mesmo fator para as duas direções são homotetias e, portanto, não modificam a forma do gráfico. Entretanto, as dimensões de Tempo e População são independentes e não há absolutamente nenhuma razão para que tenham unidades transformadas pelo mesmo fator. Enfim, a forma ``rígida'' da curva resultante não é essencial, apenas nos interessam algumas de suas propriedades topológicas: monotonicidade, tipos de curvatura e etc. que podem caracterizar uma imagem mental da função.

    (A representação cartesiana de funções, que aos olhos dos Conformistas parece ser inelutável, na verdade é uma construção arbitrária, ainda que de grande utilidade como mostra a história. Na verdade, nem sequer o espaçamento uniforme das medidas na reta real é uma necessidade. É interessante citar a possibilidade de generalizar estas ideias com a utilização de escalas não lineares e não uniformes cujas ``lentes de observação'' se modificam segundo as regiões e de acordo com a conveniência do objetivo descritivo do Modelo matemático. A escala (não linear) logarítmica é talvez o exemplo mais óbvio e mais utilizado desta estratégia.Os Métodos Assintóticos também lançam mão desta estratégia com frequência. (Lin-Segel [1990], Segel [BullMathBiol1989])).

    Assim, após uma compressão apropriada, um gráfico pontilhado pode fazer com que os pontos de gráficos discretos se aproximem o suficiente para que virtualmente descrevam uma curva contínua. Obviamente isto não a torna uma função matemática contínua e suave, apenas faz com que seu gráfico, para efeito cognitivo, se apresente mais como um traço contínuo e assim favoreça psicologicamente esta interpretação.

    É importante enfatizar que a hipótese de que um gráfico discreto (Tabela) possa ser bem representado por uma curva contínua e suave é uma hipótese (ou, ``wishful thinking'') fundamental para a construção do Modelo Matemático diferencial. A adequação do Modelo Matemático resultante a esta hipótese somente poderá ser verificada a posteriori , nunca ``demonstrada'' a priori. (Um conjunto de pontos completamente aleatórios dificilmente poderá ser associado a alguma estrutura simples que o represente razoavelmente).

    A compressão vertical do gráfico de uma dinâmica populacional somente pode ser realizada se tratamos de grandes populações, por exemplo, da ordem de 109, como as populações do Brasil, de um grande formigueiro [Gordon [1999]), do número de células do sistema imunológico ou de neurônios [Herculano-Houzel-2009] e etc.. (O número de Avogadro(número de moléculas de um gás em um mol) é da ordem de \(10^{23}\). O número de partículas no Universo, \(10^{?}\). t’Hoft, H.Fritzsch). Assim, se a unidade empregada for \(P_0 = 10^7\) (isto é, um ``lote'' de 10 milhões de indivíduos) estas populações biológicas passam a ser descritas com valores \(0 \le n \le 100\).

    A compressão horizontal, por sua vez, pode, por exemplo, aglutinar 200 pontos para a representação discreta de uma dinâmica demográfica de 100 anos com censos semestrais. A dinâmica populacional de um formigueiro, por outro lado, se analisada em um período de 5 anos (Gordon [1999]) com dados semanais apresenta um total da ordem de 250 pontos. A dinâmica imunológica é mais rápida e um período de 1 mês, pode ser registrada discretamente em intervalos de três horas o que também exige uma ordem de 250 pontos. (Perelson).

    A modificação de escalas, ou de unidades, funciona como uma lente regulável que pode, por um lado, focalizar pequenos detalhes locais, tal como faz um microscópio,(reduzindo o campo de observação), ou, por outro, possibilitar a percepção da estrutura geométrico-topológica do ``todo'' global com o afastamento do ponto de vista do observador, que aglutina pequenas variações (``borragem'') ,reduzindo, neste processo, a quantidade de informação.

    A construção de imagens contínuas pela iluminação de (invisíveis) pixels e a superposição rápida de imagens em um filme cinematográfico são exemplos comuns e práticos desta técnica de apresentar (ilusoriamente) como um contínuo aquilo que é fundamentalmente discreto.

    O problema de redução de informações, todavia, não é apenas uma estrategia psicológica humana. Hoje em dia, qualquer área científica (Social ou da Natureza) é inundada por uma massa avassaladora de informações, de tal monta que se torna patentemente impossível enfrentá-la com os métodos computacionais ou analíticos tradicionais. Em vista disso, os chamados Métodos Matemáticos de Redução, que tem por finalidade exatamente reduzir e organizar apropriadamente enormes arquivos de informações, foram rapidamente desenvolvidos nas últimas décadas tornando-se uma classe especial de métodos da Matemática Aplicada Contemporânea. Um dos métodos mais efetivos e interessantes para este fim faz uso de um processo (artificial) de difusão (um operador de difusão) que de fato associa conjuntos discretos a figuras (gráficos) matematicamente suaves. Tal como a diversão de encontrar padrões de carneirinhos em formação de nuvens. (W. C. Ferreira Jr. [2018b]). Estes Métodos serão tratados em capítulo a parte. (Kutz [2015],...).

    E, por fim, a substituição do discreto pelo contínuo e vice-versa não é uma atitude nova ou abstrusa, mas é exatamente o que se faz o tempo todo com a representação gráfica no papel de funções infinitamente diferenciáveis, como, por exemplo a função seno. (Afinal, como diz um surpreso personagem de uma peça teatral ao seu instrutor: ``A prosa é isto? Então, eu estou a pratica-la durante toda a minha vida sem o saber!'')
    
    Resolvida a questão geral da plausibilidade da aplicação da Metodologia Newtoniana à Dinâmica de grandes Populações, o segundo passo consiste na caracterização funcional da função diferenciável \(N(t)\) que descreve o estado da população em cada instante. Esta segunda etapa depende naturalmente de cada caso específico e das hipóteses biológicas particulares assumidas para esta população. Neste capitulo trataremos do problema proposto por Malthus que, apesar de (ou, por) considerar situações extremamente específicas, é fundamental para toda a Dinâmica Populacional em geral.

    É interessante ressalvar que, embora o problema abordado na questão Malthusiana trate explicitamente de Grandes populações, os conceitos que serão desenvolvidos neste mister serão úteis para o tratamento de Pequenas populações , só que, neste caso, a função \(N(t)\) será assumidamente discreta, mas com valores estocásticos.


%% CAPITULO II PARTE B


\chapter{Princípio de Malthus: Variações}
\addt

\section{VARIAÇÕES SOBRE O MODELO MALTHUSIANO: Relaxamento da Homogeneidade e Independência}


\begin{enumerate}
\item Interação Indireta: Resgate da Doutrina de Malthus- Verhulst-Tragédia dos Comuns-Hardin-Ostrom-Lotka-Volterra, Nowak \& - MetodoGeom
vHeterogeneidade: Maturidade Retardada em Modelo Discreto- Fibonacci - Euler - Equações Recirsivas - Método F. Geradoras/Momentos 
\item Heterogeneidade:Maturidade Retardada em Modelo Continuo-Diferencial - Lotka - Oscil \& Caos - (Ref: Nelson-Mikhailov/Forys) 
\item Heterogeneidade e Independência-Modelo Efetivo- Método Assint. de Múltiplas Escalas-Média Harmônica 
\item Heterogeneidade e Acoplamento Dirigido (Cadeias) - Método de Fourier/Oper.- Método F. Geradoras
\item Heterogeneidade e Acoplamento Mutuo (Difusivo)-Grafos-Metodos Oper/Fourier/F. Ger - (Gunawardena 
\item Heterogeneidade-Processos Sociais(Young) 
\item Influencia Externa- Método Oper-Método Assint.Condensação/Lagrange 
\item Influencia Externa Estocástica-Langevin
\item Influencia Externa (?) Estocástica-Modelo de Lapicque-Neuronio Carga \& Descarga - Peskin-Knight-Dayan 
\item Pequenas Populações: Modelos Probabilísticos - Mutação e Decisão: Delbruck Luria - Kirman.. - Método F. Ger.
\item Pequenas Populações: Modelos Probabilísticos: Eq. Kolmogorov-Fokker-Planck- Método F. Geradoras. 
\end{enumerate}

\begin{quotation}
    O Modelo Malthusiano é uma pedra angular a partir do qual, seguindo os Princípios de Comenius e de Ockham, procede-se com o relaxamento parcimonioso, progressivo e argumentativo das restrições impostas pelas hipóteses de Homogeneidade e Independência que o caracterizam, resultando deste procedimento Modelos Matemáticos mais inclusivos e representativos de uma gama ampliada de fenômenos biológicos.
\end{quotation}
 
\section{INTERAÇÃO INDIRETA UNIFORME: Resgate da Doutrina de Malthus, o Modelo (Malthus-) Verhulst e a Tragédia dos Comuns}

\begin{citacao}
    ``Assuming then, my postulata as granted, I say, that the power of population is  indefinitely greater than the power in the Earth to produce subsistence for man. Population when unchecked, increases in a geometrical ratio. \textbf{Subsistence increases only in arithmetical ratio}''.

    (Th. R. Malthus ``Population: The first Essay'', London, june 7, 1798, Chap.I, pg. 5)
\end{citacao}

    A ressalva sobre a lenta (Aritmética) produção de meios de subsistência que Malthus tão bem enfatizou logo no início de seu trabalho (``\textit{Subsistence increases only in arithmetical ratio}'') foi amplamente ignorada em detrimento do aspecto mais espetacular do crescimento Geométrico (exponencial) da população. Esta flagrante injustiça intelectual com relação ao exposto na sua Doutrina foi ideologicamente explorada ad nauseam ao longo da história e suas previsões catastróficas imputadas a uma má fé do seu autor. 

    Sob o ponto de vista científico e pedagógico a identificação pura e simples da ``Doutrina Malthusiana'' com a equação diferencial
    \[\dfrac{dN}{dt} = \nu N,\]
    e sua solução exponencial
    \[N(t) = N_0 e^{\nu t},\]
    eliminou uma excelente oportunidade para abordar diversos aspectos interessantes e profundos sugeridos neste trabalho, o que somente veio a ocorrer muito mais tarde.
    
    Nesta seção, a Doutrina de Malthus será apresentada matematicamente em sua versão original que leva em conta a ressalva sobre as taxas distintas de \textbf{reprodução} populacional a \textbf{produção} de nutrientes.

    Inicialmente, observa-se que sendo o transcurso do tempo representado discretamente em períodos regulares, a expressão ``\textit{Arithmetical ratio}'' utilizada por Malthus significa, de acordo com a linguagem da época, que denotando \(R(k)\) por medida da quantidade de recursos de subsistência no instante \(k\) então \(R(k) = R_0+\lambda k\), ou seja, ela é representada por uma ``\textit{progressão aritmética}'', enquanto que a população \(P(k) = \lambda^k P_0\) é uma ``\textit{progressão geométrica}''. Em um modelo diferencial com o tempo registrado continuamente a produção ``aritmética'' de recursos toma a forma: \(R(t) = R_0 + \lambda t\).

    Embora Malthus não tenha argumentado explicitamente a respeito da produção ``aritmética'' de recursos, a hipótese de uma taxa média constante para produção de nutrientes em uma região e durante um longo período de tempo é plausível porque toda a matéria orgânica, de uma forma ou de outra, provem da ação solar (fotossíntese) cujos raios, a menos de variações sazonais, incidem anualmente em intensidade regular sobre a Terra. Portanto, se a unidade de tempo k for de um ano, esta hipótese significa que a quantidade total  de alimentos produzida a cada ano (taxa de produção anual) é constante ao longo de décadas. De fato, todos os nutrientes orgânicos na Terra são originários da fotossíntese realizada por plantas e fitoplânctons no oceano:
    \[\begin{array}{c}
    \mbox{Luz solar } + 12 H_2O + 6 CO_2 \\
    \Rightarrow \\
    \underbrace{6O_2 + 6 H_2O + C_6H_{12}O_6}_{\mbox{Glicose}} \\[0.5cm]
    (\mbox{v. P. Nelson})
    \end{array}\]

    A análise dos aspectos biológicos intrínsecos ao modelo Malthusiano, enfatiza duas características cruciais da população:
    \begin{enumerate}
    \item \textbf{Homogeneidade} dos indivíduos quanto à sua fertilidade e/ou morbidade, uma vez que, independente de qualquer quantidade de indivíduos desta espécie que são acrescentados ou retirados desta (grande) população, o valor de
    \[\dfrac{1}{N} \dfrac{dN}{dt}\]
    permanece constante.
    \item \textbf{Isolamento Social} (Não Interação entre Indivíduos): Em decorrência de um argumento semelhante ao empregado no item 1) não há influência de um indivíduo sobre outro no que diz respeito à sua capacidade reprodutiva ou chance de óbito.
    \end{enumerate}


    Observemos que em Dinâmica Populacional o individuo é representado pelo seu ``átomo'' reprodutivo. Isto é, no caso de uma população de bactérias que se reproduzem assexuadamente, o individuo (``átomo'') é uma bactéria, enquanto que no caso de uma população humana (ou de qualquer espécie cuja reprodução é sexuada) o ``átomo'' da população é o ``casal'' representado por um indivíduo do sexo feminino sob a hipótese tácita de equilíbrio entre as populações dos dois sexos. 

    A reprodução biológica exige um grande investimento de nutrientes e somente pode ocorrer caso o suprimento de recursos disponíveis para um organismo apresente um saldo mínimo com respeito ao fluxo necessário para a manutenção de sua sobrevivência basal. Se a taxa de fornecimento de nutrientes for menor do que o requerimento basal, a sua chance de óbito aumenta e a sua fertilidade diminui ou cessa totalmente. Em resumo, é biologicamente plausível a hipótese de que a taxa de reprodução per capita \[\dfrac{1}{N} \dfrac{dN}{dt} = r\]
    desta população seja uma função monotônica (mas com saturação, i.e., uma curva logística) do ``\textit{excesso de oferta de nutriente}'', isto é,
    \[r = \varphi(\lambda - \alpha N),\]
    onde \(\alpha\) é a taxa de consumo basal de nutrientes de cada organismo e \(\lambda\) a taxa temporal de produção total de nutrientes. (Isso não significa, necessariamente, que \(\varphi(0) = 0\), mas podemos redefinir o consumo basal como sendo exatamente este valor. O estudo sobre a dependência da fertilidade e mortalidade individual com respeito à disponibilidade de nutrientes é um tema à parte e de importância óbvia.

\begin{quotation}
    (Sobre a abordagem populacional deste tema consulte David Pimentel - R. Hopfenberg - Human Population Numbers as a Function of Food Supply, - pp online 2001, e sob o ponto de vista individual consulte o seguinte artigo e suas referências, A. Bose \& al. - Phenotypic traits and resource quality as factors affecting male reproductive success in a toadfish, Behavioural Ecology 29(2), 2018, 496-507 e G. West - Scales, 2019, .....holandês...??).
\end{quotation}

    Seguindo a severa e sábia admoestação do frei William de Ockham representaremos as argumentações biológicas acima por intermédio da função matemática monotônica mais simples possível e conveniente,
    \[r = r_0 (\lambda - \alpha N),\]
    onde \(r_0\) é uma constante de proporcionalidade. Esta hipótese pode ser argumentada matematicamente com base em uma aproximação linear de uma função monotônica (desconhecida, mas suposta diferenciável) para valores não muito grandes de população, um cenário em que o efeito de saturação não se faz notar. 

\begin{quotation}
    (O conselho de William Ockham, corroborado pelo filósofo Ludwig Wittgenstein para que uma representação matemática seja sempre parcimoniosa pode ser expressa na forma mais direta: ``Se não souber o que dizer (ou como se justificar biologicamente), cale a boca. Diga apenas aquilo que for possível argumentar'').
\end{quotation}

    Portanto, o Modelo simples resultante de uma representação matemática da Doutrina do Rev. Malthus toma a seguinte forma:
    \[\dfrac{dN}{dt} = r(N) N = r_0 (\lambda - \alpha N) N = \lambda r_0 \left(1 - \dfrac{N}{K}\right)N\]
    que, na literatura em geral é conhecida como representante do Modelo de Verhulst.

\begin{quotation}
    (Na verdade, o modelo do matemático belga Pierre F. Verhulst (1804-1849) foi apresentado em um artigo de 1845 na Academia de Ciências da Bélgica com um objetivo declarado: Escrever uma equação diferencial que alegadamente representaria uma dinâmica populacional e exibisse um efeito de saturamento. Após tentar várias expressões matemáticas, ele optou pela quadrática, sem qualquer argumentação biológica e simplesmente pelo fato de que ela cumpria o seu objetivo de tranquilizar a assustada realeza Europeia com a ``anunciada explosão Malthusiana''. Considerando que argumentos biológicos suficientes para justificar este modelo diferencial já se encontravam claramente expostos no livro de Malthus, é injusto atribuir tal modelo a Verhulst e não ao próprio Malthus. A propósito, Verhulst era considerado um matemático competente que nunca abandonou de fato a sua especialidade, mas que ganhou sua fama imortal com a repercussão deste simples trabalho).
\end{quotation}

    É fácil verificar que para \(N < K\) a população cresce (\(\frac{dN}{dt} > 0\)), pois \(r_0 (1=\frac{N}{K}) > 0\) e, para \(N < K\) a população decresce, o que significa que \(K = \frac{\lambda}{\alpha}\) é o valor de \textbf{saturação} para o crescimento da população, também denominado ``\textbf{Capacidade de Suporte}'' do meio ambiente, (um termo devido ao biólogo Raymond Pearl (1879-1940)) que representa o tamanho da população para a qual a capacidade de produção de recursos da região geográfica é exatamente igual à necessidade \textbf{basal} total da população, \(\lambda = \alpha K\). Nesta população a taxa de produção de nutrientes é suficiente apenas para a manutenção da população. Observa-se que o próprio modelo não permite crescimento ilimitado da população (efeito de saturação) e, portanto, a hipótese linear para a função de reprodução \(\varphi\), faz-se razoável se \(K\) não for muito grande.

    A classe de modelos que exibem uma saturação populacional após crescimento monotônico, (exatamente os modelos procurados por Verhulst), é importante para a Biomatemática, especialmente o modelo acima que apresenta a expressão analítica mais simples dentre eles e permite uma interpretação biológica clara em termos de seus parâmetros constitutivos, \(r_0, \lambda, K\).

    É importante observar que este modelo não pressupõe explicitamente uma interação direta entre indivíduos da população. Uma interação todavia existe, mas é indireta e decorre do fato de todos participarem de um mesmo recurso finito, uns em detrimento de outros. Neste caso, como não há discriminação biológica entre os participantes, o modelo supõe que todos eles tem acesso e capacidade iguais (``\textit{democráticas}'') de exploração dos recursos. Expandindo a expressão \(r_0 (1-\frac{N}{K})N\), obtemos um termo de mortalidade de segundo grau em \(N\) (isto é, \(-\alpha N^2\)) que é explicitamente relacionado a um efeito de interação, pois a taxa per capita de mortalidade para este termo \(\red \frac{1}{N} (-\alpha N^2) = -\alpha N\) varia (linearmente) com o tamanho da população!

    O Modelo não discrimina termos lineares de reprodução (\(\nu\)) ou mortalidade(\(\mu\)), (isto é, \(r_0 \lambda = \nu - \mu\)), mas seria razoável considerar que todo o termo \(r_0\lambda\) é associado à fertilidade, enquanto \(-\alpha N\) se refere à mortalidade, ambos \textit{per capita}.

    É interessante observar também que, qualquer que seja o método de interação entre indivíduos, ele deve depender essencialmente de interações binárias (muito mais prováveis do que interações tríplices, isto é, de três ao mesmo tempo, ou de quatro, e etc). Supondo que todas as interações binárias sejam igualmente possíveis, sem preferencias, e com resultados iguais, é razoável supor que ocorram a uma taxa proporcional ao números de pares possíveis, ou seja, da ordem do quadrado do número de indivíduos da população.

\begin{quotation}
    A questão de exploração simultânea de recursos comuns por várias populações tem importância no planejamento econômico e social de recursos naturais. Este problema foi discutido e várias de suas consequências analisadas pelo ecólogo Garrett Hardin em um histórico e influente artigo ``\textit{The Tragedy of Commons}'' Science 1968 que é leitura  obrigatória em Ecologia. Consulte também Simon Levin - A Fragile Dominion: Complexity and Commons, Perseus 1998 e artigos (em geral não matemáticos) da publicação aberta (online) Ecology \& Society. O problema da pesca em alto mar foi analisado também com base nestes argumentos por Donald Ludwig, cujo trabalho influenciou na convenção de um tratado mundial de pesca da baleia em 1976. (Ref. C. Clark - Mathematical Bioeconomics, 1990, D.Ludwig - artigos). Este tema envolve naturalmente a necessidade de considerar comportamentos sociais que determinam as decisões sobre a ação dos indivíduos que são agentes da exploração de recursos, assim como seus aspectos econômicos o que torna o seu estudo importante, multidisciplinar e progressivamente mais complexo.Para uma referencia inicial sobre o aspecto de psicologia social envolvido nesta importante questão, consulte: ....S. Levine-J. Tennenbaum \& al. - The logic of Universalization guides moral judgement - PNAS, 20Oct. 2020.).
\end{quotation}

%Exercícios:

\begin{exercise}
    Leia o artigo de Garret Hardin e algumas referências posteriores sobre o tema ``\textit{Tragedy of Commons}''. Consulte o texto de Simon Levin e visite a página deste importante autor. Consulte também o artigo de Luwig - Walter e Holling no primeiro número da revista (online - acesso livre) Ecology and Society de 1997. Faça um resumo comentado de aproximadamente 20 linhas sobre o tema.
\end{exercise}


    \begin{exercise}
    Considere \(n\) populações com tamanhos \(N_k\) que subsistem com a exploração peculiar de um portfólio de recursos com taxas de consumo basal distintas. Adapte o argumento acima para obter um sistema de \(n\) equações diferenciais para as funções \(N_k(t)\) que represente este Modelo de Malthus-Verhulst com Saturação. (Sistemas deste tipo tem sido empregados por Martin Nowak para a descrição de competições e evolução de populações. Ref: M. Nowak - Evolutionary Dynamics, Harvard UP 2010).
    \end{exercise}

    \begin{exercise}
    Obtenha uma solução explícita em termos de funções elementares para a equação diferencial de Riccatti \(\dfrac{du}{dt} = au + bu^2\) dividindo por \(u^2\) e escrevendo uma equação linear para \(v = \frac{1}{u}\) que pode ser resolvida explicitamente.
    \end{exercise}

    \begin{exercise}
    Escreva o Modelo de Malthus-Verhulst na forma adimensional e explique o resultado. 
    \end{exercise}

    \begin{exercise}
    Considere um conjunto de indivíduos \(P_0\) ``fundadores'' de uma população no instante \(t=0\) cuja dinâmica é regida por um Modelo de Malthus-Verhulst. Considere que para a dinâmica da população resultante o termo \(-\alpha N\) designa a mortalidade per capita da população. Determine o tempo médio de sobrevivência da população fundadora desta dinâmica. (\textbf{Sugestão}: Considere a subpopulação \(P(t)\) dos ``fundadores'', \(P(0) = P_0\) e verifique que \(\frac{dP}{dt} = -\alpha NP\). Refaça o argumento utilizado para determinar o tempo médio de sobrevivência do Modelo de Malthus para esta subpopulação, mas observe que para isto será necessário dispor da função \(N(t)\) que estabelece a população total - deles e de seus descendentes - em que eles ``vivem'').
    \end{exercise}

    \begin{exercise}
    \begin{description}
    \item (a) Faça um esboço geométrico para a dinâmica da equação que descreve o Modelo de Malthus-Verhulst interpretando \(N(t)\) como uma trajetória que se move com velocidade \(\frac{dN}{dt}\) sobre a reta coordenada horizontal (chamada reta de fase) segundo um campo de velocidades pré-estabelecido sobre a reta (chamada espaço de fase). Este campo de velocidades é determinado pelo gráfico cartesiano da função \[F(N) = r_0\left(1-\dfrac{N}{K}\right)N\]
    representado acima da reta de fase e a velocidade do ponto ocupando a posição \(N\) é dada pelo valor de \(F(N)\): Para a direita se positivo e para a esquerda se for negativo com intensidades respectivas. Observe que há duas posições estacionárias (isto é, posições onde a velocidade é nula) e que uma delas (a origem) é repulsiva (isto é, as velocidades em posições próximas tendem a afastar as trajetórias da origem, dita instável) e a outra (\(N = k\)) é atrativa (i.e., as velocidades em posições próximas tendem a dirigir as trajetórias de volta à posição estacionária original, dita estável).
    \item (b) Com base nesta dinâmica na reta de fase faça um esboço dos gráficos cartesiano das curvas \(N(t) ( (t, N(t)) )\) observando que o formato da curva depende do ponto de partida \(N(0) = 0\). Determine quais delas são ``logísticas'' isto é, tem forma de ``S'' enquanto que as outras tem forma de ``C''. Se uma população com esta dinâmica inicia com \(N(0) = N_0 < K\) determine o tempo \(t(N)\) que ela leva para atingir os valores \(N > N_0\). Mostre que ela leva um tempo infinito para atingir \(N = K\) e, portanto, jamais atingirá um valor \(N > K\). Explique!
    \item Analise uma dinâmica na reta determinada por um campo de velocidades da forma \[\dfrac{dn}{dt} = \cos(n) \sqrt[4]{1-\cos(n)},\]
    determine as posições estacionárias atrativas e repulsivas e mostre que o tempo de percurso até a origem de uma trajetória que se inicia em seu campo de atração é finito.
    \end{description}
    \end{exercise}

    \begin{exercise}
    \begin{description}
    \item (a) Mostre que a dinâmica ``Caricatura de Malthus''
    \[\dfrac{dN}{dt} =  F(N),\]
    onde o gráfico cartesiano da função \[F(N)\] consiste de uma reta \(y = r_0 N\) para \(0 le N \le K\) e outra reta \(y =-\lambda N + (r_0 + \lambda) K\), para \(N \ge K\), também produz uma dinâmica de saturação como o Modelo de Malthus, mas é linear por pedaços, o que pode ser uma vantagem na hora de obter uma solução explicita, mesmo que por pedaços. Analise esta questão. (Ideia inventada por Joseph B. Keller e seu aluno John Rinzel para analisar ondas em sistemas neurais na década de 1970).
    \item (b) Analise o modelo ainda mais caricatural em que \(F(N) = a > 0\), para \(N < K\) e \(F(N) = b < 0\), para \(N > K\) e \(F(K) = 0\). Mostre que este modelo exibe um efeito de saturação e, portanto, é da classe de Malthus-Verhulst, mas que não é mais parcimonioso do que o modelo quadrático comparando o número de parâmetros necessários para defini-los adimensionalmente.
    \end{description}
    \end{exercise}

    \begin{exercise}
    Considere uma população cuja mortalidade se dá na forma (não Malthusiana) chamada dinâmica de Monod-Holling II:
    \[\dfrac{dN}{dt} = -\mu \dfrac{N}{A+N}, N(0) = N_0.\]
    (Segundo o bioquímico Jacques L. Monod (1910-1976) e o ecólogo Crawford S. Holling (1920-2019)). Mostre que o tempo de vida média dos indivíduos desta população depende do valor inicial \(N_0\) o que significa uma interferência mútua entre eles. Observe que a taxa de mortalidade per capita deste modelo, \(\frac{1}{N} \frac{dN}{dt}\) é decrescente com a densidade populacional, ao contrário do modelo Malthusiano. O que se pode concluir deste fato que, neste caso, a interação entre seus indivíduos, é (em suma) prejudicial.
    \end{exercise}

    %\begin{quotation}
    Aspectos negativos da interação foram enfatizados por entusiastas da Teoria Evolutiva e identificados como um processo competitivo de Seleção Natural em que os ``mais fortes'' prevaleceriam populacionalmente com a extinção dos ``mais fracos''. Entretanto, aspectos cooperativos que favorecem o desenvolvimento de uma população como um todo foram apontados ideologicamente e ecologicamente pelo príncipe russo revolucionário Peter Kropotkin (``Mutual Aid - A Factor in Evolution Theory'', 1902) e desenvolvidos, também ideologicamente, pelo ecólogo quaker W. C. Allee (1885-1955) em ``Cooperation among Animals, with Human Implications'' (1951) e vários outros durante o século XX. O chamado Efeito Allee será tema de uma das seções do Capítulo III (Princípios de Interação) e o fenômeno de Cooperação será especificamente tratado nos Capítulos Fins sobre Matemática Biológica, Morfogênese, ``Quorum Sensing'' e Dinâmica Coletiva. (Th Seeley, D. Sumpter, D. Gordon, L. Dugatkin).
    %\end{quotation}

\section{O Modelo de Compartilhamento de Recursos: A Tragédia dos Comuns e a Tragédia das Revisões}

    Conforme o direito britânico e seus descendentes, qualquer recurso que é disponível livremente a todos indivíduos de uma comunidade (como, por exemplo, o ar que se respira, a água de rios e fontes, raios solares, peixes do oceano, e etc) são considerados ``Recursos/Bens Comuns'' (``Commons''). Suponhamos que exista um Recurso orgânico que se reproduza ``aritmeticamente'' proporcionalmente ao fornecimento constante de energia solar na Terra.

    Está claro que se a (taxa de) retirada total do Recurso pelas duas populações se mantiver abaixo da reposição do mesmo, haverá sempre disponibilidade para todos os co-participantes. A regulamentação legal de um teto de taxa de extração para cada co-participante de um Recurso de forma que este não se extinga é, em geral, resultado de longas negociações e tratados. (A pesca da Baleia é um dos exemplos típicos. v. S. Levin.).

    Entretanto, imagine que uma das populações participantes do tratado desconfie que a outra esteja burlando o acordo e, portanto, decida também fazer o mesmo para auferir a maior quantidade de recursos antes que eles se extingam. Esta atitude provocará um desequilíbrio que certamente levará à extinção do Recurso e se ele for o único das populações envolvidas, todas elas serão também extintas. A ocorrência deste cenário foi denominada ``Tragédia dos Comuns'' pelo ecólogo Garrett Hardin em um artigo na revista Science em 1967 que ganhou uma enorme notoriedade e deu origem a vários trabalhos que objetivam organizar a administração de recursos comuns. Dentre estes, destaca-se o trabalho de Elinor Ostrom (1933-2012) a quem foi atribuido o premio Nobel de Economia em 2009. (E. Ostrom- Governing the Commons - The evolution of institutions for collective action, Cambridge Univ.Press 1990). 

%Exercício: 

\begin{exercise}
    Analise a questão de sobrevivência de duas populações que utilizam um único Recurso Comum. Se uma das populações burlar consistentemente um ``acordo"de compartilhamento, analise os cenários que levam à extinção populacional de ambas.
\end{exercise}

    Nesta era de revisões históricas o argumento irrefutável de Hardin foi imerso em outras considerações alheias ao tema ecológico que intencionavam uma tentativa de ``cancelamento intelectual'' do já falecido autor. A acusação que lhe é imputada se refere ao fato de que uma das soluções óbvias para o problema dos Comuns, consiste na possibilidade (ou quase certeza) de que em um acordo sobre exploração de Recursos Comuns pares com maior força política excluiriam o acesso de algumas populações mais fracas aos recursos em disputa. Esta é uma questão politica e econômica importante, mas que não faz parte do problema ecológico específico e a sua ocorrência não pode ser imputada a quem a descobre ou analisa. A solução ideal da questão ampla não existe, mas o assunto pode ser melhor discutido desde que o fenômeno sócio-ecológico seja bem entendido.

    Uma referência particularmente interessante que trata da questão dos Recursos Comuns sob o ponto de vista da Psicologia Social de uma maneira científica é o recente e interessante artigo: S. Levine - J. Tennenbaum \& al - The logic of Universalization guides moral judgement, Proc.Nat.Acad.Sci. USA, 10 Oct 2020.

    Esta questão sob o ponto de vista da Teoria Dinâmica dos Jogos tem sido amplamente estudada pela escola vienense de Karl Sigmund, Martin Nowak e outros (M. Nowak - Evolutionary Dynamics, Harvard UP, K. Sigmund - The Calculus of Selfishness, Princeton UP e Herbert Ginits - Game Theory Evolving, Princeton UP 2000). 
. 
\section{VARIAÇÕES: HETEROGENEIDADE por MATURIDADE RETARDADA - Modelo Discreto de Fibonacci-Euler}

    A apresentação do modelo de Fibonacci como um mero ``Quebra-Cabeça'' referindo-se pitorescamente a uma população de coelhos, e não como um sisudo ``Modelo Matemático'', certamente foi uma das razões para que os importantes argumentos utilizados na sua formulação passassem desapercebidos durante cinco séculos. Ainda hoje, este problema é comumente apresentado em textos elementares apenas como uma inócua curiosidade de almanaque o que induz o/a leitor/a a um descaso infeliz da questão que elimina uma excelente oportunidade para a compreensão de alguns princípios fundamentais de dinâmica populacional.

    Somente em 1748/60, Leonhard Euler (descobre e) generaliza os argumentos de Fibonacci introduzindo o conceito de ``faixas etárias'' que refina naturalmente o conceito de maturidade para diversos graus e os representa como estados biológicos (idade) dos indivíduos do qual dependerão, não apenas a sua fertilidade, mas também a susceptibilidade à morte. Com isto, Euler propôs um Modelo Demográfico que ainda hoje é amplamente empregado.(Caswell [2000]). O Modelo de Euler, assim como o Modelo de Fibonacci, representam o tempo discretamente, o que é sugerido pelo fato de que o censo de populações humanas (ou ecológica) é sempre registrado em intervalos regulares de tempo. O modelo de Euler é amplamente empregado com múltiplas variações nos textos de Caswell [2000] e Keyfitz.).

%Exercícios: Extensões do Modelo de Fibonacci

    \begin{exercise}
    Considere a seguinte modificação natural do Modelo de Fibonacci: A população é constituída de ``casais'' (o que pode ser interpretado como o número de fêmeas da população supondo-se que haja um equilíbrio entre as populações dos dois sexos regulamentado de alguma forma pela biologia reprodutiva da espécie). Considere que em cada período entre dois ``censos'', \(k\) e \(k + 1\), a reprodução obedeça a uma taxa \(\alpha\), isto é, se \(M\) for o número de casais maduro no censo \(k\) então eles produzirão \(\alpha M(k)\) casais no censo \(k + 1\). Suponha também que ocorra uma mortalidade que será a uma taxa de \(\mu_1\) para os imaturos e \(\mu_2\) para os casais maduros. Por exemplo, se \(M(k)\) for o número de casais maduros no censo \(k\), então, a proporção destes que sobreviverão para o próximo censo \(k + 1\) será de \((1 - \mu_2)\). 
    \begin{description}
    \item (a) Escreva, argumentando, um Modelo recursivo para esta população
    \item (b) Escreva a solução desta recursão em termos de funções elementares. (Sugestão: Estude a seção no texto Bassanezi-Ferreira 1988 - relativo a equações de recursão e o Método Operacional para resolve-las explicitamente)
    \item (c) Analise a possibilidade de crescimento exponencial desta população, ou de extinção (isto é, de \(\displaystyle\lim_{k \to \infty} P(k) = 0)\).
    \end{description}
    \end{exercise}

    \begin{exercise}
    Modelo de Plantas anuais. Considere uma espécie de plantas que a cada ano \(k\) tem \(P(k)\) exemplares durante a primavera. Estas plantas produzirão durante o verão seguinte \(s P(k)\) sementes que serão lançadas ao solo durante o outono, quando as plantas secarão. Estas sementes lançadas hibernarão durante o inverno e apenas uma fração \(f_1\) delas sobreviverá e destas apenas uma fração \(g_1\) germinará produzindo plantas na próxima primavera no ano \(k + 1\). (Neste caso, as sementes que não germinaram morrem).

    \begin{description}
    \item (a) Escreva um modelo para a dinâmica da população P(k) destas plantas na primavera do ano \(k\).
    \item (b) Generalize o Modelo para sementes que tem a capacidade de hibernar dois invernos seguidos (com mortalidades específicas para o primeiro e segundo inverno) para germinarem apenas no segundo ano seguinte (com taxa especifica). Neste caso, dentre as sementes que não germinaram no primeiro inverno, algumas morrem e outras hibernarão no próximo inverno.
    \item (c) Escreva um modelo com muitas hibernações.
    \textbf{Observação}: A hibernação alongada (isto é, a não germinação de todas as sementes na primeira primavera seguinte) é uma ``estratégia'' de sobrevivência para épocas de estiagem que certamente ocorrerão durante alguns anos e que poderiam levar à extinção da espécie que não ``guardasse'' sementes para o próximo ano.
    \end{description}
    \end{exercise}
    
    \begin{exercise}
    Considere diversos ``graus de maturidade'' de uma população de ``casais'' generalizando o Modelo de Fibonacci onde \(P(k)\) é esta população. Neste caso considere, digamos \(100\) ``faixas etárias'' de maturação cada uma com duração de um ano e com taxas (``frações'') específicas de mortalidade e reprodutibilidade.
    
    \item (a) Escreva um Modelo recursivo para a dinâmica desta População que, a proposito é semelhante ao Modelo de Euler de 1760.
    \item (b) Escreva a solução elementar desta recursão em termos de funções elementares.(Considere, ``ingenuamente'', que a obtenção de raízes de polinômios seja uma tarefa elementar e simples!).
    \item (c) Convencido/a de que a obtenção dos valores de raízes de polinômios de grau superior não é tarefa simples, reduza suas ambições e Analise teoricamente apenas a possibilidade de explosão populacional e de extinção em termos matemáticos com a informação da localização destas raízes com relação ao disco unitário no plano complexo. (Consulte Bassanezi-Ferreira a respeito e o Método de Mikhailov).
    \item (d) É historicamente interessante registrar que, embora o artigo de Euler tenha sido publicado nos Anais da Academia de Ciências de Berlim e precedido à publicação do livro de Malthus por 40 anos, o seu impacto foi quase nulo, tanto que o próprio Darwin, somente faz referência rápida a este trabalho em um ligeiro comentário na pg. 428 da 5a. edição do ``\textit{Origin of Species}''. Darwin revisou infatigavelmente o seu famoso texto inúmeras vezes desde sua publicação inicial em 1859 até a sexta edição de 1876 e não sendo muito afeito à Matemática (isto, por palavras dele próprio!- [autobiografia]) é possível que a referência ao artigo de Euler tenha sido resultado de alguma de suas muitas correspondências, alguma delas, talvez originárias de matemáticos. Esta é uma boa questão histórica para ser resolvida analisando o enorme arquivo de correspondência de Darwin. (Darwin[...]). Verifique a correção desta afirmação histórica com subsídios a favor ou contra ela.
    \item (e) É de se ressaltar que àquela época o conceito (pelo menos implícito) de Modelo Matemático que descreve hipóteses e conclusões em linguagem matemática era já bem conhecido como consequência das bem sucedidas e inúmeras aplicações do Cálculo de Newton e Leibniz, notadamente à Mecânica (Celeste ou Terrestre), em cuja arte Leonhard Euler foi um mestre insuperável. Assim, não seria por falta de bons exemplos a imitar que Malthus deixou de formular um Modelo mais preciso para a sua teoria. De qualquer forma, desde o princípio do século XIX o crescimento geométrico, ou exponencial, de uma população tornou-se sinônimo de um processo ``Malthusiano''.
    \end{exercise}

\section{VARIAÇÕES- Heterogeneidade- Maturidade Retardada em Modelo Contínuo de Lotka}

    Como já foi discutido, uma das dificuldades para aplicação do Modelo de Malthus a certas populações consiste no fato de que a sua expressão diferencial
    \[\dfrac{dP}{dt}(t) = (\nu - \mu) P(t)\]
    estabelece que a taxa total de nascimentos no instante \(t\), ou seja, \(\nu P(t)\), é considerada como proveniente de todos os indivíduos presentes na população neste exato momento. Entretanto, em algumas populações, um processo de maturidade razoavelmente longo deve ser levado em conta, de tal forma que um indivíduo recém nascido somente ingressará na população fértil após um período de tempo. Digamos que este período seja representado por \(T > 0\). Então, a taxa de natalidade no instante \(t\) deve se referir à população que já estava presente no instante \(t-T\) e que não pereceu durante o período \([t-T, t]\), ou seja, \(e^{-\mu T} P(t-T)\). Por outro lado, a mortalidade (malthusiana) no instante \(t\) depende exatamente de quantos indivíduos existem no instante \(t\) (``para morrer, basta estar vivo''). Levando em consideração estes argumentos, podemos re-escrever o Modelo Malthusiano com Maturação da seguinte maneira:
    \begin{eqnarray}
    \dfrac{dP}{dt}(t) &=& \mu P(t) + \nu e^{-\mu T} P(t-T) \nonumber \\ &=& -\mu P(t) + \nu^\ast P(t-T).
    \end{eqnarray}
    Esta equação, sob o ponto de vista simbólico, é apenas uma pequena variação da equação original, o nos induz a crer que uma solução explícita dela seja também facilmente obtida em termos de uma função elementar. Infelizmente, tal conclusão está muito longe de ser correta. De fato, é possível prever a dificuldade inerente a este Modelo com um argumento matemático formal muito simples. Observemos que a função incógnita \(P(t)\) no modelo de Malthus é submetida a uma operação funcional linear obtida da soma de duas operações lineares: derivação de primeira ordem \(\frac{d}{dt}\) e o produto por um número
    \[(-k): L = \left(\frac{d}{dt}\right) + (-k).\]
    Esta simples equação de Malthus original,
    \[\left(\frac{d}{dt}-k\right) N = 0\]
    é, por isso, denominada de equação diferencial ordinária de primeira ordem.

    Por outro lado, no Modelo Malthusiano com Maturidade a função incógnita é submetida a uma operação linear obtida da soma de três operações lineares: derivação de primeira ordem \(\frac{d}{dt}\), produto por um número \(\mu\) e uma ``ingênua'' operação de ``\textbf{deslocamento}'' \(E\). Esta operação \(E\) é definida operacionalmente da seguinte maneira: aplicada a uma função qualquer \(f\) e calculada em \(t\) é: \((Ef)(t) = f(t-T)\). Apesar de sua simples aparência, ela não é tão inócua neste contexto. E, para avaliarmos isto, basta lembrarmos que o Teorema de Expansão de Taylor nos dá a seguinte expressão:
    \begin{eqnarray}
    (Ef)(t) &=& f(t-T) \nonumber \\
    &=& \displaystyle\sum_{k=0}^{\infty} \left(\dfrac{-T^k}{k!}\right) \left(\dfrac{d}{dt}\right)^k f(t)
    \end{eqnarray} 
    o que, para nosso espanto é uma operação diferencial de ordem infinita! 
    
    Equações em que a função incógnita comparece calculada em intervalos de tempos distintos (mas fixos) são chamadas na literatura em inglês de ``Equações com Retardamento'' ou, sob influência da terminologia russa de ``Equações com desvio de argumento''. (Ref. MacDonald, Erneux, Nelson) e serão tratadas em outros contextos com maior detalhe. 

    Portanto, a Equação Malthusiana com Maturidade é uma equação diferencial ordinária de ordem INFINITA, e não poderá ser tratada matematicamente com apenas uma ligeira modificação da teoria empregada no estudo da equação diferencial original, embora resulte de uma pequena variação do Modelo de Malthus. Este fato ilustra de maneira exemplar que a aplicação do princípio de Occam é indispensável para o exercício relevante da Matemática Aplicada.

    O estudo da equação diferencial com retardamento será feito no capítulo sobre Modelos Fisiológicos, quando este fenômeno tem presença quase inevitável e produz efeitos surpreendentes tais como oscilações e comportamento caótico, impensáveis em um simples modelo Malthusiano. 

\section{VARIAÇÕES: Heterogeneidade Paralela Não Acoplada: Modelos Efetivos \& Média Harmônica}

    Em praticamente todas as situações, a amostra de material radioativo analisada quanto ao seu decaimento não é formada de uma substância pura, mas consiste de uma mistura de substancias que decaem de maneira ligeiramente distinta tanto temporal quanto ao subproduto. (Este fato é análogo a uma possível variação de mortalidade em subpopulações de uma população biológica). É natural então indagar sobre o efeito que tais variações podem causar quando consideramos toda a amostra como uma única substancia. Podemos identificar esta classe de heterogeneidade designando-a paralela e independente, pois cada substancia procede como se fosse isolada das outras.
    
    É claro que se soubermos a composição das ``impurezas'' da amostra o problema se resume em simplesmente calcular o resultado para cada uma delas. Entretanto, na maioria dos casos é conhecida apenas a existência e uma avaliação quantitativa das impurezas e o problema consiste em fazer o melhor possível com esta informação parcial com relação à determinação de seu efeito no processo dinâmico. A influencia de ``pequenas'' variações de parâmetros constitutivos no resultado final de um Modelo Matemático é tratada no Capítulo sobre Métodos Assintóticos e Princípios Probabilísticos em particular a Análise de Sensitividade que se refere à uma estimativa de erro relativo da medida de interesse com respeito à variação do parâmetro constitutivo. Por exemplo, considerando-se o tempo médio de sobrevivência \(T(\mu)\) como sendo a medida de interesse sobre o Modelo e o parâmetro constitutivo de controle \(\mu\), a sensitividade relativa de \(T\) com relação a pequenas variações de \(\mu\) é determinada pela aproximação linear representado no Cálculo pela derivada da seguinte maneira:
    \begin{eqnarray}
    \Delta T(\mu_0) &=& T(\mu_0+\delta)-T(\mu_0) \nonumber \\
    &\sim& \dfrac{1}{T(\mu_0)} \dfrac{\partial T}{\partial \mu}(\mu_0) \delta,
    \end{eqnarray}
    o que neste caso, é dada por:
    \begin{eqnarray}
    \dfrac{1}{\dfrac{1}{\mu_0}} \dfrac{-1}{\mu_0^2} \delta &=& \dfrac{1}{\mu_0} \delta.
    \end{eqnarray}
    Este resultado demonstra uma maior sensitividade com respeito a pequenas variações em coeficientes 0 pequenos. Em outros casos interessa-nos analisar a sensitividade logaritmizada, em que tanto a medida de interesse quanto a medida do parâmetro é feita na escala logarítmica. No exemplo acima, ela seria definida pela expressão:
    \begin{eqnarray}
    \dfrac{\partial \ln(T)}{\partial \ln(\mu)} = -1.
    \end{eqnarray}
    (H. Caswell - Matrix Population Models, Sinauer 2020).

    Entretanto, interessa-nos, em particular e muito mais, determinar um novo Modelo simplificado, denominado ``\textbf{Modelo Matemático Efetivo}'' que tenha características análogas ao Modelo original, mas cujos parâmetros constituídos sejam constantes e calculados como ``médias'' dos valores do parâmetro constitutivo variável e cuja solução represente também uma boa aproximação da solução do Modelo original. Os Métodos Matemáticos especificamente adequados para a representação do efeito de pequenas heterogeneidades (conhecidas apenas quanto às suas médias) por intermédio do conceito de Modelo Efetivo são denominados ``\textbf{Métodos de Homogeneização}'' e constarão como temas importantes do Capítulo sobre Métodos Assintóticos. A construção de Modelos reduzidos no sentido ``Efetivo'' é tema importante do capítulo ``Princípios (gerais) de Redução''. Por enquanto, o assunto será introduzido com exemplos simples nos exercícios abaixo.

%Exercícios: 

    \begin{exercise}
    Considere uma População (não homogênea segundo a dinâmica Malthusiana), com \(N(t)\) indivíduos no instante \(t\) constituída de subpopulações homogêneas, \(N_k(t)\), \(N(t) = \sum_{k} N_k(t)\), cujas dinâmicas são descritas pelos Modelos Malthusianos:
    \[\dfrac{dN_k}{dt} = -\mu_k N_k.\]
    Mostre que o tempo de vida médio da população total é
    \[\tau = \sum \tau_k = \sum \dfrac{N_k(0)}{N(0)}\dfrac{1}{\mu_k}.\]
    Conclua que uma ``homogeneização'' (este é o termo técnico utilizado para designar estes procedimentos) da população total segundo o Modelo Malthusiano deve utilizar a constante de Malthus
    \[\mu_{eff} = \left(\dfrac{N_k(0)}{N(0)}\dfrac{1}{\mu_k}\right)^{-1}\]
    obtida pela \textbf{média harmônica} das constantes das subpopulações. O modelo Malthusiano
    \[\dfrac{dN_{eff}}{dt} = -\mu_{eff} N_{eff},\]
    é chamado ``Modelo Efetivo''.
    \end{exercise}

    \begin{exercise}
    Analise a aproximação das soluções do modelo efetivo com respeito à sua solução exata:
    \[N(t) = \sum N_k(0) e^{-\mu_k t}\]
    e as condições para que seja uma boa aproximação. 
    \end{exercise}
    
    \begin{exercise}\(\ast\)
    Considere uma população contínua que seja distribuída continuamente segundo uma característica \(x\) na forma \(N(t, x)\), onde o ``Número de indivíduos com característica \(x_1 \le x \le x_2\) no instante \(t\)'' é dado por:
    \[\int_{x_1}^{x_2} N(t, x)  dx\]
    e suponha que a mortalidade Malthusiana varie na forma \(\mu(\epsilon x)\) lentamente (isto é, \(\left|\dfrac{d\mu(s)}{ds}\right| \approx 1\) e \(0 < \epsilon << 1\) (bem pequeno). Obtenha o Modelo Efetivo para esta população.
    \end{exercise}

\section{VARIAÇÕES: Heterogeneidade - Cadeias Sequencialmente Acopladas de Decaimento}

    A heterogeneidade no problema de decaimento é uma questão inevitável no estudo de transmutação nuclear quando sabemos que toda substância se transmuta em outra substancia que tem caraterísticas distintas da anterior e daí por diante, como uma cadeia. Portanto, para analisar a dinâmica de decaimento de uma amostra inicial, mesmo que pura, será necessário considerar que ao longo do tempo, várias outras substâncias em série estarão presentes e, portanto, o problema é de fato heterogêneo.

    Em muitos casos a transmutação que nos interessa é apenas aquela referente a uma substância inicial e outra final que é produzida não diretamente, mas após vários estágios intermediários muito rápidos, cujos detalhes não nos interessam conhecer. Assim, é natural procurar por um modelo simplificado que ``homogenize'' a etapa intermediaria no sentido de que o processo entre as etapas inicial e final possa ser ``aproximadamente'' descrito com um ``\textbf{Modelo Efetivo}'' que se refira, simplificadamente, apenas a estes dois estágios de interesse. O Método matemático que produz este Modelo Efetivo é usualmente denominado de ``\textbf{Homogeneização}'' e tem por objetivo reduzir o modelo com uma perda de acuracidade em troca de um enorme ganho de ``realismo'' porque nem sempre os estágios intermediários são completamente conhecidos e/ou de interesse. 

    O decaimento espontâneo (i.e., uma transformação sem causa aparente, mas com resultado previsível) é um dos fenômenos mais universais e descreve a dinâmica de todas as estruturas, físicas, químicas e biológicas. Até mesmo os prótons, uma das estruturas mais estáveis do Universo, que tem meia vida de 1030 anos, também decaem! (E, é bom lembrar que a idade da terra é estimada em \(5 \cdot 10^9\) anos!) (H. Fritzsch).

    Reações químicas em que uma molécula ``espontaneamente'' se transforma em uma ou mais moléculas de diferentes espécies são denominadas ``reações autocatalíticas'' e podem ser representadas por modelos matemáticos semelhantes ao do decaimento radioativo, isto é Malthusiano. (Observemos que em reações químicas ocorrem apenas modificações das associações de átomos que são preservados no processo, ao contrario das reações nucleares em que os átomos são transmutados). (Bialek, Nelson, Phillips).

    Consideremos então um modelo que contemple toda uma série de transmutações (ou de reações autocatalíticas) representado esquematicamente na forma
    \[\ldots \substack{\longrightarrow}
    A_{-1} \substack{\mu_{-1} \\ \longrightarrow}
    A_{0} \substack{\mu_{0} \\ \longrightarrow}
    A_{1} \substack{\mu_{1} \\ \longrightarrow}
    \ldots
    A_{N} \substack{\mu_{N} \\ \longrightarrow}
    \ldots
    \]
    e descrito pelo seguinte sistema de equações diferencias (malthusianas) acopladas:
    \begin{eqnarray}\label{eq:sistedmalthusianasacopladas}
    \dfrac{dA_k}{dt} &=& \mu_{k-1} A_{k-1} - \mu_k A_k.
    \end{eqnarray} 

    De acordo com a interpretação do modelo Malthusiano, os parâmetros \(\frac{1}{\mu_k}\) indicam o tempo médio de permanência de átomos (moléculas) na categoria \(A_k\). Escrevendo o vetor de dimensão infinita
    \[X = (\ldots, A_{-1}, A_0, A_1, \ldots, A_k, A_{k+1}, \ldots)^t\]
    o sistema (infinito) de equações diferenciais acima pode ser sucintamente descrito na forma:
    \[\dfrac{dX}{dt} = M X\]
    onde \(M\) é uma matriz infinita, bidiagonal. 

    A rigor este sistema é infinito, mas se o tempo de reação de algumas das etapas, \(\frac{1}{\mu_{0}}\) e \(\frac{1}{\mu_{N+1}}\), forem incomparavelmente maiores do que outras intermediárias, \((\frac{1}{\mu_{k}}, 1 \le k \le N\) podemos escolher observar o fenômeno em uma escala de tempo pequena comparada com \(\frac{1}{\mu_{0}}\) e \(\frac{1}{\mu_{N+1}}\) quando as duas extremidades permanecerão praticamente estáveis. Com isto, o sistema \ref{eq:sistedmalthusianasacopladas} deverá se referir apenas aos termos \(1 < k < N\), suplementados pelas seguintes equações das extremidades:
    \begin{eqnarray}
    \dfrac{dA_1}{dt} &=& \mu_1 A_1 \\
    \dfrac{dA_N}{dt} &=& \mu_{N+1} A_{N+1}
    \end{eqnarray}
    e a equação matricial \(\dfrac{dX}{dt} = \overline{M} X\) se referirá a um vetor em \(\mathbb{R}^N\) e a matriz \(M\) será bidiagonal de ordem \(N \times N\).

%Exercícios:

    \begin{exercise}
    \begin{description}
    \item (a) Descreva as Matrizes \(M\) e \(\overline{M}\) para sistemas de reações autocatalíticas, respectivamente, dos casos finito e infinito. 
    \item (b) Considere um processo de decaimento radioativo de uma amostra de Carbono 14. O \(C_{14}\) é um isótopo instável com um excesso de dois nêutrons no núcleo com relação ao Carbono 12, \(C_{12}\), que tem seis prótons e seis nêutrons e que é nuclearmente muito mais estável e mais comum. Apesar desta diferença nuclear, eles são quimicamente idênticos e, portanto, são utilizados sem distinção em processos bioquímicos de organismos vivos. O elemento carbono é um dos constituintes fundamentais no metabolismo dos organismos, junto com hidrogênio, oxigênio, fósforo, e outros. O \(C_{14}\) da biosfera (Terra e sua atmosfera), é produzido a uma taxa que pode ser considerada constante como resultado do bombardeio de raios cósmicos sobre os átomos de nitrogênio abundantes na alta atmosfera. Uma amostra típica de \(C_{14}\), um mol, \((14g)\), contém uma população em torno de \(10^{25}\) átomos. O processo de decaimento de um átomo \(C_{14}\) consiste em transmutar-se espontaneamente, ``sem essa nem aquela'', de volta ao átomo de nitrogênio, ou seja, de maneira totalmente ``aleatória'', o que significa ser imprevisível determinar (prática e mesmo teoricamente) quando isto ocorrerá com um átomo individual. Entretanto, observa-se que em uma grande população deles o processo coletivo se dá tal de acordo com o Modelo de Rutherford, ou seja, Malthusianamente: Se \(C(t)\) for a população de Carbono 14 no instante \(t\), então,
    \[\dfrac{dC}{dt} = -\mu C,\]
    onde \(k\) é a constante de decaimento.
    \end{description}
    \end{exercise}

    \begin{exercise}
    Em Físico-Química a tradição histórica ainda domina, razão porque é mais comum utilizar o conceito de ``meia-vida'', \(T_{1/2}\), para caracterizar esta dinâmica, e que é definido da seguinte maneira: ``A Meia vida do C14 é o tempo (característico desta substancia) necessário para que uma amostra (qualquer) dela se reduza à sua metade por conta do decaimento radioativo''. 
    
    \begin{description}
    \item (a) Mostre que esta é uma boa definição, ou seja, que não depende do tamanho da amostra, e relacione o parâmetro ``meia vida'' \(T_{1/2}\) de \(C_{14}\) com a sua ``vida média'' \(\mu^{-1}\).

    \item (b) Obtenha a ``meia vida'' do \(C_{14}\) na literatura e determine a sua constante \(\mu\) de decaimento e a sua vida média em horas.
    
    \item (c) Considere a dinâmica do sistema de Carbono \(14\) total da biosfera \(C(t)\), produzido a uma taxa constante pelo processo já descrito e continuamente em decaimento. Mostre que esta dinâmica pode ser descrita na forma
    \[\dfrac{dC}{dt} = p - \mu C,\]
    e mostre que um equilíbrio é atingido.
    
    \item (d) Com base neste equilíbrio, (que depois de milhões de anos de `vida' da Terra, para todos os efeitos já deve ter sido atingido!), cujo valor você pode, e deve, descobrir qual seja na literatura, determine a sua taxa de produção \(p\) na alta atmosfera.
    
    Neste equilíbrio participam também todos os seres vivos, já que estão em contínuo intercâmbio com a matéria orgânica do ambiente. Após a morte de um organismo, este intercambio é interrompido, enquanto a concentração de \(C_{14}\) em seus restos mortais continua decaindo. Portanto, analisando a concentração \(c_0\) de \(C_{14}\) de restos arqueológicos de matéria orgânica, pode-se avaliar o tempo \(T\) decorrido desde sua morte. 

    \item (e) Descreva como medir o tempo com a concentração \(c_0\), ou seja, obtenha a função \(T(c_0)\).
    
    É necessário ressaltar que o conceito de ``tempo médio de vida, ou de sobrevivência, ou de permanência'' dos indivíduos em uma população em extinção \(\mu^{-1}\), é diferente do conceito de ``tempo de meia-vida'' , este último utilizado no modelo de decaimento radioativo, cujo significado é: ``período de tempo necessário para que ocorra o decaimento da metade da `população'''. Relacione os dois conceitos.
    \end{description}
    \end{exercise}
 
\section{VARIAÇÕES: Heterogeneidade - Compartimentos Multiplamente Acoplados - A Face Primeira da Difusão}

    Processos de Difusão são Modelos Matemáticos para a descrição de fenômenos macroscópicos fundamentais em Dinâmica de Populações distribuídas e se encontram nos mais variados campos da ciência. A Teoria e os Métodos matemáticos desenvolvidos para a sua representação e análise são fundamento para uma ampla parte da Matemática contemporânea originados pontualmente com a publicação do tratado de Jean-Baptiste Joseph Fourier (1768-1830) ``\textit{Théorie Analytique de la Chaleur}'', em 1822, certamente um dos textos mais citados da literatura matemática. 

    Em muitos contextos, mas não todos, Processos de Difusão são notoriamente associados a comportamentos microscópicos aleatórios e esta será a abordagem pela qual introduziremos o presente modelo com base na interpretação probabilística do Modelo de Malthus. 

    A motivação que nos levará a um modelo microscópico probabilístico para a Difusão tem origem em um fenômeno chamado Movimento Browniano. Este fenômeno foi descoberto por Robert Brown quando experimentava a sua curiosidade com a novidade tecnológica do século XVII, observando ao microscópio a movimentação frenética e aparentemente sem propósito de partículas de pólen suspensas em água. Este fenômeno foi encarado como um mistério de locomoção biológica por mais de dois séculos até que Albert Einstein (1879-1955) e o polonês Marian Smoluchowski (1872-1917), que a princípio não conheciam o trabalho de Brown, apresentaram modelos matemáticos que previam movimentos microscópicos desta natureza desde que a teoria atômica da matéria, que não era unanimidade naquela época, fosse levada a sério. Um trabalho experimental exaustivo realizado em seguida por Jean B. Perrin (1870-1942) comprovou sem qualquer sombra de dúvida os resultados de Einstein-Smoluchowski e com isso, estabeleceu a teoria atômica da matéria, não apenas como uma boa hipótese de trabalho, mas como a descrição microscópica correta da natureza. (ref. Haw, Frey, Einstein,Ulam). A explicação fenomenológica de Einstein-Smoluchowski para o movimento irregular de partículas microscópicas em suspensão em um fluido baseia-se na variação (imprevisível) da taxa de colisões das moléculas do fluido que se movem por agitação térmica, com as partículas de pólen, ligeiramente maiores.

    Por outro lado, e ao contrário do aspecto \textbf{involuntário} do fenômeno físico observado em partículas inanimadas, microrganismos (e mesmo organismos superiores como o homo sapiens) quando em procura de nutrientes em uma região do espaço no qual não há qualquer informação sobre a localização de suas fontes, realizam \textbf{deliberadamente} um movimento aleatório em tudo semelhante ao Movimento Browniano. Isto significa que o comportamento básico de procura empregado por organismos (inteligentes e, muito inteligentes!) é a estrategia de ``varredura'' por movimentos aleatórios rápidos que vasculha uma região de razoável extensão em pouco tempo, em vez de ``perderem tempo com planejamento racional'' às escuras. Possivelmente daí é que se origina o dito popular ``\textit{De tanto pensar morreu um burro}'' (talvez de fome), o que justificaria a ubiquidade do movimento aleatório microscópico na natureza. (Berg, Gordon, Seeley, Sci. Am...)

    Analisaremos este processo em um Modelo especulativo (``Toy Model'') unidimensional representado por um tubo (seccionalmente uniforme) onde estão localizados os indivíduos de uma grande população que realizam movimentos espaciais \textbf{individuais e independentes} (isto é, sem choques e influencia entre eles) na projeção longitudinal do tubo. Consideremos que este tubo seja virtualmente fatiado em seções regulares que se constituirão em compartimentos contíguos cujas populações estejam submetidas a uma dinâmica Malthusiana em que os fenômenos de ``mortalidade e reprodução'' são substituídos por ``movimento à direita ou à esquerda'' com a mesma densidade de probabilidade \(\mu = \nu\). Assim, se \(P(t)\) for a população de partículas neste compartimento no instante \(t\), a sua população interior perderá indivíduos segundo uma dinâmica Malthusiana com a taxa \(-2\mu P\), sendo \(\mu P(t)\) para cada lado. Ao mesmo tempo, este mesmo compartimento receberá indivíduos provenientes dos compartimentos contíguos, à sua direita e à sua esquerda, devido ao mesmo motivo.

    Consideremos agora que o tubo completo seja virtualmente fatiado em N compartimentos adjacentes, (com populações \(P_k(t), 0 \le k \le N)\), de tal maneira que cada compartimento \(P_k\) tenha um vizinho à esquerda \((P_{k-1})\) e outro à direita \((P_{k+1})\), que receberão os indivíduos que escapam dela \((P_k)\), enquanto que a própria recebe aqueles que escaparam das suas vizinhas. (As fatias no começo, \(P_0\), e do fim, \(P_N\), terão apenas vizinhas, respectivamente, à direita e à esquerda) segundo o esquema gráfico:
    \[\ldots\
    \substack{\mu \\ \longleftarrow \\[-0.1cm] \longrightarrow \\ \mu}\ P_{k-1}\
    \substack{\mu \\ \longleftarrow \\[-0.1cm] \longrightarrow \\ \mu}\ P_{k}\
    \substack{\mu \\ \longleftarrow \\[-0.1cm] \longrightarrow \\ \mu}\ P_{k+1}\
    \substack{\mu \\ \longleftarrow \\[-0.1cm] \longrightarrow \\ \mu}\ \ldots
    \]
    Desta forma, podemos escrever um Modelo de Dinâmica Populacional para o conjunto de compartimentos \(P_k (0 < k < N)\) constituído de \(N\) Modelos Malthusianos acoplados:
    \begin{eqnarray}\label{eq:moddinamicapcompartimentos}
    \dfrac{dP_k}{dt} &=& -2\mu P_k + \mu P_{k-1} + \mu P_{k+1}.
    \end{eqnarray}

    Digamos, em particular, que o tubo esteja obstruído nas suas duas extremidades. Neste caso, teremos as seguintes equações para os compartimentos extremos que completarão o sistema:
    \begin{eqnarray}\label{eq:moddinamicapcompartimentossistema}
    \dfrac{dP_0}{dt} &=& -\mu P_{0} + \mu P_{1} \\
    \dfrac{dP_N}{dt} &=& -\mu P_{N} + \mu P_{N-1}
    \end{eqnarray}

    Esta hipótese obstrutiva sobre as fronteiras do tubo denomina-se, por motivos óbvios, ``\textit{Fronteira Reflexiva}'' e, por motivos históricos também de ``\textit{Condição de Neumann}'' (Carl Neumann, matemático alemão do seculo XIX). Neste caso, os compartimentos extremos têm conexões apenas com uma vizinha e está fechada na face final.

    Uma outra condição de fronteira de interesse que pode ser imposta à esquerda ou à direita, denomina-se, por motivos óbvios, ``\textit{Fronteira Absorvente}'' e, por motivos históricos ``\textit{Condição de Dirichlet}'' (P. L. Dirichlet, matemático alemão, século XIX), que, por exemplo, se for à esquerda toma a seguinte forma: 
    \begin{eqnarray}\label{eq:moddinamicapcompartimentosesquerda}
    \dfrac{dP_0}{dt} &=& -2\mu P_{0} + \mu P_{1}
    \end{eqnarray}
    ou seja, \(P_0\) perde indivíduos para a lateral esquerda (``O resto do universo'') e para a lateral direita \((P1)\), e recebe da lateral direita.

    As condições de Reflexão ou Absorção podem ser impostas em cada extremidade, dependendo do caso a ser estudado, ou seja, quatro situações. 

    Em qualquer caso, o sistema de Equações diferenciais ordinárias que representa estes Modelos pode ser escrito na forma vetorial-matricial de maneira formalmente muito simples e apropriada para o seu estudo representando simultaneamente todas as populações dos compartimentos em um instante qualquer por um vetor
    \[X(t) = (P_0(t), P_1(t), \ldots, P_N(t))^t.\]
    Em todas as quatro possibilidades de condições de fronteira nas extremidades, o Modelo resultante poderá ser genericamente representado na seguinte forma matricial
    \[\dfrac{dX}{dt} = SX\] 
    onde \(S\) será uma matriz simétrica (tridiagonal) facilmente determinada a partir da descrição acima. 

    \begin{exercise}
    Escreva a matriz \(S\) para as quatro situações indicadas e determine se, de fato, ela é sempre simétrica. 
    \end{exercise}

    \begin{exercise}
    Considere um sistema periódico, isto é, de tal forma que uma extremidade é conectada à outra. Neste caso o compartimento \(P_0\) terá \(P_N\) à sua esquerda e \(P_N\) terá \(P_0\) à sua direita. Mostre que ainda assim podemos escrever o modelo na forma \(\frac{dX}{dt} = SX\), com \(S\) simétrica.
    \end{exercise}

    Modificando ligeiramente a maneira de se escrever o sistema \eqref{eq:moddinamicapcompartimentos}, podemos entender melhor diversos aspectos deste Modelo, em particular, a razão pela qual ele representa de farto um \textbf{Processo de Difusão} caracterizado por ser uma \textbf{Dinâmica homogenizadora}.

    Rearranjando os termos da direita de \eqref{eq:moddinamicapcompartimentos}, temos:
    \begin{eqnarray}
    \dfrac{dP_k}{dt} &=& \mu (P_{k-1} + \mu (P_{k+1}-P_{k}).
    \end{eqnarray} 

    Observa-se que nos dois termos à direita, temos uma comparação de populações entre os compartimentos laterais (\(P_{k-1}\) e \(P_{k+1})\) e o compartimento central \((Pk)\).
    
    Com isto, verificamos que, por exemplo, \((P_{k-1}-P_{k})\) será positivo se \(P_{k} \le P_{k-1}\), ou seja, se o compartimento à esquerda tiver mais indivíduos do que o compartimento central \(P_{k}\). Portanto, como termo positivo da equação diferencial que contribui para o cálculo de \(\frac{dP_{k}}{dt}\), isto significa que nesta interface à esquerda haverá um ``fluxo líquido'' (no sentido contábil) do compartimento que tem mais indivíduos, no caso \(P_{k-1}\), para o compartimento que tem menos, \(P_k\). Obviamente, isto não significa que não haverá intercambio entre os dois compartimentos, apenas que no ``\textit{final das contas}'' quem tem menos ganha e quem tem mais perde. Repetindo-se o mesmo argumento para o caso em que o sinal for contrário, isto é, se \(P_{k-1}-P_{k} < 0\) e, também, para o outo termo de comparação \(P_{k+1}-P_{k}\), concluímos que o Processo de Difusão descrito pelo modelo acima tem um efeito (``Robin Hood'') homogeneizador para a distribuição de populações ao longo do tubo, produzindo um ``fluxo líquido'' de onde tem mais para onde tem menos. (Mais uma vez, observe que este é o efeito final, porque ``microscopicamente'' há indivíduos se movimentando para a direita e para a esquerda em todo o tempo).

    Embora esta tenha sido uma conclusão sobre o comportamento do Modelo de Difusão acima apresentado, podemos utilizar esta propriedade ``homogeneizadora dinâmica'' como a própria caracterização fundamental de qualquer processo que se denomine ``Difusão'' ou, do próprio conceito de ``Difusão''. 

    Re-escrevendo também o sistema \eqref{eq:moddinamicapcompartimentos} de uma nova maneira \[\dfrac{dP_k}{dt} = \mu (P_{k+1}-2P_{k} + P_{k} = \mu [(P_{k+1}-P_{k}) - (P_k - P_{k-1})\] 
    verificamos que o termo à direita da equação é uma diferença de segunda ordem, (diferença da diferença) que usualmente denotamos pelo operador de diferença \(\Delta^2\). (Em alguns textos o operador de diferença de primeira ordem é representado na forma de um ``\textit{gradiente}'' \(\nabla\), de onde, \(\Delta^2 = \nabla^2\)). Assim, se interpretarmos \(P_0, P_1, \ldots, P_N\) como uma sequência \(P\), podemos escrever o sistema acima na forma:
    \[\dfrac{dP}{dt} = \mu \Delta^2 P.\] 

    Esta representação tem um significado importante porque sinaliza uma maneira de estabelecer uma conexão entre o modelo espacialmente discreto (apresentado acima) e um modelo contínuo descrito por uma equação diferencial parcial de difusão, que terá a forma
    \[\dfrac{\partial \rho}{\partial t} = D \dfrac{\partial^2 \rho}{\partial x^2}\]
    para a função de distribuição \(\rho(x,t)\). (ref. Bassanezi-Ferreira 1988).
    
    Uma terceira e importante maneira de entender o efeito homogenizador deste Modelo é facilmente obtida rearranjando os termos à direita em \eqref{eq:moddinamicapcompartimentos} da seguinte forma:
    \[\dfrac{dP_k}{dt} = 2\mu \left(\dfrac{P_{k-1} + P_{k+1}}{2}-P_{k}\right).\] 

    Neste caso, se vê, claramente, que há uma comparação entre o valor do \(k\)-ésimo compartimento e a \textbf{média} (aritmética) dos valores dos compartimentos adjacentes e a equação tende a ajustar estes valores pois, se \(P_k\) for menor do que a média o termo será positivo e a dinâmica indica um crescimento no seu valor (\(\frac{dP_k}{dt} > 0\)). Por outro lado, se o  valor de \(P_k\) for maior do que a média de seus vizinhos, teremos \(\dfrac{P_{k-1}+P_{k+1}}{2}-P_{k} < 0\) e, portanto, a dinâmica do sistema indica um decréscimo de \(P_k(t)\), pois, \(\frac{dP_k}{dt} < 0\) . 

    Com esta interpretação confirmamos mais uma vez que um processo difusivo tem sempre a tendencia de homogenizar os valores do sistema, ou seja, uniformizá-lo. 

    Na verdade, esta é a definição geral de um processo difusivo. 

\subsection{Modelo de Difusão em Grafos-Redes Conectadas}

    Os argumentos acima podem ser aplicados ``\textit{ipsis litteris}'' a sistemas multiconectados onde os compartimentos com populações \(P_k\) estão localizados nos vértices de uma rede (grafo) e as conexões (lados) são as interfaces. 

%Exercícios:

    \begin{exercise}
    Desenhe um gráfico com 5 componentes e conexões múltiplas e estabeleça um Modelo de Difusão para este caso representando- na forma vetorial-matricial.
    \end{exercise}

    Anteriormente, mostramos que o Modelo Populacional Malthusiano pode ser encarado como uma grande quantidade de experimentos individuais, simultâneos e independentes e, portanto, passível de ser interpretado como um processo probabilístico para cada partícula individualmente. É natural portanto que estendamos o mesmo argumento para este modelo de Difusão. Por exemplo, podemos interpretar o parâmetro \(\mu^{-1}\) como o tempo médio de permanência de uma partícula em qualquer um dos compartimentos.

%Exercícios:

    \begin{exercise}
    Determine a probabilidade de que uma partícula do compartimento \(P_k\) se transfira para o compartimento adjacente \(P_{k-1}\) (ou para \(P_{k+1}\)) durante o intervalo de tempo de comprimento \(T\).
    \end{exercise}

    \begin{exercise}
    Considere um sistema finito de compartimentos \(P_k(t), 0 \le k \le 2m+1\), conectados segundo um modelo de Difusão, sendo que as extremidade à esquerda de \(P_0\) e à direita de \(P_{2m+1}\) sejam absorventes. Supondo que no instante zero, exista apenas um único indivíduo e este ocupa o compartimento central \(P_m\), qual a probabilidade deste indivíduo se perder por uma das extremidades até o instante \(t\)? Ou seja, qual o tempo médio de permanência deste individuo no sistema?
    \end{exercise}

\subsection*{Extremidades Infinitas:}

    Se:
    \begin{enumerate}
    \item A sequencia de compartimentos acoplados é muito grande,
    \item Os indivíduos se encontram inicialmente apenas em compartimentos centrais, 
    \item Nas extremidades seja válida a condição de absorção (nula) e 
    \item O tempo de observação seja em uma escala tal que uma proporção muito pequena de indivíduos possa alcançar as extremidades (isto é, o efeito das extremidades não seja sentido na região e no período de tempo de observação), é razoável considerar um sistema infinito de compartimentos, o que pode, eventualmente, simplificar o tratamento matemático do problema.

    Neste caso, um sistema infinito de compartimentos \(\{P_k\}\) conectados linearmente mantém constante sua população, ou seja,
    \[P_0(0) = \sum_{-\infty}^{+\infty} P_k(t)\]
    é constante o que, em particular, significa que
    \[\lim_{k \to \pm\infty} P_k(t) = 0.\]

    Argumentos semelhantes podem ser utilizados com apenas uma ou outra extremidade infinita.
    \end{enumerate}

%Exercício: 

    \begin{exercise}
    Mostre que a quantidade total de partículas é mantida no modelo de Difusão infinito,e também naqueles em que as condições laterais são de reflexão (Neumann) ou periódica. 
    \end{exercise}

    \begin{exercise}
    Mostre que se pelo menos uma das condições laterais é de Absorção (Dirichlet), então a quantidade total de partículas do sistema decresce exponencialmente. 
    \end{exercise}

    Consideremos, agora, que no instante \(t = 0\) uma ``grande'' população de \(N_0\) partículas seja colocada no compartimento central \(k = 0, (P_0(0) = N_0)\) enquanto todos os outros compartimentos estão vazios; \(P_k(0) = 0, k \ne 0\). Iniciando-se o processo de difusão no sistema de compartimentos, o valor de \(P_k(t)\) será determinado pelas equações \eqref{eq:moddinamicapcompartimentos}, para \(t > 0\). Este processo de difusão sendo constituído por vários processos de Poisson acoplados, pode, ele mesmo, ser considerado como uma grande quantidade de experimentos, simultâneos, para cada uma das partículas.

    Assim, definindo \(p_k(t)\) como sendo a Probabilidade de que uma partícula que inicia sua trajetória no compartimento central \(k = 0\), no instante \(t = 0\), esteja no \(k\)-ésimo compartimento no instante \(t\), podemos calcular esta probabilidade utilizando sua definição frequentista na forma:
    \[p_k(t) = \dfrac{P_k(t)}{P_0(0)}.\]

    Este cálculo pode ser realizado por vários Métodos (Função Geradora, Método Fourier, Método de Green/Operacional e etc.) que serão apresentados em outro capítulo.

    Naturalmente, um modelo matemático em tudo semelhante ao que foi apresentado acima pode ser utilizado para a descrição da difusão do calor ao longo de uma barra transversalmente homogênea, repartindo-a virtualmente em fatias finas adjacentes e supondo que a ``lei de Newton-Fourier'' para a condução de calor se aplique em cada interface. 

\section{VARIAÇÕES: Influência Externa Determinística-Método Operacional de Green-Laplace}


    Populações, biológicas ou não, em geral não são isoladas e estão sujeitas a influências externas nas formas de imigração e emigração. Analisemos uma modificação do modelo de Malthus onde este este processo ocorre e pode ser representado matematicamente pelo modelo auto explicativo:
    \[\dfrac{dP}{dt} = rP+f(t).\]
    A função \(f(t)\) representa um ``fluxo'' com a dimensão de variação populacional (\(NT^{-1} = \left[\frac{dP}{dt}\right]\)) de imigração quando positiva e, de emigração quando negativa, já que é uma parcela do cálculo de \(\frac{dP}{dt}\) e \(r = \nu-\mu\). Observe-se que este termo \(f(t)\) é completamente independente do estado do sistema (isto é, de \(P(t)\)), daí, o conceito de que representa ``influências exteriores''. 

    Uma expressão explícita (em termos operacionais) para o cálculo de \(P(t)\), chamada Fórmula de Green, pode ser escrita para esta equação e tem um papel importante em diversas questões de Matemática Aplicada, razão pela qual apresentaremos um método para a sua obtenção que pode ser facilmente generalizado para situações muito mais complexas.

    Analisando o problema matematicamente, verificamos que a função incógnita \(P(t)\) é submetida a uma operação linear da forma \(L = \frac{d}{dt}-r\) (onde \(L\varphi = \frac{d\varphi}{dt}-r\varphi\)). Uma das maneiras genéricas de resolver a equação \(Lu = f\), é a obtenção de um operador \(A\) que seja inverso à direita de \(L\), de tal forma que, aplicada à função dada \(f\), nos dê a solução \(u = Af\) pois, \(L(Af) = f\). Esta é apenas uma visão global do problema que deve ser suplementada por técnicas específicas para a chegarmos ao resultado desejado. Para isto, observemos, inicialmente, que são conhecidas as inversas à direita tanto para ao operador derivação (isto é, o operador integral via Teorema Fundamental do Cálculo) como para a multiplicação por um número \(r\). Mas para que esta informação tenha alguma utilidade no presente contexto é necessário ``fatorar'' o operador \(L\) como ``composição'' destes operadores; a sua expressão original como soma delas, \(L = \frac{d}{dt} - r\), não serve para este propósito. A modificação apropriada do operador \(L\) será obtida lançando mão de um procedimento simples, mas astuto, que faz uso de uma propriedade fundamental:
    \[\dfrac{d}{dt}(e^{-rt}) = -r e^{-rt},\]
    isto é, a atuação de \(\frac{d}{dt}\) em uma função exponencial se resume a uma simples multiplicação algébrica ou, em outras palavras, as funções exponenciais são auto-funções do operador linear \(\frac{d}{dt}\). Com isto, a regra de Leibniz para derivação de produto de funções,
    \[(\frac{d}{dt}) (e^{-rt}\varphi = e^{-rt}(L\varphi),\]
    produz, imediatamente, a expressão operacional:
    \[L = e^{rt} \frac{d}{dt} e^{-rt}.\]

    \textbf{Atenção}: esta igualdade é ``operacional'' e significa que \(L\) tem o mesmo efeito que aplicar sucessivamente as operações funcionais na seguinte \textbf{ordem}:
    \begin{description}
    \item (1) Multiplicação pela função \(e^{-rt}\);
    \item (2) Derivação, \(\frac{d}{dt}\), (do resultado da operação anterior);
    \item (3) Multiplicação (do resultado da operação anterior) pela função \(e^{rt}\). (Faça algumas experiências para se familiarizar com o significado operacional desta expressão).
    \end{description}

    Escrevendo, agora, a equação original na forma
    \[Lu = e^{rt} \frac{d}{dt} e^{-rt} u = f,\]
    podemos invertê-la (à direita) da seguinte maneira (``sucessivamente de fora para dentro: método cebola''):
    \begin{eqnarray}
    \left(\dfrac{d}{dt}\right) e^{-rt} u &=& e^{-rt} f \Rightarrow \nonumber\\
    e^{-rt} u &=& u(0) + \int_{0}^{t} e^{-rs} f(s)\ ds \Rightarrow \nonumber\\
    u &=& e^{rt} u(0) + e^{rt} \int_{0}^{t} e^{-rs} f(s)\ ds, \nonumber
    \end{eqnarray}
    de onde tiramos a Fórmula de Green:
    \begin{equation}\label{eq:formulagreen}
    P(t) = e^{rt} P(0) + \int_{0}^{t} e^{r(t-s)} f(s)\ ds
    \end{equation}

    A função
    \begin{equation}\label{eq:funcaogreen}
    e^{r(t-s)} = G(t,s)
    \end{equation}
    é denominada ``Função de Green'' (em homenagem a George Green (1793-1841)), mas também tem outros nomes significativos: ``Função de Memória'', ``Função de Influência'', ``Propagadora'', ``Função de Causalidade'' e etc., que atestam a sua múltipla ocorrência e importância.

    Observemos que a primeira parte da solução, \(e^{rt} P(0)\), é a solução da equação Malthusiana original (dita, equação homogênea sem influencia externa) e, portanto, o termo integral se refere inteiramente à contribuição da influência externa. De fato, este termo pode ser visto como uma soma destas influências, \(f(s)\ ds\), ao longo do intervalo \([0,t]\), onde o efeito no instante \(t\) é resultado da ``soma contínua'' das influências exercidas ``no passado'' \(f(s)\ (0 \le s \le t)\) intermediada pela função de Green, \(G(t,s) = e^{r(t-s)}\), como um fator de ponderação. Esta interpretação explica a utilização dos termos, ``função de influência'', ``memória'', ``propagação'' e etc. Em termos populacionais, o termo (operador) \(e^{r(t-s)}\) multiplicado (aplicado) a uma população (no caso \(f(s)\ ds\)) produz os seus descendentes vivos depois de um período de tempo \((t-s)\) de acordo com o próprio modelo de Malthus. Verificamos, assim, que esta expressão teria sido obtida por uma simples argumentação de ``contabilidade demográfica'':
\begin{quotation}
    ``A População no instante \(t\) é igual à soma da expansão malthusiana da população inicial no período \([0,t],\ e^{rt} P(0)\), com a expansões malthusianas durante os respectivos períodos \([s,t]\) das subpopulações que foram continuamente acrescentadas nos instantes \(s\), \(e^{r(t-s)} f(s)\ ds\), (se \(f(s) > 0\)) e subtraindo ainda as expansões das subpopulações que foram continuamente retiradas nos instantes \(s\), \(e^{r(t-s)} f(s)\ ds\), (se \(f(s) < 0)\)''. 
\end{quotation}

%Exercícios: 

    \begin{exercise} \quad
    
    \begin{description}
    \item (a) Considere uma população Malthusiana que inicia sua história com \(P_0 = P(0)\) indivíduos (vivos!) ``colonizadores'' submetida a uma taxa específica de mortalidade \(\mu\) e de natalidade \(\nu\). Determine o número total de nascimentos \(N(t)\) e o de óbitos \(M(t)\) durante o período \([0,t]\) nesta população.
    
    \item (b) Mostre que a subpopulação de sobreviventes dentre os indivíduos colonizadores \(P_0\) no instante \(t\) é \(e^{-\mu t} P(0)\) e que, em geral, os sobreviventes no futuro \(t+T\) da população \(P(t)\) existente no instante \(t\) é \(e^{-\mu T} P(t)\), e que os não sobreviventes são \((1-e^{-\mu T}) P(t)\).
    \end{description}
    \end{exercise}
    
    \begin{exercise}
    Considere uma população malthusiana com \textbf{altíssima} taxa de mortalidade, sem procriação. Argumente, matemática e demograficamente, porque um sistema com estas características é denominado como de ``\textit{Memória Curta}''. Se este sistema for alimentado por uma fonte externa \(f(t)\) de valor limitado, \((|f(t)| \le M)\) mostre que o ``grosso'' da \(t\) população
    \[P(t) e^{-\mu t} P(0) + \int_{0}^{t} e^{-\mu(t-s)} f(s)\ ds,\]
    para \(t\) distante da origem é descrito pela expressão: \(P(t) \approx \mu^{-1} f(t)\) a menos de erro exponencialmente pequeno. (\textbf{Sugestão}: Como \(1 \lll \mu\), valores da forma \(e^{-\mu t} P(0)\) são exponencialmente pequenos e na integral apenas a parte próxima de \(t\) (isto é, para \(s \sim t\)) o integrando \(e^{-\mu(t-s)} f(s)\) efetivamente contribui. Diz-se que os valores da integral estão ``condensados'' na extremidade \(t\). Com esta argumentação substitui-se a função \(f(s)\) por sua expansão de Taylor nas proximidades de \(t: f(s) = f(t)+(s-t) f'(t)+o(s-t)^2\) e avaliamos que apenas o termo \(f(t)\) terá contribuição não exponencialmente pequena). Esta argumentação faz parte de um conjunto de técnicas matemáticas importantes denominados ``\textbf{Métodos Assintóticos}'' a serem tratados com maiores detalhes em outro capítulo.
    \end{exercise}

    \begin{exercise}
    Obtenha uma expressão ``Fatorada'' para o operador
    \[\left(\dfrac{d}{dt} + \mu(t)\right),\]
    onde \(\mu(t)\) é uma função de \(t\) e, com isso, obtenha uma fórmula de Green para o Modelo Malthusiano com influência externa
    \[\dfrac{dN}{dt} = -\mu(t)N + f(t).\]
    \end{exercise}

\section{VARIAÇÕES: Interação Neuronal- Modelo ``Integrate and Fire'' de Lapicque para a Dinâmica de uma População Neuronal}

    O Neurônio é uma célula extremamente complexa em seus aspectos fisiológicos de natureza químico-física. Entretanto, considerando-se o neurônio como um individuo de uma população, verifica-se que a essência de sua participação no conjunto pode ser ``caricaturizada'' de maneira radicalmente simplificada, mas ainda relevante para o estudo de diversos aspectos da Neurologia. (Isto é, ``sem jogar fora o nenê junto com a água do banho'', como dizia Mark Kac). Este Modelo foi idealizado por L. E. Lapicque (1866-1952), em 1907, e, recentemente, tem sido utilizado em muitos Modelos Matemáticos para a representação da Dinâmica de uma População Neuronal (J. P. Keener - J. Sneyd - Mathematical Physiology, Springer 1998, Peter Dayan - Theoretical Neuroscience, Cambridge UP2012, Ch. S. Peskin - Lecture on Mathematical Aspects of Heart Physiology, Courant Inst. 1975). 

    O Neurônio de Lapicque será definido por um aspecto biológico denominado ``carga'' medido por um parâmetro \(0 \le \xi \le 1\) (fisiologicamente associado a uma concentração de íons de Cálcio em um neurônio biológico) podendo apresentar três ``\textit{estados dinâmicos}''. O estado de um neurônio determina a sua capacidade de interação com outros neurônios segundo as seguintes regras que procuram mimetizar caracterizadamente as três etapas essenciais da dinâmica da célula:
    \begin{enumerate}
    \item \textbf{Estado Refratário} (``\textit{Integrate}''): Quando \(0 \le \xi \le 1\). Neste caso, ocorre uma dinâmica Malthusiana de recarregamento \(\frac{d\xi}{dt} = \frac{1}{T}\xi\), e representa uma absorção progressiva de íons do meio em que está imerso. No estado refratário um neurônio não recebe influencia e nem transmite influencia para outros neurônios. A dinâmica refratária é acrescentada da condição de que, atingindo plena carga, \(\xi = 1\), o neurônio se estabiliza e passa ao Estado Reativo estacionário descrito em seguida.
    \item \textbf{Estado Excitável}: Quando \(\xi = 1\), o estado dinâmico do neurônio se mantém constante caso \textbf{todos os outros} neurônios conectados a ele estejam em estado Refratário ou também Excitável. Este é o único estado em que o Neurônio pode receber influencia de outro neurônio ao qual ele esteja conectado como estabelece a próxima regra:
    \item \textbf{Estado de Descarga} (``\textit{Firing}''): Este é o único estado de um Neurônio em que ele pode influenciar outro neurônio e isto ocorre com outros neurônios em estado reativo que estejam conectados a ele, no que passam também ao estado de Descarga. O Estado de Descarga é é uma dinâmica Malthusiana, \(\frac{d}{dt} = \frac{1}{\tau}\) em escala de tempo muito mais curta do que a Recarga (i.e., \(\tau \ll T\)) e considerada instantânea em alguns modelos e que leva o Neurônio de volta a um Estado refratário com carga \(\xi = 0\).
    \item \textbf{Influência Estocástica}: O Estado de Descarga também pode ocorrer na subpopulação de Neurônios Excitável mesmo que suas conexões estejam inertes e se dá segundo um modelo Malthusiano - Poisson (coletivo-individual).
    \end{enumerate}

    Observe-se que neste Modelo a dinâmica Malthusiana determinística comparece de duas formas distintas; uma delas intrinsecamente e outra com respeito a uma subpopulação de neurônios. A dinâmica Malthusiana populacional de descarga pode ser ``individualizada'', mas neste caso, probabilisticamente segundo o Modelo de Poisson, conforme vimos acima. 

    A Neurobiologia de um sistema é descrita pela Dinâmica de uma grande população de Neurônios, heterogênea segundo os estados dinâmicos individuais, e cujos indivíduos interagem entre si conforme as regras de Lapicque. Uma população Neuronal de Lapicque pode ser representada discretamente ou continuamente. A construção do Modelo contínuo exige a representação matemática da população neuronal heterogênea que é Tema do próximo capítulo. (B. W. Knight \& L. Sirovich, Dayan).

    Um Modelo Caricatural discreto de um população neuronal de Lapicque pode ser representado pela estrutura de um Autômato Celular, tema que será discutido no capitulo sobre Princípios Discretos Computacionais.

\section{VARIAÇÕES: Influência Externa Estocástica- Modelo de Langevin e a ``Redução em Média'' da Dinâmica Populacional}

    Um modelo de Mortalidade Malthusiano \(\frac{dN}{dt} = -\mu N\) é frequentemente perturbado por pequenas retiradas e acréscimos de indivíduos (descrita por uma taxa, respectivamente negativa ou positiva \(\xi(t)\) gerada por efeitos externos sobre o qual pouco se conhece, exceto algumas informações estatísticas. O Modelo resultante \(\frac{dN}{dt} = -\mu N + \xi\), pode ser resolvido por intermédio de uma Fórmula integral de Green (Método de Green) se a função \(\xi(t)\) for conhecida na sua descrição clássica, isto é, ponto a ponto. Entretanto, em muitos casos a descrição desta influência externa \(\xi(t)\) é apenas conhecida de maneira incompleta. Portanto, neste caso a descrição clássica da função \(N(t)\), também está fora de questão. Apesar disso, resta a possibilidade de obter alguma informação, ainda que também incompleta (sob o ponto de vista clássico) para a função \(N(t)\), mas ainda assim útil. A abordagem desta questão é originalmente devida ao físico francês Paul Langevin (1872-1946) no começo do século XX que pretendia modelar o mesmo Movimento Browniano estudado ao mesmo tempo por Albert Einstein e por Marian Smoluchowskii sob pontos de vistas totalmente distintos. A abordagem de Langevin procura descrever a dinâmica de uma pequena partícula (de tamanho quase macroscópico) em consequência da ação de choques aleatórios de uma (``\textit{nuvem}'') de moléculas microscópicas que, por sua vez, realizam movimentos aleatórios devido a uma agitação térmica. O objetivo deste modelo é utilizar um conceitos de ``\textbf{descrição em média de uma função}'' de tal maneira que, tal tipo de informação (não clássica) da dinâmica \(N(t)\) pudesse ser calculada a partir de uma informação (apenas) em média disponível sobre a influência externa \(\xi(t)\). Curiosamente, um trabalho matemático semelhante, mas com motivações totalmente diversos destes, foi desenvolvido na mesma época por Louis Bachelier sob orientação de Henri Poincaré e tratava especificamente das variações da Bolsa de Paris. (M. Haw, S. Brush, H. Berg, ...).

    A apresentação dos argumentos e técnicas que levam à formulação desta importante classe de Modelos Matemáticos é um dos temas abordados no capítulo sobre Princípios Probabilísticos.

\chapter*{BIBLIOGRAFIA:} 

M. Abramowitz-I.Stegun-ed.-Handbook of Mathematical Functions-Formulas,Graphs and Mathematical Tables, NBS 1964-online

D.Aldous-Stochastic Models and Phylogenetic Trees-From Yule to Today, Stat.Sci. 2001 

W.C.Allee-Animal Aggregations, U.Chicago Press 1939. 

W.C.Allee-Cooperation Among Animals, H.Schumann 1951. 

E.Amaldi-Radioactivity: A Pragmatic pillar of probabilistic Conceptions, pp.1-28 in Proc.Int.School Phys. E.Fermi-Corso 72–Problemi dei Fondamenti della Fisica-ed. G. Toraldo di Francia, North-Holland 1979. 

N.Arley-Introduction to Stochastic Processes and Cosmic Radiation, J.Wiley 1948 

N.Bacaer-Short History of Mathematical Population Dynamics, Springer 2011 OL 

A.L. Barabasi-Burst, 2010 - Nature 435, 2005 

H.C. von Baeyer-Information-The new Language of Science, Harvard Univ.Press 2003.

Th. Bayes-An Essay towards solving a problem in the doctrine of chances,Phil.Trans.Royal Soc., 53, 370-418, (1763) 

H.Bateman-The Probability Variations in the Distribution of Particles, Phil.Mag. 20 (1910), 698-704. 

J.Beekman-Two Stochastic Processes, Halsted 1974. 

H.Berg-Random Walks in Biology, Princeton UP 1993. 

E.Borel-Le Hasard, 1924 

E.Borel-Radioactivité Probabilité et Determinism, Oeuvres vol. 4, 1972, pg. 2189-2196. 

K.E.Boulding-Foreword to Malthus’ Essay, U.Michigan 1959 

B.Brecht-Galileo, A Play, Grove 1966 

J.Browne-Darwin’s Origin of Species, 2 vol. 

S.G.Brush-Randomness and Irreversibility, AHES 5 1968, 1-36, 12, 1974, 1-80. 

S.G.Brush 

P.Buhlmann-Toward Causation and External Validity, Proc.Nat.Acad.Sci(pnas)2020 

D.Calvetti-E.Somersalo-An Introduction to Scientific Bayesian Computing, Springer-Verlag 2008. 

S.Chandrasekhar-Newton’s Principia for the Common Reader. Oxford UPress, 1995. 

J.E.Cohen-How many People can Earth Support, Norton 1995 

J.E.Cohen-Population Growth and Earth’s Human Carrying Capacity, Science (269), 1995, 341- 

J.M.Cushing-Integrodifferential Equations and Delay Models in Population Dynamics, Springer Lect.Notes Biomath. 20 1977-SIAM 1998

H.Caswell-Matrix Population Models, 3rd. Ed. Sinauer, 2005. 

H.Cramer-Historical review of Filip Lundberg’s works on risk theory, pg. 1288-1294 in Collected Works of Harald Cramer. 

Charles Darwin-The Origin of Species and the Descent of Man, 1859 (The Modern Library-Random House s/d) 

L. Curtis-Concepts of Exponential Law prior to 1900, A.J.Ph. 46(9), 1978, 896-

O. Darrigol-....History of Electrodynamics....., Oxford UP 

O.Darrigol-World of Flows- History of Fuid Dynamics.... 

M.Delbrück-Statistical Fluctuations and Autocatalytic Reactions, J.Chem. Phys. 1940 

J.Diamond 

O.Diekmann-An Invitation to Structured (Meta) Population Models, Springer Verlag Lect BioMath. 96, 1993, pg.162-175. 

O.Diekmannn-H.Tiemme-Mathematical Biology....Springer-Verlag 

K. Doya \& al. editors-Bayesian Brain, MIT Press 2007. 

E.B.Dynkin-Random Walks, in ``Mathematical Conversations'', ed. E.B.Dynkin-V.A.Ouspensky, Dover 2006. 

E.B.Dynkin-A.V.Yushkevich-Markov Processes, Plenum 

N.Eldredge- Entrevista Revista Pesquisa da Fapesp- ``Sobre a capacidade de suporte da Terra'' 
T.Erneux-Delay Differential Equations, Springer 

L.Euler - ``Recherches générales sur la mortalité et la multiplication du genre humain- Acad.Sci Berlin 1760- trad. Theor.Popul.Biol. 1:307-314, 1970, \& pg. 83-91 in Smith-Keyfitz(1977)

Leonhardi Euleri-Opera Omnia -ser.I vol 7- pg. 345-352 

L.Euler-Introductio Analysin Infinitorum, 1748- trad. inglês– Springer-Verlag 

G.T.Fechner-KollektivemassLehre, 1897 

W.Feller-An Introduction to Probability Theory and its Applications, 2 vol. J.Wiley 1966 

W.Feller-On the Integral of Renewal Theory, Ann.Math.Stat. 12 (1941), 243-267, pg. 131-156, in Smith-Keyfitz(1977). 

W.Feller-On the logistic law of growth and its empirical verification in biology, Acta Biotheor. 5, (1940): 51-66 

E.A.Fellmann-Leonhard Euler (Biography)-Birkhauser 2007 

W.C.Ferreira Jr.-Dinâmica de Populações: De íons a sapiens, online- Revista ComCiência 

W.C.Ferreira Jr.-The Multiple Faces of Diffusion, 2011 

WCFerreira Jr.-O Silêncio dos Conformistas, Conf. Enc.Biomat I, 2017 e 2018-prelo. 

R.P.Feynman-The Concept of Physical Law, MIT Press 

L.Fibonacci-Liber abbaci di Leonardo Pisano, 1202 (online) 

Ph.Flajolet-R.Sedgewick-Analytic Combinatorics, Cambridge UP, 2009 

H. von Foerster-Some remarks on changing populations, The Kinetics of Cell Proliferation, pg. 382-407, 1959. 

H.Fritzsch-The Creation of Matter, BB 1984 (pg. 165-The decay of the proton) 

Galileo Galilei-Dialogo,Fiorenza 1632 (trad. Dialogue Concerning the Two Chief World Systems, UnivCalif.Press1967) 

C.W.Gardiner-Handbook of Stochastic Methods for Physics,Chemistry and Natural Sciences, Springer 1985 

M. Gellman-The Quark and the Jaguar, Norton 1985.pg.132 

N.Gershenfeld-Mathematical Methods for Information Technology, MIT Press 

G.Gigerenzer-P.M.Todd-editors-Simple Heuristics that make Us Smart, Oxford Univ.Press 1999. 
P.Glimcher-Decision, Uncertainty and the Brain, MIT Press 2004. 

R.Graham-D.Knuth-O.Patashnik-Matemática Concreta: Fundamentos para a Teoria de Computação, Livr.Tecno-Cientifica 1988 

John Graunt-Natural and Political Observations Mentioned in a Following Index and Made upon the Bills of Mortality, 1662-pg. 12-20 in Smith-Keyfitz(1977). R.Gregory-Eye and the Brain-The Psychology of Seeing, Princeton UP 2015 

K.P.Hadeler-Pair Formation in Age-Structured Populations, Acta Appl.Math. 14, (!989), 91-102. 

E. Halley-An Estimation of the Degree of Mortality of Mankind, PhilTr RS,1693. 

W.D.Hamilton-The Moulding of genes by Natural Selection, J.ThBiol 12 (1966), 12-45. 

I.Hacking-The Emergence of Probability, 

G.Hardin-The Tragedy of Commons, Science 1968 

G.Hardin-Living within Limits, Oxford U.P.1993

S. Herbert-Darwin Malthus and Natural Selection, J.Hist.Biol. 4 (1971) 209-217 

S. Herculano-Houzel S (2009) The human brain in numbers: a linearly scaled-up primate brain. Frontiers of Hum Neurosci 3:31 
E.Hopf- On causality, statistics and probability, J. of Math. and Physics, vol 13, 1934.1763 

F.Hoppensteadt-Mathematical Theories of Populations, SIAM 1972 

David H. Hubel-Eye, Brain and Vision, Sci. Am. 1988 

M.Kac-Lectures on Probabilistic Methods, 1958 

D.Kahneman - D.Slovic - A. Tversky - Judgement under uncertainty: Heuristics and Biases, Cambridge Univ. Press 1982. 

J. P. Keener-J.Sneyd-Mathematical Physiology, 2 vol. Springer 2008. 

J. B. Keller-Mortality rate versus age, Th.Pop.Biol. 65 (2004) pg.113. 

N. KEYFITZ - H.Caswell- Applied Mathematical Demography, 3rd Ed.\_SV2005 

B. KEYFITZ - N. KEYFITZ - The McKendrick Partial Differential Equation and its Uses in Epidemiology and Population Study, Math.Comp.Mod. (1997):26, 1-9. 
N.Keyfitz-Reconciliation of Population Models:Matrix, Integral and partial fraction, JRStat.Soc. A, 130, 1967, 61-83 

N.Keyfitz-World Population and Ageing, UChicago Press 1990. 

N.Keyfitz-J.Beekman-Demography through Examples, Springer 1984. 

M.Kline-Mathematical Thought from Ancient to Modern Times, Oxford UP 1970 

E.V.Koonin-A.Novozhilov-G.Karev-The Biological Applications of Birth \& Death Processes, Briefings in Bioinform. 7(1), 2010, 70-85. 

M.Kot-Elements of Mathematical Ecology, Cambridge U.Press 2001 

H.Kragh-The Origin of Radioactivity: From solvable problem to unsolved Non Problem, Arch.Hist.ExactSci. 50(3-4), 1997, 331-358. 

Peter Kropotkin-Mutual Aid:A Factor of Evolution, 1902 

K.Lange-Applied Probability, Springer Verlag 2010 

P.H.Leslie-On the use of matrices in certain population mathematics, Biometrika, 33 :183-212, 1945. 

R.Lewontin-D.Cohen-................PNAS 1959 

C.C.Lin-L.A.Segel-Mathematics Applied to Natural Sciences, SIAM 1990 

A.J.Lotka-Elements of Physical Biology, 1924, Dover 1956. 

D.LUDWIG-Stochastic Population Theories, Springer-Verlag Lect. Notes in Biomath. 3, 1974 

D.LUDWIG-The Distribution of Population Survival Times, Am.Nat.147, (1996), 506-520. 

S.Luria-M.Delbruck-Mutation of Bacteria, Genetics 28 (1943), 491-511 

DJC McKay-Information Theory, Inference and Learning Algorithms, Cambride Univ. Press 2003. 
N.N.MacDonald-Biological delay systems, Cambridge U.P. 1990 -(Time lags in Biological Models SVLectBiomath 27, 1978). 

Thomas Robert Malthus- Population: The First Essay, London 1798 

R.M.May-When two and two do not make four: nonlinear phenomena in ecology, Proc.R.S.London B 228(1986), 241-66. 

R.M.May-Stability and Complexity in Model Ecosystems, Princeton U.P. 1974. 

E.Mayr-The Growth of Biological Thoght, Harvard U.Press 1982 

A.G.McKendrick-Applications of mathematics to medical problems. Proc.Edinburgh Math.Soc., 44: 98-130, 1926 

George A. Miller-The Magical Number Seven, Plus or minus Two: Some Limits on Our Capacity for Processing Information, Psych.Rev. 63(1956), 81-97: ``My problem, ladies and gentleman is that I have been persecuted by an integer'' 

J.Monod-Le hasard et la necessité, Ed. du Seuil 1970-(trad. Chance and Necessity:An Essay on the Natural Philosopy of Modern Biology- Vintage 1971) J.Monod-The Growth of Bacterial Culture, Ann.Rev. Microbiol. 1949, 3, 371-394. 

P.W.Nelson-A.Perelson 

J. Pearl-D.Mackenzie-The Book of Why: The New Science of Causation and Effect, Basic Books 2018 

A.Perelson-P.W.Nelson-.... HIV Virus Dynamics.....SIAM Rev 1999 

A.S.Perelson-P.W.Nelson-The Mathematics of HIV Infection, in J.Sneyd-ed-An Introduction to Mathematics of Biology, AMS 2001 
Charles S.Peskin-Mathematical Aspects of Heart Physiology, Lect. Courant Inst.-NYU 1978- AMS2008

Physics Web-Bismuth break half-life.....online: http://physicsweb.org/article/news/7/716 

PhysicsWeb- Carbon clock could show the wrong time, 10 May 2001-online: http://physicsweb.org/article/news/5/5/7/1J. von Plato-....... David Pimentel-R.Hopfenberg-Human Population Numbers as a Function of Food Supply, -pp online 2001 

E.Pitacco - \& al. editors-Modelling Longevity for Pensions, Oxford UP 2009 

S.D.Poisson-La proportion des Naissances des Filles et des Garçons, Memoire de l’Acad. des Science, 08 février 1829. https://babel.hathitrust.org/cgi/pt?idmdp.39015011958983;view1up;seq489 

G.Polya-Patterns of Plausible Inference, Princeton Univ. Press 1968. 

Th.Porter-A Statistical survey of Gases: Maxwell’s Social Physics, Hist.Studies in the Phys.Sci. 12(1) 1980, 77-116 

L.Redniss-Radioactive–Marie and Pierre Curie; A Tale of Love and Fallout, 2010 

E.Renshaw-Modelling Biological Populations in Space and Time, Cambridge U.P., 1991. 

E.Renshaw-Stochastic Population Processes, Oxford UP 2011 

M.Rose-The Evolution of Ageing since Darwin, J.Gen. 87(4) 2008 

W.Rundell-Determining the birth function for an age structure population, Math Popul Studies, 1: 337-395, 1989. 

P.Samuelson-Resolving a historical confusion in population analysis, Human Biology, 48: 559-580, 1976 \& pg.109-129 in Smith \& Keyfitz (1977). S. Schweber - The Origin of Origins, J. of the Hist Biol 10 (1977), 229-310. 
Scientific American–The Mind’s Eye, Readings from Sci. Am. 1986 
Scientific American- Image, Object Illusion, Readings from Sci. Am. 1974 

A. Shapiro-ed.-The Oxford Handbook of Visual Illusion, Oxford UP 2017 

D. Smith - N.Keyfitz - editors-Mathematical Demography-Selected Papers, Springer Verlag 1977.

J.Sneyd-ed-An Introduction to Mathematics of Biology, AMS 2001 

J.Sung-J.Yu-Population System Control Springer 1988- rev. J.Cohen SIAM Review 1990 

J.Sung \& al - Population System Control\_Math.Comp.Mod. 11(1988) 11-16. 

L. Szilard - Ageing Process, Proc.Nat.Acad,Sci., USA 1959 

S. Ulam - Marian Smoluchowski and the Theory of Probability in Physics, Am.J.Phys. 25 (1957), 475-481. 

J. van Brackel-Radioactivity as Probability , Arch.Hist.Exact Sci. 31 (1985), 369-385. 

N. van Kampen-Stochastic Methods in Physics and Chemistry,North-Holland 1985. 
J. von Plato-Creating Modern Probability-Mathematical Physics Perspective, Cambridge UP 1994 

P. Vorzimmer-Darwin Malthus and Natural Selesction, J.Hist.Ideas 30 (1969), 527-542. 

Howard Wainer-Picturing the Uncertain World-How to understand, Communicate and Control Uncertainty through Graphical Display, PrincetonUP 2009 Howard Wainer-Graphic Discovery, Princeton UP 2005 

A.R.Wallace-Contribution to the Theory of Natural Selesction, 1870. 

N.Wax-editor-Selected papers on Noise and Stochastic Processes, Dover 1954 

E.Widiger-ed.-The Five Factor Model, Oxford Univ.Press 2014 

R.M.Young-Malthus and the Evolutionist’-Common Context, pg. 23-55 in - Darwin’s Metaphor, 

R.M.Young-ed. CUP1985. R.Zwanzig - ... Verhulst logistic Equation... PNAS . 

\chapter{APÊNDICE}

\section{O EFEITO KANIZSA, A METODOLOGIA ANALÍTICA DE GALILEO, E SUA EXTENSÃO NEWTONIANA:}

Redução e Síntese: Uma Curva Suave em lugar de uma Enorme Tabela Discreta (Kanizsa), sua Representação Cartesiana Funcional (Galileo) e sua Caracterização como Solução de uma Equação Diferencial (Newton) 

\begin{citacao}
    ``Eu tenho uma maior admiração por aquele que, pela primeira vez, imaginou e construiu um instrumento musical que seria um tosco protótipo da harpa, do que pelos admiráveis artesãos que aperfeiçoaram este instrumento até a forma graciosa e a sonoridade perfeita que hoje o caracteriza''.

Galileo 
\end{citacao}

    A Metodologia de Galileo que busca sintetizar (e generalizar) funcionalmente a correspondência de uma Tabela de dados experimentais deu início a uma revolução científica quando a sua perspicácia (e conhecimentos de Geometria Elementar) detectou (aproximadamente) as propriedades de uma parábola nas imagens de trajetórias de balas de canhão que ele exaustivamente analisou. (G. Galilei - Dialogo). Daí, à uma formulação analítica para esta trajetória foi um pequeno passo em vista da recentemente desenvolvida Geometria Analítica de Renée Descartes. (Uma coincidência que não pode passar despercebida).

    Entretanto, se a ideia de Galileo era brilhante, por outro lado, a sua implementação era difícil, mesmo com a Geometria Analítica de Descartes, pois não havia um procedimento padrão para caracterizar a função que bem representasse uma Tabela de dados. Além disso não havia ainda uma ``biblioteca de funções'' suficientemente grande que permitisse encontrar esta representante, visto que as funções disponíveis à época eram apenas as Elementares Algébricas, i.e., obtidas como resultado de uma sequencia finita de operações de soma, produto, composição e potencias racionais aplicadas às funções básicas {funções constantes e função identidade}.

    Com a invenção do Cálculo Diferencial a ``biblioteca de funções'' aumentou consideravelmente pois incluiria a operação (transcendental) de soma infinita (além das operações algébricas) e, não menos importante, disponibilizou o emprego da Metodologia Newtoniana que caracteriza uma função de uma maneira extremamente sintética em termos da  solução de uma equação diferencial. (Compare a expressão aritmética da função exponencial, \(f(x) = \frac{1}{k!} x^k\), com a sua ultra-sintética descrição diferencial, \(\frac{df}{dx} = f,\ f(0) = 1\)).

    A Metodologia Newtoniana para a Matemática Aplicada se constitui, portanto, de duas partes: 

    1) Os Princípios que permitem ``encriptar'' uma função representativa de um Modelo Matemático na forma de solução de uma Equação Diferencial e,
    
    2) Os Métodos analíticos que são instrumentos necessários para a ``abertura destes códigos'', ou seja, os métodos de resolução de equações diferenciais. (O Presente texto está organizado segundo esta Metodologia Newtoniana).

    Os Modelos Matemáticos da Mecânica Newtoniana formulados a partir do século XVIII são expressos na forma funcional, cujas funções são definidas como soluções de equações diferenciais. A Metodologia Newtoniana para a construção de Modelos Matemáticos tornou-se o paradigma predominante da Matemática Aplicada devido à sua capacidade de síntese que estava imersa na Teoria do Cálculo Diferencial e Integral inventado por Newton e Leibniz. É natural portanto que a aplicação desta metodologia na formulação de um modelo demográfico fosse encarado como indispensável.

    Entretanto, para que isto fosse possível o modelo matemático adequado para representar ``o tamanho \(N\)'' da população em cada instante \(t\), \(N(t)\), teria que ser uma função contínua e diferenciável. Ora, mas diriam os apressados da objetividade, esta pretensão é um rematado contra-senso já que, notoriamente (!), a função \(N(t)\) varia aos saltos, ``de um em um'' e, portanto, é irremediavelmente descontínua. 

    Na verdade, esta é uma objeção pertinente e bem fundamentada o que a faz merecer uma abordagem séria e cuidadosa muito embora na maioria das vezes os textos usuais de Matemática Aplicada apelem para um silêncio conformista da audiência. (W. C. Ferreira J - Conferencia/Artigo: O Silêncio dos Conformistas, 2018).
    
    A argumentação mais simples e direta que sugere a transição do discreto para o contínuo (e para a suavidade do diferenciável) na representação de uma Dinâmica Populacional tem um fundamento essencialmente cognitivo e também é relevante para diversos outros contextos semelhantes na Matemática Aplicada.
    
    Iniciemos pela observação inequívoca de que em toda a Matemática, dispomos apenas de representações gráficas descontínuas de funções (quaisquer que sejam elas) e, somente por uma ilusão de ótica (efeito Kanizsa) é que mentalmente completamos algumas sequencias pontilhadas por linhas (imaginárias) ``contínuas e suaves''. Ou seja, a continuidade e a suavidade é apenas uma construção mental que decorre de uma estratégia inata do cérebro humano para fazer sentido dos sinais que a retina lhe envia associando-os a memórias (imagens mentais) de elementos mais organizados, ou estruturados. O cérebro é patentemente incapaz de discriminar uma grande quantidade de informações não estruturadas (segundo George Miller [1956], \(7 \pm 2\)) e, por esta razão, a (enorme quantidade de) informação contida em um gráfico pontilhado tem que ser ``reduzida'' para ser ``entendida'', isto é, memorizada por intermédio de algumas de suas características mais proeminentes. De fato, a representação gráfica-visual de uma tabela numérica foi uma das técnicas mais importantes inventadas pela ciência em geral, uma vez que, com isto, toda esta capacidade de percepção visual/mental torna-se disponível. É difícil imaginar que o extraordinário desenvolvimento da ciência nos últimos séculos, especialmente da Matemática, pudesse prescindir desta singular sinergia entre símbolos e representação visual.Uma outra maneira de organizar uma grande quantidade de dados ``desconexos'' como, por exemplo, uma série aleatória de dígitos (CPF, Telefone e etc.) é associá-los à uma simples melodia que tem estrutura e, portanto, é de mais fácil memorização, ainda que a associação seja completamente desprovida de sentido. A associação de um gráfico pontilhado a uma curva contínua é apenas uma das estratégias que a mente utiliza para ``reduzir'' uma grande massa de informações a um tamanho que pode ser ``arquivado'' e classificado segundo algumas poucas características (Sci.Am. [1974], [1986], Hubel [1988], Gregory [2015], Shapiro [2017], Widiger [2014]).

    Naturalmente, para que esta suavização de descontinuidades seja possível, é necessário que o conjunto de pontos não apresente ``grandes'' espaços interstícios e que, de fato, ele disponha de alguma estrutura propícia e não completamente aleatória. Para minimizar a dispersão dos pontos, ou seja, para promover uma ``aglutinação'' dos pontos do gráfico, usualmente lançamos mão da liberdade de escolha de unidades para as medidas das variáveis numéricas Tempo e População. Por exemplo, um aumento da unidade Tempo resulta em uma compressão linear do gráfico no sentido horizontal. Efeito semelhante na ordenada vertical é obtido com a modificação da unidade de População. Estas deformações do gráfico não modificam aspectos topológicos que representam informações essenciais como região de crescimento e de curvaturas (isto é, os sinais da primeira e segunda derivadas). Enfim, na representação gráfica, a forma ``rígida'' da curva resultante não é essencial, apenas suas propriedades topológicas: monotonicidade, curvatura e etc. 

    Como as duas dimensões (Tempo e População), a princípio, nada tem a ver uma com a outra, isto é, são independentes, não há absolutamente nenhuma razão para que tenham unidades transformadas pelo mesmo fator e este fato será utilizado em várias situações para enfatizar aspectos distintos do Modelo Matemático. Um gráfico pontilhado pode, portanto, após compressão apropriada nas duas direções, fazer com que os pontos de gráficos discretos se aproximem o suficiente para que virtualmente descrevam uma curva contínua. Obviamente isto não a torna uma função matemática contínua, apenas faz com que, para efeito cognitivo, seu gráfico se apresente como um traço contínuo e assim favoreça psicologicamente esta interpretação.

    De qualquer forma, é bom ressaltar que a hipótese de que um gráfico discreto possa ser bem representado por uma curva contínua e suave é uma hipótese (ou, ``wishful thinking'') fundamental para a construção do Modelo Matemático diferencial. A adequação do Modelo Matemático resultante desta hipótese somente poderá ser verificada a posteriori, nunca ``demonstrada'' a priori. Um conjunto de pontos completamente aleatórios dificilmente poderá ser associado a alguma estrutura simples que o represente razoavelmente. Mesmo assim, veremos que há casos em que isto é possível.(....[..] ).

    É interessante citar a possibilidade de generalizar estas ideias com a utilização de escalas não lineares cujas ``lentes de observação'' se modificam segundo as regiões e de acordo com a conveniência do objetivo descritivo do Modelo matemático. A escala (não linear) logarítmica é talvez o exemplo mais comum desta estratégia.Os Métodos Assintóticos também lançam mão desta estratégia com frequência.(Lin-Segel[1990], Segel [BullMathBiol1989]). 

    A compressão vertical do gráfico de uma dinâmica populacional somente pode ser realizada se tratamos de grandes populações, por exemplo, da ordem de \(10^9\), como as populações do Brasil, de um grande formigueiro [Gordon [1999]), do número de células do sistema imunológico ou de neurônios [Herculano - Houzel - 2009] e etc.. (O número de Avogadro é da ordem de \(10^{23}\). Assim, se a unidade empregada for \(P_0 = 10^7\) (isto é, um ``lote'' de 10 milhões de indivíduos) estas populações biológicas passam a ser descritas com valores \(0 \le n \le 100\).

    A compressão horizontal, por sua vez, pode, por exemplo, aglutinar 200 pontos para a representação discreta de uma dinâmica demográfica de 100 anos com censos semestrais. A dinâmica populacional de um formigueiro, por outro lado, se analisada em um período de 5 anos (Gordon [1999]) com dados semanais apresenta um total da ordem de 250 pontos. A dinâmica imunológica é mais rápida e um período de 1 mês, pode ser registrada discretamente em intervalos de três horas o que também exige uma ordem de 250 pontos. 

    A modificação linear de escalas, ou de unidades, funciona como uma lente regulável que pode, por um lado, focalizar pequenos detalhes locais, tal como um microscópio, (reduzindo o campo de observação), ou, por outro, possibilitar a percepção da estrutura geométrico-topológica do ``todo'' global com o afastamento do ponto de vista do observador, que aglutina pequenas variações (``borragem'') ,reduzindo, neste processo, a quantidade de informação. A construção de imagens contínuas pela iluminação de (invisíveis) pixels e a superposição de imagens em um filme cinematográfico são exemplos comuns e práticos desta técnica. 

    O problema de redução de informações, todavia, não é apenas uma estrategia psicológica humana. Hoje em dia, qualquer área científica (Social ou da Natureza) é inundada por uma massa avassaladora de informações, de tal monta que se torna patentemente impossível enfrenta-la com os métodos computacionais ou analíticos tradicionais. Em vista disso, os chamados Métodos Matemáticos de Redução, que tem por finalidade exatamente reduzir e organizar apropriadamente enormes arquivos de informações, tem sido rapidamente desenvolvidos nas últimas décadas tornando-se uma classe especial de métodos da Matemática Aplicada Contemporânea. Um dos métodos mais efetivos e   interessantes para este fim faz uso de um processo (artificial) de difusão (um operador de difusão) que de fato associa conjuntos discretos a figuras (gráficos) matematicamente suaves. (W. C. Ferreira Jr. [2018b]). Estes Métodos serão tratados em capítulo a parte. (Kutz [2015],...).

%xxxxxxxxxxxxxxxxxxxxxxxxxxxxxxxxxxxxxxxxxxxxxxxxx 

\section{A Metodologia de SÍNTESE FUNCIONAL DE DADOS EXPERIMENTAIS: Galileo-Newton-Kanizsa \\ A Estética e a Associação (Mnemônica) como Economia de Informação}

\begin{citacao}
    ``The ability to describe underlying patterns from data has been called the fourth paradigm of scientific discovery. [However, according to Kanizsa, that is exactly what our cognitive senses authomatically do all the time since immemorial eras. Besides, people forget to point out that Galileo, Kepler, Huygens, Euler and Rutherford have done just that centuries ago]''. Primeira parte. J.G. Hey Anthony \& al; Report Microsoft Res., Redmond WA 2009.
    
    Segunda parte paráfrase anônima.
\end{citacao}





\chapter{HETEROGENEIDADE E O PRINCÍPIO DE CONSERVAÇÃO ``apud Eulero''}



\section{INTRODUÇÃO HISTÓRICA: O Modelo Discreto de Euler para Heterogeneidade Populacional (\(\sim\)1750)}

\begin{citacao}
``Leiam Euler, ele é o mestre de todos nós''. Pierre Simon de Laplace.
\end{citacao}

Leonhard Euler (\(\star\)1707 - Basileia, Suíça - \(\dag\)1783 - S. Petersburgo, Rússia), o mais prolífico e um dos mais criativos matemáticos em todas as épocas, conhecido por sua modéstia e lisura de caráter, tem o seu nome relacionado à concepção e ao desenvolvimento de importantes áreas da Matemática e de suas Aplicações. (E. Fellman, N. Bogolyubov). Partícularmente notáveis são as suas ideias e técnicas que simultaneamente fundamentaram a descrição matemática da Dinâmica do Meio Contínuo e da Dinâmica de Grandes Populações em meados do século XVIII.

Ao longo dos últimos 250 anos, os métodos e conceitos criados por Euler nesta empreitada foram amplamente generalizados e se tornaram ferramentas matemáticas fundamentais na formulação de Modelos destinados a descrever o inclusivo tema da Dinâmica de ``Grandes'' Populações que compreende uma vasta gama de aplicações específicas em Física, Química, Demografia e especialmente em Biologia de Populações.

O objetivo deste capítulo é apresentar os elementos básicos destas ideias em uma linguagem contemporânea tomando como um protótipo concreto delas o Modelo Matemático de demografia publicado por Euler em um artigo de 1748 e que também fez parte seus textos pedagógicos de Matemática como um exemplo de aplicação.

A definição de População segue o conceito matemático de conjunto, cujos elementos são indivíduos indistinguíveis entre si para os propósitos em questão e dotados de alguma característica comum que define a sua pertinência a ela. {\small\color{red!50} (Tal como ocorre no conceito matemático, o critério de pertinência de um indivíduo a uma população deve ser verificável, inequívoco e, de certa maneira, é a única característica que o distingue dos demais indivíduos do Universo. Modelos fuzzy, que não serão abordados aqui, substituem a condição categórica de pertinência por uma medida de ``possibilidade``).}

A informação fundamental que se deseja determinar sobre uma População é o seu tamanho, ou seja, a cardinalidade do conjunto de seus indivíduos.

Todavia, ao contrário do que ocorre com os conjuntos matemáticos, a mensuração do tamanho de ``Grandes'' Populações, como já foi observado no Capítulo I, não é necessariamente expressa pela cardinalidade simples do conjunto de seus indivíduos, (i.e., o ``número inteiro de cabeças''), mas em termos de alguma unidade definida por um ``grande lote'' de indivíduos. Mesmo assim, em linguagem matemática corrente, nos referimos a esta medida como o ``número de indivíduos'' da População.

Por exemplo, sob um ponto de vista de Modelos Matemáticos, registrar a população do Brasil como sendo constituída por 205.435.731 indivíduos às 10h:55min:37s5fs em 16/04/2018 é completamente sem sentido, se é que há alguma utilidade nesta ``precisão''. Em geral, é mais do que suficiente e razoável descrevê-la, por exemplo, com a expressão \(20,435\ (\pm 10^{-3}P\) onde \(P\) é um ``lote'' de \(10^{7}\) indivíduos, que toma o lugar de uma unidade de população.

Isto significa, em particular, que o tamanho de uma população será, em geral, denotada por um número \textbf{real} \(N\).

\comentario{(A escolha dos números reais e não dos números racionais para a medida do tamanho de grandes populações, com já foi comentado, decorre do efeito visual de Kanizsa).}

A Dinâmica Populacional é uma área da Matemática Aplicada que consiste na descrição de seu tamanho ao longo de um intervalo determinado de tempo, \([0, T]\), na forma de uma \textbf{função} \(n(t)\), segundo a Metodologia de Galileo.

Um Modelo de Dinâmica Populacional consiste em: ``\textit{Uma caracterização matemática da função \(n(t)\) que representa a evolução temporal do tamanho da População, isto é, sua Dinâmica}''.


Tratando-se de ``Grandes Populações'' e do estudo de uma Dinâmica que envolve variações relativamente ``Pequenas'' de seu tamanho na escala de tempo \(T\) de interesse, o ``Efeito de Completamento Visual de Kanizsa'' nos \textbf{sugere} que \(n(t)\) seja representadas por uma função de variável contínua (real) diferenciável o que possibilita a sua caracterização matemática na forma de uma solução de uma Equação Diferencial segundo a Metodologia de Newton. O Modelo mais fundamental e simples (minimalista) de Dinâmica Populacional de Malthus-Euler com representação diferencial foi abordado no capítulo anterior e trata de uma população ``homogênea'' e ``não interativa'' na qual ocorrem unicamente processos vitais de reprodução e morte.

Em termos mais específicos, uma população malthusiana é aquela em que o tempo médio de espera para um indivíduo se reproduzir \(\nu^{-1}\) e a sua expectativa média de sobrevivência \(\mu^{-1}\) são características biológicas únicas da população, o que automaticamente possibilita a sua descrição pelo fundamental Modelo matemático de Malthus-Euler:
\begin{equation}\label{eq:Malthus-Euler}
\dfrac{1}{n}\dfrac{dn}{dt} = \nu-\mu.
\end{equation}


As Dinâmicas Populacionais para as quais o efeito Kanizsa não é assumido (por intenção ou imprecisão) são descritas por funções \(n(k)\) de variável inteira \(k \in \mathbb{N}\) e valores \(n\) reais, em que os respectivos Modelos Matemáticos são representados por Equações Recursivas que caracterizam Newtoniamente a função \(n(k)\), como no caso de Fibonacci e Malthus-Euler.

\comentario{(Dinâmicas de pequenas populações também podem ser descritas probabilisticamente segundo o Princípio de Poisson-Kolmogorov por uma sequência infinita de funções discretas \(P_n(t) \), a \textbf{Probabilidade} da População ter \(n\) indivíduos no instante \(t\). Neste caso, o Modelo Matemático é representado por um sistema também infinito de equações diferenciais ordinárias, tema a ser tratado especificamente no capítulo sobre Princípios Probabilísticos.)}

A condição de homogeneidade para a população malthusiana é, obviamente, muito radical para que seja pressuposta na maioria dos fenômenos biológicos uma vez que as características vitais de mortalidade e natalidade são fortemente dependentes de vários fatores biológicos individuais, especialmente a idade, e, portanto, altamente variáveis para indivíduos de uma grande população humana.

A ideia fundamental utilizada por Euler para o tratamento da dinâmica de populações heterogêneas é muito bem exemplificada em seu modelo demográfico e consiste em simplesmente parcelar a população total em subpopulações de tal forma que para cada uma delas a hipótese de homogeneidade vital seja plausível e, assim, admita uma descrição segundo o Modelo malthusiano.

Enfim, a estratégia de Euler é representar a descrição de uma população heterogênea em termos de subpopulações que podem ser consideradas homogêneas segundo o Modelo malthusiano.

Embora esta ideia seja simples, a sua formulação apropriada em linguagem matemática não é tarefa para amadores e sua realização constitui um dos Princípios básicos da Matemática Aplicada contemporânea. 

\comentario{(Observamos que, neste modelo, permanece a condição de não-interatividade entre os indivíduos exigida pelo Modelo minimalista malthusiano; apenas a homogeneidade da população total será relaxada. A inclusão de interações será tema de um capítulo posterior).}

Assim, por exemplo, o modelo demográfico de Euler para a população brasileira poderia considerar os seus 220 milhões de indivíduos em subpopulações referentes a faixas etárias (digamos, de 0 a 5 anos, de 5 a 10 anos e daí por diante), com tamanhos respectivos denotados por \(n_i(t)\), indivíduos com idade na faixa \([i, i+5)\) no instante \(t\), de tal forma que \(n = \sum_{i=0}^{95} n_i\) (excetuando os centenários!). A descrição completa de uma população como união de uma quantidade finita de subpopulações disjuntas é denominada \textbf{Estruturação Discreta} e representada por uma matriz, neste caso \((n_0, \ldots, n_k, \ldots, n_{95})\), onde cada coordenada \(n_i\) se refere ao tamanho da respectiva sub-população.

Considerando assim, parcimoniosamente, apenas um aspecto da heterogeneidade humana, Euler estabeleceu que os parâmetros populacionais vitais de reprodução e mortalidade seriam completamente determinados e uniformes para cada subpopulação das estreitas faixas etárias, o que permite a aplicação do \textbf{Modelo Minimalista malthusiano} separadamente à cada uma delas. {\tiny \color{red!50} (Aliás, uma ideia cujo embrião já se encontra de forma quase explícita no problema de Fibonacci, 500 anos antes, quando ele separa uma população em duas subpopulações, de imaturos e reprodutores!).}

\comentario{Uma vez inventada a ideia da Estruturação Etária, é quase imediata a tentação de estender a estratégia distributiva de Euler e estruturar uma população segundo outros critérios, de maneira a aperfeiçoar a justificativa de homogeneidade malthusiana para as subpopulações resultantes. Por exemplo, a população brasileira poderia ser estruturada não apenas segundo faixas etárias, mas também segundo a sua situação econômica, estabelecendo-se faixas de poder aquisitivo, \([a, a+1), 0 \le a < A\) (em unidade monetária apropriada) que fosse registrável para cada indivíduo. Com isto, a população total seria estruturada segundo uma \textbf{matriz} \(M\) cujas entradas \(m_{ia}\) representariam a população com idade na faixa etária \([i, i+1)\) e poder aquisitivo \([a, a+1)\) e os respectivos parâmetros malthusianos seriam representados na forma \(\mu_{ia}\) e \(\nu_{ia}\).

Analogamente, a estruturação segundo a localização geográfica em um plano é de grande interesse em muitas situações, mas exigiria a introdução de duas novas ``dimensões'', cada uma delas referente a uma das coordenadas. A estruturação de uma população segundo \(m\) critérios independentes exige, portanto, uma Tabela numérica \(m\)-dimensional para representá-la. Embora a estruturação múltipla seja indispensável em situações importantes, especialmente com respeito à localização espacial (v. Capítulo sobre Princípios de Discretização), torna-se claro que a tentação generalizadora é rapidamente refreada diante das óbvias complicações resultantes do aumento das coordenadas que, portanto, devem ser parcimoniosamente acrescentadas.}

Desta forma, um modelo demográfico de Euler, de 1748, estabelece de partida uma \textbf{Estruturação Discreta Etária} da população considerando um número finito \(m+1\) de faixas etárias, regularmente espaçadas segundo uma unidade de tempo. Em linguagem matemática contemporânea, o \textbf{Estado} da população (isto é, a totalidade das informações disponíveis) neste modelo é descrito em qualquer instante por uma grande matriz (\textbf{vetor}) \(N = (n_0, n_1, \ldots, n_m) \in \mathbb{R}^{m+1}\), em que \(n_i\) representa o número de indivíduos da subpopulação encontrada na \(i\)-ésima faixa etária, \(0 \le i \le n\), em que o tamanho da população total é \(n(t) = \displaystyle\sum_{i=0}^{n} n_i(t)\).

Ao contrário do modelo simples em que todas as informações sobre a população se resumiam a um único número \(n\), neste caso, dizemos que o ``\textbf{Estado}'' da população estruturada é representada por uma matriz coluna \(N \in \mathbb{R}^{n+1}\).

\comentario{(Embora os vetores representativos de uma população estruturada sempre tenham coordenadas com valores racionais positivos, a sua representação matemática em um espaço vetorial mais inclusivo (\(\mathbb{C}^{m+1}\), por exemplo) permite a utilização da estrutura vetorial destes espaços e, portanto, uma maior liberdade operacional).}

Naturalmente, as Dinâmicas destas \(n+1\) subpopulações não são desacopladas, isto é, independentes, pois o envelhecimento contínuo ``transporta'' todos os indivíduos sobreviventes de uma faixa etária para a seguinte em cada unidade de tempo (por exemplo, no intervalo de cinco anos). Além disso, a primeira faixa etária \(N_0\) (de indivíduos com idade entre 0 e 5 anos) tem um papel especial neste modelo uma vez que é a única faixa etária que recebe os indivíduos recém-nascidos.

Consideremos, agora, a unidade de tempo igual ao mesmo intervalo que define uma faixa etária, no caso, cinco anos. Portanto, a evolução temporal (história) desta população poderá ser visualizada como a trajetória de um ponto \(N(t)\) no espaço \(\mathbb{R}^{m+1}\), ou seja, uma função \(N: \mathbb{N} \to \mathbb{R}^{n+1}\).

Construído o cenário matemático apropriado para descrever o fenômeno populacional, resta estabelecer o Modelo Matemático para a Dinâmica que deverá gerar a evolução temporal de \(N(t)\) em \(\mathbb{R}^{n+1}\).

Assumindo, portanto, uma homogeneidade malthusiana para cada uma destas faixas etárias caracterizadas pelos parâmetros vitais \(\nu_i, \mu_i\) (na unidade de tempo já estabelecida), concluímos que entre os instantes \(t\) e \(t+1\) os seus sobreviventes são em número de \(e^{\mu_i}\ n_i(t)\) indivíduos que passarão a ser contabilizados na faixa etária seguinte, ou seja, \(i+1\)-ésima no momento \(t+1\). Portanto, \(n_{i+1}(t+1) = e^{-\mu_i} n_i(t)\), para \(i \ge 0\).

Por outro lado, neste mesmo intervalo de tempo (entre \(t\) e \(t+1\)) nascem \(\displaystyle\sum_{i=0}^{n} e^{\nu_i} n_i(t)\) indivíduos, que serão contabilizados na \textbf{primeira} faixa etária no instante \(t+1\), ou seja, \(n_0(t+1) = \displaystyle\sum_{i=0}^{n} e^{\nu_i} n_i(t)\). Como todos os sobreviventes de \(n_0(t)\) foram transferidos após este intervalo de tempo para a segunda faixa etária no instante \(t+1\), temos, \(n_1(t+1) = e^{-\mu_0} n_0(t)\).

Reunindo todas estas informações, podemos escrever o Modelo de Euler-Malthus na seguinte formulação operacional:
\[N(t+1) = \mathcal{L}(N(t)).\]

A evolução temporal desta população é, portanto, representada pela trajetória descrita por um ponto \(N(t)\) em \(\mathbb{R}^{m+1}\) \textbf{gerada} recursivamente a partir do seu Estado inicial \(N(0)\) pela aplicação sucessiva do operador \(\mathcal{L}\). Este operador \(\mathcal{L}\) é denominado \textbf{Gerador} da Dinâmica do sistema e, neste caso, é representado por uma matriz quadrada positiva de ordem \(m+1\), denominada \textbf{Matriz de Leslie}. (v. H. Caswell).

O operador \(\mathcal{L}\) pode ser interpretado como o ``\textit{motor}'' da dinâmica do modelo de Euler e a sua estrutura matemática reflete todas as características biológicas vitais (mortalidade e fertilidade) da população estruturada em faixas etárias.

Esta é uma formulação do Modelo (discreto) de Euler em uma nomenclatura contemporânea.

\comentario{A escolha de uma estrutura discreta (i.e., aritmética) para a formulação de seu modelo demográfico original, provavelmente foi motivada por uma deferência especial de Euler ao fato de que poucos dominavam os métodos (e quase ninguém entendia a teoria) do Cálculo Diferencial àquela época, e os demógrafos, provavelmente, muito menos. A boa intenção de Euler, todavia, resultou-se desastrosa sob o ponto de vista da propaganda uma vez que, por motivos opostos, nem demógrafos nem matemáticos se interessariam pelo seu trabalho ao longo dos dois séculos seguintes!}

É razoável supor que estreitando as faixas etárias (e aumentando o seu número) a hipótese de homogeneidade vital se torne cada vez mais acurada já que os indivíduos de cada uma destas subpopulações exibirão uma menor variação de idades e uma menor heterogeneidade vital.

Seguindo com este argumento, seria imaginar a formulação de um modelo demográfico no limite em que ``\textit{subpopulações com homogeneidade vital perfeita}'' seriam atingidas e, no qual, tanto as faixas etárias quanto o tempo tomariam um sentido de ``\textit{variação contínua}''.

A interpretação do ``valor limite'' da função, todavia, traria dificuldades consideráveis pois, a rigor, cada ponto da reta real positiva (idade) representaria uma subpopulação etária (?) e a soma de todas elas careceria de sentido, pois teríamos uma quantidade não enumerável delas.

A solução de Euler para esta dificuldade conceitual foi desenvolver de saída um modelo demográfico contínuo utilizando para isso as mesmas ideias e técnicas empregadas por ele na formulação do seu Modelo para a Dinâmica de Fluidos.

Apresentaremos, abaixo, o modelo demográfico contínuo como possivelmente Euler o teria feito caso se dirigisse a um público mais restrito e com maior habilidade Matemática.

O conceito de ``\textit{População Continuamente distribuída}'' (ou, ``\textit{População Estruturada}'', segundo alguns autores contemporâneos) e com dependência contínua com relação ao tempo introduzido por Euler, mostrou-se um instrumento de grande versatilidade demonstrada pela sua posterior utilização na construção de Modelos matemáticos para a descrição de coleções de indivíduos das mais variadas naturezas, desde a escala molecular (reações químicas e fluidos) até à escala astronômica (galáxias), passando por partículas quase macroscópicas (poluição), células (neurobiologia, imunologia), organismos microscópicos (vírus, bactérias, fungos) e ``macroscópicos'' (ecologia, demografia). Esta enorme abrangência exemplifica a capacidade de síntese e de representação que esta classe de Modelos oferece, razão da sua centralidade na Matemática Aplicada moderna e, particularmente, na Biomatemática. (W. C. Ferreira Jr. - ``\textbf{Dinâmica de Populações}; De angströms a km, de íons a sapiens'' Revista Comciencia 10/02/2002 online: \href{http://www.comciencia.br/comciencia/}{http://www.comciencia.br/comciencia/}, Lin-Segel).

Além disso, segundo a Metodologia de Newton, que foi abraçada e desenvolvida com entusiasmo por Euler, um Modelo Matemático deve ser preferencialmente descrito por funções de variáveis contínuas (reais) e caracterizadas matematicamente como solução de uma equação diferencial, o que submete o problema à poderosa teoria do Cálculo com sua conhecida capacidade de produzir uma enorme biblioteca de funções.

Este é o tema da próxima seção.

\section{DISTRIBUIÇÃO (ESTRUTURAÇÃO) CONTÍNUA DE POPULAÇÕES HETEROGÊNEAS E A FUNÇÃO DENSIDADE}

Aumentando a quantidade de faixas etárias (e estreitando a amplitude delas) no modelo demográfico discreto, é altamente plausível que a hipótese de homogeneidade malthusiana também se torne progressivamente mais adequada para cada faixa separadamente. Por outro lado, esta vantagem biológica é rapidamente descontada pela dificuldade matemática da análise e do cálculo algébrico-numérico com matrizes que a maior dimensão do Modelo acarreta.

\comentario{(Além disso, o conceito de ``grande população'' para estreitas faixas etárias pode perder o significado).}

A primeira opção para evitar esta dificuldade seria considerar uma situação limite em que a população seria distribuída continuamente ao longo da idade e a representação de seu estado em cada momento realizada por uma função de variável real contínua. Embora este cenário limite fosse sugestivo, ele conduzia às dificuldades já mencionadas de interpretação.

A estratégia matemática de Euler para contornar estas questões foi direta e aborda de saída uma formulação contínua para o modelo populacional como, de resto, já havia sido feito por ele no seu modelo de Dinâmica de Fluidos.

\comentario{De fato, um modelo discreto não apresenta, em princípio, nenhuma vantagem ou precedência conceitual sobre um modelo contínuo que justifique a derivação de um a partir do outro. Na verdade, o enorme desenvolvimento e sucesso do Cálculo nas mãos de matemáticos habilidosos, como o próprio Euler, tornou vantajosa a representação de difíceis problemas discretos na forma de problemas diferenciáveis equivalentes.

Uma das principais estratégias comumente utilizada em Matemática Discreta para a representação de uma grande tabela (ou, sequencia) numérica por intermédio de uma única função analítica é denominada \textbf{Método da Função Geradora} que, à propósito, foi também amplamente utilizado, se não inventado, pelo mesmo Euler. [D. Knuth \& al. - Matemática Discreta]. A representação da Função Fatorial-Gama de Euler na forma integral
\[\Gamma(n) = n! = \displaystyle\int_{0}^{\infty} t^n\ e^{t}\ dt,\]
é o exemplo típico e mais bem sucedido deste procedimento.

Segundo George Polya, ``\textit{a função geradora é uma maneira conveniente de transportar uma grande quantidade de objetos em uma única sacola}'', o que, não somente tem vantagens mnemônicas, mas também operacionais. Entretanto, se por um lado, a representação de uma sequência por uma Função Geradora pode ser conveniente sob o ponto de vista matemático, por outro, em geral, ela não representa conceitos biológicos de maneira natural o que dificulta a sua interpretação como um modelo populacional contínuo. A estrategia de Euler para a representação de uma grande população continuamente distribuída introduz um argumento revolucionário totalmente distinto do Método de Função Geradora que foi inventado \textit{ab ovo} por ele especificamente para este fim.}

Utilizando uma linguagem matemática contemporânea para as ideias de Euler, introduziremos, inicialmente, o conceito de \textbf{Espaço Etário} contínuo, (ou de Estrutura Etária contínua) \(A = \{x \in \mathbb{R}, x \ge 0\}\), onde cada indivíduo da população é registrado/representado por um ponto na posição respectiva de sua idade \(x \ge 0\). A população se torna, portanto, ``virtualmente'' representada pelo conjunto de todos os pontos que registram as idades respectivas de seus indivíduos na semirreta real positiva \(A\) que, desta forma, pode ser visualizada como uma ``nuvem'' contínua sobre a reta real. É exatamente esta representação mental, induzida pelo Efeito de Completamento Visual (descrito com detalhes dois seculos depois de Euler pelo psicólogo Gaetano Kanizsa) que sugere fortemente a seguinte abordagem para a formulação de um Modelo Matemático contínuo para a Dinâmica Populacional.

Para representar matematicamente toda a informação sobre a população em estudo (isto é, o ``\textbf{Estado}'' da população), Euler, em um lance de característica genialidade, introduziu o conceito de ``\textbf{Função Densidade}'' (ou, \textbf{Função de Distribuição}) \(\rho: A \to \mathbb{R}^{+}\), cujo significado (como Modelo Matemático) é definido pela seguinte interpretação de suas integrais:
\[\displaystyle\int_{x_1}^{x_2} \rho(x)\ dx\] ``Quantidade de indivíduos da população cujas idades estão entre \(x_1\) e \(x_2\)''.

A população total \(N\), em particular, é dada pela integral em toda a semi-reta positiva:
\[N = \displaystyle\int_{0}^{\infty} \rho(x)\ dx.\]

Neste Modelo, o \textbf{Estado} de uma população heterogênea, isto é, o conjunto completo de informações que são necessárias e suficientes para descrever a sua Dinâmica, será representado por uma função de variável contínua, \(\rho: A \to \mathbb{R},\ x \mapsto \rho(x)\), segundo a interpretação definida acima, e não mais por uma matriz de números.


\comentario{A grande síntese desta ideia consiste na representação do Estado de uma grande população heterogênea por um único objeto matemático, a função densidade \(\rho(x)\), o que possibilitará sua caracterização por intermédio de uma equação diferencial, segundo o figurino da Metodologia de Galileo e Newton. A representação etária de uma grande população como um fluido contínuo (mais do que uma ``nuvem'' ligeiramente esparsa de pontos) é psicologicamente sugerida (ou mesmo imposta) pelo Efeito Kanizsa de completamento da imagem.}


A função densidade \(\rho(x)\) encapsula, portanto, não apenas a informação fundamental que nos \textbf{interessa} conhecer sobre a população (i.e., o número total de indivíduos), mas também as informações que são \textbf{necessárias} para que o argumento de Malthus possa ser aplicado com mais propriedade na formulação de um modelo matemático para a sua dinâmica.

\comentario{Esta situação é muito semelhante ao que ocorre na Mecânica Newtoniana, onde as posições de um sistema de partículas em cada instante é um ``Estado'' insuficiente para descrever a sua própria dinâmica. Para obter uma dinâmica \textbf{autônoma} ou Markoviana (isto é, cuja evolução de um Estado dependa apenas do Estado presente) Newton mostrou a necessidade de \textbf{acrescentar às posições das partículas} também as suas velocidades. A descoberta do Estado suficiente, por Newton, denominado \textbf{Espaço de Configuração} (que consta da posição e da velocidade), foi a ideia fundamental para a formulação de uma recursão infinitesimal autônoma na forma das ``Leis diferenciais da Dinâmica''.}

A evolução temporal (história) desta população será, agora, descrita matematicamente por uma função \(\rho(x,t)\) que, para cada instante \(t\), representa o Estado momentâneo da referida população, ou seja, \[\displaystyle\int_{x_1}^{x_2} \rho(x, t)\ dx,\] o ``\textit{Número de indivíduos da população que tem suas idades entre \(x_1\) e \(x_2\) no instante \(t\)}''.

Utilizando uma nomenclatura matemática análoga ao modelo discreto, diz-se que a evolução temporal de uma população continuamente distribuída é descrita por uma trajetória no espaço de funções densidade representada por uma função \(\rho(x,t)\), de tal forma que, a cada instante \(t\) está definida uma função ``da variável \(x\)'' \(\rho_t(x) = \rho(x, t)\).

Para definir matematicamente esta função \(\rho(x, t)\), segundo a Metodologia Newtoniana, será necessário caracterizá-la como solução de uma equação diferencial. Portanto, ao contrário do modelo discreto de Euler, em que o Estado da População era representado por um ponto \(N \in \mathbb{R}^{n+1}\), e a sua Dinâmica descrita por uma trajetória neste mesmo espaço gerada recursivamente na forma do algoritmo \(N(t+1) = \mathcal{L} N(t)\), no caso contínuo, o Estado da População é representada por uma função densidade ``da variável \(x\)'', e a trajetória se dará em um Espaço Funcional de dimensão \textit{infinita} \(E = \{\sigma: A \to \mathbb{R}\}\), em termos de um \textit{algoritmo infinitesimal} que gera esta trajetória, isto é, representada por uma equação diferencial da forma \[\dfrac{\partial \rho}{\partial t} = \mathcal{L} \rho,\] em que o ``\textit{Gerador}'' da dinâmica \(\mathcal{L}\) é agora uma operação funcional (diferencial e/ou integral) em \(E\).

Este será o tema da próxima seção.

\comentario{
\subsubsection*{OBSERVAÇÕES}:

\begin{enumerate}
\item A recorrência histórica cíclica do Modelo de Euler

O modelo de dinâmica populacional de Euler (1760) é uma lição de Matemática Aplicada em seu melhor estilo e um dos primeiros trabalhos de ``Biomatemática'' quando este termo ainda teria que esperar por mais de dois séculos para ser definido. Embora o artigo original de Euler tenha sido citado por ninguém menos do que Charles Darwin em um dos mais famosos textos da Ciência, ``\textit{The Origin of Species}'', mesmo assim, a sua importância ou sua mera existência, foi misteriosamente ignorada em todos os níveis e áreas da Biologia durante os dois séculos seguintes à sua publicação. Basta mencionar que a mesma equação tem sido ``redescoberta'' , sem referências ao trabalho de Euler, inúmeras vezes por eruditos personagens como, por exemplo, os importantes (proto)biomatemáticos A. Lotka (\(\sim\)1907), W. Kermack \& A. McKendrick (\(\sim\)1920), H. von Foerster (\(\sim\)1958) e, certamente, por muitos outros. O artigo original de Euler foi reproduzido no primeiro número da prestigiosa revista Theoretical Population Biology, em 1970, durante a comemoração (atrasada) do bicentenário de sua publicação, talvez para fazê-la, enfim, universalmente conhecida de uma vez por todas. Sob o ponto de vista matemático este interessante modelo também foi totalmente ignorado nos últimos 200 anos. Para verificar este fato, basta constatar a sua inexistência na maioria dos textos matemáticos clássicos de Equações Diferenciais Parciais e de Física Matemática. (Riemann, Goursat, Smirnov, Courant-Hilbert, Petrovskii, Garabedian, Evans, e etc.)

\item A Generalização do Conceito de Função

Ao contrário da prescrição clássica (atribuída a P. L. Dirichlet, 1828) para a definição de uma função, em que ela é determinada pelo conjunto de todos os seus valores pontuais, na teoria de Euler, a caracterização de uma função densidade \(\rho(x)\), se dá pelo conhecimento do valor de suas integrais em uma classe de ``conjuntos testes'' ou, sob um ponto de vista do modelo matemático, como depositária da informação sobre o ``número de indivíduos'' (subpopulação) de cada intervalo \([x_1, x_2] \subset \mathbb{R}^{+}\).

O conceito de função densidade segundo Euler é um autêntico ``\textit{ovo de Colombo}'', se lembrarmos que, desde a formulação da Mecânica de Partículas de Newton para a Mecânica de Fluidos, passaram-se mais de meio século sem que os notáveis intelectos do período resolvessem a dificuldade de representar matematicamente a Dinâmica de uma grande quantidade de partículas, como, por exemplo, de um fluido! Apesar da clareza da ideia de Euler, muitos textos contemporâneos de Mecânica do Contínuo e Biomatemática ainda persistem definindo o significado da função densidade ``pontualmente'' com o enigmático ``limite experimental'':
\[\displaystyle\lim_{\substack{x \in \Omega \\ \operatorname{vol}(\Omega) \to 0}} \dfrac{\mbox{N. partículas em } \Omega}{\operatorname{vol}(\Omega)}.\]
Obviamente, este ``limite experimental'' não tem o sentido matemático do termo pois, caso contrário, para um volume extremamente pequeno (menor do que ``o de um átomo de hidrogênio'', por exemplo) esta fração não teria sentido algum. Mas, se este ``limite'' não for matemático, o que ele poderia ser então? O conforto da tradição de um lado e o ``\textit{silêncio dos conformistas}'' do lado discente, ajudou a preservar esta incongruência por gerações. (W. C. Ferreira Jr. - \textbf{O Silêncio dos Conformistas}, Conf. ERMAC-RS03 Dez. 2020 - I Encontro de Biomat. IMECC-Unicamp, 2017). Uma das poucas discussões lúcidas a respeito desta enganosa ``definição experimental``, encontra-se em Lin-Segel, 1974).

O conceito operacional de função introduzido por Euler, (isto é, em que uma função é definida matematicamente pelo valor de suas integrais) era tão revolucionário, que foram necessários aproximadamente 200 anos para a sua incorporação formal às estruturas da Matemática com os trabalhos de Laurent Schwartz (c.1950) precedido por Kurt-Otto Friedrichs, Sergei L. Sobolev e Israel M. Gelfand, (c.1930). O texto representativo da escola de Gelfand: ``Funções generalizadas'', 5 volumes, é uma notável expressão da Análise Matemática ``sensata'' do século XX, enquanto que o texto de Schwartz, ``\textit{Théorie des Distribuitions}'', Hermann, 1950, é uma abstração típica da escola francesa!

A generalização formal do conceito relacional clássico de função foi motivada pela bem sucedida utilização da famosa ``função de Dirac'' que se tornou o exemplo fundamental e protótipo de uma ``função'' sem qualquer representação clássica pontual. L. Schwartz, muito apropriadamente, designou suas funções de ``distribuições'', embora não tenha sequer mencionado Euler na atribuição de créditos, preferindo muito mais basear suas motivações nos trabalhos de Paul A. M. Dirac (\(\sim\)1920), aproveitando, obviamente, o glamour que a Física Quântica já desfrutava na época. A teoria de funções generalizadas e algumas de suas aplicações são encontradas nos clássicos de Gelfand \& al., e nos textos pedagógicos de V. S. Vladimirov - ``\textbf{Equations of Mathematical Physics}'', MIR, 1984, e ``\textbf{Generalized Functions in Mathematical Physics}'', MIR, 1979).



\item O Conceito de Estado na Física Estatística: James C. Maxwell e Josiah W. Gibbs.

 O desenvolvimento imediato das ideias de Euler para a representação matemática de populações se deu em grande parte ao seu interesse na Mecânica de Fluidos, que foi o objeto principal dos seus estudos. Posteriormente, durante a segunda metade do século XIX os métodos de Euler foram empregados na formulação da Mecânica Estatística segundo James Clerk Maxwell, J. Willard Gibbs e Ludwig Boltzmann, dentre outros. O objetivo original da Mecânica Estatística era explicar os fenômenos descritos pela Termodinâmica contínua e macroscópica como manifestações médias do movimento newtoniano de grandes populações (\(\sim10^{23}\) - Número de Avogadro) de partículas microscópicas segundo uma hipótese atômica da matéria ainda pouco aceita àquela época. Para se ter uma ideia da ousadia destes personagens, é importante lembrar que até a primeira década do século XX, a hipótese atômica da matéria (tão antiga quanto Demócrito) ainda era considerada como, no máximo, um modelo matemático abstrato e, às vezes, conveniente, mas ideologicamente combatida com vigor por figuras proeminentes da Ciência. O trágico suicídio de Boltzmann, em 1906, pouco antes da comprovação cabal da materialidade dos átomos por Jean Perrin, é atribuído, em parte, às contrariedades que esta disputa lhe impôs. (ref. C. Cercignani, E. Broda, S. Brush).

Dentre as várias e importantes contribuições teóricas de Maxwell, a introdução dos conceitos de \textit{Espaço de Fase e de População distribuída} no espaço de fase foi fundamental para o desenvolvimento da Mecânica Estatística. O princípio básico da Mecânica Newtoniana estabelece que o estado mecânico completo de uma partícula pontual (isto é, o conjunto de informações necessárias para prever algoritmicamente a trajetória futura de seu próprio estado mecânico) é constituído por sua posição espacial e velocidade. Isto compreende seis medidas contínuas no espaço, representável, portanto, na forma \((x, v) \in \mathbb{R}^6\), denominado Espaço de Fase, por Maxwell.

Assim, para descrever o Estado Mecânico completo de uma enorme ``população'' de \(N\) partículas microscópicas (\(\sim N = 10^{23}\) moléculas), pela teoria de Newton, seria necessário um espaço de dimensão colossal \(\mathbb{R}^{3N} \times \mathbb{R}^{3N} = \mathbb{R}^{6N}\). Obviamente esta quantidade de informações seria flagrantemente tão impossível de armazenar como inútil de se conhecer para o objetivo macroscópico em vista.

\comentario{(Observe-se, a propósito, que o problema gravitacional newtoniano somente é completamente solúvel por métodos elementares para o caso de dois (\(2 \times 10^0\)) corpos/partículas, conhecido como problema de Kepler, enquanto que o problema de três (03) corpos é insolúvel, mesmo com respeito a questões mais simples e, em particular, não solúvel explicitamente com funções elementares. (Ref. Moser, 1978, Diacu-Holmes, 1997, Barrow-Green, 1997).}

A estratégia idealizada por Maxwell para descrever satisfatoriamente esta enorme quantidade de informações microscópicas foi reduzi-las drasticamente até o ponto que ainda restasse informação suficiente para seus propósitos macroscópicos. Este é um exemplo de uma importante estratégia de Matemática Aplicada denominada como \textbf{Método de Redução} que segue exatamente a sugestão ``psicológica'' do efeito de Kanizsa.

De qualquer forma, ambos adaptaram as já centenárias ideias introduzidas por Euler em sua Dinâmica de Fluidos, interpretando o estado mecânico de um gás formado por \(N\) (\(\sim 10^{23}\)) partículas como uma enorme \textit{população contínua} de pontos \((x_k, v_k)_{1 \le k \le N} \in \mathbb{R}^{3} \times \mathbb{R}^{3} = \mathbb{R}^{6}\), cada um deles representando o estado mecânico de uma das \(N\) partículas. Em seguida, este conjunto de \(N\) informações vetoriais discretas é visualizado como uma ``nuvem contínua'' de pontos no Espaço de Fase \(\mathbb{R}^{6}\) e, com base em tal interpretação, ``homogeneiza-se'' esta massa gigantesca de informações discretas por intermédio de uma (única) função contínua de densidade \(\rho(x,v)\), cujo significado é definido pela descrição:
\[\displaystyle\iint_{\Omega_{v_1}} \rho(x,v)\ dx\ dv,\]
``\textit{a `Quantidade' de partículas cujos estados mecânicos \((x, v) \in \Omega \times \Lambda\)}''. (O. Buhler, A. Chorin \& O. Hald).

\item \textbf{A Função Densidade} como um ponto no espaço de funções

O modelo demográfico original de Euler descreve o Estado Dinâmico de uma população por um vetor \(N = (N_0, N_1, \ldots , N_i, \ldots, N_n)\) cujas respectivas coordenadas são identificadas por um índice discreto \(i\). Assim, a evolução temporal de uma população deste modelo discreto é completamente representável pela trajetória (poligonal) de um ponto \(N\) no Espaço de Aspecto \(E \in \mathbb{R}^{n+1}\), que pode ser descrita pela função \(N: \mathbb{N} \to \mathbb{R}^{n+1}\).

O modelo de tempo contínuo mantém a mesma interpretação do modelo original de Euler representando, agora, o seu Estado dinâmico por uma função densidade do espaço vetorial \(\mathcal{F} = \{h: \mathbb{R}^{n+1} \to \mathbb{R}\}\) para cada momento \(t\), por meio de uma função \(\rho(x,t)\), em que para cada \(t\), \(\rho_t(x) \in \mathcal{F}\).

O espaço vetorial funcional \(\mathcal{F}\), em que estão as possíveis funções densidade, é o espaço de Estado da Dinâmica (uma função de um espaço vetorial funcional \(\mathcal{F}\)) o que aproveita tanto a estrutura matemática de seu ``habitat'' (uma estrutura vetorial) quanto a sua visualização geométrica na forma de um ponto em \(\mathcal{F}\).

Assim, para preservar a mesma interpretação geométrica e matemática do modelo discreto, a função densidade \(\rho(x)\) será, também, interpretada como um \textbf{vetor} cujas ``coordenadas'' são, agora, identificadas por um ``índice contínuo`` \(x\), de tal forma que a cada \(x\) corresponde um valor \(\rho_x\). Desta maneira, a evolução temporal de uma população \(\rho(x,t)\) pode ser interpretada como uma curva contínua (e suave) em um espaço vetorial funcional \(\mathcal{F}\) que a cada instante \(t\) determina um ponto \(\rho(t) = \rho(x, t) \in \mathcal{F}\) (i.e., uma ``função de \(x\)''). Esta generalização formal na interpretação do modelo matemático tem um significado de grande alcance que possibilitará sua abordagem matemática no contexto da Análise Funcional conveniente para o desenvolvimento de Métodos Operacionais. (Chorin \& Hald).
\end{enumerate}
}


\section{ESPAÇO DE ASPECTO: REPRESENTAÇÃO DE HETEROGENEIDADES MÚLTIPLAS E CONTÍNUAS}

\comentario{
A Física Estatística, estabelecida por Gibbs, Maxwell e Boltzmann, dentre outros, ainda no século XIX foi, certamente, a mais vistosa descendente imediata das ideias introduzidas por Euler, no século XVIII, e de tal forma que muitas vezes é citada como sendo a própria fonte delas. Entretanto, constitui uma grave injustiça histórica e um engano pedagógico, não reconhecer que no longínquo ano de 1748, Euler utilizou, explicitamente, um conceito abstrato de estrutura etária que, sob certos aspectos, era mais ousado do que os trabalhos de Física desenvolvidos no século seguinte. A razão de fundo para tal equívoco histórico se deve, em grande parte, à já citada obscuridade que envolveu o artigo de Euler causada, possivelmente, por uma infeliz confluência: os biólogos não lhe deram atenção por conta de sua linguagem matemática (apesar da boa intenção de Euler em torná-lo simples), enquanto que os matemáticos não se deixaram impressionar pela sua simplicidade matemática. De qualquer maneira, não parece ser uma simples coincidência histórica e nem surpreendente que o fundador da Dinâmica do Meio Contínuo e da Dinâmica de Populações, tenha sido o mesmo personagem e que o modelo demográfico (1748) tenha sido desenvolvido na mesma época que as famosas Equações de Euler para a Dinâmica dos Fluidos (1754). (Ref: E. Fellmann, N. Bogolyubov, N. Keyfitz).

Para apreciarmos a importância e o alcance desta ideia, consideremos o problema de descrever uma grande população de indivíduos. À primeira vista, a ingenuidade nos leva a supor que, a rigor, seria indispensável nomear indivíduo a indivíduo e registrar todas as suas características individuais ao longo do tempo. Mas, obviamente, isto é tarefa para organismos de ``Inteligência'' e controle politico-social e não para Modelos Matemáticos de grandes populações. A Stasi, por exemplo, o `eficientíssimo' organismo de espionagem da antiga DDR-Alemanha Oriental, tinha uma pasta com informações incrivelmente detalhadas para cada um e de todos(!) os seus cidadãos. Este arquivo, que pode parecer a mais perfeita e completa informação sobre a população do país tornou-se, todavia, inútil pois, para examiná-la adequadamente, esta agência necessitaria de um corpo de funcionários muito maior do que sua própria população, na época, de aproximadamente \(10^6\) indivíduos! De fato, ao final desta loucura expiatória, quase todo mundo espionava todo mundo dentro da DDR e não se pode espantar com o fato de que tanta ``eficiência`` tenha falhado na previsão do seu colapso político! [ref. M. Wolf - \textbf{The Man without a Face}, e ``\textbf{A Vida dos Outros}'', um filme]. Em termos, este contrassenso é semelhante ao do personagem Funes, de José Luís Borges, que tinha uma memória tão boa e detalhista que, para relatar um dia qualquer do seu passado, perdia um dia todo do presente! Enfim, um dos procedimentos mais fundamentais e indispensáveis na abordagem científica de qualquer fenômeno natural é discriminar sensatamente a quantidade e qualidade de informação arquivada o que consiste na difícil administração de duas condições conflitantes: reduzi-a, radicalmente, para se tornar processável, mas com critério, para que possa conter alguma utilidade. Este procedimento pode ser denominado ``\textbf{Princípio de Mark Kac}'' segundo o autor da espirituosa frase: ``\textit{É preciso descartar muita informação, mas com o cuidado de não lançar fora o bebê junto com a água do banho}''.
}

O conceito de representação de uma população em um \textbf{Espaço de Aspecto} contínuo A, que generaliza biologicamente o conceito de Espaço Etário de Euler foi (re)introduzido na Biomatemática moderna somente na década de 1970, mais de duzentos anos depois do trabalho demográfico de Euler. (L. A. Segel in di Prima R. ed. 1974).

Um \textbf{Espaço de Aspecto} é a estrutura matemática contínua \(\mathbb{R}^{n}\) em que estão representados todos os indivíduos de uma população heterogênea segundo \(n\) aspectos biológicos individuais que os distinguem e que são continuamente mensuráveis de tal forma que cada indivíduo é registrado por um ponto \(x = (x_1, \ldots , x_n) \in \mathbb{R}^n\) onde \(x_k\) é a medida correspondente ao \(k\)-ésimo aspecto biológico sobre o mesmo.

O Modelo original de Euler registra uma única informação mensurável para cada indivíduo, a idade, obviamente representada em \(\mathbb{R}^n\) (ou em \([0, L] \subset \mathbb{R}^+\), para uma idade limite \(L\)).

A escolha do elenco de informações (aspectos) cujas medidas individuais serão registradas é crucial para a definição do Modelo de Dinâmica Populacional e será discutido caso a caso.

Se, em outras circunstâncias, for necessário que estes mesmos indivíduos sejam descritos também segundo suas localizações no plano, (digamos, com coordenadas cartesianas \(x_1\) e \(x_2\)) então o registro completo de cada indivíduo deverá consistir de três ``coordenadas'' \((x_1, x_2, x_3)\), onde \(x_3\) representará a idade. Outros aspectos mensuráveis como peso, carga viral e etc., se necessário conhecê-los, podem ser acrescentados à estrutura do modelo o que resultará, ao final, em \(n\) medidas individuais registradas como um conjunto ordenado representável na forma de um ponto \(x = (x_1, \ldots, x_m)\) no Espaço \(\mathbb{R}^m\).

Portanto, uma vez escolhidos os \(n\) aspectos numericamente mensuráveis que importam descrever sobre indivíduos de uma população, cada um destes passa a ser representado, ou registrado, por um ponto \(x = (x_1, \ldots, x_n) \in \mathbb{R}^n\) no chamado Espaço de Aspecto \(\mathbb{R}^n\), que é a estrutura matemática onde estão depositadas estas informações.

Esta representação obviamente não é necessariamente biunívoca e, com esta abstração, passamos a analisar os pontos do Espaço de Aspecto e não, especificamente, os indivíduos, uma vez que todas as informações desejadas e necessárias sobre a população estão registradas no conjunto de pontos deste Espaço. Entretanto, sob o ponto de vista intuitivo é conveniente associar cada ponto a um único indivíduo ``localizado'' em seu respectivo registro o que não é de todo absurdo considerando que, se mensurados com precisão suficiente, não haveria possibilidade de ambiguidades.

\comentario{(Por exemplo, considere o registro na reta real das idades ``precisas até o mínimo segundo'' dos indivíduos da população. Em \(\mathbb{R}\) há pontos mais do que suficientes para registrar biunivocamente a idade de todos os indivíduos de qualquer população).}

Naturalmente, para evitar um ``Efeito Pandora'' que pode resultar do entusiasmo com este procedimento, é necessário lembrar do Princípio de Parcimônia de Ockham que sugere a mais reduzida dimensão possível para o Espaço de Aspecto de um Modelo Matemático.

O processo de escolha do Espaço de Aspecto na construção de um Modelo Matemático de Dinâmica Populacional apresenta sempre uma tensão entre o desejável (a ser conhecido) e o necessário (para definir a Dinâmica) que é exemplarmente ilustrado pelo modelo demográfico de Euler. Embora o objetivo primordial de Euler com este modelo fosse descrever apenas a evolução do número total de indivíduos da população humana, mesmo assim ele verificou que seria \textbf{necessário} que o \textbf{Estado da População} em cada instante também registrasse a sua distribuição etária para que o modelo malthusiano pudesse ser utilizado com plausibilidade.

O objetivo deste capítulo é apresentar Modelos Matemáticos de Populações continuamente distribuídas (ou, ``estruturadas'') em Espaços de Aspecto enquadrando os conceitos e argumentos clássicos de Euler na linguagem matemática contemporânea que permita o seu emprego e análise para uma ampla variedade de fenômenos populacionais das mais diversas origens.

Uma vez definida a maneira de descrever a Dinâmica Populacional segundo a Metodologia de Galileo (isto é, representando-a na forma funcional como densidades variáveis no Espaço de Aspecto \(\rho(x, t)\)), o próximo passo consiste em expressar hipóteses biológicas na forma de condições matemáticas que resultem na caracterização destas funções como soluções de Equações Diferenciais e Integrais segundo a Metodologia Newtoniana.

Os \textbf{Princípios de Conservação}, introduzidos por Euler no século XVIII, constituem a Metodologia mais empregada e eficiente para caracterizar matematicamente a evolução temporal das funções densidade \(\rho(x, t)\).

\comentario{
\subsection{Apêndice: Espaços de Aspecto}

\subsubsection{Espaços de Aspecto Curvilíneos:}

O modelo de Espaço de Aspecto global pode, eventualmente, assumir uma estrutura matemática distinta do \(\mathbb{R}^n\), não pelo gosto da generalização, mas pela própria natureza do problema. Por exemplo, um processo dinâmico de cicatrização de tecidos planos (Murray, 1989), pode ser modelado com uma dinâmica de populações de células alongadas (chamadas fibroblastos que tem a tarefa de reconstruir rupturas de tecidos) sobre as quais é necessário descrever pelo menos duas características:
\begin{description}
\item a) A posição espacial de seu centro, digamos \(x \in \mathbb{R}\) e,
\item b) A sua orientação definida por um ângulo \(\theta \in S^1\).
\end{description}

Neste caso, o Espaço de Aspecto global para representar esta população é naturalmente a superfície de um cilindro, ainda que localmente seja um \(\mathbb{R}^2\).

\subsubsection{Espaços de Aspecto de Dimensão Infinita:}

Uma outra extraordinária, mas inevitável, generalização do conceito de Espaço de Aspecto foi introduzida por Alan Perelson e Lee Segel, em 1989, na formulação de modelos matemáticos para o sistema imunológico cuja população é constituída de uma enorme quantidade e variedade de células. Neste caso, as células devem ser necessariamente caracterizadas pela forma exterior de seu sítio reativo que determina completamente o comportamento interativo dela no sistema. Assim, o aspecto que caracteriza o indivíduo tem, de alguma maneira, que representar a ``forma'' de seu sitio reativo o que, matematicamente, somente é possível por intermédio de uma função e não um número. A rigor, portanto, o Espaço de Aspecto da população das células do modelo matemático do sistema imune deveria ser um Espaço Funcional de dimensão infinita (``Shape Space'').

Recentemente, esta estratégia também foi utilizada na construção de um modelo para o fenômeno de mimetismo em que o comportamento do predador (pássaro) depende de maneira crucial do reconhecimento dos padrões de asas de suas presas (borboletas). [Ferreira-Marcon, 2014]. Embora este conceito generalizado seja importante para analisar o modelo, ele apresenta dificuldades analíticas insuperáveis. Com o objetivo de torná-lo matematicamente tratável, alguns recursos matemáticos, derivados de sua interpretação, podem ser utilizados para simplificá-lo consideravelmente sem que se diminua a sua relevância. Por exemplo, o reconhecimento de Padrões geométricos das asas de borboletas tóxicas por seus predadores, pode ser aproximadamente realizado de maneira efetiva em um subespaço de dimensão finita (e razoavelmente pequena) constituído da percepção aguda de caracteres desenvolvidos evolutivamente pelo seu sistema cognitivo visual. Matematicamente, isto significa projetar o Espaço de Aspecto em um sub-espaço de pequena dimensão que concentra uma grande quantidade de informações sobre o padrão geométrico, que é suficiente para discriminá-lo com a acuracidade que o fenômeno exige. Esta estrategia é exatamente aquela utilizada na reconstituição policial do retrato falado de um criminoso por testemunhas ou, na elaboração de caricaturas excepcionalmente minimalistas e explica a extraordinária capacidade para a avaliação da intenção alheia desenvolvida pelo sistema cognitivo humano e de alguns animais com uma ``leitura facial''. (L. Sirovich-Eigenfaces, M. Kac - \textbf{Can you hear the shape of a drum}? AMM 1967, Widiger, 2017, George Miller (\(7 \pm 2\)), 1956, Ch. Darwin - \textbf{Observation about the expression of emotions by animals}).
}

\section{O PRINCÍPIO DE CONSERVAÇÃO apud EULER}

\subsection{INGREDIENTES FUNDAMENTAIS DE UM MODELO POPULACIONAL: Espaço de Aspecto e Densidade, Fluxo, e Fonte}

\begin{citacao}
``Sábios são aqueles que desvendam a complexidade do mundo por intermédio de coisas simples, ao contrário dos estultos que teimam em obscurecer as coisas mais simples''. ~Confucio.
\end{citacao}

O Modelo Populacional de Euler em sua versão diferencial, apesar de suas peculiaridades simplificadoras (como um Espaço de Aspecto unidimensional), pode ser considerado como o protótipo de uma abrangente teoria, uma vez que já apresenta em sua formulação o conceito de Espaço de Aspecto como um cenário de fundo apropriado onde são definidos os ingredientes fundamentais para o desenvolvimento do Princípio geral de Conservação.

A origem do conceito do \textbf{Princípio de Conservação} é uma observação absolutamente óbvia que não deve surpreender ninguém, a menos que seja por sua simplicidade.

Este Princípio, a bem da verdade, trata de uma ``\textit{Contabilidade}'' matemática instantânea dos indivíduos pertencentes a sub-populações definidas por regiões \(\Omega \subset A\) delimitadas do Espaço de Aspecto \(A\) e não exatamente de uma ``Conservação''. O termo ``conservação'' se refere mais ao fato de que há um controle dos processos que regulam o ganho e perda de indivíduos destas subpopulações por intermédio de suas representações matemáticas e à sua aplicação especifica à Dinâmica de Fluidos no que se refere à massa material do meio.

A representação matemática adequada para estes processos foi o ``\textit{Ovo de Colombo}'' da teoria de Euler, e a metodologia proveniente destas ideias constitui uma síntese de linguagem, conceitos e objetos matemáticos que possibilitam a descrição matemática de uma notável variedade de fenômenos naturais.

O Princípio de Conservação, portanto, tem por objetivo estabelecer um Modelo Matemático para a Dinâmica de uma População Heterogênea segundo uma distribuição contínua (estruturação) em um Espaço de Aspecto \(A \subset \mathbb{R}\) que, a cada instante \(t\), é completamente descrita pela sua densidade \(\rho_t(a) = \rho(a, t)\) como função de \(a \in A\) segundo a interpretação:
\[\displaystyle\int_\Omega \rho(a,t)\ dt,\]
``a Quantidade de indivíduos registrados na região \(\Omega \subset A\) do Espaço de Aspecto no instante \(t\)''.

A definição de um Espaço de Aspecto e, consequentemente, de sua função densidade estabelece assim o primeiro e mais fundamental ingrediente do Modelo geral de Euler.

Para estabelecer a contabilidade da sub-população definida por \(\Omega\), analisaremos a expressão:
\[\displaystyle\int_{t_1}^{t_2}\int_{\Omega} \rho(a,t)\ da\ dt,\]
a ``Variação do tamanho da subpopulação definida por \(\Omega\) no intervalo de tempo entre \(t_1\) e \(t_2\) ou em sua versão instantânea:
\[\displaystyle\dfrac{d}{dt}\int_{\Omega} \rho(a,t)\ da = 
\displaystyle\int_{\Omega} \dfrac{\partial}{\partial t}\rho(a,t)\ da,\]
a ``Taxa de variação do conteúdo populacional definido pela região \(\Omega\) no...?????

\comentario{
Naturalmente, uma derivada positiva indica uma taxa de crescimento instantâneo da subpopulação contida em \(\Omega\), no instante \(t\), e uma derivada negativa significa uma taxa de decréscimo instantâneo da mesma subpopulação, no instante \(t\). A variação de população é resultado de uma contabilidade final entre ganhos e perdas de indivíduos registrados em \(\Omega\).

Portanto, termos positivos que contribuírem para definição da expressão \[\displaystyle\int_{\Omega} \dfrac{\partial}{\partial t}\rho(a,t)\ da\]
se referem a uma taxa ``líquida'' de \textbf{introdução} de indivíduos ``registrados'' na região fixa \(\Omega\) do Espaço de Aspecto, enquanto que os termos negativos representam a taxa ``líquida'' de \textbf{retirada} de indivíduos ``registrados'' na mesma região.}

O passo crucial na formulação do Princípio de Conservação consiste em representar matematicamente os fenômenos biológicos que poderiam causar a variação do tamanho da população contida (i.e., registrada) em uma região fixa (mas suficientemente genérica) \(\Omega\) do Espaço de Aspecto. Estes fenômenos podem ser classificados de acordo com duas origens bem distintas e peculiares quanto à sua interpretação populacional, uma delas relativa à fronteira e a outra relativa a processos interiores da região:

1\textordmasculine\ - ``\textit{Passagem}'' de indivíduos através da fronteira \(\partial\Omega\) denominada ``Fluxo'' (para dentro, se positiva e, para fora, se negativa) representada matematicamente por integrais de superfície que utilizam informações contidas apenas na fronteira da região estudada,

2\textordmasculine\ - ``\textit{Nascimentos}'' (se acréscimos) ou ``\textit{Mortes}'' (se decréscimos) de indivíduos com respeito à região \(\Omega\) representados matematicamente por integrais de volume de funções denominadas ``\textbf{Densidade de Fontes}''.

\comentario{Observe-se que os termos ``Passagem'', ``Nascimento'' e ``Morte'' estão ressaltados em itálico para enfatizar que se referem aos registros dos indivíduos no Espaço de Aspecto e não necessariamente a uma presença física deles, embora utilizemos uma linguagem sugestiva que se refere à posição em \(A\) como se fosse de fato uma localização geográfica, e se refere ao aparecimento como se fosse nascimento e ao desaparecimento como se fosse a própria morte do indivíduo correspondente.

Por exemplo, tratando-se de uma reação química (digamos, a combustão de hidrogênio em reação com o oxigênio cujo resultado é a água) a densidade \(\rho(o, t)\) do oxigênio livre varia durante o processo de combustão, mas não por desaparecimento físico dos átomos de oxigênio e sim pela sua condição química de estar ``livre'' e que determina a sua pertinência à respectiva subpopulação.}

Em uma primeira abordagem pedagógica, utilizaremos o Modelo original de Euler como protótipo para exemplificar o Princípio de Conservação devido à conveniente vantagem de se tratar, neste caso, de um Espaço de Aspecto unidimensional \(A = \mathbb{R}^+ \subset \mathbb{R}\), o que simplifica consideravelmente a geometria do problema.

Em uma seção posterior, apresentaremos estes mesmos conceitos em Espaços de Aspecto \(A \subset \mathbb{R}^n\) multidimensionais que exigem construções geométricas mais sutis para representar conceitos análogos.

Portanto, no que se segue imediatamente, os conjuntos \(\Omega \subset A \subset \mathbb{R}\) para ``testes'' de variação de conteúdo populacional serão intervalos \(\Omega = [x_1, x_2]\), cujas fronteiras são constituídas pelos pontos extremos \(x_1\) e \(x_2\) e que no modelo demográfico contínuo de Euler representarão faixas etárias.

\subsection{O CONCEITO DE FLUXO \(J\) no Modelo unidimensional de Euler}

Obviamente, a passagem de indivíduos através de uma fronteira é resultado de sua movimentação no Espaço de Aspecto, ou seja, da variação das medidas de seus registros.
 
Para melhor entender a representação matemática desta ideia, imaginemos que a ``movimentação dos indivíduos'' no Espaço de Aspecto unidimensional analisando a sua passagem nos dois sentidos através de um ponto \(x\) fixo, tanto para à direita (direção positiva) como para à esquerda dele.

Denotando \(J^+(x)\) a taxa de passagem na direção positiva e \(J^-(x)\) a taxa de passagem na direção negativa, então \(J(x) = J^+(x) - J^-(x)\) é o ``\textbf{saldo líquido}'' deste processo em que o sinal tem significado de orientação. Assim, se \(J(x) > 0\), haverá um ``saldo líquido'' de taxa de passagem através de \(x\) no sentido positivo (muito embora alguns indivíduos possam eventualmente ter passado na direção negativa). Se, em outro caso, \(J(x) < 0\), o ``saldo líquido'' da taxa de passagem através de \(x\) será no sentido negativo.

Para a formulação do Princípio de Conservação (contabilidade) a introdução de \(J^+\) e \(J^-\) é meramente pedagógica, pois nos interessará apenas o ``saldo líquido'' de passagem \(J\) e não a ``fulanização'' deste tráfego.

Definiremos, então, o conceito de Fluxo através de um ponto \(x\) no Espaço de Aspecto unidimensional \(A \subset \mathbb{R}\), como sendo o ``\textit{saldo líquido da taxa de passagem}'' de indivíduos através de \(x\) no instante \(t\), que será denotado por \(J(x,t)\) e entendido, pelo argumento acima, como sendo\[J(x) = J^+(x) - J^-(x).\]

Portanto, se na fronteira à esquerda do conjunto \(\Omega = [x_1, x_2]\), isto é, no ponto \(x_1\), o Fluxo for positivo \(J(x_1, t) > 0\), isto significa que há uma contribuição instantânea para a taxa de aumento da subpopulação de \(\Omega\), enquanto que se for negativo \(J(x_1, t) < 0\), então há uma taxa de retirada de indivíduos da subpopulação de \(\Omega\). Por outro lado, se na fronteira à direita \(x_2\) do conjunto \(\Omega = [x_1, x_2]\) o fluxo for positivo \(J(x_2,t) > 0\), isto significa que há uma retirada instantânea de indivíduos da subpopulação de \(\Omega\), enquanto que se for negativo \(J(x_1,t) < 0\), então há uma taxa de acréscimo de indivíduos da subpopulação de \(\Omega\).

Assumindo a definição acima e a sua interpretação decorrente, se a variação da população contida em \(\Omega\) for unicamente causada pelo trânsito (intercâmbio) de indivíduos através de sua fronteira \(\partial\Omega = \{x_1, x_2\}\), (sem a ocorrência de ``morte'' ou ``reprodução'' interna de indivíduos), podemos representar matematicamente este fato com a EQUAÇÃO INTEGRAL DE CONSERVAÇÃO (Unidimensional):
\[\begin{array}{rcl}
\displaystyle\dfrac{d}{dt}\int_{\Omega} \rho(a,t)\ da
&=& \displaystyle\int_{\Omega} \dfrac{\partial}{\partial t}\rho(a,t)\ da \\[0.5cm]
&=& -J(x_2,t) + J(x_1,t) \\[0.5cm]
&=& \displaystyle\int_{x_1}^{x_2} -\dfrac{\partial J}{\partial x}\ dx
\end{array}\]
ou
\[
\displaystyle\int_{\Omega} \left(\dfrac{\partial \rho}{\partial t} + \dfrac{\partial J}{\partial x}\right) = 0,\ \forall\ 0 < x_1 < x_2, t > 0.
\]

Assumindo, agora, a \textbf{continuidade} da expressão \(\left(\dfrac{\partial \rho}{\partial t} + \dfrac{\partial J}{\partial x}\right)\), é possível concluir (via Argumento de Localização de Euler (v. Apêndice) que esta função é identicamente nula para \(x > 0, t > 0\), ou seja, a função densidade pode ser caracterizada matematicamente pela seguinte Equação Diferencial, denominada EQUAÇÃO DIFERENCIAL DE CONSERVAÇÃO (Unidimensional):
\[\dfrac{\partial \rho}{\partial t} + \dfrac{\partial J}{\partial x} = 0,\ \forall\ x > 0, t > 0.\]

É importante ressaltar que a continuidade da expressão
\[
\dfrac{\partial \rho}{\partial t} + \dfrac{\partial J}{\partial x},
\]
em todo o Espaço de Aspecto e para todo o instante \(t > 0\), é condição essencial para a obtenção do Princípio de Conservação na forma de uma Equação Diferencial.

\comentario{
\subsubsection*{OBSERVAÇÕES}:

\begin{enumerate}
\item Atente para os sinais dos Fluxos em cada extremidade do intervalo e suas interpretações.
\item O termo ``Conservação'' neste caso se justifica porque associamos à não ocorrência de ``mortes'' ou ``nascimentos'' como sendo a preservação, em algum ``lugar'' (ou forma), dos indivíduos representados na região \(\Omega\).
\item A Equação Integral de Conservação é mais geral do que a Equação Diferencial porque a precede na argumentação e a implicação entre as duas depende da continuidade do termo integrando da equação integral.
\item A Equação Integral de Conservação tem um caráter não local e é definida para ``todos os intervalos de tempo \([t_1, t_2]\) e todas as regiões \([x_1, x_2]\), enquanto que a Equação Diferencial é local sendo definida instantaneamente e pontualmente em \((x,t)\).
\item Em alguns casos relevantes para a Matemática Aplicada (especialmente em Dinâmica de Gases), a continuidade do integrando da Equação integral será explicitamente violada, devido a fenômenos dinâmicos intrínsecos (denominados ``choques'') que levam à formação de variações bruscas e com limite em tempo finito de descontinuidades de saltos da função de fluxo \(J\). Nestes casos, a equação integral permanece válida, mas a equação diferencial somente se aplica nas regiões exteriores ao ``choque'', o que usualmente ocorre em pontos que se movimentam. Para uma análise elementar deste fenômeno consulte R. Bassanezi - W. C. Ferreira Jr ou G. Whitham.
\item A preferência pela Equação Diferencial se deve ao fato de que não é necessário especificar intervalos de tempo ou regiões do espaço de aspecto, mas apenas pontos.
\end{enumerate}
}


\comentario{
\subsubsection{Apêndices}

A - \textbf{Argumento de Localização de Euler}: (Quase óbvio, mas indispensável!)

Se uma função contínua \(h: I = [A, B] \to \mathbb{R}\) for tal que para todo sub-intervalo \([a, b] \subset [A, B]\) tenhamos \(\displaystyle\int_{a}^{b} h(x)\ dx = 0\), então \(h\) é identicamente nula em \(I\).

\textbf{Demonstração}: Basta supor o contrário. Digamos que \(m = h(x_0) > 0\), para algum \(x_0 \in (A, B)\). Pela continuidade de \(h\), existe um subintervalo \([x_0-\delta, x_0 +\delta] \subset I\), em que \(h(x) > \dfrac{1}{2} h(x_0)\) e, portanto,
\[
\displaystyle\int_{x_0-\delta}^{x_0+\delta} h(x)\ dx \ge 2\delta \dfrac{m}{2}
= \delta m > 0
\]
é uma contradição. \\ (Se \(h(x_0) < 0\) considere \(-h(x)\)).

B - \textbf{Exemplo}: Fluxo do Modelo Demográfico de Euler

O modelo demográfico de Euler distribui a sua população em um Espaço de Aspecto etário \(A = \mathbb{R}^+\) e, portanto, todos os indivíduos (i.e., seus registros de idade) se movimentam para a direita (sentido positivo) sempre com velocidade unitária, isto é, envelhecem 1 dia por dia. Portanto, o fluxo através de cada ponto \(x\) será sempre no sentido positivo.

O total dos indivíduos que passarão por uma idade fixa \(x > 0\), no período de tempo \([t-\Delta t, t]\), será dado por
\[
\displaystyle\int_{x-\Delta t}^{x} \rho(s, t-\Delta t)\ ds
\]
(isto é, todos os indivíduos que se tornarão mais idosos do que a idade \(x\), estão registrados na faixa etária \([t , t +\Delta t]\) no instante \(t\). Portanto, a taxa de passagem de indivíduos através do ponto \(x\) (para a direita) por unidade de tempo, isto é, o Fluxo \(J(x,t)\), pode ser obtido da seguinte forma:
\[J(x,t) =\displaystyle \lim_{\Delta t \to 0} \dfrac{1}{\Delta t} \int_{x-\Delta t}^{x} \rho(s, t-\Delta t)\ ds = \rho(x,t).\]

O Princípio de Conservação diferencial para o Modelo de distribuição etária de Euler, considerando apenas envelhecimento, sem mortalidade ou reprodução, pode então ser representado pela equação diferencial parcial de 1\textordfeminine\ ordem, na forma:
\[\dfrac{\partial \rho}{\partial t} + \dfrac{\partial J}{\partial x} = 0,\ \forall\ x > 0, t > 0.\]

Observe-se que o ponto \(x\), neste caso, é sempre maior do que zero. Em \(x = 0\), o fluxo é singularmente distinto, uma vez que representará a taxa de nascimentos (que só ocorrem com esta idade!) e será tratado logo adiante.

C - \textbf{Relação Constitutiva para o Fluxo}

Ressalte-se que, em geral, o Fluxo \(J\) não é, necessariamente, uma função diretamente dependente de \((x,t)\) mas sim, indiretamente, ou seja, ela pode ser obtida como resultado de uma operação funcional \(\Phi\) aplicada a \(\rho\) na forma \(J(x,t) = \Phi[\rho](x,t)\).

Este funcional pode ser classificado em quatro tipos:

\begin{enumerate}
\item ``\textbf{Pontual}'': quando \(J(x,t)\) depende apenas do valor de \(\rho(x,t)\);
\item ``\textbf{Infinitesimal}'' ou ``\textbf{Local e Instantâneo}'': quando \(J(x,t)\) depende apenas do valor de \(\rho(x,t)\) em alguma vizinhança de \(x\) e \(t\), como no caso de derivadas.

Um exemplo particular sendo o importante fluxo de difusão, definido como \(J = -D \dfrac{\partial \rho}{\partial t}\) a ser tratado em próximos capítulos.)

\item ``\textbf{Não Local}'': quando \(J(x,t)\) depende de valores de \(\rho(s,t)\), para \(s\) em um intervalo finito; e
\item ``\textbf{Não Instantâneo}'' ou seja, \textbf{com memória}: quando \(J(x,t)\) depende de valores de \(\rho(x,\tau)\), para \(t - T \le \tau \le t,\ T > 0\).
\end{enumerate}


Exemplos de todos estes casos serão apresentados mais adiante.

D - \textbf{Princípio de Leibniz}: ``\textit{O Mundo Infinitesimal é plano e linear}''

Em textos de Matemática Aplicada e, principalmente, de Física, é comum encontrar frases ligeiras como ``... Desprezando termos de segunda ordem ...'' como justificativa na obtenção de algum resultado limite que envolve uma integral ou o cálculo de uma derivada.

Em poucos, ou quase nenhum destes casos, apresenta-se alguma indicação matemática ou argumento de evidência que torne esta afirmativa mais plausível ou aceitável além da força de autoridade (``\textit{Magister Dixit}'') que não é desprezível, já que tem sua origem em ninguém menos do que G. Leibniz (séc. XVII) um dos fundadores do Cálculo! Apesar disso, e surpreendentemente, com menos frequência, ainda se ouve alguma expressão de desconforto intelectual por parte de uma audiência conformista, e isto não se deve à trivialidade da questão e nem à sua total compreensão por todos/as.

\comentario{(A explicação para este fenômeno social pode estar na pedagogia pouco sutil instituída por d’Alembert que alegadamente repreendia seus pupilos mais inconformistas com a (para)frase: ``Cale a boca, continue calculando incessantemente e a fé um dia lhe chegará. Se ela não vier, desista da Matemática e vá procurar a sua vocação em outro lugar'').}

O mais notável é que uma argumentação razoavelmente simples e intuitiva a respeito pode ser apresentada na maioria dos casos.

Um exemplo importante por si mesmo e o protótipo destes argumentos é representado pela definição usual de comprimento de curvas e de área para superfícies ditas suaves (isto é, que admitem tangência de retas e, respectivamente, de planos em cada ponto).

Observe-se que a definição de tangência, segundo o conceito de Leibniz, é analiticamente expressa como a existência de uma aproximação linear local ótima. No caso de uma variável, isto pode ser descrito na forma:
\[f(x_0+h) = f(x_0)+Ah+h\epsilon(h),\]
onde \(\epsilon(h) \to 0\) quando \(h \to 0\). O ``erro'' \(h\epsilon(h)\) é denotado pelo símbolo \(o(h)\) que significa um erro de ordem menor do que de \(h\). A aproximação linear \(Ah\), com esta ``precisão'', se existe, é única, e claro, é a própria derivada da função, ou seja, \(f'(x_0) = A\)). (Ref. W. C. Ferreira Jr. - \textbf{As múltiplas faces da derivada I}, Ciência e Natureza - UFSM, 2014).

A \textbf{definição} do conceito geométrico de curvas retificáveis está associada ao conceito de mensuração de seu comprimento e deriva da percepção geométrica altamente intuitiva de que a razão entre o (pretendido) comprimento \(\Delta s\) de um pequeno arco em torno de um ponto e o comprimento (efetivamente mensurável) \(\Delta l\) do segmento obtido da sua projeção sobre a reta tangente neste ponto se aproxima de \(1\), ou seja,
\[\dfrac{\Delta s}{\Delta l} \to 1,
\mbox{ para } \Delta s \to 0,\]
o que é equivalente a dizer que
\[\Delta s = \Delta l + o(\Delta l).\]

\comentario{Esta afirmação também pode ser entendida como uma aproximação em escala logarítmica, ou seja,
\[\displaystyle\lim_{\Delta l \to 0} \log(\Delta s) - \log(\Delta l) = 0 \mbox{ (Verifique)}.\]}

Esta \textbf{afirmação} (axiomática) se constitui na própria definição do conceito da função comprimento de uma curva \(s\).

Uma ``medida'' \(s\), que satisfizer esta condição ``local'', será determinada por um limite integral como soma Riemanniana de segmentos lineares, ou seja, reduz-se (no limite) à medida já definida de comprimento de segmentos lineares, que, em princípio, é a única geometricamente definível.

Portanto, com base nesta ``linearização local'' e na sua interpretação geométrica, o comprimento de uma linha suave é definida da seguinte maneira:
\[\begin{array}{rcl}
s
&=& \displaystyle \lim \sum \widehat{(s_{k+1}-s_k)} \\
&=& \displaystyle \lim \sum [\overline{(l_{k+1} - l_k)} + \epsilon_k(\delta) \delta] \\
&=& \displaystyle\lim \sum [\overline{(l_{k+1} - l_k)}],
\end{array}\]
onde \(\widehat{(s_{k+1}-s_k)}\) é o ``comprimento do arco'', \([\overline{(l_{k+1} - l_k)}]\) é o comprimento do segmento retilíneo projetado sobre a tangente.

A justificativa da igualdade no limite decorre do fato de que os erros de ordem \(o(h) = \epsilon_k(\delta)\delta\) se aproximam uniformemente de zero (quando no limite integral o segmento de máximo comprimento se aproxima de zero), de onde temos:
\[\left|\displaystyle\sum [\epsilon_k(\delta)]\delta\right|
\ge
\left(\max |\epsilon_k(\delta)| \right) \sum \delta \to 0.\]

\comentario{Observe-se que este argumento não é uma ``demonstração matemática'' porque, afinal, estamos tratando de um conceito exterior à Matemática (comprimento de curvas que, em princípio, ainda não foi definido); o objetivo é exatamente o de tornar plausível esta definição! Além disso, que para a lisura do argumento acima, é necessário que o erro local seja considerado como de segunda ordem e uniforme com relação ao ponto central.}

Sob um ponto de vista matemático, a linearização de um argumento decorre da possibilidade de substituir o valor de uma função \(f\) (cujo gráfico é retificável, ou seja, diferenciável) nas imediações \(x_0 + \delta = x\) de um ponto \(x_0\), por uma aproximação linear, cujo erro é de ``ordem infinitesimal superior'' a \(\delta\), o que, na verdade, é exatamente a definição de derivada segundo Leibniz: \(f(x) = f(x_0 + \delta) = f(x_0) + A\delta + o(\delta)\), em que \(o(\delta)\) é uma função avaliada na forma \(|o(\delta)| \le \delta\ \epsilon(\delta)\) onde \(\epsilon(\delta) \to 0\) quando \(\delta \to 0\) e, obviamente, \(A = f'(x_0)\).

Assim,
\[\begin{array}{rcl}
& & \displaystyle\int_{a}^{b} f(x)\ dx \\
&=& \displaystyle\lim_{\max \Delta_k \to 0} \sum_{k=0}^{N} f(x_k) \Delta_k \\
&=& \displaystyle\lim \sum_{k=0}^{N} f(\xi_k) \Delta_k \\
&=& \displaystyle\lim \sum_{k=0}^{N} \{f(\xi_k) +f'(\xi_k) (x_k-\xi_k) \\
&& + o(x_k-\xi_k)\} \Delta_k
\end{array}\]
se \(|x_k-\xi_k| < \max|\Delta_k|\), pois supondo que a estimativa seja uniforme, podemos escrever:
\[\begin{array}{l}
\left|\displaystyle \sum_{k=0}^{N} \{f(\xi_k) +f'(\xi_k) (x_k-\xi_k) + O(x_k-\xi_k) \} \Delta_k\right| \\
\le C\max|\Delta_k|,
\end{array}\]
de onde vem o argumento.

Enfim, em linguagem mais coloquial, este resultado implica que é possível substituir a função integrando por uma outra em cada parcela da soma integral, se o erro cometido for uniformemente de ordem menor do que \(\max|\Delta_k|\) e isto significa substituir localmente \(\Delta_k\) por sua aproximação linear.

O Cálculo de razões no limite (derivadas) também pode ser substituído por sua aproximação linear, ou seja, formalmente:
\[\begin{array}{rcl}
\displaystyle\lim \dfrac{\Delta f}{\Delta g} 
&=& \displaystyle \lim_{\delta \to 0} \dfrac{f(x_0+\delta)-f(x_0)}{g(x_0+\delta)-g(x_0)} \\[0.5cm]
&=& \displaystyle \lim_{\delta \to 0} \dfrac{f'(x_0) \delta + o(\delta)}{g'(x_0) \delta + o(\delta)} \\[0.5cm]
&=& \displaystyle \lim_{\delta \to 0} \dfrac{f'(x_0) \delta}{g'(x_0) \delta}
\end{array}\]


Estes fatos genéricos são a explicação da afirmação também genérica comumente utilizada de que (graças à concepção Leibniziana do Cálculo) ``\textit{Localmente e Instantaneamente o Universo é linear}'', e felizmente!

Argumentos baseados em Análise Dimensional clássica (que está baseada na mudança de escalas uniformemente relacionadas) também é um procedimento de ``linearização'' que fornece uma primeira aproximação quantitativa do fenômeno. A substituição de escalas uniformes por escalas não uniformes é uma extensão não linear destes argumentos e será tratada nos Métodos Assintóticos.
}


\subsection{O FLUXO DE TRANSPORTE Unidimensional: Movimento Induzido e Determinístico}

O Fluxo em um Espaço de Aspecto é resultado da variação temporal dos registros mensuráveis dos indivíduos, isto é, da movimentação dos pontos de um Espaço de Aspecto. Matematicamente, a representação deste processo pode ser convenientemente descrito por trajetórias geradas por campos vetoriais como um Sistema Dinâmico. Assim, em muitos casos, o conceito de Fluxo pode ser reduzido ao conceito de um campo vetorial \(v(x)\) e às soluções de seus respectivos Problemas de Valor Inicial (Cauchy),
\[
\dfrac{dx}{dt} = v(x),\ x(0) = \alpha
\]
que denotaremos por \(\varphi(t,\alpha)\). A função \(\varphi(t,\alpha)\) é denominada \textit{função de fluxo} na teoria de Sistemas Dinâmicos. O termo ``Fluxo'' é utilizado com precedência de muitos séculos na Dinâmica de Fluidos e este será o seu sentido intencionado durante o presente texto.

O Modelo demográfico contínuo de Euler, por exemplo, tem o Fluxo gerado pelo campo vetorial \(v(x) = 1\), cuja função de fluxo é representada pela expressão elementar \(x = \varphi(t, \alpha) = \alpha+t\). (Verifique e interprete).

Assim, sob este ponto de vista, o ingrediente fundamental do Modelo Dinâmico passa a ser o ``\textbf{Campo Vetorial}'' e sua função de fluxo no Espaço de Aspecto que resultam na formulação do conceito de Fluxo de Transporte a ser derivado em seguida. Antes, apresentaremos uma classificação de Campos Vetoriais que se reflete no tipo de \textbf{Fluxo de Transporte} que lhe é associado.

O exemplo mais comum destes consiste em Campos Vetoriais que podem ser interpretados como uma velocidade \(v(x)\) imprimida em qualquer indivíduo que esteja localizado no ponto \(x\). Neste caso, o movimento é interpretado como involuntário e determinístico e resulta em um ``arrasto'' ao longo de uma ``corrente'' (função de fluxo) associada ao ponto \(x\), como se fosse um rio.

O campo, neste caso, independe da variável \(t\), e diz-se que é ``\textbf{intrínseco}'', também denominado ``\textbf{autônomo}'', ou seja, não apresenta influência exterior regulada pela variável \(t\) como seria, em geral, se \(\vec{v}(x,t)\). Uma manipulação matemática pode reduzir um problema não autônomo a um problema autônomo considerando \(t\) como uma coordenada \(x_{n+1} = t\) extra, mas não abordaremos este fato enquanto considerarmos modelos sem influência exterior).

Neste caso, utilizando um argumento análogo àquele empregado na formulação do Modelo demográfico de Euler, o Fluxo de Transporte \(J(x,t)\) resultante de um campo de velocidades \(v(x)\) no Espaço de Aspecto será definido (convincentemente) na forma
\[J(x,t) = \rho v.\]

{\small\color{blue}
\subsubsection*{Exercício:}

Argumentos matemáticos que corroboram a boa definição interpretativa de Fluxo de Transporte:
\[J(x,t) = \rho v.\]

\begin{exercise}
Considere um campo \(v(x)\) e dois pontos \(x_1 < x_2\), assim como a função de fluxo \(\varphi(t, x)\). Como as trajetórias dos pontos \(a = \varphi(t, x_1)\) e \(b = \varphi(t, x_2)\) se movimentam com o campo, nenhum outro ponto cruza com eles. (Unicidade local de solução do Problema de Cauchy). Portanto, o tamanho da população no intervalo móvel \([a, b]\) se mantém constante e igual a \(\displaystyle\int_{a}^{b} \rho(x,t)\ dx\) e sua derivada deve ser nula. Calculando, matematicamente, a derivada desta integral, argumente que o Fluxo de Transporte deve ser representado segundo a expressão de Fluxo de Transporte:
\[J(x,t) = \rho v.\]
\end{exercise}
}



Portanto, o Princípio de Conservação para a função densidade \(\rho(x,t)\) que descreve uma população distribuída em um Espaço de Aspecto unidimensional \(A\) no qual age um Campo de Velocidades \(v(x)\), é representado na forma:
\[\dfrac{\partial \rho}{\partial t} + \dfrac{\partial }{\partial x}(\rho v)= 0,\]
em que \(J = \rho v\) é o Fluxo denominado Fluxo de Transporte resultante do campo de velocidade \(v\).



\subsection{O CONCEITO DE FONTE: A Representação Matemática de ``Nascimento'' e ``Morte'' no interior da região \(\Omega \subset A\)}

A variação populacional em uma região \(\Omega\) (medida como uma taxa instantânea) será interpretada como um processo de Nascimento (se positivo) e Morte (se negativo) quando for representada por uma integral de volume em \(\Omega\) cuja função integrando será, portanto, interpretada como uma densidade de fonte.

\comentario{(Observe o contraste com a definição de Fluxo que faz uso apenas de informações na fronteira de \(\Omega\) e, portanto, é associada à ideia de ``Passagem'' de indivíduos através da fronteira).}

Assim como no caso do Fluxo, a abstração das causas microscópicas deste conceito permite que o utilizemos em uma ampla gama de situações específicas, sem perda de sua interpretação intuitiva.

A este conceito é dado o nome apropriado de ``\textbf{Densidade de Fonte}'' ou, simplesmente, ``\textbf{Fonte}'', e é designado, genericamente, por uma função \(f(x,t)\) que tem a seguinte interpretação como modelo matemático:
\[
\displaystyle\int_{t_1}^{t_2}\int_{x_1}^{x_2} f(x,t)\ dx\ dt =
\displaystyle\int_{x_1}^{x_2}\int_{t_1}^{t_2} f(x,t)\ dt\ dx
\]
``\textit{o \textbf{Saldo líquido} da população que foi originada do/no interior do intervalo \([x_1, x_2]\) durante o intervalo de tempo \([t_1, t_2]\)}''.

O emprego do termo densidade (de fonte) é justificado pelo fato de que a expressão que define sua interpretação no modelo matemático é uma população e se refere a uma integral da expressão
\[\displaystyle\int_{t_1}^{t_2} f(x,t)\ dt,\]
ao longo da região \([x_1, x_2]\).

Por outro lado, como esta integral se dá entre os limites \(t_1\) e \(t_2\), o significado de \(f(x,t)\) pode ser associado também a uma taxa instantânea.

A integral entre \(t_1\) e \(t_2\) da expressão:
\[\left(\displaystyle\int_{x_1}^{x_2} f(x,t)\ dx\right)\]
implica que esta, tem um sentido de taxa populacional, uma vez que a integral em um intervalo de tempo para esta expressão, isto é:
\[\displaystyle\int_{t_1}^{t_2}\left(\int_{x_1}^{x_2} f(x,t)\ dx\right) dt\]
foi definida como sendo uma população.

Para melhor entender o conceito de ``saldo líquido'' da origem interior de indivíduos, utilizamos um artifício semelhante ao empregado para a análise correspondente do fluxo.

Considerando \(f^+ (\ge 0)\), a densidade de taxa de introdução de indivíduos e \(f^- (\le 0)\), representando a densidade de taxa de retirada de indivíduos, escrevemos \(f = f^+-f^-\).

Portanto, se \(f(x,t) > 0\) em uma região \(\Gamma\) do espaço de aspecto, então
\[\displaystyle\int_\Gamma f(x,t)\ dx > 0\]
e isto representa uma taxa instantânea de acréscimo da subpopulação de \(\Gamma\) no instante \(t\).

Se, por outro lado, \(f(x,t) < 0\) em uma região \(\Gamma\), então \(\displaystyle\int_\Gamma f(x,t)\ dx < 0\) e o número positivo \(-\displaystyle\int_\Gamma f(x,t)\ dx\) representa uma taxa instantânea de retirada da subpopulação de \(\Gamma\) no instante \(t\). Naturalmente, \(f(x,t)\) pode, em geral, mudar de sinal no interior de uma região \(\Omega\) em \(A\) e em cada uma das sub-regiões em que preserva o sinal pode ser interpretado adequadamente.

\comentario{
Ressalte-se mais uma vez que, genericamente, os processos de intercambio com o exterior (tanto na fronteira quanto ao longo de uma região) não se referem propriamente aos indivíduos físicos, mas sim ao registros pontuais de seus aspectos que se modificam com o tempo e, com isso, a sua representação no espaço de aspecto.
}

A definição representativa destes objetos como modelos matemáticos segue, mais uma vez, um argumento típico de Euler já utilizado para a definição do conceito de função densidade, ou seja, utilizando suas integrais, e não os seus valores pontuais para defini-las, já que tanto o fluxo quanto a fonte são também conceitos de densidade.

\comentario{
Analogamente ao campo de velocidades, no caso geral o ingrediente Fonte \(f(x,t)\) não é, a rigor, uma função diretamente dependente de \((x,t)\) mas sim, resultante da aplicação de um funcional \(\Psi\) à função densidade \(\rho\), ou seja, \(\Psi[\rho](x,t) = f(x,t)\). Da mesma forma, este funcional pode ser de uma das classes funcionais mencionadas naquela situação.
}

Uma vez construído o cenário e os objetos matemáticos apropriados para representar a Biologia do fenômeno em vista, a afirmação ``A taxa de variação instantânea dos indivíduos contidos no conjunto \([x_1, x_2]\) no instante \(t\) é igual ao ``saldo líquido'' da taxa de indivíduos intercambiados com o exterior, (1) na fronteira deste conjunto \& produzidos (ou retirados) do seu interior (2)'', pode ser representada na forma matemática de um PRINCÍPIO GERAL DE CONSERVAÇÃO INTEGRAL UNIDIMENSIONAL
\[\begin{array}{rcl}
& & \dfrac{d}{dt}\left(\displaystyle\int_{x_1}^{x_2} \rho(x,t)\ dx\right) \\[0.5cm]
&=&
\displaystyle\int_{x_1}^{x_2} \dfrac{\partial}{\partial t}\rho(x,t)\ dx \\[0.5cm]
&=&
J(x_1,t) - J(x_2,t) - \displaystyle\int_{x_1}^{x_2} f(x,t)\ dx
\end{array}\]
\(\forall\ 0 < x_1 < x_2\) e \(t > 0\).


\comentario{
\subsubsection*{OBSERVAÇÕES}:

\begin{enumerate}
\item Observe-se que a primeira igualdade é uma simples operação matemática, enquanto que a segunda é uma afirmação com base nas interpretações de Fonte e de Fluxo.

\item Para o exemplo demográfico de Euler, toma-se sempre \(x_1 > 0\) evitando assim tratar dos nascimentos biológicos que não ocorrem no ``interior'' do espaço de aspecto, mas excepcionalmente como um fluxo positivo de entrada no ponto \(x_1 = 0\).
\end{enumerate}
}

Fazendo uso do Teorema Fundamental do Cálculo e assumindo a continuidade da derivada do Fluxo \(J(x,t)\), reescrevemos a equação acima na forma
\[\begin{array}{rcl}
\displaystyle\int_{x_1>0}^{x_2} \left(\dfrac{\partial \rho}{\partial t} + \dfrac{\partial J}{\partial x} - f\right)\ dx = 0
\end{array}\]
para quaisquer \(0 < x_1 < x_2\) e \(t > 0\).

\textbf{Assumindo a continuidade} da expressão
\[\left(\dfrac{\partial \rho}{\partial t} + \dfrac{\partial J}{\partial x} - f\right)\]
em \(x\), para todo \(t > 0\), concluímos (via argumento de localização de Euler) que ela é identicamente nula. Neste caso, a afirmação acima assume a forma equivalente de uma Equação Diferencial Parcial, denominada PRINCÍPIO GERAL DE CONSERVAÇÃO DIFERENCIAL UNIDIMENSIONAL:
\[\dfrac{\partial \rho}{\partial t} + \dfrac{\partial J}{\partial x} = f,\]
para quaisquer \(x > 0\) e \(t > 0\).

\subsection{O MODELO DE EULER}

Assumindo, segundo Euler, o Modelo malthusiano de mortalidade, os indivíduos da (pequena) subpopulação registrada em \([x, x+\delta]\) (aproximadamente \(\rho(x,t)\delta\), para \(\delta\) positivo e bem pequeno) tem (aproximadamente) uma mesma taxa de mortalidade característica da idade \(x\), ou seja, \(\mu(x)\). Portanto, de acordo com o modelo malthusiano, a expressão \(\displaystyle\int_{x_1}^{x_2} \mu(x)\ \rho(x,t)\ dx\), representa a taxa de mortalidade para a subpopulação contida no intervalo \([x_1, x_2]\), no instante \(t\) e, assim, no modelo de Euler a Fonte é representada por \(f(x,t) = -\mu(x)\ \rho(x,t)\), com o sinal negativo, pois ``retira'' indivíduos da faixa analisada (já que \(\mu\) e \(\rho\) são não-negativos).

Substituindo os termos particulares do Modelo demográfico de Euler, obtemos a Equação Diferencial Parcial de Euler:
\[\dfrac{\partial }{\partial t} \rho(x,t) = -\dfrac{\partial }{\partial t} \rho(x,t) - \mu(x)\ \rho(x,t),\]
para \(x > 0\) e \(t > 0\).

Para completar o Modelo Demográfico de Euler, é necessário incluir os nascimentos biológicos nesta população, ou seja, é necessário analisar o que ocorre no ponto peculiar \(x = 0\) do espaço etário onde são introduzidos os (registros dos) recém-nascidos.

Lembrando-se do significado de \(\rho(0, t) = J(0, t)\): ``o Fluxo de passagem de indivíduos através do ponto \(x = 0\)'' escrevemos, coerentemente, o termo que introduz os nascimentos segundo uma hipótese malthusiana de natalidade:
\[\rho(0, t) = \displaystyle\int_{0}^{\infty} \nu(s) \rho(s,t)\ ds.\]

Derivando a última expressão com relação a \(t\) e substituindo o termo integrando de acordo com a expressão anterior, a Dinâmica do ``Estado da População'', representada pela derivada temporal da função (``vetor'') densidade \(\rho_x(t)\), toma a forma do modelo populacional de Euler como a \textbf{Recursão Infinitesimal}:
\[\dfrac{\partial \rho}{\partial t} (x) = -\dfrac{\partial \rho}{\partial t} (x,t) - \mu(x)\ \rho(x,t),\]
para \(x>0\) e \(t \ge 0\) e,
\[\dfrac{\partial \rho}{\partial t} (x) = \displaystyle\int_{0}^{\infty} \nu(x) \left[\dfrac{\partial \rho}{\partial x} (s,t) - \mu(s)\ \rho(s,t)\right]\ ds,\]
para \(x=0\) e \(t \ge 0\). 

Em resumo: \(\dfrac{\partial \rho}{\partial t} = \mathfrak{L}[\rho]\), para \(x \ge 0\) e \(t \ge 0\).

Uma característica notável do Modelo de Euler é o fato de que o Gerador (infinitesimal) da dinâmica populacional \(\mathfrak{L}\), definido no quadro acima, atua no espaço de configuração \(M = \{u: \mathbb{R}^+ \to \mathbb{R}\}\), \(\mathfrak{L}: M \to M\), e é um operador linear integro-diferencial, o que faz dele um operador \textbf{não local} pois, em particular, o valor de \((\mathfrak{L}\rho)(0,t)\) depende de \(\rho(s,t)\) para todos os valores de \(s\). Esta característica peculiar de \(\mathfrak{L}\) torna o Modelo de Euler um exemplo à parte com relação às equações funcionais (diferenciais e integrais) clássicas da Física-Matemática Aplicada e talvez explique a sua ausência dos textos usuais de Métodos Matemáticos.

{\small\color{blue}
\subsubsection*{Exercícios}

\begin{exercise}
Mostre que o operador de Euler \(\mathfrak{L}\) é, de fato, Linear.
\end{exercise}

\begin{exercise}
Se \(\mu(x,t)\) e \(\nu(x,t)\) são funções que dependem do instante \(t\) (o que significa a ação de uma influência externa sobre a dinâmica), obtenha o operador linear de Euler \(\mathfrak{L}(t)\) que agora também depende de \(t\).
\end{exercise}

\begin{exercise}
Se \(\mu\) e \(\nu\) são constantes, obtenha o Modelo malthusiano para a população total.
\end{exercise}

\begin{exercise}
Considere a ``coorte'' de indivíduos \(m_{[x_1,x_2]}(t) \displaystyle\int_{x_1(t)}{x_2(t)} \rho(x,t)\ dx\), onde
\(\dfrac{dx_1}{dt} = 1 = \dfrac{dx_2}{dt}\). Interprete esta expressão. Mostre que não há acréscimo ou decréscimo de indivíduos neste conjunto ``em movimento''. (As extremidades estão se movimentando com os indivíduos no Espaço de Aspecto etário). Derive a função \(m_{[x_1,x_2]}(t)\) com relação a \(t\) e, com base nas interpretações adequadas, obtenha a equação de Euler para \(\rho(x,t)\) em \(x > 0\), \(t > 0\) e, de sobra, a confirmação de que o fluxo é descrito pela relação constitutiva \(J = \rho\).
\end{exercise}
}



\section{O PRINCÍPIO DE CONSERVAÇÃO EM ESPAÇOS DE ASPECTO DE DIMENSÃO SUPERIOR}

\begin{citacao}
``A história das grandes ideias e descobertas usualmente apresentam três etapas. A primeira se caracteriza pela indiferença ou mesmo hostilidade de muitos à nova ideia que é reputada insensata, ou mesmo idiota. Caso esta ideia produza algum resultado interessante, ela passa a ser considerada simples e trivial. Em uma terceira etapa, se ela realmente se mostra brilhante, então muitos/as dirão que esta era uma ideia antiga já conhecida deles/as há muito tempo.''. Anônimo
\end{citacao}


A estrutura geométrica simples do Espaço unidimensional facilita enormemente a introdução dos conceitos que definem os ingredientes básicos de um Princípio de Conservação. Entretanto, a utilidade do conceito geral de Espaço de Aspecto impõe a necessidade de tratar as ideias de Euler em contextos geométricos mais includentes.

Neste capítulo, abordaremos os Espaços de Aspecto representados em \(\mathbb{R}^n\) que abrangem uma considerável variedade de aplicações, seguindo de perto a intuição associada ao plano bidimensional e ao espaço tridimensional. Neste caso, o primeiro ingrediente matemático de um Princípio de Conservação está automaticamente definido: o Espaço de Aspecto (\(A \subset \mathbb{R}^n\)) e suas densidades, constituídas de funções \(\rho(x)\) com valores reais não negativos e integráveis em subconjuntos ``regulares'' de \(A \subset \mathbb{R}^n\).

O segundo ingrediente do Princípio de Conservação em dimensão superior, o Fluxo, exige um tratamento mais cuidadoso que será apresentado em seguida.

\subsection{O MODELO MATEMÁTICO GERAL DE DENSIDADE DE FLUXO (CORRENTE) \(J\)}

O mais sutil dos três ingredientes que compõe um Princípio de Conservação é certamente o conceito de \textbf{Fluxo} que tem por tarefa descrever o trânsito de indivíduos através das fronteiras \(\partial \Omega = S\) de regiões \(\Omega\), e que será representado matematicamente por integrais restritas à esta superfície \(\partial \Omega = S\).

Os argumentos a seguir se restringirão ao espaço tridimensional cuja familiaridade e intuição indicará claramente o caminho formal para a respectiva definição em espaços de dimensão superior.

Seguindo a estratégia ``distributiva'' de Euler, a função representativa do Fluxo será definida, não por valores pontuais, mas pelas suas integrais de superfície, o que lhe confere uma interpretação como uma densidade superficial.

Para que as integrais de fronteira sejam definidas, os conjuntos testes  nos quais o Princípio de Conservação será aplicado consistirão de regiões com fronteiras orientadas e suaves, ou seja, que admitem em cada ponto \(s \in \partial\Omega\) um plano tangente que varia continuamente com o ponto \(s \in S = \partial\Omega\), exceto em algumas poucas ``quinas''. 

\comentario{
(Textos usuais se restringem ao tratamento de regiões na forma de paralelepípedos, mas aqui aproveitaremos a oportunidade para introduzir conceitos geométricos em regiões mais gerais - V. Exercício abaixo).
}

O argumento que conduz ao Princípio de Conservação Diferencial é fundamentado no mesmo Lema de Localização de Euler para o qual basta considerar regiões regulares, o que, felizmente, nos exime de analisar integrais em fronteiras ``rugosas'' ou fractalizadas.

Na verdade, basta a verificação de ``contabilidade'' em superfícies poliedrais formadas por faces planas já que uma integral de superfície é definida com o limite de integrais sobre reticulados poliedrais que a envolve progressivamente. Em vista disso, a interpretação do conceito de fluxo é completamente apreendida em uma análise do seu efeito para pequenas superfícies planas.

\comentario{
A respectiva generalização matemática deste conceito foi denominada por L. Schwartz de ``Courant'' (ou, ``Corrente``, em português), também caracterizado por suas integrais de superfície e não por seus valores pontuais. [Schwartz - \textbf{Théorie des Distributions}, 1950]. Em textos de Geometria Diferencial este e outros conceitos relacionados são representados pela estrutura de formas diferenciais. (M. Perdigão, B. Dubrovnin)

O Conceito generalizado de Fluxo exige o tratamento de questões geométricas mais delicadas que envolve o conceito de tensores e nem sempre são necessárias no estudo de Dinâmica de Populações, a menos de questões especiais em Dinâmica do Meio Contínuo e de Fisiologia. (Gurtin). Por este motivo, nos restringiremos ao estudo de Fluxos de transporte que constituem a maioria dos casos de interesse para a Dinâmica de Populações e admitem uma apresentação pedagógica mais intuitiva.
}

\subsubsection{FLUXO DE TRANSPORTE}

Como o \textbf{Fluxo de Transporte} é resultado do movimento causado por um campo de velocidades \(v(x,t)\) definido no Espaço de Aspecto, o ingrediente fundamental deste objeto passa a ser o \textbf{Campo de Velocidades} e sua função de fluxo \(x = \varphi(t, \alpha, t_0)\) que representada pela solução geral do Problema de Valor Inicial (Cauchy) \(\dfrac{dx}{dt} = v(x,t)\), com \(x(t_0) = \alpha\).

\comentario{
\subsubsection{Apêndice: Campos Funcionais e não autônomos}

1 - Um campo de velocidades \(v(x)\) que independe da variável \(t\), é dito \textbf{autônomo} na terminologia de Sistemas Dinâmicos e pode ser interpretado como \textbf{intrínseco}, já que uma dependência de \(t\) pode ser interpretado como resultado de uma influencia do exterior ao Espaço de Aspecto, onde o valor de \(t\) é determinado. Uma simples, mas artificial, manipulação matemática pode reduzir o tratamento dinâmico de um campo sempre ao caso autônomo introduzindo uma variável ``espacial extra'' \(x_{n+1} = t\).

2 - O campo de velocidades representado na forma \(v(x,t)\) não é, em muitos casos, uma função \textbf{direta} da posição \(x\) do espaço de Aspecto e do instante \(t\), isto é, não é diretamente calculável dado o ponto e o instante, mas pode depender da densidade local da população, ou seja, da função \(\rho\). Neste caso, a função \(v(x,t)\) é resultado de uma operação funcional aplicada sobre a função densidade \(\rho\) e, a rigor, deveria ser denotada por \(v[\rho](x,t)\) e não \(v(x,t)\).

Uma operação funcional \(\Phi\) aplica-se a uma função \(\varphi\) e obtém-se como resultado \(\Phi[\varphi]\) um número ou uma outra função. Por exemplo, a derivada \(\Phi = \dfrac{d}{dt}\) é uma operação funcional que aplicada a uma função \(\varphi\) produz uma outra função enquanto que a integral definida \(\displaystyle\int_{0}^{1} \varphi(s)\ ds\) é uma operação funcional que aplicada a uma função \(\varphi\) dá como resultado um número. Há três tipos de operações funcionais básicas:

\begin{enumerate}
\item \textbf{Pontual}: Em que os valores da função resultante \(\Phi[\varphi](x)\) depende apenas dos respectivos valores pontuais de \(\varphi(x)\). Por exemplo: \(\Phi[\varphi](x) = 2 \varphi^2\).

\item \textbf{Local}: Em que os valores da função resultante \(\Phi[\varphi](x)\) depende apenas dos valores de \(\varphi\) em uma vizinhança ``infinitesimal'' do ponto \(x\). Por exemplo: \(\Phi[\varphi](x_0) = \dfrac{d\varphi}{dx}(x_0)\). (Impossível calcular a derivada de uma função em um ponto conhecendo-se apenas o seu valor neste ponto!)

\item \textbf{Não Local}: Em que os valores da função resultante \(\Phi[\varphi](x)\) depende dos valores de \(\varphi\) em um intervalo completo em torno do ponto \(x\). Por exemplo: \(\Phi[\varphi](x_0) = \dfrac{1}{2h} \displaystyle\int_{x_0-h}^{x_0+h} \varphi(s)\ ds\) (Média local da Função).
\end{enumerate}

A dependência do Campo de Velocidades em relação ao instante \(t\), \(v(x,t)\), se refere, em última análise, a influências externas ao Espaço de Aspecto, uma delas sendo a própria densidade.
}


{\small\color{blue}
\subsubsection*{Exercícios}

\begin{exercise}
*Interprete as equações diferenciais para a função \(\rho(x,t)\) abaixo como um Princípio de Conservação identificando o Fluxo de Transporte.

\begin{enumerate}
\item \(\dfrac{\partial \rho}{\partial t} + 2 \dfrac{\partial \rho}{\partial x} = 0\),
\item \(\dfrac{\partial \rho}{\partial t} + \dfrac{\partial}{\partial x} \left(\dfrac{1}{2}\rho^2\right) = 0\) (Equação de Burgers),
\item \(\dfrac{\partial \rho}{\partial t} + \dfrac{\partial^2}{\partial x^2} \left(g(t)\rho\right) = 0\),
\item \(\dfrac{\partial \log\rho}{\partial t} + \dfrac{\partial \log\rho}{\partial x} = 0\).
\end{enumerate}
\end{exercise}
}

Analisemos geometricamente no espaço tridimensional a expressão que será interpretada como um Fluxo de transporte.

A passagem de indivíduos através da fronteira \(\partial \Omega = S\) de uma região regular \(\Omega\) em \(\mathbb{R}^3\) será representada de forma global por uma integral de superfície genericamente descrita na forma de um limite integral de somas de Riemann
\[\displaystyle\int_{S} g\ dS = \lim_{\max{d_k} \to 0} \sum_{k=1}^N g(s_k) |\Delta \Sigma_k|,\]
em que \(s_k\) são pontos sobre a superfície e \(|\Delta \Sigma_k|\), as áreas de pequenos polígonos \(\Delta \Sigma_k\) sobre o plano tangente a \(S\) em \(s_k\), circundando o próprio ponto \(s_k\), e \(d_k\), o diâmetro de \(\Delta \Sigma_k\). As somas são realizadas em um poliedro formado pelos fragmentos de planos \(\Delta \Sigma_k\) que tangenciam, reticulam e envolvem completamente a fronteira. Este poliedro se aproxima da fronteira, segundo o limite integral de Riemann, i.e., com o refinamento progressivo da reticulação da superfície, que é indicado pelo símbolo \(\max\{d_k\} \to 0\).

A expressão integral faz uso apenas dos valores de uma função \(g\), nas imediações da fronteira e, como ao final esta integral representará um conceito de taxa de variação populacional, interpretaremos o seu efeito como um processo de ``passagem'' de indivíduos através da fronteira. Sendo a integral superficial, \(g\) será também encarada como uma densidade de superfície.

O objetivo agora é propor uma expressão para a função \(g\) que corrobore a interpretação reservada para ela (taxa de passagem de indivíduos) e, para isto, substituiremos este conceito não linear pela sua aproximação linear tal como justificado no Apêndice sobre a linearização local de Modelos.

Para isto, consideremos os pequenos polígonos planos \(\Delta \Sigma_k\) em torno de pontos \(s_k \in S = \partial \Omega\), orientados por um respectivo vetor normal \(\vec{N_k}\) à superfície, neste ponto, determinando uma orientação da fronteira \(\partial \Omega\), no sentido exterior à região \(\Omega\), e que denominaremos de positiva.

O fluxo, através deste pequeno fragmento plano \(\Delta \Sigma_k\), será causado, naturalmente, pela passagem de indivíduos ``arrastados'' em trajetórias segundo o campo vetorial através dele.

Para que um cenário seja visualizado, consideremos que, nesta pequena superfície, o produto interno \(v(s_k, t) \cdot N_k\) seja positivo, o que será interpretado como a passagem de uma trajetória na direção positiva da superfície do polígono, no sentido exterior à região.

\comentario{
(Observe que, de acordo com a Geometria Analítica elementar, \(v(s_k, t) \cdot N_k\) é a componente da velocidade na direção da normal \(N_k\) e, para um campo fixo \(v\), o produto escalar \(v \cdot N\) atinge o seu máximo quando a normal \(N\) for colinear e de mesmo sentido do campo.)
}

Assim, \(v(s_k, t) \cdot N_k > 0\) pode ser interpretado como indicador da passagem instantânea de todas as trajetórias através do fragmento no sentido positivo, isto é, para fora da região \(\Omega\). Por continuidade, este sinal é preservado para instantes em um intervalo \((t - \delta, t)\). Portanto, neste intervalo de tempo, o volume formado pelas trajetórias que atravessam o polígono \(\Delta \Sigma_k\) no sentido positivo é (linearmente em \(\delta\)) aproximado por \(\delta v(s_k, t) \cdot N_k|\Delta \Sigma_k| = \Delta V_k\), e o mesmo pode ser afirmado para a sub-população que a atravessa e que é também (linearmente em \(\delta\)) aproximada por \(\rho(s_k, t) \Delta V_k = \rho(s_k, t) \delta v(s_k, t) \cdot N_k|\Delta\Sigma_k|\). Assim, a taxa de passagem de indivíduos na direção positiva pelo fragmento plano \(\Delta\Sigma_k\) será \(\rho(s_k, t) v(s_k,t) \cdot N_k|\Delta\Sigma_k|\).

Argumento análogo ocorre quando \(v(s_k, t) \cdot N_k = 0\), mas indicando, neste caso, a passagem de indivíduos no sentido oposto ao da Normal \(N_k\), isto é, para o interior da região \(\Omega\).

Os termos da soma para os quais \(v(s_k, t) \cdot N_k = 0\), não contribuirão para a soma Riemanniana e podem ser interpretados como movimentos tangenciais ao longo da fronteira que não contribuem (no limite) para a passagem de indivíduos através da fronteira, para dentro ou para fora.

Assim, a soma integral de Riemann pode ser interpretada como uma Taxa de passagem líquida de indivíduos pela fronteira do poliedro envolvente à região \(\Omega\) e a mesma interpretação será atribuída ao seu limite integral.

A função de valores reais \(j(x, t, \rho, N) = -\rho(x,t) v(x,t) \cdot N(x)\), definida sobre \(S = \partial \Omega\) é, desta forma, interpretada como uma densidade superficial e sua integral de superfície representará uma taxa de passagem líquida pela superfície, no sentido de sua orientação interior.

Observe-se que o \textbf{vetor Fluxo}, definido como
\[\vec{J}(x, t, \rho) = -\rho(x,t) \vec{v}(x,t)\]
é linearmente dependente da densidade e do campo de velocidades.

Assim, a integral de superfície:
\[\begin{array}{rcl}
& &-\displaystyle\int_{\partial \Omega} \rho v \cdot d\vec{S} \\[0.5cm]
&=& \displaystyle\int_{\partial \Omega} \vec{J} \cdot d\vec{S} \\[0.5cm]
&=& -\displaystyle\lim_{n \to \infty} \sum_{k=0}^{n} \rho(s_k) (N_k \cdot v(s_k)) \Delta\Sigma_k
\end{array}\]
será \textbf{definida} como a taxa ``líquida'' de passagem de indivíduos através da fronteira, no sentido interior à região \(\Omega\), se positivo, e exterior, se negativo.

Neste contexto, o ``vetor'' Fluxo de transporte \(\vec{J} = -\rho \vec{v}\) é, a rigor, representante de um funcional linear (\(J(\vec{N}) = -\rho \vec{v} \cdot \vec{N}\)) e, portanto, deve ser encarado como um tensor ou, como uma matriz linha (isto é, um funcional linear) e não exatamente como um ``campo vetorial''.

É importante frisar, sob o ponto de vista metodológico, que a argumentação apresentada acima não é, de forma alguma, uma demonstração matemática, pois, trata-se de interpretações que utilizam elementos exteriores à Matemática. Esta argumentação é uma maneira de apresentar evidências de que a integral acima pode, de fato, representar, de maneira plausível, os conceitos geométricos/biológicos abordados. A expressão para o fluxo é, portanto, proposta com (\textbf{fortes}) evidências corroborativas de interpretação, mas não matematicamente deduzida, o que seria impossível, já que faz uso de conceitos exteriores à Matemática.

\subsection{O MODELO MATEMÁTICO DE FONTE}

A formulação e definição do conceito de Densidade (volumétrica) de Fonte, que deverá representar a produção, ou desaparecimento, de indivíduos do interior de conjuntos, será necessariamente representada por uma \textbf{integral de volume}.

Este conceito, geometricamente mais simples do que o conceito de Fluxo, tem a sua argumentação semelhante àquela apresentada no caso unidimensional. A fonte será representada por funções de valores reais \(f(x,t)\), embora nem sempre diretamente dependente das variáveis \((x,t)\).

A definição da função fonte como modelo matemático, também segue os argumentos de Euler e é expressada por intermédio de integrais volumétricas sobre todos conjuntos \(\Omega\) (regulares) que nos interessam testar:
\[\displaystyle\int_{\Omega} f(x,t) dx\]
``a Taxa de produção `líquida' de indivíduos no interior de \(\Omega\) no instante \(t\)''.

\comentario{
\subsubsection*{OBSERVAÇÕES}:

\begin{enumerate}
\item A explicação sobre o significado da expressão ``Taxa de produção líquida'', apresentada para a fonte no Modelo Unidimensional, pode ser repetida aqui.

\item Se \(f(x)\) for de fato uma função direta de \(x\), ela é intrínseca. A dependência com relação à variável \(t\), significa que \(f(x,t)\) sofre uma influência externa ao Espaço de Aspecto.

\item A fonte pode ser definida por uma ``\textbf{Relação constitutiva}'', isto é, como resultado da aplicação de um funcional em \(\rho\), de tal forma que \(f(x,t) = \Psi[\rho](x,t)\).

\item Se o valor de \(f(x,t)\) depende apenas do valor \(\rho(x,t)\), então o funcional é pontual, isto é, da forma: \(f(x,t) = \varphi(\rho(x,t))\). Este caso é exemplificado pela mortalidade no modelo demográfico de Euler, onde \(f(x,t) = -\mu(x,t) \rho(x,t)\), \(f = -\mu\rho\).

\item Em algumas situações de interesse, o valor de \(f(x,t)\) depende de uma gama de valores \(\rho(y, t)\), para \(y\) distantes de \(x\). Neste caso, a relação constitutiva pode tomar uma forma ``não-local'', comumente representada por expressões integrais do tipo,
\[f(x,t) = \displaystyle\int_{\Omega} K(x, y, t) \rho(y, t)\ dy,\]
onde \(K(x, y, t)\) é interpretada como uma ``medida de influência'' de indivíduos da posição \(y\) sobre a fonte (produção/retirada de indivíduos) na posição \(x\).

Fontes não locais ocorrem, naturalmente, em espaços de aspecto não geográficos quando indivíduos ``distantes'' podem estar, na verdade, fisicamente próximos (Por exemplo em um Modelo de canibalismo com a população distribuída em espaço de ``tamanho''. G. Odell \& Wm. Fagan, 1994). Todavia, estes modelos também ocorrem em espaços de aspecto geográficos, quando, por exemplo, os indivíduos têm capacidade de se movimentar por grandes distâncias, em uma escala de tempo muito menor do que da própria da dinâmica populacional e, portanto, pode-se supor que a influência seja instantânea. Um exemplo típico deste fato é um modelo para a dinâmica florestal, que evolui na escala de meses, ou anos, enquanto uma semente pode ser transportada a quilômetros em poucos minutos, por vento, animais, veículos e etc. e são depositadas diretamente no interior da região. (M. Kot, R. Nathan, D. Mistro).
\end{enumerate}
}

\subsection{O PRINCÍPIO DE CONSERVAÇÃO GERAL}

Uma vez determinados os seus três ingredientes básicos \(((A,\rho), J, f)\), um Princípio de Conservação de Euler é auto-explicativo segundo as interpretações apresentadas e pode
ser imediatamente formulado da seguinte na sua \textbf{forma integral}:
\[\dfrac{d}{dt} \displaystyle\int_\Omega \rho(x,t)\ dx = \int_\Omega -J \cdot d\vec{S} + \int_\Omega f\ dx,\ \forall\ t, \forall\ \Omega.\]

Passando-se a primeira derivada para o interior da integral e transformando (via Teorema de Gauss) a integral de fluxo de fronteira para uma integral de volume, podemos escrever o Princípio de Conservação geral na forma Integro-diferencial:
\[\displaystyle\int_\Omega \dfrac{\partial \rho}{\partial t} (x,t) + \operatorname{div}(J) - f\ dx = 0,\ \forall\ t, \forall\ \Omega.\]
onde \(\Omega\) é um conjunto regular.

\comentario{
Esta formulação do Princípio de Conservação geral não parece ``prática'', uma vez que envolve a verificação da integral de uma expressão em todas as regiões fechadas contidas no espaço de aspecto, e para todos os instantes! Entretanto, a sua generalidade é importante sob o ponto de vista conceitual e teórico e, mesmo útil em muitas situações práticas, por exemplo na concepção de alguns métodos numéricos. (v. R. LeVeque).
}

Se o integrando do Princípio de Conservação Integral for contínuo, podemos estabelecer O PRINCÍPIO DE CONSERVAÇÃO DIFERENCIAL:
\[\dfrac{\partial \rho}{\partial t} (x,t) = -\operatorname{div}(J) + f.\]

\comentario{
A condição de continuidade desta expressão pode parecer uma daquelas inoportunas filigranas matemáticas mas, como já observamos anteriormente, há casos especiais, e importantes, da Matemática Aplicada, em que a formulação integral faz sentido, enquanto que a correspondente equação diferencial parcial deixa de ser uma descrição razoável, por razões muito ``práticas''. O protótipo de todos estes casos é o fenômeno de ondas de choque em dinâmica de gases, que foi primeiramente estudado com as equações de Euler por ninguém menos que Riemann. (G. F. B. Riemann - (H. Weber) - \textbf{Die Partielle Differential Gleichungen der Mathematischen Physik}'', 1876 - R. Courant - K.-O. Friedrichs - \textbf{Supersonic Flow and Shock Waves}, J. Wiley, 1948, Ya. B. Zeldovich - Yu. P. Rizer - \textbf{Elements of Gasdynamics and the Classical Theory of Shock Waves}, Acad. Press, 1968, A. J. Chorin - J. E. Marsden - \textbf{A Mathematical Introduction to Fluid Mechanics}, 3rd ed. Springer 2000. Veja também R. Bassanezi - W. C. Ferreira Jr, EDO, 1988).

Em muitos casos importantes, especialmente na Dinâmica de Fluidos e na Física, o princípio de conservação diferencial é válido em toda a região, a menos de delgadíssimas interfaces de separação, dentro das quais se desenvolve algum fenômeno extraordinariamente complexo comparado ao modelo inicial. Por exemplo, em dinâmica de gases, no interior de um ``choque'', desenvolvem-se processos térmicos violentos, inclusive de ionização, onde, obviamente, não são razoáveis as considerações que nos convencem do modelo de Euler. Entretanto, a relativa ``magreza'' da região de interface ou, de fronteira, onde efeitos mais complicados ocorrem, torna conveniente considerá-la como de fato uma fronteira sem conteúdo volumétrico (isto é, um ponto na reta, uma linha no plano ou, uma superfície no espaço e etc.). Mas, por outro lado, o processo interno não pode ser ignorado e, portanto, é necessário admitir um salto de descontinuidade das funções de fluxo, em que se considera a existência de fontes internas à interface. Nestes casos, se a interface é descrita por um ponto em uma reta, é parte do modelo estabelecer o salto de descontinuidade do fluxo que é definido e denotado na forma:
\[[J]_{x_0} = \displaystyle\lim_{h \downarrow 0} (J(x_0+h) - J(x_0-h)).\]

Um problema adicional é que nem sempre a localização do ponto de descontinuidade (choque) é conhecido `\textit{a priori}', mas é uma incógnita, importante, do problema.

Em muitas outras situações, a interface é uma fronteira fixa e estabelecida, o que ocorre em problemas de dinâmica de populações biológicas (v. Fagan \& al).

O modelo geral de conservação, expresso na forma diferencial, será repetidamente invocado nestas notas, daqui por diante, como uma das principais fontes de Equações Diferenciais Parciais (e integro-diferenciais) da Matemática Aplicada; as equações específicas, como já foi observado, serão caracterizadas pelas relações funcionais constitutivas para o Fluxo e para a Fonte.

Um Princípio de Conservação, acrescido das relações constitutivas para fluxo e fonte, reduz a incógnita do problema à função densidade.
}


\subsection{EXEMPLOS DA APLICAÇÃO DO PRINCÍPIO DE CONSERVAÇÃO}

\subsubsection{MODELO DE AGREGAÇÃO \& COAGULAÇÃO (Fragmentação/Fissão) de Smoluchowski- Modelos de Canibalismo}

Em vista das ideias empregadas na construção do modelo populacional de Euler, torna-se claro que argumentos semelhantes podem ser aplicados a outros problemas de dinâmica populacional, em que a ``reprodução e a mortalidade'' (fonte) dependem de alguma característica dos indivíduos que seja continuamente mensurável. Um dos modelos clássicos que fez uso desta argumentação foi desenvolvido pelo físico-químico polonês Marian Smoluchowskii (\(\sim\)1916/17) para a descrição de fenômenos de coagulação em um processo particulado. Neste caso, a população é constituída de agregados (``clusters'') de pequenas partículas (células, ou organismos) formados por fusão de dois outros agregados ou por fissão (fragmentação) deles mesmos. Neste modelo interessa descrever a população de agregados e não apenas o número deles, mas, também, a maneira como eles estão distribuídos segundo seus tamanhos, ou seja, ``quantos de cada tamanho''. Portanto, é natural que cada agregado (o ``indivíduo'' desta população) seja registrado segundo este aspecto, isto é, o seu tamanho. Supondo que as partículas sejam pequenas, comparadas ao tamanho típico destes agregados, podemos considerar que a medidas destes tamanhos sejam representáveis continuamente por um número real positivo. (Ref. Chandrasekhar, Redner, Wattis).

A bem da verdade, Smoluchowski, tal como Euler em seu famoso artigo demográfico, descreveu seu modelo original utilizando tamanhos apenas discretos de agregados, o que, a rigor, resulta em um conjunto infinito de equações diferenciais ordinárias não lineares acopladas, onde cada uma delas descreve a dinâmica da população de agregados de um determinado tamanho. Este procedimento é razoável quando as partículas/organismos são grandes comparadas aos tamanhos dos agregados. Neste caso, o número efetivo de equações diferenciais é finito e pequeno o que facilita o seu tratamento, sob o ponto de vista analítico e numérico.

Recentemente, as ideias de Smoluchowski tem sido empregadas na formulação de modelos matemáticos para a descrição de fenômenos de agrupamentos de células, microrganismos e de animais presentes em variados temas, da oncologia e morfogênese fisiológica à ecologia da predação e sociobiologia (Ref. Gyllenberg, Sumpter, Ruxton...).

Portanto, fica claro que o Espaço de Aspecto é \(A = \mathbb{R}^+\), onde registramos o aspecto tamanho do indivíduo (agregado) e a respectiva Função de Distribuição / Densidade \(\rho(x,t)\) interpretada, como sempre, por intermédio das integrais: 
\[\displaystyle\int_{a}^{b} \rho(s,t)\ ds,\]
``a Quantidade de Agregados com `tamanhos' entre \([a, b]\)'').

O próximo passo é caracterizar os demais ingredientes:

\begin{enumerate}
\item O Fluxo \(J\): que causa a ``movimentação'' de agregados neste espaço (ou seja, a sua taxa de crescimento ou decrescimento de volume) e;

\item A Função de Fonte \(f\): que representa a produção ou perda de agregados com tamanhos na faixa \(a < s < b\).
\end{enumerate}

As diversas classes de hipóteses sobre como se dá cada um destes processos, levam a diferentes modelos mas, todos eles obtidos do mesmo Princípio de Conservação,
\[\dfrac{\partial \rho}{\partial t} = \operatorname{div}(J) + f.\]  

O modelo geral de agregação e fragmentação se constitui em um problema matemático extremamente difícil de analisar, muito embora a sua construção resulte de uma aplicação imediata do Princípio de Conservação. Para que o tratamento analítico da questão seja possível e esclarecedor, é necessário considerar, inicialmente, hipóteses simplificadoras para os processos microscópicos de agregação e fragmentação.

Um conjunto de hipóteses simplificadoras. mas que ainda abrange situações de interesse. é supor que os agregados estejam suspensos em um meio formado por \(N\) partículas e que o processo de agregação se dê somente pela captação destas partículas soltas e nunca pela aglutinação de dois agregados e que a fissão destes, se dê sempre por perdas unitárias e nunca pela fragmentação em dois agregados complementares quaisquer. Além disso, é interessante considerar a hipótese de que a agregação e a fragmentação unitária ocorra apenas na superfície dos agregados e, portanto, que o processo se dê proporcionalmente à sua área.

Como um agregado esférico de volume \(x\) tem um raio proporcional a \(\sqrt[3]{x}\) e sua área exterior é proporcional a \(\sqrt[3]{x^2}\), um agregado registrado com o tamanho \(x\) tem um crescimento (velocidade positiva no espaço de aspecto) proporcional a \(N\sqrt[3]{x^2}\) e um decrescimento proporcional a \(\sqrt[3]{x}\), ou seja, o campo de velocidades no espaço de aspecto seria determinado por \(v(x) = (\beta N - \alpha) \sqrt[3]{x^2}\). Por outro lado, o número total de partículas soltas \(N(t)\), não havendo introdução e nem retirada exteriores delas, a dinâmica total seria regulada por:
\[\dfrac{dN}{dt}
= \displaystyle\int_{0}^{\infty} \alpha x^{\frac{2}{3}} \rho(x,t)\ dx
- \displaystyle\int_{0}^{\infty} \beta N x^{\frac{2}{3}} \rho(x,t)\ dx.\]

O Princípio de Conservação aplicado à população de agregados, portanto, resultaria na seguinte equação:
\[\dfrac{\partial \rho}{\partial t} + \dfrac{\partial J}{\partial x} = 0,\]
onde \(J = \rho (\beta N - \alpha) x^{\frac{2}{3}}\), que é uma Equação Diferencial Parcial de 1\textordfeminine\ Ordem, mas acoplada a uma equação integro diferencial para \(N(t)\). Observe que tanto \(\rho(x,t)\) quanto \(N(t)\) são incógnitas do problema.

{\small\color{blue}
\subsubsection*{Exercício}

\begin{exercise}
Considerando apenas fusão, com uma taxa que independe dos tamanhos envolvidos, e que não haja aumento ou diminuição de tamanho por crescimento ou decrescimento contínuo, escreva a equação para coagulação: Argumente tanto quanto lhe parecer interessante e escreva um modelo de coagulação em que ocorra não apenas fusão, mas também fissão dependentes do tamanho (como?) e crescimento contínuo, por exemplo, ``malthusiano'' ou, com saturação.
\end{exercise}
}

Equações semelhantes para modelos de precipitação de solventes foram descritas por R. Becker e W. Doering, em 1935, por I. M. Lifschitz, V. V. Slyozov e C. Wagner, em 1961 (modelo LSW), (Fife \& Penrose), para dinâmica de agregados celulares por, O. Diekmann, na década de 1970 (Diekmann), para dinâmica de reprodutiva de colônias de abelhas africanizadas, por D. C. Mistro e W. C. Ferreira Jr., em 1997 e, para o desenvolvimento de doenças em plantações (L. A. Kato \& W. C. Ferreira Jr, 2005) e muito(a)s outro(a)s.

O estudo de dinâmica populacional de insetos (mantidae/ ``louva-deus'') e outras populações, em que o canibalismo é um comportamento predominante (e onde, como em qualquer lugar e circunstância, é muito mais frequente que o ``marmanjo'' ataque o ``fedelho'', tanto mais quanto maior a diferença de tamanho, daí a necessidade de se descrever sua estrutura etária ou, volumétrica) foi estudado em diversos trabalhos de Biomatemática como, por exemplo, W. Fagan , G. Odell e O. Diekmann. (``Cannibalism is more frequent than we think'' - G. Odell).

\subsubsection{EQUAÇÃO DE LIOUVILLE: Dinâmica Não linear e Modelo Linear de Koopman}

Consideremos o estudo de uma epidemia em uma grande população humana e que o conhecimento que se deseja obter sobre ela em cada uma de sua ocorrência, denominado Estado da Epidemia, são as duas seguintes medidas: $S$, o ``número de indivíduos susceptiveis'' e $I$, o ``número de indivíduos infecciosos''. O modelo matemático SIS para a evolução desta classe de epidemias estabelece um Algoritmo Infinitesimal (isto é, uma equação diferencial {\red ordinária ???}) \textbf{não linear} que permite descrever a evolução temporal do Estado \(e = (S, I)\) da Epidemia em termos do próprio estado atual, ou seja,
\begin{eqnarray} %% Aqui deveria ser EDP?
\dfrac{dS}{dt} &=& \lambda S I \\
\dfrac{dI}{dt} &=& \lambda S I - \mu I,
\end{eqnarray}
com as interpretações usuais para os parâmetros: \(\lambda\) é a taxa de contágio e \(\mu\) a mortalidade malthusiana de infectados/infecciosos.

O modelo consiste em definir um campo vetorial
\[\begin{array}{rcl}
F: E = \mathbb{R}^2 & \to & \mathbb{R}^2 \\
(S, I) & \mapsto & (-\lambda S I, \lambda S I - \mu I),
\end{array}\]
no Espaço \(E = \{(S, I)\}\)  que ``propulsiona'', deterministicamente, o estado do sistema de um ponto do espaço para outro. O estudo determinístico deste modelo procura, portanto, descrever a trajetória \(e(t) = (S(t), I(t))\) de uma epidemia, ao longo do tempo, cujo estado ``inicial'' é dado (com absoluta certeza) por \(e_0 = (S_0, I_0)\).

Entretanto, deve-se admitir que, o dado inicial é, em geral, conhecido apenas incompletamente e, assim, a sua evolução \(e(t) = (S(t), I(t))\), também transporta esta incerteza ao longo do tempo, sendo natural, portanto, analisar esta questão.

Uma das maneiras de estudar o efeito da incerteza sobre o estado de uma epidemia é procurar descrevê-la, não pelo seu estado preciso, mas por intermédio de uma função densidade \(\rho\), definida no Espaço de Estados da epidemia, que tem por objetivo fornecer a frequência com que este estado se encontraria em regiões do referido Espaço ou, em outras palavras:
\begin{equation*}
\int_{\Omega} \rho(e)\ de
\end{equation*}
a ``Expectativa de que o estado \(e\) da Epidemia se encontre no conjunto \(\Omega\); isto é, que \(e \in \Omega\)''.

A analogia desta questão com o Princípio de Conservação, segundo Euler, é impossível de ser ignorada.

Aprofundando esta analogia, consideremos a função densidade deste caso, como a representação de uma população de ``possibilidades'' (ou, na linguagem frequentista, de ``experimentos'') de estados da epidemia, registrados no Espaço de Aspecto \(E = \{(S, I)\}\). Utilizando a linguagem de Euler, estamos tratando de uma População de Estados de uma epidemia SIS, em que o Espaço de Aspecto é exatamente \(E\). Portanto, segundo o próprio modelo SIS, a movimentação (determinística) dos pontos deste espaço de aspecto é uma consequência do efeito do campo \(F\), ou seja, há um processo de transporte neste espaço patrocinado por \(F\). Uma vez que estes estados não aparecem nem desaparecem, isto é, são conservados de fato, e que não há interação entre eles (já que são fictícios), podemos descrever a evolução da função densidade, na seguinte forma, segundo a aplicação do Princípio de Conservação para a função \(\rho(S, I, t)\):
\begin{equation}
\dfrac{\partial \rho}{\partial t} + \operatorname{div}(\rho F) = 0.
\end{equation}
ou seja,
\begin{equation}
\dfrac{\partial \rho}{\partial t} - \dfrac{\partial}{\partial S} (\lambda\rho SI) + \dfrac{\partial}{\partial I} [\rho (\lambda SI-\mu I)] = 0.
\end{equation}

Esta Equação Diferencial Parcial, EDP de Liouville, é de primeira ordem, linear na incógnita \(\rho\), e descreve a evolução probabilística da epidemia (Ludwig, 1974). Operacionalmente, esta equação pode ser re-escrita como uma dinâmica no espaço funcional de configurações \(M = \{\rho\}\), onde se representam todas as densidades na forma:
\[\dfrac{\partial \rho}{\partial t} = L\rho,\]
onde \(L: M \to M\) é o operador \textbf{Linear} de Liouville:
\begin{equation}
L[\rho] = \dfrac{\partial}{\partial S} \Big[(\lambda SI) \rho\Big] + \dfrac{\partial}{\partial I} \Big[-(\lambda SI-\mu I) \rho\Big].
\end{equation}

Observe que o problema determinístico, com condições iniciais \((S_0, I_0)\), se resume em analisar a evolução temporal da densidade inicial, dada por \(\rho_0(e) = \delta(S - S_0, I - I_0)\), onde \(\delta\) é a função generalizada de Dirac.

A abordagem apresentada para o modelo de epidemias é, apenas, um exemplo de uma estratégia que pode ser aplicada a qualquer sistema dinâmico.

A teoria abstrata de Sistemas Dinâmicos é descrita por intermédio de dois ingredientes fundamentais:

\begin{enumerate}
\item \textbf{Espaço de Fase}: que é, em geral, um subconjunto do \(\mathbb{R}^n\) (ou uma cópia local dele) em que se registra os Estados do Sistema; e
\item \textbf{Campo Vetorial} \(v: \mathbb{R}^n \to \mathbb{R}^n\): que dá origem a trajetórias no espaço de fase, por intermédio de uma equação diferencial ordinária \(\dfrac{dx}{dt} = v(x)\).
\end{enumerate}

O Problema de Cauchy
\[\dfrac{dx}{dt} = v(x),\ x(0) = x_0,\]
determina uma função solução da forma
\[x = \varphi(x_0, t),\]
que pode ser interpretada para cada \(x_0\), fixado como uma trajetória \(\varphi_{x_0}(t)\) da ``partícula'', denominada \(x_0\) (sua posição inicial) ou, para cada \(t\), como um mapeamento entre a posição inicial \(x_0\) de cada partícula e sua posição no instante \(t: \varphi_t(x_0) = x,\ \varphi_t: \mathbb{R}^n \to \mathbb{R}^n\), que é, nas melhores condições (assumidas em \textit{default}), diferenciavelmente inversível entre seu domínio e contradomínio.

O sistema dinâmico acima é, portanto, totalmente determinístico, uma vez que, dado o ponto inicial \(x_0\), consequentemente, o estado \(x\) do sistema é determinado por \(x = \varphi(x_0, t)\), em qualquer instante \(t\).

Em muitas questões, todavia, há uma incerteza com respeito à localização \(x_0\) e, em muitos casos, esta dinâmica é tão complicada em seus detalhes determinísticos individuais que é suficiente conhecer o efeito da dinâmica ``média'' sobre conjuntos. A dinâmica individual de pontos significa um excesso de informações, difícil de obter e inútil de se ter.

Digamos que a posição inicial \(x_0\) seja tentativamente obtida como resultado de uma enorme quantidade de experimentos (ou medidas independentes), cada uma delas resultando em uma posição \(x_0\). Interpretando, frequencialmente, os dados experimentais, a função de densidade, que representa continuamente esta ``nuvem'' de pontos no espaço de fase \(\mathbb{R}^n\), pode ser interpretada como uma função de distribuição de probabilidade \(\rho_0(x)\), no seguinte sentido:
\[\displaystyle\int_{\Omega} \rho_0(x)\ dx,\] a ``Probabilidade de que a condição inicial \(x_0 \in \Omega\)'', ou ainda, o ``número de pontos iniciais que se encontram dentro de \(\Omega\)''). 

Considerando que cada condição inicial \(x_0\) determina uma trajetória possível do sistema, interpretamos o consequente movimento desta nuvem de pontos iniciais, como um transporte efetuado pelo campo vetorial \(v: \mathbb{R}^n \to \mathbb{R}^n\). 

Portanto, considerando esta ``nuvem'' de pontos como situada em um meio contínuo, podemos garantir que a situação de incerteza do sistema no instante \(t\) será dada por uma função de distribuição \(\rho_t(x) = \rho(x,t)\).

Observando que estes pontos (``dados'') não aparecem ou desaparecem do espaço de fase, podemos escrever a dinâmica de \(\rho\) por um princípio de conservação com fluxo \(J = \rho v(x)\), que é um funcional (pontual) linear com relação a \(\rho\) e de onde escrevemos a famosa Equação de Liouville:
\begin{equation}
\dfrac{\partial \rho}{\partial t} + \operatorname{div}(v\rho) = 0,
\end{equation}
que é, necessariamente, uma EDP linear, ao contrário do sistema dinâmico que, em geral, é não linear.

Levando em conta a interpretação do modelo e do fato de que não há interação entre os pontos, é razoável escrever a solução desta equação na forma:
\[\rho(x,t) = \rho(\varphi_t^{-1}(x), t).\]

Portanto, a solução deste problema matemático é equivalente à obtenção da inversa da função \(\varphi_t\), o que pode ser (e de fato é) interessante, mas apenas indica a enorme dificuldade da questão.

Sendo este um problema matemático de grande interesse, especialmente devido a esta interpretação, abordaremos alguns métodos de tratamento analítico dela no capítulo adequado. (Ref.  Chorin-Hald, 2005; Zwanzig, Lasota \& Mackey, Cvitanovic, Mesic, Kutz, Bassanezi \& Ferreira, ...)

\subsubsection{TRANSPORTE E DINÂMICA VITAL}

Consideremos, por simplicidade operacional, um meio físico (que pode ser um rio ou um tubo) ao longo do qual escoa um líquido com velocidade estabelecida \(v(x,t)\) e cujos indivíduos (partículas poluidoras, animais aquáticos e ribeirinhos, ou microrganismos) estão em suspensão e são carregados por esta corrente. Se não houver perda ou ganho (contabilístico) de indivíduos ao longo deste trajeto, a Equação de Conservação (isto é, a equação diferencial parcial para a função densidade \(\rho(x,t)\)) pode ser imediatamente escrita:
\begin{equation}
\dfrac{\partial \rho}{\partial t} + \dfrac{\partial }{\partial x} (\rho v) = 0.
\end{equation}

Entretanto, por exemplo, em um modelo de dispersão de poluição é, por vezes, indispensável considerar que as partículas suspensas são depositadas no leito do rio a uma taxa proporcional à densidade, o que pode ser descrito por uma Fonte negativa \(f(\rho) = -\mu \rho\), com a seguinte interpretação
\[\displaystyle\int_{x_1}^{x_2} \mu \rho(x,t)\ dx,\]
a ``Taxa de deposição de partículas no trecho \([x_1, x_2]\)''. 

Ainda em um modelo desta natureza, o efeito de emissões poluidoras ao longo do rio também é um fator a ser considerado, o que, no caso, poderia ser descrito por uma função \(\lambda(x,t)\) com o seguinte significado:
\[\displaystyle\int_{x_1}^{x_2} \lambda(x,t)\ dx,\]
a ``Taxa de emissão de poluentes no trecho \([x_1, x_2]\)''.

Com isto, o modelo matemático tomaria a forma:
\begin{equation}
\dfrac{\partial \rho}{\partial t} + \dfrac{\partial }{\partial x} (\rho v) = -\lambda.
\end{equation}

No caso de microrganismos, existe a possibilidade da ocorrência de uma reprodução vital de indivíduos ao longo do habitat.

Consideremos, então, um processo vital de reprodução e morte, segundo o modelo de Verhulst, ou seja, malthusiano para populações rarefeitas e assumindo um efeito de saturação para altas densidades. Se descrevermos a capacidade de suporte do meio por uma função densidade \(k(x)\), com a interpretação:
\[\displaystyle\int_{x_1}^{x_2} k(x)\ dx,\]
a ``Capacidade de suporte da região compreendida pelo intervalo \([x_1, x_2]\)'', então em um pequeno segmento \([x, x+dx]\) (``infinitesimal'') do habitat, a taxa de variação da população por efeitos vitais pode ser descrito (em primeira ordem em \(dx\)) na forma:
\[r\rho(x,t)\ dx \left(1 - \dfrac{\rho(x,t)\ dx}{k(x)\ dx}\right) = r\rho \left(1 - \dfrac{\rho}{k}\right)\ dx\]
de onde vem que a fonte vital do modelo Verhulst é representada na forma \begin{equation}
f(\rho)(x,t) = r\rho \left(1 - \dfrac{\rho}{k}\right).
\end{equation}

Recolhendo os ingredientes e considerando o Princípio de Conservação correspondente a uma população de um habitat unidimensional submetida a um transporte com velocidade \(v(x,t)\) e desenvolvendo-se segundo uma dinâmica vital de Verhulst, obtemos o modelo matemático:
\begin{equation}\label{eq:dvverhulst}
\dfrac{\partial \rho}{\partial t} + \dfrac{\partial }{\partial x} (v\rho) = r\rho \left(1 - \dfrac{\rho}{k}\right),
\end{equation}
que é uma EDP de primeira ordem e não linear para a função incógnita \(\rho(x,t)\).

Obviamente, um habitat tem início e fim, digamos, o intervalo \(a \le x \le b\), e a descrição do modelo matemático se restringe a um período de tempo finito, digamos \([0, T]\). A EDP obtida do Princípio de Conservação, na verdade, rege o funcionamento do sistema apenas internamente a esta região de espaço-tempo e isto deve ser especificado por \eqref{eq:dvverhulst}, com \(a \le x \le b\) e \(0 \le t \le T\).

Portanto, para descrevermos a evolução da função densidade \(\rho_t(x)\), é necessário acrescentar ao problema, a maneira como o meio exterior interage com o sistema nas fronteiras \(x = a\) e \(x = b\) (denominada condições de fronteira) e, também, a seu estado inicial de partida \(\rho(x, 0) = \rho_0(x)\) (denominada condição inicial). As condições de fronteira mais comuns serão analisadas em outro item. (Kenkre)

\subsubsection{TAXIA: Modelo Keller-Segel}

Organismos dispersos em uma região espacial estão quase sempre efetuando movimentos constantes em busca de melhores condições ou em fuga de piores condições. Uma das observações mais impressionantes e que tem intrigado os biólogos há muito tempo, se refere ao movimento de populações de microrganismos (a famosa Escherichia Coli, por exemplo) quando dispersas em um meio liquido (Berg, Lin-Segel, Segel). Se este meio apresenta uma suspensão heterogênea de algum nutriente útil para estes microrganismos, verifica-se que, por ``um passe de mágica'', eles se dirigem em ondas na direção de maior concentração nutritiva. Como explicar este comportamento coletivo sem assumir que estes simples organismos disponham de uma capacidade ``cognitiva'' excepcional, ou melhor, como explicar este fenômeno sem abandonar o Princípio de Parcimônia de Ockam? A tensão entre a plausibilidade (\(\sim\)parcimônia) das hipóteses biológicas e a sua suficiência para a descrição do fenômeno é evidente neste caso como, de resto, em todas as formulações de modelos matemáticos na Biologia.

Há dois modelos matemáticos que apresentamos como explicação deste fenômeno, ambos, assumindo uma capacidade intrínseca de processamento muito rudimentar e plausível para organismos elementares. Um deles assume a existência de uma sinalização entre indivíduos, inconsciente, claro, mas que permite a realização desta façanha e será analisada no capítulo de Difusão. O modelo que apresentaremos aqui foi desenvolvido por Evelyn Fox-Keller e Lee A. Segel, em 1970, e se constitui em um dos trabalhos fundamentais da Biomatemática contemporânea.

Os organismos em questão são unicelulares e, como se pode observar no microscópio, ele dispõe de sensores em toda a sua membrana externa, capazes de detectar a concentração do nutriente em cada ponto da membrana em termos da taxa de reação entre o sensor e o nutriente. Estas células também dispõe de flagelos que funcionam como motores a hélice e que podem impulsioná-los em qualquer direção (Bonner, Berg). O comportamento microscópico destes organismos é o seguinte: O organismo testa a concentração em todos os seus sensores. Por intermédio de um processo interno (ainda desconhecido, mas plausível) os flagelos da direção oposta a dois sensores são ativados com maior ou menor intensidade dependendo da discrepância da medida entre os dois sensores antípodas. Assim, um movimento final do organismo se dará na direção de maior concentração do nutriente, pelo menos por um pequeno percurso.

O comprimento deste percurso (ou do tempo de ativação dos flagelos) depende da discrepância entre as concentrações detectadas pelos respectivos sensores antípodas mas de uma maneira regulada por um dos princípios mais fundamentais e universais da psicologia de organismos denominada de ``\textbf{Lei de Weber-Fechner}''. Esta ``lei'' é amplamente ilustrada no nosso dia a dia com a ``receita para o escaldamento de um sapo vivo'' (em termos de sensação térmica) ou, ``o risco de se assentar em uma poltrona ocupada no escurinho do cinema'' (em termos de percepção de luminosidade). A sua forma original foi amplamente verificada experimentalmente e proposta pela primeira vez na segunda metade do século XIX (\(\sim\)1870) pelos fisiologistas/psicólogos alemães Ernst H. Weber, Gustav Fechner, H. von Helmholz e Wihelm Wundt sendo, posteriormente, aperfeiçoada por Stanley S. Stevens, um século depois (\(\sim\)1970), continuando ainda como tema de estudo. Simplificadamente, esta ``lei'' estabelece que a percepção de um estímulo sensorial (táctil-térmico, olfativo, visual, ...) ocorre quando a variação logarítmica \(\dfrac{\Delta S}{S}\), da intensidade de um determinado estímulo \(S\), ultrapassa um valor limiar \(\lambda\) (característico do estímulo e dos sensores), isto é, \(\dfrac{\Delta S}{S} > \lambda\), onde \(S\) é a intensidade do estímulo de fundo. Além disso, a reação ao estímulo depende do valor \(H\left(\dfrac{\Delta S}{S} - \lambda\right)\), onde \(H\) é a função de Heaviside (\(H(x) = 0\), para \(x = 0\) e \(H(x) = 1\), para \(x > 0\)).

Uma vez efetuado um pequeno percurso retilíneo proporcional à reação Weber-Fechner causada pela percepção da discrepância de concentração do nutriente, o organismo novamente cessa o seu movimento e procede a uma nova avaliação sensorial do ambiente após o qual o processo se repete. Com este procedimento calibrado, a observação macroscópica da população de organismos apresenta um evidente movimento contínuo na direção de maior concentração do nutriente. Sob o ponto de vista do Princípio de Conservação segundo Euler, é possível descrever macroscopicamente o fenômeno como um fluxo de transporte na direção do gradiente da concentração da referida substância. Assim, considerando \(\eta(x)\) como sendo a densidade do nutriente (função densidade de um população de moléculas!), a população de microrganismos descrita pela função densidade \(b(x,t)\) que se movimenta no espaço de aspecto (espaço físico) segundo um fluxo \(J = b\left(\chi \dfrac{\nabla\eta}{\eta}\right)\), de onde vem a equação de Keller-Segel:
\begin{equation}
\dfrac{\partial b}{\partial t} + \operatorname{div}\left[b\left(\chi \dfrac{\nabla\eta}{\eta}\right)\right] = 0.
\end{equation}


O fenômeno em questão é denominado de \textbf{Quimiotaxia}, uma vez que há um movimento (``taxia'') na direção de concentração Química. Analogamente, existe uma enorme variedade de tipos de Taxia, que se referem aos mais diversos tipos de estímulos, FotoTaxia, GyroTaxia, TermoTaxia, PresaTaxia, HaptoTaxia, ..., todos eles passiveis de serem analisados por argumentos semelhantes (Murray, Segel, ...). O modelo completo de Keller-Segel também inclui uma dinâmica da ``população'' do nutriente que varia, uma vez que está sendo consumida pelos microrganismos. Em outros casos particularmente importantes, a substância química de concentração \(\eta(x,t)\) não é exatamente um nutriente mas um estimulador (sinalizador) secretado pelos próprios microrganismos, assunto que será tratado mais adiante.

É importante observar que neste modelo há a presença de duas populações constituídas de indivíduos (moléculas e microrganismos) completamente distintos, em natureza e tamanho, mas que todavia interagem entre si.

\subsubsection{DIFUSÃO COMO TAXIA de DESAGREGAÇÃO}

Os Modelos de Difusão são fundamentais para toda a Biomatemática e serão tratados em capítulo a parte. Entretanto, uma de suas múltiplas interpretações (não encontrada em textos da praça) decorre da aplicação dos conceitos desenvolvidos no item anterior sobre taxia, o que motiva a sua inclusão nesta oportunidade.

Em termos gerais, um Processo de Difusão é uma dinâmica temporal no Espaço de Configuração \(M = \{\rho\}\) de densidades que tem o efeito de ``homogenização'' monotônica, ou seja, \(\rho_t(x) \in M\) descreve um Processo de Difusão temporal com relação ao tempo \(t\), se \(\rho_{t+\delta}(x)\) é ``mais homogênea'' do que \(\rho_t(x)\), para \(\delta > 0\). O conceito de Dinâmica de ``Homogenização'' será discutido, com maiores detalhes, no capítulo que tratará especificamente de Difusão. Aqui, nos basta observar que um processo de Difusão tende a ``dispersar`` picos relativos de concentrações da população em questão. (V. também W. C. Ferreira Jr. - \textbf{The Multiple Faces and Feats of Diffusion}, 2019)

Consideremos, então, uma população distribuída no espaço físico com densidade \(\rho\) constituída de indivíduos que têm ``horror'' a aglomeração e que são dotados de capacidade intelectual suficiente para avaliar a concentração de indivíduos em sua vizinhança e reagir a esta informação, segundo a ``Lei'' de Weber-Fechner, movimentando-se na direção contrária ao gradiente de \(\rho\). Em linhas gerais, este processo pode ser interpretado como um modelo de ``Quimiotaxia'' de Keller-Segel com relação à concentração de indivíduos e será descrito da seguinte forma:
\begin{equation}
\dfrac{\partial \rho}{\partial t} + \operatorname{div}\left[\rho\left(-\chi \dfrac{\nabla\rho}{\rho}\right)\right] = 0.
\end{equation}

Re-escrevendo a equação dinâmica para \(\rho(x,t) = \rho_t(x)\), obtemos o Modelo de Difusão:
\begin{equation}
\dfrac{\partial \rho}{\partial t} = \operatorname{div} \left(\chi\nabla\rho\right)
\end{equation}
e, se o coeficiente \(\chi\) (dito de Difusão e usualmente denotado por \(D\)) for independente do local, do tempo e da concentração, obtemos, finalmente, a equação clássica de Difusão:
\begin{equation}
\dfrac{\partial \rho}{\partial t}
= \chi\ \nabla \rho
= \chi\ \sum_{k=1}^{n} \dfrac{\partial^2 \rho}{\partial x_k^2}.
\end{equation}
em que \(\nabla = \displaystyle\sum_{k=1}^{n} \dfrac{\partial^2}{\partial x_k^2}\) é o operador diferencial de Laplace, o mais importante da teoria e aplicações de EDP.

Re-escrevendo a Equação de Difusão clássica como um Princípio de Conservação, podemos obtê-la, definindo um (funcional local) Fluxo da forma:
\begin{equation}\label{eq:difprincipioconservacao}
J = -\chi\ \nabla \rho.
\end{equation}

O modelo fundamental para o estudo da Difusão em Físico-Química é fundamentado em uma hipótese para o Fluxo de moléculas nesta forma, que é denominada de ``Lei de Fick'' (Os químicos e físicos professam muita fé na existência de ``Leis `Pétreas' da Natureza'' que, a rigor, são apenas ``Modelo/Hipóteses'' convenientes para descrevê-la). A Equação \eqref{eq:difprincipioconservacao}, acima, é mais conhecida em livros de EDP como \textbf{Equação do Calor}, já que é o modelo obtido quando a Termodinâmica é tratada fenomenologicamente, isto é, quando se trata o calor como um fluido. Neste caso, o Fluxo de Calor (energia térmica cuja densidade é \(Q(x,t)\)) é representado pela ``Lei de Newton'' \(J = -c\ \nabla Q\), uma precursora óbvia da lei de Fick. (P. J. Nahin - \textbf{Hot Molecules and Cold Electrons}, Princeton - UP, 2020).

É importante ressaltar que, o mesmo modelo matemático populacional representado pela equação de Difusão, pode ser interpretado, macroscopicamente, como resultante do comportamento microscópico de partículas sem qualquer iniciativa ou, de indivíduos com alta capacidade intelectual.

\subsubsection{DINÂMICA DO PULSO ARTERIAL}

Analisaremos, agora, rapidamente, um modelo de fluxo sanguíneo em vasos arteriais, assumindo que o sangue é um fluido de densidade constante \(\rho_0\) e ``ideal'', isto é, sem viscosidade. Este modelo é útil para uma análise preliminar de alguns aspectos dinâmicos da circulação (Keener-Sneyd).

\comentario{(Para remediar uma eventual ofensa causada pela radicalidade desta simplificação, observa-se que um modelo mais inclusivo do sangue deveria considerar um fluido não apenas viscoso mas, anisotrópico e não newtoniano, cujo modelo matemático pode igualmente ser estabelecido segundo os métodos de Euler, mas cuja complexidade permitiria poucas chances de uma análise matemática relevante).}

Um dos problemas fundamentais na fisiologia dinâmica do sistema circulatório diz respeito à análise do pulso arterial que decorre basicamente da interação entre a elasticidade das paredes arteriais e a pressão do fluxo sanguíneo, cujos aspectos mais fundamentais podem ser analisados com um modelo relativamente simples como aquele que apresentaremos. Apenas os princípios básicos da formulação do modelo matemático serão abordados, deixando o seu estudo mais detalhado para o(a) leitor(a) interessado(a) que se dispuser a consultar as referências abaixo.

Consideremos, então, um fluxo unidimensional do sangue ao longo da coordenada \(x\) de uma artéria com velocidade \(v(x,t)\) e pressão \(p(x,t)\) em um tubo cujas seções tem área \(A(x,t)\) em cada ponto \(x\) e instante \(t\). (Observe que as artérias são elásticas e a área seccional varia ao longo do comprimento e do tempo). O fluxo do sangue será de transporte com densidade \(J = \rho_0v\), e o fluxo total unidimensional através de uma seção será \(j(x,t) = \rho_0 v A\).

Portanto, em princípio, o modelo estabelece \textbf{três} funções incógnitas: \(A(x,t)\), \(p(x,t)\) e \(v(x,t)\) o que significa, segundo o ``Princípio de Compatibilidade'', que necessitaremos \textbf{três} equações para determiná-las completamente.

O primeiro aspecto considerado é, naturalmente, a conservação de massa sanguínea que descreveremos da maneira usual:
\begin{eqnarray}
& & \dfrac{d}{dt} \int_{x_1}^{x_2} \rho_0 A(x,t)\ dx \\
&=& j(x_1,t) - j(x_2,t) \\
&=& - \int_{x_1}^{x_2} \dfrac{\partial }{\partial t} \rho_0 v(x,t)\ A(x,t)\ dx,
\end{eqnarray}
de onde tiramos a primeira equação
\begin{equation}
\dfrac{\partial A}{\partial t} + \dfrac{\partial }{\partial x} (vA) = 0.
\end{equation}

A segunda consideração será o princípio mecânico de conservação de quantidade de movimento (ou, segunda lei de Newton) aplicada à massa de sangue de um segmento genérico \([x_1, x_2]\) da artéria na direção longitudinal. Observe que neste caso a quantidade de movimento (longitudinal) deste segmento é dada por:
\begin{equation}
\int_{x_1}^{x_2} q(x,t)\ dx,
\end{equation}
onde \(q(x,t) = \rho_0 A(x,t) v(x,t)\) é a densidade da quantidade de movimento, cujo fluxo de transporte é dado por \(J = vq\). Entretanto, a segunda lei de Newton afirma que há também um segundo fluxo de quantidade de movimento na fronteira do segmento causado pela força de pressão exercida sobre ela (as forças internas de pressão se anulam) de onde vem o Princípio de Conservação para a Quantidade de Movimento:
\begin{eqnarray}
\dfrac{d}{dt} \int_{x_1}^{x_2} \rho_0 A(x,t) v(x,t)\ dx \\
= J(x_1,t) - J(x_2,t) + p(x_1,t) A(x_1,t) \\
  - p(x_2,t) A(x_2,t)
\end{eqnarray}
implicando em
\begin{eqnarray}
\int_{x_1}^{x_2} \dfrac{\partial}{\partial t} (\rho_0 A(x,t) v(x,t))\ dx = \\
-\int_{x_1}^{x_2} \dfrac{\partial}{\partial x} \left[p(x_1,t) A(x_1,t) - p(x_2,t) A(x_2,t)\right] dx
\end{eqnarray}
de onde, fazendo uso do argumento de Euler, e já utilizando a primeira equação, obtemos a equação (\textbf{Exercício})
\begin{eqnarray}
\dfrac{\partial v}{\partial t} + \dfrac{1}{2} \dfrac{\partial}{\partial x} v^2 = \dfrac{1}{\rho_0} \dfrac{\partial p}{\partial x}.
\end{eqnarray}

Para completar o sistema de \textbf{três} equações a \textbf{três} incógnitas, utilizaremos uma relação constitutiva entre a pressão que o fluido exerce sobre as paredes longitudinais dos vasos e a sua dilatação seccional. Um dos modelos mais simples é representado por uma relação linear (``Lei'' de Hooke):
\begin{eqnarray}\label{eq:III}
A - A_0 = c(p - p_0)
\end{eqnarray}
onde \(A_0\) é a seção constante e não distendida da artéria, \(p\) é a pressão sobre a parede e \(p_0\) a pressão atmosférica e, finalmente, \(c\) uma constante de elasticidade.

Substituindo esta relação constitutiva \eqref{eq:III} nas equações anteriores, obtemos um sistema de duas equações com duas funções incógnitas \((v, p)\) que determina um sistema hiperbólico de EDP que pode ser analisado pela teoria desenvolvida para as Equações de Euler para a Dinâmica de Fluidos iniciada pelas famosas notas das aula de G. B. Riemann em 1855, redescobertas por Richard Courant e Kurt-Otto Friedrichs na década de 1940 como programa de guerra, desenvolvida por Peter Lax, e bem expostas por Whitham, Lax, e LeVeque). Esta linha histórica demonstra a importância do tema.

\subsubsection{Referências:}

J. Keener - J. Sneyd - \textbf{Mathematical Physiology}, Springer , 1998.

T. J. Pedley - \textbf{The Fluid Mechanics of Large Blood Vessels}, Cambridge U.Press 1980.

J. Lighthill - \textbf{Mathematical Biofluidynamics}, SIAM 1975.

M. Anliker et.al. - \textbf{Nonlinear Analysis of flow pulses and shock waves in arteries}, Z.Ang.Math.Phys. 22, (!971), (I): pg. 217-246, (II): pg. 563-581.


\subsubsection{DIFUSÃO: Modelo Macroscópico de Movimento Microscópico Orientado-Taylor \& Kac}

Nesta seção, apresentaremos um outro modelo matemático para o fenômeno de Difusão, totalmente oposto ao anterior sob o ponto de vista microscópico, mas que, surpreendentemente, é representado, macroscopicamente, pela mesma equação diferencial parcial. Este modelo tem origem nos trabalhos de L. Boltzmann, escritos ao final do século XIX, mas a sua ``popularização'' somente ocorreu com os trabalhos de Mark Kac, na década de 1960, embora tivesse sido considerado também por G. Taylor em dinâmica dos fluidos.

Consideremos uma população de partículas que trafegam em um tubo (unidimensional) com velocidade constante \(v\) mas separadas em duas subpopulações; uma que se movimenta para a direita e outra para a esquerda, com respectivas densidades \(\rho^+(x,t)\), \(\rho^-(x,t)\). Consideramos que não ocorra choque entre elas, mas que eventualmente elas se choquem elasticamente (isto é, sem perda de energia) com impurezas distribuídas ao longo do tubo, o que imediatamente modifica a direção de seu movimento para o sentido oposto. Consideremos que estes choques com impurezas ocorram segundo um modelo probabilístico de Poisson, ou seja, com uma dinâmica malthusiana. O objetivo final do modelo será obter uma equação (macroscópica, isto é, sem se referir diretamente a estes eventos) para a densidade total da população, \(\rho(x,t) = \rho^+(x,t) + \rho^-(x,t)\). 

\comentario{(A separação das duas populações é um mero artificio introduzido para desenvolver a argumentação do modelo).}

As equações para cada população será obtida utilizando um princípio de conservação em que o fluxo é de transporte (respectivamente) \(J^{\pm} = \pm v\rho^\pm\) e há fontes negativas de perda e positivas de ganho por conta dos choques que modificam as direções de percurso. Assim, temos:
\begin{eqnarray}
\dfrac{\partial \rho^+}{\partial t} &=& -\dfrac{\partial }{\partial x} (v\rho^+) - \lambda\rho^+ + \lambda\rho^- \\
\dfrac{\partial \rho^-}{\partial t} &=& -\dfrac{\partial }{\partial x} (v\rho^-) - \lambda\rho^- + \lambda\rho^+,
\end{eqnarray}
onde \(\lambda^(-1)\) é a constante de tempo de Malthus, isto é, o tempo médio de espera para que ocorra um choque. (ou, \(\exp(-\lambda t)\), a Probabilidade de que uma partícula não sofra um choque e, assim, mantenha sua direção de movimento durante todo o período de tempo \(t\)).

Escreveremos este sistema de duas equações operacionalmente, utilizando as notações simplificadas \(\partial_t = \dfrac{\partial }{\partial t}\) e \(\partial_x = \dfrac{\partial }{\partial x}\), e na forma:
\[\left[\begin{array}{cc}
L^+ & -\lambda \\
-\lambda & L^-
\end{array}\right]
\left[\begin{array}{c}
\rho^+ \\
\rho^-
\end{array}\right] = 0.\]


Observando que os operadores constantes da matriz operacional 
\[M = \left[\begin{array}{cc}
L^+ & -\lambda \\
-\lambda & L^-
\end{array}\right]\]
são comutativos (especialmente, \(L^+ L^- = L^- L^+\)), aplicaremos à igualdade do sistema a matriz adjunta de cofatores de \(M\) (isto é, regra de Crammer sem a divisão pelo determinante), de onde obteremos (pela própria regra de Crammer) um sistema desacoplado:
\[\left[\begin{array}{cc}
\det(M) & 0 \\
0 & \det(M)
\end{array}\right]
\left[\begin{array}{c}
\rho^+ \\
\rho^-
\end{array}\right] = 0.\]
de onde obteremos a equação satisfeita por \(\rho\):
\[\det(M) \{\rho\} = 0\]
onde, naturalmente,
\[\det(M) = L^+ L^- - \lambda^2 = \partial_t^2 - v^2\partial_x^2+2\lambda \partial_t,\]
ou ainda, na forma clássica, a equação de Kelvin (ou, do telégrafo)
\begin{eqnarray}
\dfrac{\partial \rho}{\partial t} = \left(\dfrac{v^2}{2\lambda}\right) \dfrac{\partial^2 \rho}{\partial x^2} + \left(\dfrac{-1}{2\lambda}\right) \dfrac{\partial^2 \rho}{\partial t^2}
\end{eqnarray}


Para completar o modelo matemático, consideraremos o caso em que a frequência de choques \(\lambda\) é extremamente alta, assim como a velocidade \(v\), de tal forma que \(\dfrac{1}{2\lambda} \ll 1\), enquanto \(\dfrac{v^2}{2\lambda} = D \approx 1\). Nestas condições, o modelo matemático pode ser reduzido, eliminando-se o ultimo termo da equação de Kelvin o que resultará na equação de difusão:
\begin{eqnarray}
\dfrac{\partial \rho}{\partial t} = \left(\dfrac{v^2}{2\lambda}\right) \dfrac{\partial^2 \rho}{\partial x^2}.
\end{eqnarray}
(A equação completa de Kelvin é tipicamente uma equação de propagação de ondas \(\partial_t^2\rho = v^2 \partial_x^2\rho - 2\lambda \partial_t\rho\)) com amortecimento viscoso (\(-2\lambda\partial_t\rho\)), o que representa, em princípio, fenômenos físicos completamente distintos daqueles descritos pela equação de difusão.

Surpreendentemente, todavia, há uma sobreposição assintótica no comportamento destes dois modelos para valores extremos de parâmetros, tal como aquele considerado acima).

Desde a década de 1990, estas ideias tem sido grandemente generalizadas e empregadas na construção de modelos matemáticos destinados a representar populações biológicas que se movimentam no espaço físico alternando percursos retilíneos com períodos de rápidas rotações em torno de um ponto fixo (``Run and Tumble''). O modo rotacional do movimento destina-se a recolher informações da vizinhança que serão utilizadas no próximo lance de corrida. Esta estratégia de movimentação exibe um certo caráter universal em biologia de populações, tendo sido observadas em diversas espécies de organismos microscópicos e macroscópicos, assim como no movimento de células epiteliais (reparadoras de tecidos) o que faz de seu estudo um tema com importância singular em Biomatemática. O espaço de aspecto adequado para representar este fenômeno no plano é \(\Omega = \{(x_1, x_2, \theta) \in \mathbb{R}^2 \times S_1\}\), onde cada indivíduo é caracterizado por sua posição no espaço físico \((x_1, x_2)\}\) e pela sua orientação \(\theta\). A densidade que descreverá o estado desta população será uma função \(\rho(x_1, x_2, \theta, t)\), \(2\pi\)-periódica na terceira variável. É interessante observar que, à semelhança das equações de Boltzmann para a Mecânica Estatística, estes são também modelos não-locais pois, indivíduos que se encontram fisicamente próximos mas distantes no espaço de aspecto (isto é, com orientações muito distintas) interagem entre si, o que nos levará inevitavelmente a equações integro-diferenciais para \(\rho(x_1, x_2, \theta, t)\). (ref. W. C. Ferreira Jr - \textbf{As Múltiplas Faces da Difusão}, Hillen \& Othmer, Mogilner).

Para uma interessante descrição histórica da relação entre a equação de difusão proposta por J. B. Fourier (1768 -1830), em 1807, para a transmissão do calor e a equação do telégrafo derivada por Oliver Heaviside 1830 e analisada por William Thompson (Lord Kelvin), em 1857, que possibilitou a comunicação transoceânica consulte o texto: Paul J. Nahin - \textbf{Hot Molecules, Cold Electrons}: From the Mathematics of Heat to the development of the Transatlantic Telegraph Cable, Princeton Univ. Press, 2020)

\subsubsection{MODELOS DE TRÁFEGO}

Diversos fenômenos interessantes de trafego em uma autoestrada tem sido estudados como uma população de veículos que se movimentam em um meio unidimensional com densidade \(\rho(x,t)\) e segundo um campo de velocidades dado por um funcional (pontual) da densidade \(v(\rho)\). Esta função \(v(\rho)\) é determinada experimentalmente e tem valor máximo para densidade nula e decresce até zero para uma densidade limite.

A equação de conservação é imediatamente escrita na forma:
\begin{eqnarray}
\dfrac{\partial \rho}{\partial t} + \dfrac{\partial }{\partial x} (\rho\ v(\rho)) = 0.
\end{eqnarray}

Esta simples EDP se refere a uma função incógnita escalar \(\rho(x,t)\) definida em um espaço unidimensional e apresenta um caráter não-linear o que faz dela um protótipo para o estudo de diversas questões importantes da classe geral de EDP denominadas Sistemas Hiperbólicos que representa uma enorme variedade de fenômenos denominados ``Choque'' que no caso de trafego são de fato relacionados a ``choques de veículos''. A ocorrência deste fenômeno depende da geometria de inflexão da curva função de velocidade \(v(\rho)\) o que, portanto, pode ser encarado como fator de controle (dinâmico) da velocidade máxima permitida em trechos dependendo da concentração de veículos na vizinhança. (v. Whitham, 1974, Bassanezi-Ferreira, 1988).

\subsubsection{TRANSPORTE FLUVIAL E SEDIMENTAÇÃO DE PARTICULADOS SUSPENSOS}

Consideremos um rio não turbulento como um canal retangular retilíneo com correnteza uniforme de margem a margem, de tal maneira que a seção longitudinal central representa todo o sistema e tem a forma de um retângulo de altura \(h\) e comprimento infinito: \((x_1, x_2) \in \mathfrak{R} = \mathbb{R} \times [0, h]\). Assim, a velocidade da correnteza será descrita pela função vetorial \(\vec{v}_r(x_1, x_2) = (V_r(x_1, x_2), 0)\) em que \(V(0) = 0\), ou seja, a velocidade no fundo do rio é nula e \(V(h)\) é a velocidade da sua superfície, em geral a máxima.

Consideraremos que a função \(V_r(x_1, x_2) \ge 0\), que descreve a correnteza do rio é conhecida e, portanto, o rio será representado pela sua seção longitudinal central.

Suponhamos, agora, que este rio contenha uma grande quantidade de partículas em suspensão (pequenas, mas macroscópicas) que são transportadas pela correnteza, mas que não influem nela, descritas pela função de densidade \(\rho(x_1, x_2, t)\), e que são submetidas também a um processo de sedimentação vertical que as deposita no leito do rio, de onde não mais se desprendem.

Portanto, as partículas já depositadas abandonam a sua condição de partícula em suspensão, contabilizada pela função densidade. A velocidade de queda vertical das partículas será denotada por \(\vec{v}_s(x_1, x_2) = (0, -V_a(x_1, x_2))\), \(V_a(x_1, x_2) \ge 0\), e pode ser analisada de diversas maneiras. Em qualquer caso, a velocidade resultante de uma partícula em suspensão será obtida pela soma vetorial \(v(x_1, x_2) = \vec{v}_s + \vec{v}_r = (V_r(x_1, x_2), -V_a(x_1, x_2))\).

Suponhamos que o movimento de correnteza do rio e a sedimentação ocorrem em uma escala de tempo razoavelmente pequena comparada à escala de tempo de processos de difusão molecular, o que nos leva a considerar apenas as dinâmicas de transporte e sedimentação.

Uma das maneiras de simplificar radicalmente este modelo é considerar um rio ``raso'' e verticalmente uniforme, de tal maneira que possamos representá-lo unidimensionalmente pela coordenada \(x_1 = x\), com uma correnteza \(\vec{v}_r(x) = (V_r(x))\). Neste caso, a sedimentação será descrita como um processo malthusiano que depende apenas da concentração e se dá com uma fonte (negativa) da forma \(-\mu\rho(x,t)\) (L. Leopold - \textbf{Fluvial Processes in Geomorphology}, Dover, 2020).

Modificações dos modelos descritos acima podem ser utilizadas para a representação de diversos processos de transporte e sedimentação de particulados, poluidores ou aluviões em correntes fluviais ou atmosféricas. A análise de processos em uma longa escala de tempo, como a nuvem radioativa resultante do desastre de Chernobyl - URSS, 1986 (número não identificado de mortos) ou de Brumadinho - MG, 2019 (259 mortos)) exige a consideração de fenômenos difusivos, a serem abordados no próximo capítulo.

{\small\color{blue}
\subsubsection*{Exercícios}

\begin{exercise}
*Considere \(\rho(x, v, t)\) a densidade de partículas de massa \(m\) e carga elétrica \(e\), distribuídas no espaço de fase (posição-velocidade \((x, v) \in \mathbb{R}^6\)), submetidas a um campo elétrico potencial ambiente \(E(x,t) = \nabla\phi\). Mostre que no espaço de fase a ``movimentação do aspecto'' \((x, v)\) de cada partícula se dará segundo o campo vetorial \(V(x, v, t) = (v, \frac{e}{m} \nabla\phi)\), e obtenha, como consequência a importante equação de Vlasov (1940):
\[\dfrac{\partial \rho}{\partial t} + v \dfrac{\partial \rho}{\partial x} + \dfrac{e}{m} \nabla\phi(x,t)\dfrac{\partial \rho}{\partial v} = 0,\]
para a dinâmica de plasmas eletrostáticos.

\comentario{(Se o campo elétrico ambiente não for muito forte ou se a população de partículas for muito densa, o campo elétrico \(E(x,t)\) deverá levar em consideração também da distribuição de cargas das próprias partículas, o que será expresso de maneira auto-consistente por uma equação adicional de Poisson, razão porque o modelo completo se denomina ``Vlasov-Poisson eletrostático''. O modelo se torna consideravelmente mais complexo se levarmos em conta também o campo magnético \(B\). Este modelo tem importância fundamental para o estudo da astrofísica e da fusão nuclear. (J. H. P. Goedbloed - \textbf{Principles of Magnetohydrodynamics}, Cambridge UP - online - 2010)).}
\end{exercise}


\begin{exercise}
Considere o modelo de sedimentação. Suponha que as partículas aproximadamente esféricas, todas de massa \(m\), estejam se decantando segundo a ``lei de Stokes'', isto é, em queda com velocidade limite de queda \(v_S\) constante (que depende de \(m\), da viscosidade do fluido \(\mu\) e do raio \(r\) das partículas). Obtenha uma equação para a dinâmica desta densidade. Qual seria a condição matemática adequada para ser imposta no leito do rio \(x_2 = 0\), que represente o fato (obvio para nós mas desconhecido do modelo matemático) de que as partículas não o atravessam?
\end{exercise}


\begin{exercise}
Mesma questão anterior, mas agora suponha que as partículas sejam distintas em raio \(r\), (\(R_0 > r > r_0 > 0\)), a densidade no espaço de aspecto, \(\rho(x_1, x_2, r, t)\), e a velocidade de queda segundo a lei Stokes variável com o raio, \(v_S = \dfrac{k}{\mu r}\), onde \(g\) é  a constante da gravidade, \(\mu\) a viscosidade e \(k\) uma constante dimensional.
\end{exercise}


\begin{exercise}
* Mesma questão anterior analisando agora o decantamento de partículas suspensas na atmosfera \(\Omega = \{(R, \theta, \phi) \in [R_0, \infty) \times S_2\}\) para a distribuição \(\rho(R, \theta, \phi, t)\) incluído campo de transporte (velocidade de ventos) \(v(R, \theta, \phi, t)\). Inicie pelo modelo mais simples em que se supõe uma simetria esférica, ou seja, a função densidade depende apenas da altitude \(R\), \(\rho(R, t)\). (Este é um exemplo típico em que o espaço de aspecto [o próprio espaço físico, no caso] não é exatamente o espaço euclideano plano.)
\end{exercise}


\begin{exercise}
Considere um modelo fluvial para um rio suficientemente raso para que o consideremos unidimensional com velocidade de correnteza \(v(x,t)\) e que o decantamento ocorra segundo uma sedimentação, proporcional à densidade, malthusiano. Escreva a equação do modelo para a densidade \(\rho(x,t)\) e obtenha a solução para o caso em que um ``derrame'' de poluentes ocorre com vazão (fonte)
\[f(x,t) = F_0 [H(t) - H(t - t_0)] [H(x) - H(x - x_0)],\]
onde \(F_0\) constante e \(H(s)\) é a função de Heaviside: \(H(s) = 0, s < 0\) e \(H(s) = 1\), para \(s > 0\).

Este problema é particularmente interessante sob o ponto de vista inverso, ou seja, conhecendo-se a função velocidade, obter a localização de um emissor pontual de poluente a montante conhecendo-se a densidade em pontos à jusante.
\end{exercise}

\begin{exercise}
Descreva um modelo de sedimentação de um rio unidimensional com emissor (clandestino) localizado em algum ponto ou mesmo em alguma faixa limitada do mesmo. Considerando medidas à jusante verifique como seria possível determinar a origem e intensidade deste emissor.
\end{exercise}
}

\subsubsection{ETC.}

\begin{description}
\item Ondas de Superfície em Oceanos.
\item Equações de Euler-manuscrito
\item Equações de Navier Stokes-manuscrito
\item Modelos de Imunologia-Segel \& Perelson
\end{description}


\section{BIBLIOGRAFIA}

V.Arnold-Sur la Geometrie differentielle des groupes de Lie de dimension infinite et ses applications a l’hydrodynamique des fluides parfaits, Ann.Inst.Grenoble 16(1) (1966),
319-361.

V.I.Arnold-Metodi Matematici della Meccanica Classica, MIR 1980

J.Barrow-Greene-Poincaré and the Three Body Problem, AMS 1997

H.Berg-Random Walk in Biology, Princeton UP 1985.

J.Bertoin-Random Fragmentation and Coagulation Processes, Cambridge Univ.Press 2005

N.N.Bogolyubov \& al. ed.-Euler and Modern Science, AMS 2007

J.T.Bonner-

S.Brush-The Motion we call Heat, 2 vol.

O.Buhler-Classical Mechanics, Waves and statístical Mechanics, AMS-Courant Lectures 2004

C.Cercignani-Ludwig Boltzmann: The man who trusted atoms, Oxford U.P.

S.Chandrasekhar-Stochastic Processes in Physics, 1941, reproduzido em N.Wax ed. 1954.

A.J.Chorin-O.Hald-Stochastic Methods in Science, Springer 2005.

A.J.Chorin-On the convergence of Discrete Approximation of Navier-Stokes Equations, Math. of Comp. 23 (1969), 341-53.

A.J.Chorin-J.E.Marsden-A Mathematical Introduction to Fluid Mechanics, 3rd edition Springer 2000

M.Choppard-M.Droz-Cellular Automata, Cambridge UP

P.Civitanovic-editor- The Chaos Book, Niels Bohr Institute-Copenhague- online

R.Courant-K.-O. Friedrichs-Supersonic Flow and Shock Waves, Wiley 1948

J.Cushing-An Introduction to Structured Population Dynamics, SIAM1998.

O.Darrigol-U.Frisch-From Newton’s Mechanics to Euler’s Equations, pp.: http://www.oca.eu/etc7/EE250/texts/darrigol-frisch.pdf

O.Darrigol-Worlds of Flow-Hydrodynamics from the Bernoullis to Prandtl, Oxford U.P. 2008

S.de Groot-P.Mazur-Non Equilibrium Thermodynamics, North-Holland/Dover, 1965

F.Diacu-Ph.Holmes-Celestial Encounters-The Origin of Chaos and Stability, Princeton UP 1997

O.Diekmann-J.A.P.Heesterbeek-Mathematical Epidemiology of Infectious Diseases, J.Wiley, 2000.

O.Diekmann-H.Heesterbeek-Mathematical Tools for Understanding Infectious Disease Dynamics,Princeton UP2012

L. Euler-Principes générauxs du mouvement des fluides, Mem.Acad.Roy.Sci. Berlin 1757, 274-315. (trad.ingles em arXiv:0802.2383v1-17 Feb 2008-U.Frisch)

L.Euler-Recherches Générales sur la mortalité et la multiplication du genre humain, Acad. Sci. Belgique 1760-Trad. Ingles\_ThPop.Biol 1(3)1970, 307-314.

L.Euler-Recherches sur populations.....

L.Evans-Partial Differential Equations, Springer

W.Fagan-R.S.Cantrell-C.Cosner-How Habitat Edges Change Species Interaction, The Am. Nat. 153(2), 1999, 166-182.

A.Fasano-The ``Volume Scattering'' Effect in Liquid-Liquid Dispersion, SIAM News2001: http://www.siam.org/pdf/news/523.pdf

E.Fellman- Euler, Birkhauser

P.Fife-Some Nonclassical Trends in Parabolic and Parabolic-like Evolutions, preprint-online

P.Fife-J. Carrillo-Gradient Flows, pp

J.H.P.Goedbloed-Principles of Magnetohydrodynamics, Cambridge UP2004-online

R.Graham-D.Knuth-Matematica Discreta, JO Ed. 1980

F.Hoppensteadt-Mathematical Theories of Populations, SIAM 1975.

I.Horen\_ko \& -....Sensitivity....

D.Hubel-T.Wiesel- Eye..., W.Freeman

G.Kanizsa\_Vedere e Pensare, Mulino 1991

J.P.Keener-J.P.Sneyd-Mathematical Physiology, Springer, 1998

N.Keyfitz-Applied Mathematical Demography , Springer-Verlag, 1977

B.V.Kogan-The Dynamical Theory of Rarified Gases, Plenum 1975

Mark Kot-Elements of Mathematical Ecology, Springer 2002

A.Lasota-M.Mackey-Chaos, Fractals and Noise Stochastic Dynamics, Springer 1986.

P.D.Lax-Hyperbolic Systems of PDE, AMS

L.Leopold-Fluvial Processes in Geomorphology, Dover 2010

R. LeVeque-Hyperbolic Systems of Conservation Laws,

V.G.Levich-Physico-chemical Hydrodynamics, Prentice-Hall 1962.

J.Lighthill-

C.C.Lin-L.A.Segel-Mathematics Applied to Nature, SIAM 1990

Th.Malthus-Essays on Population... 1798 (1a. edição)

M. Marder-L.Kadanoff \& -Flow and Diffusion of High-Stakes Test Scores, PNAS2009

J.Marsden-T.Hughes- Mathematical Foundations of Elasticity, P.-Hall/Dover 1983

Igor Mesic-Applied Koopmanism,...

D.C. Mistro-L.A.D.Rodrigues-W.C.Ferreira Jr.-The Africanized honey bee dispersal: a mathematical zoom, Bull.Math.Biol 2004-

D.Mollison-editor-Epidemic Models, Cambridge Univ. Press 1995. (H.Daniels-``A Perturbation Approach to Nonlinear Deterministic Epidemic Waves'', pg-202-214)

J.K.Moser-Is the Solar System Stable?, The Mathematical Intelligencer vol. 1, 1978, 65-71

J.D.Murray-Mathematical Biology, 2 vol. Springer 2002.

G.Odell-W.Fagan-.....Cannibalism...Am.Naturalist 1994

J.C.Neu-Transport and Fluids, AMS 2010

O.Penrose-The Becker-Döring equation for the kinetics of phase transition, preprint- 2001-online-HP- Penrose- Heriot-Wats University , England

Ch.S.Peskin-Mathematical aspects of Heart Physiology, NYU\_Courant Inst. Lectures 1974

Ch.S.Peskin-Partial Differential Equations in Biology, NUYU\_Courant Inst. Lect. 1975.

Ch.S.Peskin-The Immersed Boundary Method, Acta Numerica

Ch.S.Peskin-D.M.McQueen-Fluid Dynamics of the Heart and its Valves, in pg.309-337, H.Othmer \& al.ed.-Mathematical Modeling, PHall 1997.

S.V.Petrovskii-H.Malchow-A.B.Medvinsky-Noise and productivity dependence of spatiotemporal pattern formation in a prey-predator system, Discrete \& Cot. Dyn.Syst 4, 707-713, 2002.

S.Redner-A Guide to First Passage Processes, Cambridge UP 2001

G.F.B. Riemann - Die Partiellen Differential - Gleichungen - Vorlesungen, 1860 - publicado em 1869: http://gdz.sub.uni-goettingen.de/dms/load/img/?PIDPPN234595299 e https://archive.org/details/diepartiellendi00riemgoog

L.Schwartz-Théorie des Distributions, Hermann, 2nd. Ed. 1966.

L.V.Sedov-Mécanique des Millieux Continus, 2 vo. MIR 1985

L.A.Segel-L.E.-Keshet-A Primer on Mathematical Biology, SIAM 2015

L.A.Segel-A.S.Perelson-....Shape Space....1989

L.Sirovich- artigos diversos sobre população neuronal: v. HomePage.

J.J.Stoker-Water Waves, J.Wiley 1950

J.V.Stone-Vision, MIT

G.Strang-Introduction to Applied Mathematics, Wellesley/SIAM, 1980.

R.Strichartz-Differential Equations on Fractals: A Tutorial, Princeton Univ. Press 2006

S.Vogel-Comparative Biomechanics,

N.Wax, ed.-Selected Papers on Stochastic Processes, Dover 1954.

Th.Widiger ed.-The Oxford Handbook of the Five Factor Model, Oxford UP2017

G.Whitham-Linear and Nonlinear Waves, J. Wiley, 1974.

G.Zaslavsky-Chaos, fractional kinetics, and anomalous transport, Phys.Reports 371(2002), 461-580.

Ya.B.Zeldovich-Yu.P.Rizer-Elements of Gasdynamics and the Classical Theory of Shock Waves, Acad.Press 1968

R.Zwanzig-Non-Equilibrium statistical Mechanics, Oxford UP 2001



\chapter{APÊNDICE: Fenômenos de transporte segundo a Equação de Liouville}


Dada a importância dos chamados Fenômenos de Transporte analisaremos o movimento gerado por um campo de velocidades com mais detalhes por intermédio das trajetórias de pontos no Espaço de Aspecto.

Para efeito de simplicidade intuitiva analisemos o Fluxo de Transporte resultado do movimento gerado por um Campo de Velocidades autônomo \(v(x)\) no Espaço de Aspecto.

Assim, cada ponto do espaço segue uma trajetória determinada pela solução \(x = \varphi(t, x_0)\) do Problema de Cauchy:
\[\dfrac{dx}{dt} = v(x),\ x(0) = x_0.\]
Podemos interpretar a função \(\varphi(t, x_0)\), fixando \(x_0\) e considerando a função \(t \mapsto \varphi(t, x_0)\) como a trajetória do indivíduo que ocupava a posição \(x_0\) no instante \(t = 0\).

Por outro lado, fixado \(t\) e variando \(x_0\) em um subconjunto \(\Omega_0\), podemos interpretar a função \(x_0 \mapsto \varphi_t(x_0)\) como um transporte do conjunto de pontos originalmente situados em \(\Omega_0\), no instante \(t = 0\), para a sua posição atual \(\Omega_t = \{x = \varphi_t(x_0),\ x_0 \in \Omega_0\}\). Portanto, podemos analisar as trajetórias \(\Omega_t\) dos diversos subconjuntos do Espaço de Aspecto. Em particular, o Lema de Euler nos levam à Fórmula de Liouville:
\[\dfrac{d}{dt} \int_{\Omega} \rho(x,t)\ dx = \int_{\Omega} \left(\dfrac{\partial \rho}{\partial t} + \operatorname{div}(\rho v)\right)\ dx\]
que representa a taxa de variação de indivíduos contidos no subconjunto transportado \(\Omega_t\). Se não há mortes ou nascimentos nesta população, obviamente este conteúdo é conservado pois não há passagem pela fronteira \(\partial \Omega_t\), já que os indivíduos dela se movem solidários à fronteira.

Sobre a derivação de determinante de uma matriz \(A(t)\),
\[\dfrac{d}{dt}[\det A(t)] = \operatorname{Tr}\left[A^{-1} \dfrac{dA}{dt}\right] \det(A),\]
\begin{eqnarray*}
\dfrac{d}{dt} \operatorname{Vol}(\Omega_t)
&=& \int_{\Omega_t} \operatorname{div}[v(x)]\ dx \\
&=& \int_{\Omega_t} \sum \dfrac{\partial v_k}{\partial x_k}(x)\ dx
\end{eqnarray*}
e o Teorema de Liouville
\[
\dfrac{d}{dt} \int_{\Omega_t} h(x,t)\ dx = \int_{\Omega_t} \left(\dfrac{\partial h}{\partial t} + \operatorname{div}[hv]\right)\ dx
\]
consulte Bassanezi-Ferreira, 1988 ou W. C. Ferreira jr - As múltiplas faces da derivada, Ciência e Natureza, 2016).

Por outro lado, como o termo \(\int_{\Omega} \rho(x,t)\ dx\) calcula a taxa de variação da população contida em \(\Omega_t\) fixo, concluímos que a taxa de passagem de indivíduos pela fronteira deve ser dada por \(\int_{\Omega_t} -\operatorname{div}[\rho v]\ dx\). Lembrando-se do Teorema da divergência de Gauss
\[\left(\int_{\Omega} \operatorname{div}(\vec{V})\ dx = \int_{\partial\Omega} \vec{V} \cdot d\vec{S}\right)\]
concluímos que o fluxo total sobre a fronteira \(\partial \Omega\) de um subconjunto \(\Omega\) é dada por:
\[\int_{\Omega_t} \operatorname{div}(\rho v)\ dx = \int_{\partial\Omega_t} \rho v \cdot d\vec{S}.\]
Deste argumento, concluímos que o Fluxo de Transporte causado pelo movimento gerado por um campo de velocidades \(v\) é dado por:
\begin{equation}
\vec{j}_{\mbox{Transporte}} = \rho \vec{v}.
\end{equation}
A Equação de Conservação para uma Dinâmica de População representada por Função Densidade \(\rho\) e Fluxo de Transporte gerado por um Campo de Velocidades \(v\) é denominada, em particular, EQUAÇÃO DE LIOUVILLE:
\begin{equation}
\dfrac{\partial \rho}{\partial t} = -\operatorname{div}(\rho v)
\end{equation}
ou, na forma de uma recursão infinitesimal:
\begin{equation}
\dfrac{\partial \rho}{\partial t} = \mathfrak{L} \rho = -\sum \dfrac{\partial}{\partial x_k}(\rho v_k)
\end{equation}
onde o Operador Gerador da Dinâmica no Espaço de Densidades é \(\mathfrak{L} = -\sum \dfrac{\partial}{\partial x_k} v_k\).

O operador \(\mathfrak{L}\) será linear se o campo vetorial \(v(x,t)\) for previamente determinado como função direta de \((x,t)\), sendo particularmente importante no caso de Campos Autônomos \(v(x)\).

A Equação de Liouville tem grande importância na formulação de Princípios Probabilísticos (Chorin \& Hald) e na Formulação dos Métodos Reducionistas de Koopman (Mesic, Kutz) a serem tratados em outro capítulo.

Se o Modelo Populacional dispõe de seus três ingredientes, densidade \(\rho\), Fluxo de Transporte \(j = \rho \vec{v}\) e Fonte \(f\), o Princípio de Conservação é representado por uma Equação de Liouville não Homogênea:
\begin{equation}
\dfrac{\partial \rho}{\partial t} = \mathfrak{L} \rho + f = -\sum \dfrac{\partial}{\partial x_k}(\rho v_k) + f.
\end{equation}


\subsection{Exercícios*}

\noindent \textbf{Observação}: Os exercícios abaixo devem ser trabalhados \textit{ab ovo}, ou seja, sem ``fórmulas'' de mudança de variáveis.

\begin{exercise}
Considere um tubo retilíneo descrito ao longo de seu comprimento pela coordenada \(x_1\), que apresenta uma seção transversal com área \(A(x_1,t)\). Suponha que neste tubo se movimenta um ``fluido'' de partículas com densidade \(\rho(x_1, x_2, x_3, t)\) por ação do transporte causado por um campo de velocidades \(v(x,t)\) paralelo ao tubo. Escreva uma equação dinâmica para \((x,t)\) com base no Princípio de Conservação e reduza o problema a uma questão unidimensional ao longo do tubo. (v. exemplo sobre ondas em artérias mais adiante).
\end{exercise}

\begin{exercise}
Utilizando conjuntos de teste geometricamente adequados obtenha a Equação de Liouville em coordenadas polares no plano, esféricas, cilíndricas.
\end{exercise}


\begin{exercise}
Considere um Espaço de Aspecto representado pela superfície de uma esfera e obtenha a equação de Liouville neste caso.
\end{exercise}
