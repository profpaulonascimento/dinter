
\chapter*{Apresentação da disciplina}
\addcontentsline{toc}{chapter}{Apresentação da disciplina}

\begin{description}
\item \textbf{Disciplina}:
\subitem MS680 MODELOS MATEMÁTICOS EM BIOLOGIA /
\subitem MT624 BIOMATEMÁTICA I
\item \textbf{Período}: II Semestre de 2020.
\item \textbf{Professor}: Wilson Castro Ferreira Jr -
\href{mailto:wilson@unicamp.br}{wilson@unicamp.br}
\end{description}



\section{Introdução}

Esta disciplina se refere a duas siglas, uma pertencente ao catálogo de graduação (MS680-Modelos Matemáticos em Biologia) e outra ao catálogo da pós-graduação, MT624-Biomatemática I.

Os pré-requisitos matemáticos para esta disciplina são um bom conhecimento dos conceitos e técnicas referentes ao material apresentado tradicionalmente na sequência completa de Cálculo na graduação de Matemática ou áreas afins (Funções de uma e várias variáveis) acompanhado dos elementos de álgebra linear. (Os livros P. Lax, M. Tyrrell. \textbf{Calculus}, 2 vol., Springer e G. Strang. \textbf{Introduction to Linear Algebra}, Wellesley são exemplos de excelentes textos de estudo e referência para estas áreas.)

Biomatemática é, obviamente, uma disciplina de Interface entre a Biologia e a Matemática que, necessariamente, contem elementos específicos destas duas ciências representados por Métodos Matemáticos e Fenômenos Biológicos.

O objetivo central da Biomatemática é estabelecer pontes de duas mãos entre a Biologia e a Matemática na forma de Modelos Matemáticos que permite a descrição de conceitos e fenômenos biológicos por intermédio de uma linguagem matemática.

Os Métodos Matemáticos apresentados nesta disciplina devem ser apropriados ao contexto de suas aplicações à Biologia. Por este motivo, alguns tópicos (tais como Análise Dimensional, Análise Assintótica, Ondas não lineares, e etc.) que não fazem parte da ementa das tradicionais disciplinas de Métodos de Matemática Aplicada, serão introduzidos e desenvolvidos como parte essencial do curso.

Analogamente, os Fenômenos Biológicos escolhidos para exemplificar a sua representação matemática serão apresentados ``ab ovo”, isto é, apenas com a presunção de conhecimentos formais de Biologia, Física ou Química do curso secundário. Uma introdução simples aos conceitos não matemáticos necessários para a compreensão de cada um destes tema será apresentada em cada caso. Espera-se, todavia, um interesse e uma disposição real do/as aluno/as para a leitura de material que trate de forma
acessível dos temas específicos de Biologia.

Naturalmente, nem todos os fenômenos biológicos são vantajosamente representáveis por objetos matemáticos e nem todas as teorias matemáticas são apropriadas para a representação de algum fenômeno biológico. Uma exposição pedagógica da Biomatemática exige, portanto, a escolha de áreas contíguas de uma e outra ciência cujas inter-relações exemplifiquem melhor este processo.

Enfim, a Biomatemática é uma ciência que contém elementos importantes da Matemática e da Biologia mas que tem o seu foco central nas pontes transitáveis na fronteira entre elas.

A importância de um Modelo Matemático no entendimento de um fenômeno biológico não deriva necessariamente da importância e nem da sofisticação da teoria Matemática empregada em sua formulação. Na verdade, quanto mais simples for a Matemática empregada para a compreensão de um dado fenômeno biológico, melhor será a qualidade do Modelo Matemático uma vez que poderá ser explorado com maior eficácia. Este é o Princípio de Parcimônia Ockam (``Não complique sem necessidade; a vida já é suficientemente complicada por si mesma'' ou, ``Quem ajuda a complicação não vai muito longe, ela já vem de outras formas inevitáveis'').

Por outro lado, é comum supor que a Biomatemática derive a sua relevância unicamente da importância dos fenômenos biológicos que a Matemática ajuda a entender, mas, na verdade ela estabelece pontes de duas mãos e, frequentemente, problemas matemáticos de grande interesse por si mesmos podem ser vantajosamente tratados por intermédio de conceitos biológicos provenientes da sua interpretação como um modelo matemático de fenômenos biológicos. Exemplos destes são as ``Redes Neurais'', os
``Algoritmos de busca Ant Colony'', ``Algoritmos Genéticos'' dentre vários outros.

Portanto a Biomatemática, não consiste apenas de uma Biologia Matemática, mas também de uma Matemática Biológica.

Consequentemente, o núcleo desta disciplina consiste na construção e interpretação de Modelos Matemáticos para a Biologia.

Em linhas gerais, espera-se que o/a aluno/a ao final deste curso tenha:
\begin{enumerate}
\item Adquirido conhecimento da teoria e da aplicação de alguns Métodos Matemáticos especialmente úteis em Biomatemática
\item Tomado conhecimento de alguns temas fundamentais da Biologia sob o ponto de vista desta disciplina e apreciado a sua importância cientifica intrínseca, e, principalmente,
\item Adquirido conhecimento e habilidade em representar uma ampla classe de fenômenos biológicos em linguagem matemática e, vice-versa, interpretar diversas questões matemáticas sob o ponto de vista matemático enfatizando assim um trânsito de duas mãos entre áreas especificas da Biologia e da Matemática.
\end{enumerate}

A grande maioria dos temas de Biologia que podem ser matematicamente representados enquadram-se sumariamente no título ``Dinâmica de Populações'' que significa simplesmente (e genericamente) a descrição temporal do “tamanho” de uma coleção de indivíduos. A generalidade desta expressão provem do termo ``População'' cujos indivíduos podem ser moléculas (reações enzimáticas da fisiologia), células (imunológicas, neurológicas, etc) a microorganismos (bactérias, fungos, vírus), organismos sociais (insetos) e mesmo (!) subconjuntos de homo sapiens e etc. O tratamento matemático para a descrição e estudo de fenômenos biológicos tão distintos com a utilização de uma mesma linguagem e, repetidamente, com as mesmas teorias matemáticas representa uma síntese característica da própria natureza da Biomatemática. (É mais um exemplo da capacidade de síntese da Matemática cuja ocorrência mais elementar e antiga é a própria Aritmética; uma teoria que se aprende com a contagem de ``bolinhas de gude'' em tenra infância, mas que se mostra indispensável para múltiplas situações totalmente distintas ao longo da vida que seriam impossíveis enumerar ou prever.)

\section{Programa geral}

Esta disciplina seguirá o programa de um texto de Biomatemática (cujo título provisório é ``Elementos de Biomatemática: Uma Introdução Narrativa de Princípios, Métodos e Fins'') que venho redigindo ao longo de vários anos de experiencia adaptativa de magistério nesta área, tanto na graduação quanto na pós-graduação. Estas notas são devedoras de dois textos clássicos que estabeleceram as bases da Biomatemática moderna (J. D. Murray. \textbf{Mathematical Biology}, Springer-Verlag, 1989-2002; Lee A. Segel. \textbf{A Primer of Mathematical Biology}, SIAM 1989-2016), e de várias outras referências mais específicas, notadamente J. P., Keener. J. Sneyd. \textbf{Mathematical Physiology}, Springer, 1998-20010). 

Entretanto, apesar desta sua estreita conexão com a literatura principal do assunto, o presente texto apresenta uma estrutura com características distintas que procuram torna-lo mais adequado ao contexto brasileiro, tanto no seu aspecto básico e conceitual quanto aos temas biológicos tratados que, em várias circunstâncias, se referem especificamente a questões de origem local, ainda que sempre de interesse universal. (Mimetismo Mulleriano, Dispersão de abelhas africanas, Propagação da dengue).

O Programa da disciplina espelhará a visão panorâmica do texto que é representada por três partes distintas, mas correlatas e interdependentes: Princípios, Métodos e Fins.

Os Princípios tratam de argumentos eficazes para a formulação dos Modelos Matemáticos a partir de hipóteses biológicas fundamentais. Ou seja, os Modelos constituem essencialmente um ``dicionário'' entre a linguagem biológica e as suas representações Matemáticas e servirão para a transposição de significados nas duas direções. Esta parte é o núcleo principal e as espinha dorsal do texto e do curso em torno da qual estão dispostas as outras partes.

Uma vez formulado o Modelo Matemático para um fenômeno biológico, o próximo passo é analisá-lo com as ferramentas que são disponíveis na Matemática. Entretanto, nem todos os resultados e deduções matemáticas sobre o Modelo tem interesse biológico. A análise de propriedades matemáticas que se traduzem em conhecimento do fenômeno biológico respectivo exige, não raro, a aplicação de métodos que não são os usuais da Matemática Aplicada à Física ou à Engenharia. A apresentação dos Métodos tradicionais complementados por aqueles especificamente apropriados à Biomatemática é o tema desta segunda parte.

A terceira parte da estrutura deste texto e da disciplina consiste em exemplificar a aplicação dos Princípios e Métodos a questões específicas e de interesse real em Biologia com a sua representação na forma de Modelos Matemáticos e o estudo de algumas de suas questões mais fundamentais. O escopo da escolha destes temas é abrangente e permitirá o tratamento introdutório de um enorme leque de importantes áreas da Biologia que exemplificam as várias nuances dos procedimentos descritos.
Apresentaremos a seguir uma descrição esquemática e panorâmica desta estrutura.

\begin{enumerate}
\item PRINCÍPIOS:

\begin{enumerate}
\item Princípios de Representação Matemática (Análise Dimensional):
\item Princípio de Malthus: Homogeneidade e Espontaneidade
\item Princípio de Estruturação (Espaços de Aspecto e Dinâmica de Transporte)
\item Princípio de Difusão (Heterogeneidade Dinâmica-Entropia):
\item Princípios de Interação (Interação Indireta-Competição, “Lei” de Ação de Massas, Respostas Funcionais de Holling, Redes de
Influências não locais)
\item Princípios Variacionais (Fermat, Huygens-Bellman, Gradiente, Dirichlet):
\item Princípios de Reconstrução: (“Data Driven Models”)-
\item Princípios Probabilísticos (Poisson, Markov, Langevin )
\item Princípios de Discretização (Automatos Celulares, CML -``Coupled Map Lattice'', Redes de Influência)
\end{enumerate}

\item MÉTODOS:

Métodos Funcionais (Fourier e Operacionais), Assintóticos (Perturbação, Integral, Múltiplas Escalas), Redução (DMD, Koopman), DMKD (``Data Mining and Knowledge Discovery'' e Estatísticos), Variacionais, Sistemas Dinâmicos, Analíticos, Equações Diferenciais Parciais, Computacionais e de Simulação

\item FINS

\begin{enumerate}
\item Ecologia: (Invasão, Conservação e Extinção, Predação)-
\item Fisiologia (Neurobiologia, Visão, Morfogênese, Imunologia, Sistema Circulatório, Vascularização, Motores Moleculares, Reações Enzimáticas, Ondas em Meios Excitáveis, Oncologia, Realimentação de Sinais)
\item Sociologia (Cultural, Epidemiologia, Demografia, Quorum, Cooperação)
\item Evolução (Mimetismo Mulleriano, Dinâmica)
\item Matemática Biológica- (Sociobiologia de Insetos Sociais, Redes Neurais, Microrganismos)
\end{enumerate}

\item QUESTÕES BUROCRÁTICAS:

\begin{enumerate}
\item Aulas

As aulas não serão presenciais e nem participativas, ou seja, não pretendemos utilizar ferramentas como Google Classroom. Em um primeiro momento o material será exposto capítulo por capítulo no espaço Moodle segundo as Notas de Aulas citadas acima.
    
O objetivo é complementar a apresentação da matéria das Notas de Aulas com Video-aulas em um futuro mais próximo possível.

A interação com o/as estudantes se dará por intermédio de questões que serão enviadas por email ao professor e respondidas para toda a turma, no caso de interesse geral.

\item Exercícios:

Exercícios serão resolvidos nas Notas de Aula, em Video Aulas e outros serão propostos.

\item Provas

Haverá três provas que constarão dos exercícios propostos nas Notas de Aula até a data respectiva. A resolução de todos estes exercícios deverá ser apresentada ao professor por email até a data indicada da prova. Embora o estudo em grupo (virtual) seja encorajado, será necessário que cada estudante apresente sua própria versão em um documento pdf digitalizado (manuscritos não serão aceitos!!!).

Cópias entre si ou de outras fontes (Google, textos e etc) de trechos da resolução implicará na anulação automática de todas as questões semelhantes.

\item Avaliação:

A avaliação das Provas utilizará os conceitos binários S (suficiente) e I (Insuficiente) e para a aprovação será necessária a obtenção de TRÊS Suficientes.

Caso o/a aluno/a tenha obtido apenas duas avaliações Suficientes, a aprovação dependerá da obtenção de um conceito Suficiente no Exame.

\item Datas de Provas:
\begin{enumerate}
\item 1\textordfeminine\ Prova: 29 de outubro 2020;
\item 2\textordfeminine\ Prova: 22 de dezembro de 2020;
\item 3\textordfeminine\ Prova: 14 de janeiro de 2021;
\item Exame: 21 de Janeiro de 2021
\end{enumerate}
\end{enumerate}

\end{enumerate}
