
\allowdisplaybreaks % pacote amsmath quebra de pag usando eqnarray




%%%%%%% TEOREMAS

{\theoremstyle{definition}
\newtheorem{exemplo}{\textbf{Exemplo}}[chapter]
%\newtheorem{questao}{\hspace{-0.1in}}[part]
\newtheorem{etapa}{Etapa}
\newtheorem{q}{}[chapter]
\newtheorem{algoritmo}{\color{Gray}{\sf Algoritmo}}[chapter]
\newtheorem{axioma}{Axioma}
\newtheorem{problema}{Problema}
\newtheorem{ERCN}{\color{Blue}}[chapter]
\newtheorem{er}{\color{Gray}{\sf ER}}
\newtheorem{ER}{\color{\FontERColor}{\sf ER}}[chapter]
\newtheorem{EP}{\color{\FontEPColor}{\sf EP}}[chapter]


\newtheorem{nota}{Nota}{\normalfont \em}
\setcounter{nota}{0}%
\renewcommand\thenota{\arabic{nota}}


%\pslongbox{nota}{%
%\psshadowbox[shadowcolor=gray,shadowsize=.15in,framearc=.3,framesep=10pt,doubleline=false,boxsep=false]%
%}
%
%\newcommand{\notes}[2][0.3333]{%
%\begin{minipage}[t]{#1\textwidth}
%\parbox[b]{\textwidth}{\pslongbox{%
%\centering
%\input{lgc/#2.inl}}
%\end{minipage}}

\newtheorem{questao}{\color{darkgray}{\!\!\sf A}}%[chapter]
%\newtheorem{EP}{\color{darkgray}{\sf EP}}[chapter]
%\newtheorem{exercise}{\color{Orange}{\sf\!\!\!\!A\!}}
\newtheorem{EV}{}
\newtheorem{exercicio}{\color{Gray}{\sf ER}}
\swapnumbers% modifica a posição da numeração nos newtheorem

\newtheorem{definicao}{Definição}%[chapter]
%\newtheorem{theorem}[definicao]{Teorema}
\newtheorem{corolario}{Corolário}%\theoremstyle{break}
\newtheorem{lema}{Lema}
\newtheorem{proposicao}{Proposição}
\newtheorem{observacao}{Observação}
\newtheorem{agradecimentos}{Agradecimentos}
\newtheorem{caso}{Caso}
\newtheorem{afirmacao}{Afirmação}
\newtheorem{conclusao}{Conclusão}
\newtheorem{condicao}{Condição}
\newtheorem{conjectura}{Conjectura}
\newtheorem{criterio}{Critério}
\newtheorem{prova}{\sc Prova}
\newtheorem{notacao}{\hspace{0.9in}{\sc Notação}}
\newtheorem{solucao}{Solução}
}


\usepackage{tikz} 

\usepackage{xcolor} % Required for specifying colors by name
%\definecolor{ocre}{RGB}{0,50,205} % Define the color used for highlighting throughout the book

\definecolor{ocre}{RGB}{29,92,174} % Define the color used for highlighting throughout the book

%----------------------------------------------------------------------------------------
%THEOREM STYLES
%----------------------------------------------------------------------------------------

\newcommand{\intoo}[2]{\mathopen{]}#1\,;#2\mathclose{[}}
\newcommand{\ud}{\mathop{\mathrm{{}d}}\mathopen{}}
\newcommand{\intff}[2]{\mathopen{[}#1\,;#2\mathclose{]}}
\newtheorem{notation}{Notation}[chapter]

% Boxed/framed environments
\newtheoremstyle{ocrenumbox}% % Theorem style name
{0pt}% Space above
{0pt}% Space below
{\normalfont}% % Body font
{}% Indent amount
{\small\bf\sffamily\color{ocre}}% % Theorem head font
{\;}% Punctuation after theorem head
{0.25em}% Space after theorem head
{\small\sffamily\color{ocre}\thmname{#1}
\nobreakspace\thmnumber{#2}% Theorem text (e.g. Theorem 2.1)
\thmnote{\nobreakspace\the\thm@notefont\sffamily\bfseries\color{black}---\nobreakspace#3.}} % Optional theorem note
\renewcommand{\qedsymbol}{$\blacksquare$}% Optional qed square

\newtheoremstyle{blacknumbox} % Theorem style name
{0pt}% Space above
{0pt}% Space below
{\normalfont}% Body font
{}% Indent amount
{\small\bf\sffamily}% Theorem head font
{\;}% Punctuation after theorem head
{0.25em}% Space after theorem head
{\small\sffamily\thmname{#1}
\nobreakspace\thmnumber{#2}% Theorem text (e.g. Theorem 2.1)
\thmnote{\nobreakspace\the\thm@notefont\sffamily\bfseries---\nobreakspace#3.}}% Optional theorem note

\newtheoremstyle{blacknumex}% Theorem style name
{5pt}% Space above
{5pt}% Space below
{\normalfont}% Body font
{} % Indent amount
{\small\bf\sffamily}% Theorem head font
{\;}% Punctuation after theorem head
{0.25em}% Space after theorem head
{\small\sffamily{\tiny\ensuremath{\blacksquare}}\nobreakspace\thmname{#1}\nobreakspace\thmnumber{#2}% Theorem text (e.g. Theorem 2.1)
\thmnote{\nobreakspace\the\thm@notefont\sffamily\bfseries---\nobreakspace#3.}}% Optional theorem note

% Non-boxed/non-framed environments
\newtheoremstyle{ocrenum}% % Theorem style name
{5pt}% Space above
{5pt}% Space below
{\normalfont}% % Body font
{}% Indent amount
{\small\bf\sffamily\color{ocre}}% % Theorem head font
{\;}% Punctuation after theorem head
{0.25em}% Space after theorem head
{\small\sffamily\color{ocre}\thmname{#1}\nobreakspace\thmnumber{\@ifnotempty{#1}{}\@upn{#2}
}% Theorem text (e.g. Theorem 2.1)
\thmnote{\nobreakspace\the\thm@notefont\sffamily\bfseries\color{black}---\nobreakspace#3.}
} % Optional theorem note
\renewcommand{\qedsymbol}{$\blacksquare$}% Optional qed square
\makeatother

% Defines the theorem text style for each type of theorem to one of the three styles above
\newcounter{dummy} 
\numberwithin{dummy}{section}
\theoremstyle{ocrenumbox}
\newtheorem{theoremeT}{Teorema}
\newtheorem{corollaryT}{Corolário}
\newtheorem{lemmaT}{Lema}
\newtheorem{propositionT}{Proposição}
\newtheorem{problem}{Problem}[chapter]
\theoremstyle{blacknumex}
\newtheorem{exerciseT}{EP.}%[chapter]
\newtheorem{exampleT}{ER.}%[chapter]
\theoremstyle{blacknumbox}
\newtheorem{vocabulary}{Vocabulary}[chapter]
\newtheorem{definitionT}{Definição}%[section]
\theoremstyle{ocrenum}

%----------------------------------------------------------------------------------------
%DEFINITION OF COLORED BOXES
%----------------------------------------------------------------------------------------
\definecolor{asparagus}{rgb}{0.53, 0.66, 0.42}
\definecolor{icterine}{rgb}{0.99, 0.97, 0.37}

\RequirePackage[framemethod=default]{mdframed} % Required for creating the theorem, definition, exercise and corollary boxes

% Theorem box
\newmdenv[skipabove=6pt,
skipbelow=6pt,
backgroundcolor=ocre!25,
rightline=false,
leftline=true,
topline=false,
bottomline=false,
linecolor=ocre,
innerleftmargin=3pt,
innerrightmargin=3pt,
innertopmargin=18pt,
innerbottommargin=12pt,
leftmargin=0cm,
rightmargin=0cm,
linewidth=4pt]{tBox}

% theorem box
\newmdenv[skipabove=6pt,
skipbelow=6pt,
backgroundcolor=ocre!15,
rightline=false,
leftline=true,
topline=false,
bottomline=false,
linecolor=ocre!70,
innerleftmargin=3pt,
innerrightmargin=3pt,
innertopmargin=12pt,
innerbottommargin=6pt,
leftmargin=0cm,
rightmargin=0cm,
linewidth=4pt]{ptBox}

% Exercise box  
\newmdenv[skipabove=6pt,
skipbelow=6pt,
backgroundcolor=icterine!30,
rightline=false,
leftline=true,
topline=false,
bottomline=false,
linecolor=yellow,
innerleftmargin=3pt,
innerrightmargin=3pt,
innertopmargin=18pt,
innerbottommargin=12pt,
leftmargin=0cm,
rightmargin=0cm,
linewidth=4pt]{eBox}


% Example box
\newmdenv[skipabove=6pt,
skipbelow=6pt,
backgroundcolor=orange!30,
rightline=false,
leftline=true,
topline=false,
bottomline=false,
linecolor=orange,
innerleftmargin=3pt,
innerrightmargin=3pt,
innertopmargin=18pt,
innerbottommargin=12pt,
leftmargin=0cm,
rightmargin=0cm,
linewidth=4pt]{exBox}

% Example Solution box
\newmdenv[skipabove=6pt,
skipbelow=6pt,
backgroundcolor=orange!12,
rightline=false,
leftline=true,
topline=false,
bottomline=false,
linecolor=orange,
innerleftmargin=3pt,
innerrightmargin=3pt,
innertopmargin=18pt,
innerbottommargin=12pt,
leftmargin=0cm,
rightmargin=0cm,
linewidth=4pt]{sexBox}


% Definition box
\newmdenv[skipabove=6pt,
skipbelow=6pt,
backgroundcolor=black!20,
rightline=false,
leftline=true,
topline=false,
bottomline=false,
linecolor=black!70,
innerleftmargin=3pt,
innerrightmargin=3pt,
innertopmargin=18pt,
innerbottommargin=12pt,
leftmargin=0cm,
rightmargin=0cm,
linewidth=4pt]{dBox}



% Proposition box  
\newmdenv[skipabove=6pt,
skipbelow=6pt,
backgroundcolor=ocre!30,
rightline=false,
leftline=true,
topline=false,
bottomline=false,
linecolor=ocre!70,
innerleftmargin=3pt,
innerrightmargin=3pt,
innertopmargin=18pt,
innerbottommargin=12pt,
leftmargin=0cm,
rightmargin=0cm,
linewidth=4pt]{pBox}

% Proposition Proof box
\newmdenv[skipabove=6pt,
skipbelow=6pt,
backgroundcolor=ocre!15,
rightline=false,
leftline=true,
topline=false,
bottomline=false,
linecolor=ocre!70,
innerleftmargin=3pt,
innerrightmargin=3pt,
innertopmargin=12pt,
innerbottommargin=6pt,
leftmargin=0cm,
rightmargin=0cm,
linewidth=4pt]{ppBox}

% Corollary box  
\newmdenv[skipabove=6pt,
skipbelow=6pt,
backgroundcolor=asparagus!30,
rightline=false,
leftline=true,
topline=false,
bottomline=false,
linecolor=asparagus!70,
innerleftmargin=3pt,
innerrightmargin=3pt,
innertopmargin=18pt,
innerbottommargin=12pt,
leftmargin=0cm,
rightmargin=0cm,
linewidth=4pt]{cBox}

% Corollary Proof box
\newmdenv[skipabove=6pt,
skipbelow=6pt,
backgroundcolor=asparagus!15,
rightline=false,
leftline=true,
topline=false,
bottomline=false,
linecolor=asparagus!70,
innerleftmargin=3pt,
innerrightmargin=3pt,
innertopmargin=12pt,
innerbottommargin=6pt,
leftmargin=0cm,
rightmargin=0cm,
linewidth=4pt]{pcBox}


% Creates an environment for each type of theorem and assigns it a theorem text style from the "Theorem Styles" section above and a colored box from above

\newenvironment{theorem}{\begin{tBox}\begin{theoremeT}}%{\hfill{\tiny\ensuremath{\blacksquare}}}
{\end{theoremeT}\end{tBox}}

\newenvironment{ptheorem}{\begin{ptBox}\textbf{Demonstração}:}
{\hfill{\tiny\ensuremath{\blacksquare}}
\end{ptBox}}

\newenvironment{proposition}{\begin{pBox}\begin{propositionT}}
{%\hfill{\tiny\ensuremath{\blacksquare}}
\end{propositionT}\end{pBox}}

\newenvironment{pproposition}{\begin{ppBox}\textbf{Demonstração}:}
{\hfill{\tiny\ensuremath{\blacksquare}}
\end{ppBox}}

\newenvironment{corollary}{\begin{cBox}\begin{corollaryT}}
{%\hfill{\tiny\ensuremath{\blacksquare}}
\end{corollaryT}\end{cBox}}

\newenvironment{pcorollary}{\begin{pcBox}\textbf{Demonstração}:}
{\hfill{\tiny\ensuremath{\blacksquare}}
\end{pcBox}}

\newenvironment{lemma}{\begin{pBox}\begin{lemmaT}}
{%\hfill{\tiny\ensuremath{\blacksquare}}
\end{lemmaT}\end{pBox}}

\newenvironment{plemma}{\begin{ppBox}\textbf{Demonstração}:}
{\hfill{\tiny\ensuremath{\blacksquare}}
\end{ppBox}}



\newenvironment{exercise}{\begin{eBox}\begin{exerciseT}}{\end{exerciseT}\end{eBox}}

\newenvironment{definition}{\begin{dBox}\begin{definitionT}}{\end{definitionT}\end{dBox}}

\newenvironment{example}{\begin{exBox}\begin{exampleT}}{\end{exampleT}\end{exBox}}

\newenvironment{sexample}{\begin{sexBox}\textbf{Solução}:}
{%\hfill{\tiny\ensuremath{\blacksquare}}
\end{sexBox}}

%----------------------------------------------------------------------------------------
%REMARK ENVIRONMENT
%--------------------------
\newenvironment{remark}{\par\vspace{10pt}\small % Vertical white space above the remark and smaller font size
\begin{list}{}{
\leftmargin=35pt % Indentation on the left
\rightmargin=25pt}\item\ignorespaces % Indentation on the right
\makebox[-2.5pt]{\begin{tikzpicture}[overlay]
\node[draw=ocre!60,line width=1pt,circle,fill=ocre!25,font=\sffamily\bfseries,inner sep=2pt,outer sep=0pt] at (-15pt,0pt){\textcolor{ocre}{R}};\end{tikzpicture}} % Orange R in a circle
\advance\baselineskip -1pt}{\end{list}\vskip5pt} % Tighter line spacing and white space after remark
%---------------------



% NEWCOMMAND
\newcommand{\dsum}{\displaystyle \sum}
\newcommand{\dprod}{\displaystyle \prod}
\newcommand{\dlim}{\displaystyle \lim}
\newcommand{\dliminf}{\displaystyle \liminf}
\newcommand{\dlimsup}{\displaystyle \limsup}
\newcommand{\dint}{\displaystyle \int\limits}
\newcommand{\diint}{\displaystyle \iint\limits}
\newcommand{\diiint}{\displaystyle \iiint\limits}
\newcommand{\doint}{\displaystyle \oint\limits}
\newcommand{\doiint}{\displaystyle \oiint\limits}
\newcommand{\dcup}{\displaystyle \cup}
\newcommand{\dcap}{\displaystyle \cap}
\newcommand{\dbigcup}{\displaystyle \bigcup}
\newcommand{\dbigcap}{\displaystyle \bigcap}
\newcommand{\rad}{\operatorname{rad}}
\newcommand{\dist}{\operatorname{dist}}
\newcommand{\gr}{\operatorname{gr}}
\newcommand{\graf}{\operatorname{Graf}}
\newcommand{\rot}{\operatorname{rot}}
\newcommand{\grad}{\operatorname{grad}}
\newcommand{\divergente}{\operatorname{div}}
\newcommand{\dom}{\operatorname{Dom}}
\newcommand{\cd}{\operatorname{CD}}
\newcommand{\df}{\operatorname{d}\!f}
\newcommand{\dF}{\operatorname{d}\!F}
\newcommand{\dg}{\operatorname{d}\!g}
\renewcommand{\dh}{\operatorname{d}\!h}
\newcommand{\dm}{\operatorname{d}\!m}
\newcommand{\dx}{\operatorname{d}\!x}
\newcommand{\dy}{\operatorname{d}\!y}
\newcommand{\dz}{\operatorname{d}\!z}
\newcommand{\dt}{\operatorname{d}\!t}
\newcommand{\ds}{\operatorname{d}\!s}
\newcommand{\dS}{\operatorname{d}\!S}
\newcommand{\du}{\operatorname{d}\!u}
\newcommand{\dw}{\operatorname{d}\!w}
\newcommand{\dv}{\operatorname{d}\!v}
\newcommand{\dA}{\operatorname{d}\!A}
\newcommand{\dr}{\operatorname{d}\!r}
\newcommand{\dP}{\operatorname{d}\!P}
\newcommand{\dV}{\operatorname{d}\!V}
\newcommand{\dtheta}{\operatorname{d}\!\theta}
\newcommand{\dphi}{\operatorname{d}\!\phi}
\newcommand{\drho}{\operatorname{d}\!\rho}
\newcommand{\dsigma}{\operatorname{d}\!\sigma}
\newcommand{\interior}{\operatorname{int}}
\newcommand{\sgn}{\operatorname{sgn}}

\renewcommand{\sin}{\operatorname{sen}}
\renewcommand{\cos}{\operatorname{cos}}
\renewcommand{\tan}{\operatorname{tan}}
\renewcommand{\cot}{\operatorname{cot}}
\renewcommand{\sec}{\operatorname{sec}}
\renewcommand{\csc}{\operatorname{csc}}
\renewcommand{\arcsin}{\operatorname{asen}}
\renewcommand{\arccos}{\operatorname{acos}}
\renewcommand{\arctan}{\operatorname{atan}}
\newcommand{\arccot}{\operatorname{acot}}
\newcommand{\arcsec}{\operatorname{asec}}
\newcommand{\arccsc}{\operatorname{acsc}}
\renewcommand{\sinh}{\operatorname{senh}}
\renewcommand{\cosh}{\operatorname{cosh}}
\renewcommand{\tanh}{\operatorname{tanh}}
\renewcommand{\coth}{\operatorname{coth}}
\newcommand{\sech}{\operatorname{sech}}
\newcommand{\csch}{\operatorname{csch}}
\newcommand{\asinh}{\operatorname{asinh}}
\newcommand{\acosh}{\operatorname{acosh}}
\newcommand{\atanh}{\operatorname{atanh}}
\newcommand{\acoth}{\operatorname{acoth}}
\newcommand{\asech}{\operatorname{asech}}
\newcommand{\acsch}{\operatorname{acsch}}
\newcommand{\Arg}{\operatorname{Arg}}
\renewcommand{\ln}{\operatorname{ln}}
\newcommand{\Ln}{\operatorname{Ln}}
\newcommand{\Log}{\operatorname{Log}}
\newcommand{\exterior}{\operatorname{Exterior}}
\newcommand{\ivet}{\vec{\imath}}
\newcommand{\jvet}{\vec{\jmath}}
\newcommand{\kvet}{\vec{\kappa}}
\renewcommand{\vec}{\mathbf}

%%%%%%% DEFINIÇÕES

\def\addt{
\addtocounter{part}{1}
\addtocounter{chapter}{1}
\setcounter{section}{0}
\setcounter{subsection}{0}
\setcounter{questao}{0}
\setcounter{exemplo}{0}
%\thispagestyle{empty}
}

\def\ereservado#1#2{
\noindent
\begin{minipage}[!h]{1.0\linewidth}
\begin{pspicture}(0,-#2)(1.05\linewidth,0)
\multido{\i=0+1}{#1}{%
\psset{unit=0.5cm}
\rput{0}(0,-\i){\psline[linewidth=0.7\pslinewidth,linecolor=black!40](0,0)(1.0\linewidth,0)}}
\end{pspicture}
\end{minipage}
}

\def\instrucoes{
\begin{enumerate}
\renewcommand{\baselinestretch}{1.25}
\scriptsize
\parskip = 0.01in
\item{} \underline{Desligue} o celular. \underline{Não} é permitido o seu uso durante a prova;
\item{} Durante a avaliação, a saída da sala e qualquer forma de consulta não será permitida;
\item{} A \underline{interpretação} de cada questão é parte integrante da prova;
\item{} Só serão \underline{validadas} as questões justificadas com todos os \underline{cálculos} na folha de respostas;
\item{} Seja \underline{organizado} e evite rasurar a avaliação. Para isso, resolva a avaliação a \underline{lápis} e apresente a resposta final a \underline{caneta}.
\end{enumerate}
}


\def\instrucoesead{
\renewcommand{\baselinestretch}{1.25}
\scriptsize
\parskip = 0.01in

Leia atentamente as normas que seguem sob pena de perda parcial ou total de sua questão/trabalho.
\begin{enumerate}
\item{} Não serão aceitos trabalhos idênticos aos dos demais estudantes.
\item{} Não serão aceitos trabalhos de estudantes distintos escritos com a mesma caligrafia.
\item{} Os trabalhos devem ser manuscritos e escaneados.
\item{} O trabalho deve ser enviado pelo AVA conforme prazo lá  especificado.
\item{} Não serão aceitos trabalhos enviados após o prazo.
\item{} Não serão aceitos trabalhos enviados por email.
\end{enumerate}
}


\def\cabecalho{
\noindent
\begin{pspicture}(0,-8)(1.0\linewidth,0)
\pspolygon[fillstyle=solid,fillcolor=ocre!10,linecolor=ocre!10](0,-8)(1.00\linewidth,-8)(1.00\linewidth,0)(0,0)
%\uput[r](-0.02\linewidth,-1.05){\scalebox{0.3}{\brasaoufrbtex}}
\uput[r](0.02\linewidth,-0.2){\resizebox{!}{0.118in}{\MakeUppercase{\!\textsc{\ufrbcetec}}}}
%\uput[r](0.196\linewidth,-0.9){\textsc{\bf Disciplina}:\: \emph \componentecurricular}
%\uput[r](0.196\linewidth,-1.3){\textsc{\bf Docente(a)}:\: \emph \docente}
\uput[r](0.02\linewidth,-0.67){\textbf{COD}: \ufrbcomponentecodigo}
\uput[r](0.37\linewidth,-0.67){\textbf{\ufrbcomponenteturma}}
\uput[r](0.52\linewidth,-0.67){\textbf{Data}: \underline{\hspace{0.35in}}/\underline{\hspace{0.35in}}/\underline{\hspace{0.35in}}}
\uput[r](0.02\linewidth,-1.12){\textbf{Discente}:\: \underline{\hspace{0.74\linewidth}}}
\rput(0.500\linewidth,-1.7){\resizebox{!}{0.13in}{\textsc{\avaliacao}}}
\rput(0.500\linewidth,-2.1){\anosemestre}
\uput[r](0.0\linewidth,-2.7){
\begin{minipage}[r]{0.98\linewidth}
{\cal \scriptsize
\mensagem%{21}
}
\end{minipage}
}
\uput[r](0.0\linewidth,-5.2){
\begin{minipage}[r]{0.7\linewidth}
\instrucoes
\end{minipage}
}
\uput[r](0.7\linewidth,-5.2){
\scalebox{0.9}{
\begin{minipage}[r]{0.15\linewidth}\scriptsize
\quadrodenotas
\end{minipage}
}}
\end{pspicture}
\parskip=0.1cm
}


\def\cabecalhoead{
\noindent
\begin{pspicture}(0,-8)(1.0\linewidth,0)
\pspolygon[fillstyle=solid,fillcolor=ocre!10,linecolor=ocre!10](0,-8)(1.00\linewidth,-8)(1.00\linewidth,0)(0,0)
%\uput[r](-0.02\linewidth,-1.05){\scalebox{0.3}{\brasaoufrbtex}}
\uput[r](0.02\linewidth,-0.2){\resizebox{!}{0.118in}{\MakeUppercase{\!\textsc{\ufrbcetec}}}}
%\uput[r](0.196\linewidth,-0.9){\textsc{\bf Disciplina}:\: \emph \componentecurricular}
%\uput[r](0.196\linewidth,-1.3){\textsc{\bf Docente(a)}:\: \emph \docente}
\uput[r](0.02\linewidth,-0.67){\textbf{COD}: \ufrbcomponentecodigo}
\uput[r](0.29\linewidth,-0.67){\textbf{\ufrbcomponenteturma}}
\uput[r](0.55\linewidth,-0.67){\textbf{Data}: \underline{\hspace{0.35in}}/\underline{\hspace{0.35in}}/\underline{\hspace{0.35in}}}
\uput[r](0.02\linewidth,-1.12){\textbf{Discente}:\: \underline{\hspace{0.74\linewidth}}}
\rput(0.500\linewidth,-1.7){\resizebox{!}{0.13in}{\textsc{\avaliacao}}}
\rput(0.500\linewidth,-2.1){\anosemestre}
\uput[r](0.0\linewidth,-2.7){
\begin{minipage}[r]{0.98\linewidth}
{\cal \scriptsize
\mensagem%{21}
}
\end{minipage}
}
\uput[r](0.0\linewidth,-5.2){
\begin{minipage}[r]{0.7\linewidth}
\instrucoesead
\end{minipage}
}
\uput[r](0.7\linewidth,-5.2){
\scalebox{0.9}{
\begin{minipage}[r]{0.15\linewidth}\scriptsize
\quadrodenotas
\end{minipage}
}}
\end{pspicture}
\parskip=0.1cm
}






\pagestyle{fancy} % Estilo de página {headings,myheadings,plain,empty,fancy}
\paperheight = 29.7cm % Altura do papel
\paperwidth = 21.6cm% Largura do papel
\hoffset = -1in % 1 inch + \hoffset
\voffset= -1in % 1 inch + \voffset
\topmargin = 0.25cm % Modifies the top margin of a page
\headheight = 1.5cm % Tamanho que separa o corpo do texto do cabeçalho
%\headwidth = 40pt
\headsep = 0.25cm
\oddsidemargin = 2.5cm % Adjusts the position of the left margin relative to 1 inch for even
\evensidemargin = 1.5cm % Muda a posição da margem esquerda
\marginparwidth = 2.5cm
\marginparsep = 0.2cm
%\marginparpush = -300pt
\footskip = 2.5cm
%\renewcommand{\footheight}{2.0cm}
\textheight = 25.7cm% Muda a altura do texto entres as margens superior e inferior
\textwidth = 16.6cm % Muda a largura do texto entre as margens esquerda e direita
%\unitlength = 4in
\setlength{\parindent}{0in} % Distância que o parágrafo toma da margem esquerda%
\parskip = 0.15in % Distância entre parágrafos.
\renewcommand{\baselinestretch}{1.5}
\renewcommand{\headrulewidth}{0pt}
\renewcommand{\footrulewidth}{0pt}

\fancyhf{} % Clear all fields
%\fancyheadoffset[LE]{2.0cm}


\def\docente{Paulo Henrique Ribeiro do Nascimento}
\def\componentecurricular{Cálculo e Geometria Analítica}
\def\titulodoc{Apontamentos}
\def\sizetitulodoc{7.0pt}



\fancyhead[LO]{%Ímpares
\psparallelogrambox[parallelogramsep=0,linestyle=none,fillstyle=gradient,gradbegin=ocre!90,gradend=ocre!50]{%
\resizebox{!}{0.3\headheight}{\textcolor{white}{\textsf{\textbf{\qquad\titulodoc\qquad}}}}}
}

\fancyfoot[LO]{%Ímpares
\uput[dl](\textwidth,\footskip){
\psparallelogrambox[parallelogramsep=0,linestyle=none,fillstyle=gradient,gradbegin=ocre,gradend=ocre!40]{%
\resizebox{0.35\footskip}{!}{\textcolor{white}{\textsf{\textbf{\thepage}}}}}
\uput[ur](1.4,0.5\footskip){%
\rotatebox{90}{
\psparallelogrambox[parallelogramsep=0,linestyle=none,fillstyle=gradient,gradbegin=ocre,gradend=black]{%
\resizebox{0.45\textheight}{!}{\textcolor{white}{\textsf{\textbf{\componentecurricular}}}}}}
}}}

\fancyhead[R]{
\psparallelogrambox[parallelogramsep=0,linestyle=none,fillstyle=gradient,gradbegin=ocre!90,gradend=ocre!50]{%
\resizebox{!}{0.3\headheight}{\textcolor{white}{\textsf{\textbf{\qquad\titulodoc\qquad}}}}}
}

\fancyfoot[R]{
\uput[dr](-\textwidth,\footskip){
\psparallelogrambox[parallelogramsep=0,linestyle=none,fillstyle=gradient,gradbegin=ocre,gradend=ocre!40]{%
\resizebox{0.35\footskip}{!}{\textcolor{white}{\textsf{\textbf{\thepage}}}}}}
\uput[ul](-1.02\textwidth,0.9\footskip){
\rotatebox{90}{
\psparallelogrambox[parallelogramsep=0,linestyle=none,fillstyle=gradient,gradbegin=ocre,gradend=black]{%
\resizebox{0.45\textheight}{!}{\textcolor{white}{\textsf{\textbf{\componentecurricular}}}}}
}}}


\def\tableofcontentspage{
%\sectionimage{section_head_1.pdf} % Table of contents heading image
\pagestyle{empty} % No headers
%\dominitoc
\parskip=0.1cm
\tableofcontents % Print the table of contents itself
%\cleardoublepage % Forces the first section to start on an odd page so it's on the right
%\renewcommand{\baselinestretch}{1.5}
\pagestyle{fancy} % Print headers again
}




%% TIMES FONT
\newcommand\timesfamily{\fontfamily{ptm} \selectfont}
\DeclareTextFontCommand{\texttimes}{\timesfamily}

%% PALATINO FONT
\newcommand\palatinofamily{\fontfamily{ppl} \selectfont}
\DeclareTextFontCommand{\textpalatino}{\palatinofamily}

%%%%%% Sem Serifa (Sans Serif)
% HELVÉTICA font
\newcommand\helveticafamily{\fontfamily{phv} \selectfont}
\DeclareTextFontCommand{\texthelvetica}{\helveticafamily}

%% Avante Garde font
\newcommand\avantefamily{\fontfamily{pag} \selectfont}
\DeclareTextFontCommand{\textavante}{\avantefamily}
%
%
%% CONCRETE ROMAN
\newcommand\concretefamily{\fontfamily{ccr} \selectfont}
\DeclareTextFontCommand{\textconcrete}{\concretefamily}
%
%% NEW CENTURY SCHOOL BOOK
\newcommand\newcenturyfamily{\fontfamily{pnc} \selectfont}
\DeclareTextFontCommand{\textnewcentury}{\newcenturyfamily}
%
%% BOOKMAN
\newcommand\bookmanfamily{\fontfamily{pbk} \selectfont}
\DeclareTextFontCommand{\textbookman}{\bookmanfamily}
%
%
%
%%%%%%%%%%%% Mono Espaçado (type writer)
%% Courier font
\newcommand\courierfamily{\fontfamily{pcr} \selectfont}
\DeclareTextFontCommand{\textcourier}{\courierfamily}
%
%%%%%%%%%%%% Manuscrito
%% Chancery
\newcommand\chanceryfamily{\fontfamily{prm} \selectfont}
\DeclareTextFontCommand{\textchancery}{\chanceryfamily}
%%% \noindent \scalebox{5}{\rmfamily Zaft Chancery}
%
%
%%%%%%%%%%%% Manuscrito
%% Chancery
\newcommand\calligrafamily{\fontfamily{pcg} \selectfont}
\DeclareTextFontCommand{\textcalligra}{\rmfamily}
%%% \noindent \scalebox{5}{\rmfamily Zaft Chancery}
%
% yinitial fonts.
\newcommand\yinitfamily{\usefont{U}{yinit}{m}{n}}
\DeclareTextFontCommand{\textyinit}{\yinitfamily}

\def\normalfont{\helveticafamily}




\def\myoddpage{
\begin{pspicture}(0,0)(17.2,0.5)
\uput[l](17.5,0.25){
\psframebox[%
linestyle=none,%
fillstyle=gradient,%
gradbegin=lightgray,
gradend=white,%
gradmidpoint=0,%
shadowcolor=white,%
%slopecenter={1.2},%
gradangle = 90,%
framesep=0.7pt]{\textcolor{gray}{\small \hfil \rightline{\emph{\sf \nouppercase\disciplina} \:}}}
}
\uput[r](17.0,0.25){
\psframebox[linecolor=gray,linewidth=0.5pt,framesep=2.5pt,framearc=0.5,shadowcolor=white]{\: $\thepage$ \:}}
\uput[r](16.9,0.25){
\psframebox[fillstyle=solid,%
fillcolor=white,%
linecolor=gray,%
linewidth=0.5pt,framesep=7.5pt,framearc=1.5,shadowcolor=white]{\hspace{-10pt}}}
\end{pspicture}
}


\def\myevenpage{
\begin{pspicture}(0,0)(17.2,1.5)
\uput[r](-0.25,0.05){
\psframebox[%
linestyle=none,
fillstyle=gradient,%
gradbegin=black!30,
gradend=white,%
gradmidpoint=1,%
shadowcolor=white,%
%slopecenter={1.2},%
gradangle = 120,%
framesep=0.8pt]
{\textcolor{black!50}{\small \emph{\sc \titulodoc} \hfill $\looparrowright$ \hfill \emph{\sc \disciplina} \hfil}}}
%
\uput[r](15,0.05){
\psframebox[linecolor=black!90,linewidth=0.55pt,framesep=3.5pt,framearc=0.25,fillstyle=gradient,%
gradbegin=black!10,
gradend=white,%
]{\: \thepage \:}}
%
%\uput[l](16.9,0.05){
%\psframebox[fillstyle=solid,%
%fillcolor=white,%
%linecolor=gray,%
%linewidth=0.5pt,framesep=7.5pt,framearc=1.5,shadowcolor=white]{\hspace{-10pt}}}
\end{pspicture}
}







\def\disciplina{Geometria Analítica}
\def\titulodoc{Notas de Aula}


\def\BorderColor{green!99}
\def\BorderLineColor{green!60}
\def\BackgroundColor{white}

\def\BackgroundCapaColor{green!30}
\def\BorderLineCapaColor{green!99}
\def\FontCapaColor{green!30}

\def\BackgroundProofColor{green!10}
\def\BorderLineProofColor{green}
\def\FontProofColor{green}

\def\BackgroundResColor{green!20}
\def\BorderLineResColor{green}

\def\BackgroundSolveColor{green!50}
\def\BorderLineSolveColor{green!99}
\def\FontSolveColor{white}

\def\FontERColor{green}
\def\FontEPColor{green!99}


\hyphenation{pa-râ-me-tros}


\def\solveal{\noindent
\psframebox[%
fillstyle=solid,%
fillcolor=magenta,%
framearc=0.15,%
linewidth=0.75pt,
linecolor=black!60,
%linestyle=solid,
%shadowcolor=gray,%
%shadowsize=0.05in,%
%linestyle=none,%
%fillstyle=gradient,%
%gradbegin=lightgray,
%gradend=white,%
%gradmidpoint=0,%
%shadowcolor=white,%
%slopecenter={1.2},%
gradangle = 45,%
framesep=2.7pt]{
\begin{minipage}[!h]{0.10\linewidth}\centering
\resizebox{!}{0.10in}{\textsc{\textcolor{white}{\textbf{Solução}:}}}
\end{minipage}
}\:}


\def\solve{\noindent
\psframebox[%
fillstyle=solid,%
fillcolor=\BackgroundSolveColor,%
framearc=0.15,%
linewidth=0.75pt,
linecolor=\BorderLineSolveColor,
%linestyle=solid,
%shadowcolor=gray,%
%shadowsize=0.05in,%
%linestyle=none,%
%fillstyle=gradient,%
%gradbegin=lightgray,
%gradend=white,%
%gradmidpoint=0,%
%shadowcolor=white,%
%slopecenter={1.2},%
gradangle = 45,%
framesep=2.7pt]{
\begin{minipage}[!h]{0.10\linewidth}\centering
\resizebox{!}{0.10in}{\textsc{\textcolor{\FontSolveColor}{\textbf{Solução}:}}}
\end{minipage}
}\:}

\def\proof{\noindent
\psframebox[%
fillstyle=solid,%
fillcolor=\BackgroundProofColor,%
framearc=0.15,%
linewidth=0.75pt,
linecolor=\BorderLineProofColor,
%linestyle=solid,
%shadowcolor=gray,%
%shadowsize=0.05in,%
%linestyle=none,%
%fillstyle=gradient,%
%gradbegin=lightgray,
%gradend=white,%
%gradmidpoint=0,%
%shadowcolor=white,%
%slopecenter={1.2},%
gradangle = 45,%
framesep=2.7pt]{
\begin{minipage}[!h]{0.17\linewidth}\centering
\resizebox{!}{0.10in}{\textsc{\textcolor{\FontProofColor}{\textbf{Demonstração}:}}}
\end{minipage}
}\:}


\newenvironment{worduglystyle}[1]%
{\spaceskip=.33em plus \hsize%
#1}%
{}


