
\chapter{Integração}
\addcontentsline{toc}{chapter}{Integração}
%\addt

%\input{integracaocomplexa.tex}





\chapter{Teoremas de Cauchy-Gousart e Aplicações}
\addcontentsline{toc}{chapter}{Teoremas de Cauchy-Gousart e Aplicações}
%\addt

%%%%%\input{umpoucosobrecauchy.tex} %paulo




\section{O Teorema de Cauchy-Gousart}



\subsection{Fórmula Integral de Cauchy Generalizada}






\chapter{Séries de Taylor e Séries de Laurent} % Nelian


\definicao{Série de Laurent}{def:01.03}{%Definição 3.
Uma função analítica na coroa circular (anel) $R_1 \le |z - z_{0}| \le R_2$ pode ser representada pela expressão
$$f(z) = \dsum_{n=-\infty}^{\infty} C_{n}(z-z_{0})^{n}$$
na região $R1 < R_{a} \le |z-z_{0}| \le R_{b} < R_{2}$, onde os coeficientes são dados por
$$C_{n} = \dfrac{1}{2\pi i} \doint_{C} f(z) (z-z_{0})^{n+1} dz$$
e $C$ é um contorno fechado simples na região de analiticidade encerrando a fronteira interior $|z-z_{0}| = R_{1}$. Para o contorno, considere a Figura 7.
}


