
\chapter{Equações diferenciais ordinárias}

  Em analogia à seção anterior, onde revisamos as variáveis complexas através de exemplos, aqui vamos proceder de maneira análoga. Um pouco das equações diferenciais ordinárias com o intuito de, após o método de Frobenius, apresentar as funções especiais, em especial a classe das funções hipergeométricas e seus casos particulares.

\exemplo{}{
Seja $x \in \mathbb{R}_{+}\ast$. Considere a equação diferencial ordinária não homogênea do tipo Euler
\begin{equation}
\label{eq:EDOH_euler}
x^{2} \dfrac{d^{2}}{dx^{2}} y(x) - x\dfrac{d}{dx} y(x) + y(x) = x \ln(x).
\end{equation}

Obtenha a solução dessa equação diferencial ordinária satisfazendo as condições $y(1) = 1$ e $y(e) = e$.
}


\solexemplo{
Começamos com a respectiva equação diferencial homogênea. Vamos propor uma solução na forma
\begin{equation}
\label{eq:EDOH_eulerpropostasolucao}
y_{H} (x) = x^r,
\end{equation}
em que $r$ é uma constante.

Calculando as derivadas, substituindo na equação diferencial homogênea e simplificando, temos:
$$r(r - 1) - r + 1 = 0,$$
a chamada equação auxiliar, uma equação algébrica com soluções dadas por $r_1 = 1 = r_2$, isto é, raiz dupla, de onde segue que temos apenas uma solução da equação diferencial homogênea, a saber
$$y_1(x) = x.$$

Para obter uma outra solução linearmente independente, procuramos por uma função $u(x)$ impondo que a solução $y_2(x) = x u(x)$ satisfaça a equação diferencial homogênea.

Calculando as derivadas, substituindo na equação diferencial homogênea e simplificando, podemos escrever:
$$x\dfrac{d^{2}}{dx^{2}}u(x) + \dfrac{d}{dx}u(x) = 0.$$

Para resolver esta equação diferencial (redutível), efetuamos a mudança de variável dependente $u'(x) = v(x)$.

Logo, obtemos uma equação diferencial ordinária homogênea de primeira ordem
$$x\dfrac{d}{dx}v(x) + v(x) = 0$$
cuja solução é dada por $v(x) = C/x$, em que $C$ é uma constante arbitrária. 

Voltando na variável dependente $u(x)$, obtemos uma outra equação diferencial ordinária de primeira ordem
$$\dfrac{d}{dx}u(x) = \dfrac{C}{x},$$
com solução dada por $u(x) = C \ln(x)+D$, em que $C$ e $D$ são constantes arbitrárias.

Logo, uma solução linearmente independente da equação diferencial ordinária, homogênea e de segunda ordem, é dada por
$$y_2(x) = x \ln(x).$$

Segue, então, que a solução geral da respectiva equação diferencial homogênea é dada por:
$$y_H(x) = A y_{1}(x) + B y_{2}(x) = Ax + Bx \ln(x),$$
com $A$ e $B$ constantes arbitrárias.

Enfim, devemos agora, obter uma solução particular da equação diferencial não homogênea o que será feito através do método de variação das constantes (variação dos parâmetros).

Consideremos a solução dada na forma
$$y_P (x) = A(x)x + B(x)x \ln(x)$$

Calculando as derivadas primeira e segunda, e impondo a condição (arbitrária, conveniente)
$$A'(x) + B'(x) \ln(x) = 0$$
e substituindo na equação diferencial não homogênea, podemos escrever
$$A'(x) + (1 + \ln(x))B'(x) = \dfrac{\ln(x)}{x}.$$

Resolvendo o sistema nas variáveis $A'(x)$ e $B'(x)$, temos:
$$A'(x) = -\dfrac{[\ln(x)]^2}{x} \mbox{ e } B'(x) = \dfrac{\ln(x)}{x}.$$

Integrando (não levando em consideração a constante de integração, pois queremos uma solução), obtemos:
$$A(x) = -\dfrac{1}{3}[\ln(x)]^3 \mbox{ e } B(x) = \dfrac{1}{2}[\ln(x)]^2.$$

Voltando com tais valores na expressão que fornece uma solução particular e rearranjando, podemos escrever:
$$y_P (x) = \dfrac{x}{6}[\ln(x)]^3.$$

A fim de obter a solução geral da equação diferencial ordinária não homogênea, basta adicionar a solução geral da respectiva equação homogênea com a solução particular da equação não homogênea. Logo,
$$y(x) = Ax + B \ln(x) + \dfrac{x}{6} [\ln(x)]^3,$$
sendo $A$ e $B$ constantes arbitrárias.

Devemos, agora, impor as condições a fim de determinar as constantes.

Impondo a condição $y(1) = 1$, determinamos $A = 1$, enquanto a condição $y(e) = e$, fornece $B = -e/6$. Então, a solução da equação diferencial ordinária não homogênea, satisfazendo as condições de contorno (fronteira) é dada por:
$$y(x) = \dfrac{1}{6} [6x - e \ln(x) + x [\ln(x)]^3],$$
que é o resultado desejado.
}

\indent

\indent

Como pode ser notado, os coeficientes no exemplo anterior não são constantes. Por exemplo, o que acontece em $x = 0$ que é um ponto singular, pois está multiplicando a derivada segunda? Em princípio deve ser excluído do domínio.

Para responder a essa pergunta, vamos introduzir o método de Frobenius que nos conduz a uma análise relativa às raízes da chamada equação indicial/auxiliar.

\exemplo{}{%Exemplo 2.
Seja $x \in \mathbb{R}$. Utilize o método de Frobenius para discutir a equação hipergeométrica confluente, também conhecida como equação de Kummer,
\begin{equation}
\label{eq:kummer}
x\dfrac{d^{2}}{dx^{2}}y(x) + (c-x)\dfrac{d}{dx}y(x) - ay(x) = 0,
\end{equation}
com $a, c \in \mathbb{R}$ e $c \ne -n$, sendo $n = 0, 1, 2, \ldots.$
}

\solexemplo{O método de Frobenius faz uso da metodologia das séries de potências. Vamos procurar soluções na forma
$$y(x) = \displaystyle\sum_{k=0}^{\infty} a_{k} x^{k+s},$$
com $a_{0} \ne 0$ e $s$ um parâmetro a ser determinado.

Calculando as derivadas primeira e segunda, substituindo-as na equação diferencial e rearranjando, podemos escrever:
$$\displaystyle\sum_{k=0}^{\infty} (k+s)(k+s-1+c) a_{k} x^{k+s-1} - \displaystyle\sum_{k=0}^{\infty} (k+s+a) a_{k} x^{k+s} = 0$$
que, a partir da mudança de índice, $k \to k - 1$, no segundo somatório, nos conduz a
$$\displaystyle\sum_{k=0}^{\infty} (k+s)(k+s-1+c) a_{k} x^{k+s-1} - \displaystyle\sum_{k=1}^{\infty} (k+s-1+a) a_{k-1}x^{k+s-1} = 0.$$

Separando o termo $k = 0$ no primeiro somatório e rearranjando, temos:
$$s(s-1+c) a_{0} x^{s-1} + \displaystyle\sum_{k=1}^{\infty} [(k+s)(k+s-1+c)a_{k}-(k+s-1+a) a_{k-1}] x^{k+s-1} = 0.$$

Dessa expressão e sabendo que $a_{0} \ne 0$, obtemos:
$$s(s - 1 + c) = 0$$
a chamada \textbf{equação indicial (auxiliar)}. Esta é uma equação algébrica de segundo grau, admite duas raízes que, eventualmente, podem ser complexas. 

Aqui, as soluções são dadas por $s_1 = 0$ e $s_{2} = 1 - c$.

Por outro lado, temos a chamada \textbf{fórmula (relação) de recorrência}, obtida levando em conta que a única maneira de termos um zero no primeiro membro é considerando os coeficientes nulos. Então, neste caso, temos:
$$(k+s)(k+s-1+c) a_{k} - (k+s-1+a) a_{k-1} = 0.$$

Antes de continuarmos, façamos uma observação. Aqui temos duas raízes reais da equação auxiliar, distintas ou iguais, dependendo dos valores do parâmetro $c$ e a fórmula de recorrência envolvendo dois termos, $a_{k}$ e $a_{k-1}$.

A partir de agora, devemos considerar, \textbf{separadamente}, as raízes da equação indicial. Começamos com a raiz $s_1 = 0$, que nos leva à fórmula de recorrência
$$a_{k} = \dfrac{k-1+a}{k(k-1+c)} a_{k-1},$$
com $k = 1, 2, \ldots$ e $c \ne -n$, sendo $n = 0, 1, 2, \ldots$.

Da relação anterior, devemos explicitar o coeficiente que depende de $k$ em termos de uma constante, pois a única certeza é que $a_{0} \ne 0$. Para tal, vamos explicitar alguns poucos coeficientes:
$$\begin{array}{rclclcl}
k &=& 1 \Rightarrow a_{1} &=& \dfrac{a}{c} a_{0} & & \\
k &=& 2 \Rightarrow a_{2} &=& \dfrac{1+a}{2(1+c)} \dfrac{a}{c} a_{0} &=& \dfrac{a(a+1)}{2c(c+1)} a_{0} \\
k &=& 3 \Rightarrow a_{3} &=& \dfrac{2+a}{3(2+c)} a_{2} &=& \dfrac{a(a+1)(a+2)}{3!c(c+1)(c+2)} a_{0} \\
&\vdots& && &&
\end{array}$$
de onde escrevemos, para o termo $a_{k}$ em função de $a_{0}$
$$a_{k} =
\dfrac{\Gamma(a+k)}{\Gamma(a)} \dfrac{1}{\dfrac{\Gamma(k+k)}{\Gamma(c)}} \dfrac{a_{0}}{k!}
=
\dfrac{\Gamma(c)}{\Gamma(a)} \dfrac{\Gamma(a+k)}{\Gamma(c+k)} \dfrac{a_{0}}{k!}.
$$

Note que introduzimos o conceito de \textbf{função gama} que generaliza o fatorial, definido para inteiros positivos. Por enquanto, imagine que estamos trabalhando com inteiros positivos, pois a única diferença é a defasagem de uma unidade. Essa afirmação se deve à relação $\Gamma(n + 1) = n!$.

Voltando com esses coeficientes na expressão para a solução e rearranjando, temos uma solução para a equação hipergeométrica confluente, relativa ao índice $s = 0$ da equação indicial, dada por:
$$\begin{array}{rcl}
y_1(x)
&=& a_{0} \dfrac{\Gamma(c)}{\Gamma(a)} \displaystyle\sum_{k=0}^{\infty} \dfrac{\Gamma(a+k)}{\Gamma(c+k)} \dfrac{x^{k}}{k!} \\
&=& a_{0}\ {}_1F_{1}(a; c; x)
\end{array}$$
a chamada \textbf{função hipergeométrica confluente}.

Dessa expressão, fica claro que a constante $a_{0}$ deve ser diferente de zero, pois se $a_{0} = 0$, temos apenas a \textbf{solução trivial}.

Devemos, agora, considerar a outra raiz da equação indicial, $s_{2} = 1 - c$. 
Procedendo exatamente como na primeira raiz, obtemos a segunda solução
$$\begin{array}{rcl}
y_2(x)
&=& a_{0} \dfrac{\Gamma(2-c)}{\Gamma(a-c+1)} \displaystyle\sum_{k=0}^{\infty} \dfrac{\Gamma(a-c+1+k)}{\Gamma(2-c+k)} \dfrac{x^{k+1-c}}{k!} \\
&=& a_{0}\ x^{1-c}\ {}_1F_{1}(a+1-c; 2-c; x).
\end{array}$$


A pergunta natural agora é: são estas duas soluções linearmente independentes, a fim de que tenhamos uma solução geral? Para responder a esta pergunta, devemos calcular o determinante Wronskiano
$$
W[y_1(x), y_2(x)] =
\begin{array}{|cc|}
y_1 & y_{2} \\ y'_{1} & y'_{2}
\end{array}$$

Basta que consideremos apenas o primeiro termo (o chamado \textbf{termo líder}, relativo a $k = 0$) das séries, a saber:
$$
y_{1} =
\dfrac{\Gamma(c)}{\Gamma(a)} \dfrac{\Gamma(a)}{\Gamma(c)} = 1
$$
e
$$
y_{2}
=
\dfrac{\Gamma(2-c)}{\Gamma(a-c+1)}
\dfrac{\Gamma(a-c+1)}{\Gamma(2-c)}
x^{1-c}
=
x^{1-c}
$$
que, substituídos na expressão para o Wronskiano e já calculando o determinante, fornece
$$
W[y_1(x), y_2(x)] =
\begin{array}{|cc|}
1 & x^{1-c} \\
0 & (1-c) x^{-c}
\end{array}
= (1-c) x^{-c}.
$$

Desta expressão, aparece naturalmente uma imposição a ser feita, pois somente para $c \ne 1$, temos duas soluções linearmente independentes e a solução geral da equação hipergeométrica confluente é dada por:
$$y(x) = C_1\ {}_1F_1(a; c; x) + C_2 x^{1-c}\ {}_1F_1(a+1-c; 2-c; x),$$
em que $C_1$ e $C_2$ são constantes arbitrárias.

Podemos concluir que o método de Frobenius forneceu duas soluções linearmente independentes impondo $c \ne 1$.

No caso em que $c = 1$, devemos procurar uma segunda solução linearmente independente utilizando diretamente a expressão de Frobenius generalizada ou o método de redução de ordem. Este é o caso sempre que as raízes da equação indicial são iguais.
}

\exercicio{}{%Do lar 1.
Explicite os cálculo a fim de obter a segunda solução da equação hipergeométrica confluente.
}

\exercicio{}{%Do lar 2.
Utilize o método de Frobenius para discutir e obter uma solução da equação
$$x\dfrac{d^{2}}{dx^{2}}y(x) + \dfrac{d}{dx}y(x) + xy(x) = 0,$$
a \textbf{chamada equação de Bessel de ordem zero}.
}

\exercicio{}{%Do lar 3.
\cite{makarenko1979problemas} Mostre que a solução da equação diferencial ordinária com coeficientes não constantes
$$2x^2 \dfrac{d^{2}}{dx^{2}} y(x) + (3x-2x^2) \dfrac{d}{dx} y(x) - (x+1) y(x) = 0$$
é dada por
$$y(x) = A y_1(x) + B y_2(x)$$
com $A$ e $B$ constantes reais e as duas soluções $y_1(x)$ e $y_2(x)$ dadas, respectivamente, por:
$$y_1(x) = \sqrt{x} \left[1 + \displaystyle\sum_{k=0}^{\infty} \dfrac{(2x)^{k}}{5 \cdot 7 \cdot 9 \cdots (2k+3)}\right]$$
e
$$y_2(x) = \dfrac{e^x}{x}.$$
}



Após a introdução do método de Frobenius através de uma particular equação de Bessel, vamos considerar o caso geral, pois ela apresenta as possibilidades que o método de Frobenius proporciona para obter uma ou ambas as raízes de uma equação diferencial ordinária de segunda ordem com coeficientes não constantes.

\exemplo{}{%Exemplo 3.
Equação de Bessel de ordem $\mu$.

Seja $\mu \in \mathbb{R}$ um parâmetro. Utilize o método de Frobenius para discutir a equação de Bessel de ordem $\mu$,
\begin{equation}
x^2\dfrac{d^{2}}{dx^{2}} y(x) + x \dfrac{d}{dx} y(x) + (x^2 - \mu^2) y(x) = 0,
\end{equation}
em que o parâmetro é, em princípio, arbitrário.
}

\solexemplo{Uma vez que a equação de Bessel contém um parâmetro arbitrário, justificamos a conveniência desta escolha por, a partir deste parâmetro, termos a possibilidade de englobar, numa só equação diferencial, as possibilidades, advindas do método de Frobenius, de encontrar uma ou duas soluções linearmente independentes.

Ainda que seja um caso específico, considerar uma vizinhança da origem, não perdemos generalidade, pois com uma simples mudança de variável sempre podemos conduzir a análise à origem.

Assim, consideramos uma série de potências da seguinte forma,
$$y(x) = \displaystyle\sum_{n=0}^{\infty} a_{n} x^{n+s},$$
com $a_{0} \ne 0$ e $s \in \mathbb{C}$ um parâmetro.

Calculando a primeira derivada (admitindo que a derivação sob o somatório seja possível), temos:
$$
\dfrac{d}{dx}y(x) = \displaystyle\sum_{n=0}^{\infty}(n+s) a_n x^{n+s-1}
$$
enquanto, a derivada segunda é dada por
$$\dfrac{d^{2}}{dx^{2}} y(x) = \displaystyle\sum_{n=0}^{\infty} (n+s)(n+s-1) a_{n} x^{n+s-2}.$$

Introduzindo essas duas últimas expressões na equação diferencial, obtemos, já simplificando, a igualdade:
$$
\displaystyle\sum_{n=0}^{\infty} [(n+s)^2 - \mu^2] a_n x^{n+s} + \displaystyle\sum_{n=0}^{\infty} a_n x^{n+s-2} = 0.
$$

É imediato notar que existem duas diferenças básicas relativo ao desenvolvimento em série de Taylor, a saber: (a) os índices no somatório da primeira e da segunda derivadas não são alterados e (b) somente se $s = 0$, obtemos exatamente a série de Maclaurin.

Comecemos a análise a partir dos índices. Com uma mudança de índice no segundo somatório, considerando $n \to n - 2$ (note que estamos mantendo a mesma letra para o índice, pois é um índice de soma, também chamado índice mudo), podemos escrever:
$$\displaystyle\sum_{n=0}^{\infty}[(n+s)^2 - \mu^2] a_n x^{n+s} + \displaystyle\sum_{n=2}^{\infty} a_{n-2} x^{n+s} = 0.$$

A fim de que tenhamos os índices inferiores iguais devemos separar os dois primeiros termos no primeiro somatório de modo que possamos rearranjar os demais num único somatório,
$$(s^2 - \mu^2) a_{0} x^s + [(s+1)^2 - \mu^2] a_{1} x^{s+1} + \displaystyle\sum_{n=2}^{\infty} \{[(n+s)^2-\mu^2] a_n + a_{n-2}\} x^{n+s} = 0.$$

Visto que $a_{0} \ne 0$, podemos escrever, utilizando identidade de polinômios, as seguintes igualdades:
\begin{description}
\item (a) $s^2 - \mu^2 = 0$;
\item (b) $[(s+1)^2 - \mu^2] a_{1} = 0$;
\item (c) $[(n + s)^2 - \mu^2] a_n + a_{n-2} = 0, n \ge 2$.
\end{description}

Uma breve análise dessas três igualdades. A equação em (a) é a equação indicial/auxiliar, pois só envolve o parâmetro e não envolve os coeficientes, enquanto a terceira (c), para um particular valor de $s$, é a chamada fórmula/relação de recorrência que, neste caso, relaciona o termo de ordem $n$ com o termo de ordem $n - 2$. A segunda equação, (b), aquela envolvendo o coeficiente $a_{1}$ não é conhecida com nome específico.

Parece natural começar com a análise a partir da primeira das três equações, pois não envolve os coeficientes e nos fornece as raízes diretamente, ou ainda os possíveis valores do parâmetro $s$, até então arbitrário.

Começamos por estudar os possíveis casos, ou seja, a partir da equação indicial, obtemos:
$$s = \pm\mu,$$
suas raízes, também chamados expoentes.

Substituindo $s$, já determinado, na segunda equação, (b), temos $$[(\pm\mu + 1)^2 - \mu^2] a_{1} = 0,$$
ou ainda, resolvendo a equação algébrica resultante, uma equação envolvendo um produto, temos:
$$(1 \pm 2\mu) a_{1} = 0.$$

Desta igualdade, temos duas possibilidades. A saber:
\begin{description}
\item (i) $\mu = \mp \dfrac{1}{2}, \forall\ a_1$
\item (ii) $\mu \ne \pm \dfrac{1}{2}, a_{1} = 0$.
\end{description}

Por fim, substituindo estes resultados na terceira equação (c), fórmula de recorrência, obtemos:
$$[(n \pm \mu)^2 - \mu^2] a_n + a_{n-2} = 0,$$
ou ainda, expressando $a_n$ em função de $a_{n-2}$, na forma
$$a_n = -\dfrac{a_{n-2}}{n(n \pm 2\mu)},$$
com $n \ge 2$.

Da fórmula de recorrência, é evidente que, se o parâmetro $\mu$ é um número inteiro ou semi-inteiro, vamos ter problemas, pois o denominador pode se anular.

Daqui para a frente se faz necessário uma escolha relativa ao parâmetro $\mu$, até então arbitrário. Vamos escolhê-lo, além de uma maneira conveniente, de forma didática, no sentido de considerar as possibilidades, advindas do método de Frobenius. Aqui, vamos estudar, em separado, quatro casos distintos envolvendo o parâmetro $\mu$.


I - Seja $\mu = \dfrac{1}{4}$.

As raízes da equação auxiliar são distintas,
$$s_1 = \dfrac{1}{4} \mbox{ e } s_{2} = -\dfrac{1}{4}.$$

E, visto que $\mu \ne \pm\dfrac{1}{2}$, temos, da segunda equação, que $a_{1} = 0$, e da relação de recorrência, obtemos $a_n = 0$, para todo $n$ ímpar. Diante disso, podemos escrever, a partir da relação de recorrência,
$$a_n = -\dfrac{a_{n-2}}{n \left(n\pm \dfrac{1}{2}\right)}, n = 2, 4, 6, \ldots$$
e, ainda mais, como $n$ é par, na seguinte forma:
$$a_{2n} = -\dfrac{a_{2n-2}}{n(4n \pm 1)},$$
com $n = 1, 2, \ldots$.

Então, como temos duas possibilidades (dois sinais distintos), obtemos duas soluções linearmente independentes da equação diferencial. Uma, associada à raiz $s_1 = 1/4$ e a outra, associada à raiz $s_{2} = -1/4$. Não vamos nos preocupar em expressar o coeficiente $a_{2n}$ em função de $a_{0} \ne 0$, o que será feito mais adiante.

II - Consideramos $\mu = 0$.

A equação indicial admite somente uma raiz (dupla), $s = 0$.

Da segunda equação, temos:
$$\mu \ne \pm\dfrac{1}{2}$$
e, então, $a_{1} = 0$ bem como os demais ímpares.

A relação de recorrência nos fornece
$$a_{2n} = - \dfrac{a_{2n-2}}{4n^2},$$
com $n = 1, 2, \ldots$, de onde obtemos uma só solução da equação diferencial ordinária.

Uma outra solução linearmente independente pode ser procurada através do \textbf{método de redução de ordem}.

III - Seja $\mu = \dfrac{1}{2}$.

As raízes da equação indicial são $s_1 = \dfrac{1}{2}$
e $s_{2} = -\dfrac{1}{2}$.

No particular caso em que $s = \dfrac{1}{2}$, temos que $a_{1} = 0$ , de onde todos os termos de índices ímpares serem nulos e, da relação de recorrência, podemos escrever:
$$a_n = -\dfrac{a_{n-2}}{n(n+1)},$$
com $n = 2, 3, \ldots$ e obtemos uma solução da equação diferencial. 
Por outro lado, no caso em que $s = -\dfrac{1}{2}$, temos $a_{1}$ arbitrário e a fórmula de recorrência é dada por
$$a_n = -\dfrac{a_{n-2}}{n(n - 1)},$$
com $n = 2, 3, \ldots$.

Visto que temos duas constantes arbitrárias $a_{0}$ e $a_{1}$, diferentes de zero, esta raiz (a menor), fornece uma solução geral da equação diferencial ordinária, ou ainda, duas soluções linearmente independentes (Wronskiano diferente de zero) obtidas com o desenvolvimento em série de potências.

\exercicio{}{%Do lar 4.
Mostrar que a solução obtida com a outra raiz, a maior raiz, é um caso particular daquela obtida com a outra raiz, a menor.
}

IV - Consideramos $\mu = 1$.

As raízes da equação indicial são $s_1 = 1$ e $s_{2} = -1$.

No caso em que $s = 1$, concluímos que $a_{1} = a_3 = \cdots = 0$. Logo, da relação de recorrência, obtemos:
$$a_n = - \dfrac{a_{n-2}}{n(n+2)},$$
com $n = 2, 3, \ldots$. Então, temos somente uma solução.


Por outro lado, relativamente ao caso em que $s = -1$, temos, ainda, $a_{1} = a_3 = \cdots = 0$ e, da relação de recorrência, a seguinte expressão:
$$n(n - 2)a_n = -a_{n-2}$$
que, para $n = 2$ implica $a_{0} = 0$, contrariando a hipótese $a_{0} \ne 0$. Tal expoente não fornece uma solução.


Diante dessas possibilidades, façamos um breve resumo do método de Frobenius.

O método fornece \textbf{pelo menos uma} solução da equação diferencial ordinária, escrita por meio de uma série de potências. A outra solução, em princípio, pode ser obtida através do \textbf{método de redução de ordem}. Existem casos em que o método fornece duas soluções linearmente independentes e em outros casos a relação de recorrência nem mesmo é válida! Nos casos em que o método de Frobenius não fornece duas soluções linearmente independentes emerge naturalmente um termo logarítmico, uma vez que a função $\ln(x)$ não pode ser expressa em termos de uma série de Frobenius. Como já mencionamos, para a análise em torno de um ponto singular regular no infinito, basta introduzir, no procedimento descrito, uma mudança de variável independente do tipo $z = 1/\xi$ e estudar a equação diferencial resultante em torno do ponto $\xi = 0$, a origem.
}

%23 abril 21