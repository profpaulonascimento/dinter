
\chapter{Funções Especiais}

Exatamente igual ao estudo das variáveis complexas e as equações diferenciais, as funções especiais também se constituem num extenso ramo da matemática. Nesta disciplina, vamos discutir algumas delas, divididas em duas etapas, a saber: as funções hipergeométricas e seus casos particulares e as funções de Fox que admitem, como caso particular, as chamadas funções de Mittag-Leffler que se constituem nas funções que abordaremos especificamente no cálculo fracionário, especialmente, no estudo das equações diferenciais fracionárias.

\section{A função gama e o símbolo de Pochhammer}

Devido a importância e o natural aparecimento dos conceitos de função gama e o símbolo de Pochhammer, iniciamos com a definição e a discussão de algumas propriedades, através de exemplos resolvidos e/ou exercícios deixados a cargo do estudante.

\subsection{Função gama}

A função gama se constitui numa natural extensão do conceito de fatorial onde, no lugar de um inteiro, temos, em princípio um complexo. Ainda que existam outras maneiras de apresentar a definição da função gama, optamos por aquela que é devida a Euler, em termos de uma integral imprópria.

\definicao{}{}{%Definição 1.
Seja $z \in \mathbb{C}^\ast$. Definimos a função gama por
$$\Gamma(z) = \displaystyle\int_0^\infty t^{z-1} e^{-t} dt,$$
com $Re(z) > 0$. Para $Re(z) < 0$, usamos a relação
$$\Gamma(z) = \dfrac{\Gamma(z+1)}{z},$$
com $z \neq -1, -2, -3, \ldots$.
}


É importante notar dessa relação que estende a definição para os inteiros negativos, pois exatamente nesses pontos se encontram os polos da função gama. Aqui,
vamos apenas considerar os reais, isto é, $x = Re(z)$.


\exemplo{exem:aula03.01}{
Utilize a definição da função gama, para calcular $\Gamma(1)$, $\Gamma(2)$, $\Gamma(3)$ e $\Gamma(4)$ a fim de inferir que $\Gamma(n + 1) = n!$, para $n \in \mathbb{N}$.
}

\solexemplo{
Seja $z = 1$ na definição da função gama, temos:
$$\Gamma(1) = \displaystyle\int_{0}^{\infty} t^{1-1} e^{-t} \ dt$$,
cuja integração é imediata. Logo,
$$\Gamma(1) = 1 = 1!.$$

Considerando $z = 2$, podemos escrever:
$$\Gamma(2) = \displaystyle\int_{0}^{\infty} t^{2-1} e^{-t} dt,$$
que, integrando por partes, fornece
$$\Gamma(2) = 1 = 1!.$$

Analogamente, substituindo $z = 3$, obtemos:
$$\Gamma(3) = \displaystyle\int_{0}^{\infty} t^{3-1} e^{-t} dt,$$
que, novamente, integrando por partes duas vezes, fornece
$$\Gamma(3) = 2 = 2!.$$

Por fim, para $z = 4$, temos a seguinte integral
$$\Gamma(4) = \displaystyle\int_{0}^{\infty} t^{4-1} e^{-t} dt,$$
cuja integração por partes resulta
$$\Gamma(4) = 6 = 3!.$$

Coletando os resultados podemos inferir que
$$\Gamma(n + 1) = n!$$
que é o resultado desejado.
}

\exemplo{exem:aula03.02}{
Mostre que $$\Gamma\left(\dfrac{1}{2}\right) = \sqrt{\pi}.$$
}

\solexemplo{
Como é de se esperar, esses cálculos não são tão imediatos como no caso dos inteiros. Aqui, vamos considerar o produto de duas funções gama de tal modo que devemos
calcular as integrais
$$\Gamma\left(\dfrac{1}{2}\right) \cdot \Gamma\left(\dfrac{1}{2}\right)
= \displaystyle\int_{0}^{\infty} x^{\frac{1}{2}-1} e^{-x} dx \cdot \displaystyle\int_{0}^{\infty} y^{\frac{1}{2}-1} e^{-y} dy
%
$$

Começamos por introduzir a mudança de variável $x \rightarrow x^2$ e $y \rightarrow y^2$, %onde mantivemos a mesma letra, pois são variáveis mudas, logo, já simplificando,
$$\Gamma\left(\dfrac{1}{2}\right) \cdot \Gamma\left(\dfrac{1}{2}\right)
= \displaystyle\int_{0}^{\infty}\int_{0}^{\infty} x^{-1}y^{-1} e^{-(x^2+y^2)} dx dy
= 4 \displaystyle\int_{0}^{\infty}\int_{0}^{\infty} e^{-(x^2+y^2)} dx dy
$$

Para calcular essa integral, vamos utilizar coordenadas polares no plano:
$$x = r \cos(\theta) \mbox{ e } y = r \sin(\theta),$$
de onde segue, $x^2 + y^2 = r^2$ e $dx dy = r dr d\theta$.

Assim, substituindo na expressão para o produto das duas funções gama, podemos escrever, já simplificando,
$$\Gamma\left(\dfrac{1}{2}\right) \cdot \Gamma\left(\dfrac{1}{2}\right) = 4 \displaystyle\int_{0}^{\infty} e^{-r^2} r dr \cdot \int_{0}^{\frac{\pi}{2}} d\theta$$
cujas integrações fornecem
$$\Gamma\left(\dfrac{1}{2}\right) \cdot \Gamma\left(\dfrac{1}{2}\right) = 4 \cdot \dfrac{1}{2} \cdot \dfrac{\pi}{2}$$
de onde segue
$$
\Gamma\left(\dfrac{1}{2}\right) = 4 \cdot \dfrac{1}{2} \cdot \pi = \pi.$$
que é o resultado desejado.
}


\exercicio{}{%Do lar 1.
Calcule $\Gamma\left(-\dfrac{1}{2}\right)$ e $\Gamma\left(-\dfrac{3}{2}\right)$
.
}



\exemplo{exem:aula03.0}{%3.
Seja $z \in \mathbb{C}$. Mostre a relação $z\Gamma(z) = \Gamma(z + 1)$, com $z \neq 0, -1, -2, \ldots$
}

\solexemplo{
A função gama para o argumento z + 1 é
$$\Gamma(z + 1) = \displaystyle\int_{0}^{\infty} t^z e^{-t} dt.$$

Vamos utilizar integração por partes. Consideramos
$$t^z = u \rightarrow zt^{z-1} dt = du$$
bem como
$$e^{-t}dt = dv \rightarrow v = -e^{-t},$$
de onde podemos escrever, através da expressão para integração por partes
$$\Gamma(z + 1) = -t^ze^{-t}\Bigg|^{t=\infty}_{t=0} - \int^\infty_0 (-e^{-t}) z t^{z-1} dt$$
ou ainda, na seguinte forma, já simplificando
$$\Gamma(z + 1) = z \int^\infty^0 t^{z-1} e^{-t} dt$$
que, a partir da definição da função gama, fornece
$$\Gamma(z + 1) = z \Gamma(z)$$
que é o resultado desejado.
}

\exemplo{exem:aula03.0}{%4.
Seja $\alpha > 0$. Calcule a seguinte integral
$$\Omega = \int^\infty_0 \dfrac{\alpha^x}{x^\alpha} dx.$$
}

\solexemplo{
Começamos escrevendo a integral na forma
$$\Omega = \int^\infty_0 \alpha^x x^{-\alpha} dx = \int^\infty_0 e^{x \ln(\alpha)} x^{-\alpha} dx.$$

Introduzindo a mudança de variável $x \ln(\alpha) = -t$, temos, para $x = 0$ implica $t = \infty$, bem como $\alpha < 1$, de onde segue, já simplificando
$$\Omega =
\left[\dfrac{-1}{\ln(\alpha)}\right]^{-\alpha+1} \int^\infty_0 e^{-t} t^{-\alpha} dt$$
ou ainda, utilizando a definição de função gama, na forma
$$\Omega = [-\ln(\alpha)]^{1-\alpha} \Gamma(1-\alpha).$$

Logo, a integral de partida é tal que
$$\int^\infty_0 \dfrac{\alpha^x}{x^\alpha} dx = \dfrac{\Gamma(1-\alpha)}{[-\ln(\alpha)]^{\alpha-1}},$$
com a condição $0 < \alpha < 1$.
}

É importante notar que a condição $0 < \alpha < 1$ emerge naturalmente na resolução do exemplo, uma vez que foi dado no enunciado $\alpha > 0$.

\exercicio{}{%Do lar 2.
Seja $\alpha > 0$. Mostre que
$$\Lambda = \int^\infty_0 \dfrac{t^\alpha}{\alpha^t} dt = \dfrac{\Gamma(\alpha + 1)}{[\ln(\alpha)]^{\alpha+1}}\cdot $$
Discuta o caso em que $\alpha = 1$.
}

\exercicio{}{%Do lar 3.
Seja $t > 0$. Mostre o seguinte resultado
$$\int^\infty_0 e^{-x^t} dx = \Gamma\left(\dfrac{t + 1}{t}\right)\cdot $$
}

\subsection{Símbolo de Pochhammer}

Assim como a função gama, o símbolo de Pochhammer é bastante útil na simplificação de expressões que envolvem as funções gama e o fatorial.

\definicao{}{}{%Definição 2.
Seja $a \in \mathbb{C}$. Definimos o símbolo de Pochhammer, denotado por $(a)_n$, através da expressão
$$(a)_n =
\left\{\begin{array}{rcl}
1&,& \mbox{ se } n = 0 \\
a(a + 1)(a + 2) \ldots (a + n - 1) &,& \mbox{ se } n = 1, 2, 3, \ldots
\end{array}\right.$$
}

\exercicio{}{%Do lar 4.
Seja $n \in \mathbb{N}$. Mostre que $(1)_n = n!$ e, portanto, o símbolo de Pochhammer, pode ser visto como uma generalização do conceito de fatorial.
}

\exercicio{}{%Do lar 5.
Sejam $n \in \mathbb{N}$ e $a \neq 0, -1, -2, \ldots$. Mostre que vale a relação envolvendo o quociente de duas funções gama, com as devidas condições de existência,
$$(a)_n = \dfrac{\Gamma(a + n)}{\Gamma(a)}.$$
}

\exemplo{exem:aula03.05}{%5.
Vamos mostrar a relação entre o símbolo de Pochhammer e o coeficiente binomial.
}


\solexemplo{Como é sabido, o coeficiente binomial emerge naturalmente na expansão numa série de Taylor ou Maclaurin.

Sejam $n, m \in \mathbb{N}$, com $m \ge n$. Temos,
$$\binom{m}{n} = \dfrac{m!}{n!(m - n)!}$$
que, em termos da função gama, pode ser escrito como
$$\binom{m}{n} = \dfrac{\Gamma(m + 1)}{n!\Gamma(m - n + 1)}$$
ou ainda, para $a \neq 0, -1, -2, \ldots$, na seguinte forma
$$\binom{a}{n} = \dfrac{\Gamma(a + 1)}{n!\Gamma(a - n + 1)} = \dfrac{a \Gamma(a)}{n!(a - n)\Gamma(a - n)}.$$

Enfim, podemos escrever
$$\begin{array}{rcl}
\binom{a}{n}
&=& \dfrac{a(a - 1) \ldots (a - n + 1)}{n!(a - n)(a - n - 1) \Gamma(a - n - 1)} \\
&=& \dfrac{1}{n!a(a - 1) \ldots (a - n + 1)}.
\end{array}$$

A expressão anterior pode ser escrita na forma, fatorando
$(-1)$ em todos os fatores
$$\binom{a}{n}
= \dfrac{1}{n!(-1)^n(-a)(-a + 1) \ldots (-a + n - 1)}$$
e, utilizando a definição do símbolo de Pochhammer, obtemos a relação com o coeficiente binomial, dado por
$$\binom{a}{n}
= \dfrac{1}{n!(-1)^n(-a)_{n}}$$
que é o resultado desejado.
}

\exercicio{}{%Do lar 6.
Utilize o resultado do Exemplo 5 para mostrar que vale a igualdade
$$\dfrac{\Gamma(a + 1)}{\Gamma(a - n + 1)}
= (-1)^n (-a)_n.$$
}

\exercicio{}{%Do lar 7.
Considere o parâmetro $a = \alpha - 1$ com
$\alpha \neq 0, \pm 1, \pm 2, \ldots$ no resultado do Do Lar 6 a fim de mostrar a igualdade
$$\dfrac{\Gamma(\alpha - n)}{\Gamma(\alpha)}
=
\dfrac{(-1)^n}{(1 - \alpha)_n}.$$
}

\exercicio{}{%Do lar 8.
Utilize a definição do símbolo de Pochhammer e o resultado do Do Lar 7 a fim de mostrar o seguinte resultado
\begin{equation}\label{eq:aula03.dolar08}
(a)_{-n} = (-1)^n (1 - a)_n,
\end{equation}
para $n = 1, 2, 3, \ldots$ e $a \neq 0, \pm 1, \pm 2, \ldots$.
}


\exercicio{}{%Do lar 9.
Sejam $a \neq 0, \pm 1, \pm 2, \ldots$ e $m, n \in \mathbb{N}$. Mostre que vale o seguinte resultado
\begin{equation}\label{eq:aula03.dolar09}
(a)_{m+n} = (a)_m (a + m)_n.
\end{equation}
}

\exemplo{exem:aula03.06}{%6.
Sejam $k, n \in \mathbb{N}$ tal que $0 \leq k \leq m$ e $a \neq 0, \pm 1, \pm 2, \ldots$. Mostre a seguinte relação
\begin{equation}
(a)_{m-k} = (-1)^k \dfrac{(a)_m}{(1 - a - m)_k}.
\end{equation}
}

\solexemplo{Tomando $n \rightarrow -k$ na \auteref{eq:aula03.dolar09}, obtemos
$$(a)_{m-k} = (a)_m(a + m)_{-k}.$$

Utilizando a \autoref{eq:aula03.dolar09}, podemos escrever
$$(a)_{m-k} = (-1)^k \dfrac{(a)_m}{(1 - a - m)_k}.$$
que é o resultado desejado.
}

\exemplo{exem:aula03.07}{%7.
Sejam n, k \in \mathbb{N}. Mostre que
$$(-n)_k =
\left\{\begin{array}{rcl}
\dfrac{(-1)^k n!}{(n - k)!} &,& \mbox{ se } 0 \leq k \leq n \\
0 &,& \mbox{ se } k > n
\end{array}\right.$$
}

\solexemplo{Considerando $a = 1$ na \autoref{eq:aula03.dolar09}, podemos escrever
$$(-n)_k = (-1)^k \dfrac{(1)_n}{(1)_{n-k}}.$$

Utilizando a definição do símbolo de Pochhammer, obtemos a dupla igualdade
$$(-n)_k
= (-1)^k \dfrac{\Gamma(1 + n)}{\Gamma(1 + n - k)}
= (-1)^k \dfrac{n!}{(n - k)!}$$
válida para $0 \leq k \leq n$ e, para $k > n$ é igual a zero, que
é o resultado desejado.
}


\exercicio{}{%Do lar 10.
Seja $z \neq 0,-\dfrac{1}{2},-1,-\dfrac{3}{2},-2, \ldots$. Mostre a fórmula de duplicação de Legendre
$$\sqrt{\pi}\Gamma(2z) = 2^{2z-1}\Gamma(z)\Gamma\left(z + \dfrac{1}{2}\right).$$
}


\exercicio{}{%Do lar 11.
Sejam $x \in \mathbb{R}$, com $0 < x < 1$. Mostre que
$$\Gamma(x)\Gamma(1 - x) = \dfrac{\pi}{\sin(\pi x)}$$.
}


\section{Função beta}

Uma outra função, agora com dois parâmetros, que contribui para, dentre outras, na simplificação de expressões é a função beta, também conhecida como função de Euler de primeira espécie, denotada por $\mathcal{B}(p, q)$.


\definicao{}{}{%Definição 3.
Sejam $p > 0$ e $q > 0$. Definimos a função beta através da integral $\mathcal{B}(p, q) = \int_0^1 x^{p-1} (1-x)^{q-1} dx$.
}

Ainda que a definição valha para $p \in \mathbb{C}$ e $q \in \mathbb{C}$, com $\operatorname{Re}(p) > 1$ e $\operatorname{Re}(q) > 1$, aqui, vamos nos ocupar
apenas com os reais.

Visto que os extremos da integral são zero e um, uma outra representação integral para a função beta é dada em termos das funções trigonométricas.

Introduzimos a mudança de variável $x = \sin^2(t)$ na expressão que define a função beta, podemos escrever
$$\mathcal{B}(p, q) = \int_{0}^{\frac{\pi}{2}} [\sin^2(t)^{p-1}[1 - \sin^2(t)]^{q-1} 2 \sin(t) \cos(t) dt$$
que, utilizando a relação fundamental da trigonometria e
rearranjando, permite escrever
$$\mathcal{B}(p, q) = 2 \int_{0}^{\frac{\pi}{2}} \sin^{2p-1}(t) \cos^{2q-1}(t) dt$$
que é uma outra representação para a função beta.

\exercicio{}{%Do lar 12.
Sejam p > 0 e q > 0. Mostre que vale a relação de simetria, B(p, q) = B(q, p).
}

\exercicio{}{%Do lar 13.
Sejam p > 0 e q > 0. Mostre que vale a relação entre a função beta e a função gama
B(p, q) =
\Gamma(p)\Gamma(q)
\Gamma(p + q)
.
}


\exemplo{exem:aula03.08}{%8.
Mostre que $\int_0^{\frac{\pi}{2}} \sqrt{\tan(x)} dx = \dfrac{\sqrt{2}}{2} \pi.$
}

\solexemplo{Começamos escrevendo a tangente em termos de senos e cossenos e rearranjando, obtemos:
$$\int_0^{\frac{\pi}{2}} \sqrt{\tan(x)} dx =
\int_0^{\frac{\pi}{2}} \sin^{\frac{1}{2}}(x) \cos^{\frac{1}{2}}(x) dx$$
com a representação da função beta em termos das funções trigonométricas, podemos escrever para os parâmetros
$$2p - 1 = \dfrac{1}{2}, p = \dfrac{3}{4}
\mbox{ e }
2q - 1 = - \dfrac{1}{2}, q = \dfrac{1}{4}$$
de onde segue para a integral
$$\int_0^{\frac{\pi}{2}} \sqrt{\tan(x)} dx = \dfrac{1}{2} \mathcal{B} \left(\dfrac{3}{4}, \dfrac{1}{4}\right).$$

Utilizando a relação entre as funções gama e beta, temos
$$\int_0^{\frac{\pi}{2}} \sqrt{\tan(x)} dx
=
\dfrac{1}{2}
\Gamma\left(\dfrac{3}{4}\right)
\Gamma\left(\dfrac{1}{4}\right)
$$
uma vez que $\Gamma(1) = 1$. Utilizando o resultado do exercício Do Lar 11, já simplificando, obtemos
$$\int_0^{\frac{\pi}{2}} \sqrt{\tan(x)} dx
= \dfrac{1}{2} \dfrac{\pi}{\sin(\pi/4)}
= \dfrac{\sqrt{2}}{2} \pi
$$
que é o resultado desejado.
}


\exemplo{exem:aula03.09}{%9.
Expansão de Mittag-Leffler. Seja $z \in \mathbb{C}$, com $\operatorname{Re}(z) > 0$. Mostre que
$$\Gamma(z) = \sum_{k=0}^\infty \dfrac{(-1)^k}{k!(z + k)}
+ \int_1^\infty e^{-t}t^{z-1} dt.$$
}

\solexemplo{A partir da definição da função gama, podemos separar em duas parcelas, intervalos distintos,
$$\Gamma(z) = \int_0^1 e^{-t} t^{z-1} dt +
\int_1^\infty e^{-t} t^{z-1} dt$$
onde, a segunda parcela é uma \textbf{função inteira}, analítica em todo o plano complexo.

Vamos considerar apenas a integral no intervalo finito, primeira parcela. Assim, a partir do desenvolvimento em série de Maclaurin para a exponencial, obtemos
$$\Omega(z) = \int_0^1 e^{-t} t^{z-1} dt
= \int_0^1 \sum_{k=0}^\infty \dfrac{(-t)^k}{k!} t^{z-1} dt.
$$
Permutando os sinais de integral e soma (série é uniformemente
convergente) e rearranjando, temos
$$\Omega(z) = \sum_{k=0}^\infty \dfrac{(-1)^k}{k!} t^{k+z-1} dt,$$
cuja integração na variável $t$ é imediata, logo
$$\Omega(z) = \sum_{k=0}^{\infty} \dfrac{(-1)^k}{k!(z + k)}.$$
Voltando com essa expressão na soma com as duas parcelas, podemos escrever
$$\Gamma(z) = \sum_{k=0}^{\infty}
\dfrac{(-1)^k}{k!(z + k)} +
\int_1^\infty e^{-t} t^{z-1} dt,$$
que é o resultado desejado, a chamada expansão de Mittag-Leffler para a função $\Gamma(z)$.
}


\exemplo{exem:aula03.10}{%10.
Mostre que o resíduo da função gama nos polos $z = -n$, com $n = 0, 1, 2, \ldots$, é igual a $(-1)^n/n!$.
}

\solexemplo{Lembremos que para obter o resíduo, relativo a um polo, devemos calcular o seguinte limite
$$\operatorname{Res}_{z=-k}\Gamma(z) = \lim_{z\to -k} (z + k)\Gamma(z).$$

Assim, introduzindo a expansão da função gama, conforme a expansão de Mittag-Leffler, podemos escrever
$$\operatorname{Res}_{z=-k} \Gamma(z) = \lim_{z\to -k}
\left[(z + k) \sum_{k=0}^\infty \dfrac{(-1)^k}{k!(z + k)} +
(z + k)
\int_1^\infty e^{-t} t^{z-1} dt\right].$$

Visto que o limite da soma é a soma dos limites, que a segunda parcela é uma função inteira e usando a regra de l'Hôpital para calcular o limite na primeira parcela, já simplificando, obtemos
$$\operatorname{Res}\limits_{z=-k} \Gamma(z)
= \sum_{k=0}^\infty \dfrac{(-1)^k}{k!}$$
de onde, para um $k = -n$ fixo, o resíduo é dado por
$$\operatorname{Res}\limits_{z=-n} \Gamma(z)
= \dfrac{(-1)^n}{n!}$$
que é o resultado desejado.
}


\exercicio{}{%Do lar 14.
Seja $0 \leq x \leq 1$. Mostre que
$$\Gamma\left(x + \dfrac{1}{2}\right)
\Gamma\left(x - \dfrac{1}{2}\right)
= \dfrac{\pi}{\cos(\pi x)}.$$
}



\section{Função hipergeométrica}

A importância da função hipergeométrica ou função hipergeométrica de Gauss, introduzida por Gauss em 1812, contém várias funções especiais como caso particular. é importante destacar que a função hipergeométrica é a função mais geral, contendo três parâmetros, que é solução de uma equação diferencial ordinária com três pontos singulares regulares, incluindo um no infinito.

\definicao{}{def:aula03.04}{Sejam $a, b, c \in \mathbb{C}$, com $c \neq 0, -1, -2, \ldots$. A função hipergeométrica, denotada por ${}_2F_1(a, b; c; z)$, é dada pela série de potências
$${}_2F_1(a, b; c; z) =
\sum_{k=0}^{\infty} \dfrac{(a)_k (b)_k z^k}{(c)_k k!},$$
onde $(\cdot)_k$ é o símbolo de Pochhammer.
}


Note que estamos considerando a expansão em torno de $z_0 = 0$, caso contrário, no caso de $z_0 \neq 0$, uma translação deve ser considerada, bem como no infinito uma inversão deve ser considerada. O nome série hipergeométrica está diretamente relacionado com o caso particular em que $a = c$ e $b = 1$ ou $b = c$ e $a = 1$, pois a série hipergeométrica, para $|z| < 1$, se reduz à clássica série geométrica. Pelo teste da razão a série hipergeométrica, para $|z| < 1$, converge no círculo de raio unitário.


\exercicio{}{%Do lar 15.
Utilize o resultado do Exemplo 7 a fim de mostrar que se o parâmetro $a$ ou $b$ (ou ambos) é nulo ou inteiro negativo, a série termina, isto é, temos um polinômio.
}

\exercicio{}{%Do lar 16.
Seja $|z| = 1$, circunferência de raio unitário, a série hipergeométrica é: (i.) absolutamente convergente, se $\operatorname{Re}(c - a - b) > 0$; (ii.) condicionalmente
convergente, se $-1 < \operatorname{Re}(c - a - b) \leq 0$ e (iii) divergente, se $\operatorname{Re}(c - a - b) \leq -1$.
}

\exercicio{}{%Do lar 17.
Se $\operatorname{Re}(c - a - b) > 0$ e $c \neq 0,-1,-2, \ldots$, então vale a relação
$${}_2F_1(a, b; c; 1) = \dfrac{\Gamma(c) \Gamma(c-a-b)}{\Gamma(c-a)\Gamma(c-b)}.$$
}

\exercicio{}{%Do lar 18.
Se $a = -n$, com $n \in \mathbb{N}$, mostre que
$${}_2F_1(-n, b; c; 1) = \dfrac{(c-b)_n}{(c)_n}.$$
}


\exercicio{}{%Do lar 19.
Sejam $a, b \in \mathbb{C}$ e $n \ge 0$. Mostre que
$$\sum_{k=0}^n \binom{a}{k} \binom{b}{n - k}
= \binom{a + b}{n}$$
conhecida como relação de Vandermonde.
}



\subsection{Representação integral}

Nesta seção, vamos introduzir uma representação integral para a função hipergeométrica através de uma integral e da definição da função beta.

Sejam $a, b, c \in \mathbb{R}$, com $\operatorname{Re}(c) > \operatorname{Re}(b) > 0$. Admitindo $|z| < 1$, considere a integral
$$J = \int_0^1 t^{b-1} (1 - t)^{c-b-1} (1 - zt)^{-a} dt.$$
Com tais restrições, no intervalo $0 \leq |z| \leq 1$, podemos expandir $(1-zt)^{-a}$ em uma série uniformemente convergente que substituída na expressão para $J$, fornece
$$J = \int_0^1 t^{b-1} (1 - t)^{c-b-1}
\sum_{n=0}^{\infty} \dfrac{\Gamma(a + n)(zt)^n}{\Gamma(a) n!} dt,$$
ou ainda, permutando os símbolos de integral e série e rearranjando, na seguinte forma
$$J = \dfrac{1}{\Gamma(a)} \sum_{n=0}^{\infty} \dfrac{\Gamma(a + n)}{n!} z^n
\int_0^1 t^{b+n-1} (1 - t)^{c-b-1} dt.$$

A integral remanescente pode ser expressa como uma função beta, logo
$$J =
\dfrac{1}{\Gamma(a)} \sum_{n=0}^{\infty} \dfrac{\Gamma(a + n)}{n!} z^n \mathcal{B}(b + n, c - b),$$
ou ainda, utilizando a relação entre a função beta e a função gama, na seguinte forma, já rearranjando,
$$J =
\dfrac{1}{\Gamma(a)}
\sum_{n=0}^{\infty}
\dfrac{\Gamma(a + n)\Gamma(b + n)\Gamma(c - b)}{n! \Gamma(c + n)} z^n.$$
Utilizando a definição do símbolo de Pochhammer, temos
$$J = \dfrac{\Gamma(c - b)\Gamma(b)}{\Gamma(c)} \sum_{n=0}^{\infty} \dfrac{(a)_n(b)_n}{(c)_n} \dfrac{z^n}{n!}.
$$


O somatório remanescente nada mais é que a definição da função hipergeométrica, logo
$$J = \dfrac{\Gamma(c - b)\Gamma(b)}{\Gamma(c)} {}_2F_1(a, b; c; z).$$

Assim, de posse desta expressão podemos escrever, isolando a função hipergeométrica
$${}_2F_1(a, b; c; z) = \dfrac{\Gamma(c)}{\Gamma(c - b) \Gamma(b)} \int_0^1 t^{b-1} (1 - t)^{c-b-1} (1 - zt)^{-a} dt$$
com $\operatorname{Re}(c) > \operatorname{Re}(b) > 0$, que é uma representação integral para a função hipergeométrica.

Concluímos esta seção apenas mencionando o que atende pelo nome de relação contígua que, em muitos cálculos permite simplificar as expressões. Tais relações estão associadas aos parâmetros, a, b e c, adicionado ou subtraído da unidade, isto é,
$${}_2F_1(a \pm 1, b; c; x), {}_2F_1(a, b \pm 1; c; x), {}_2F_1(a, b; c \pm 1; x)$$
ou uma combinação deles.

\exercicio{}{%Do lar 20.
Admita que os parâmetros estejam bem definidos para verificar a relação contígua
$$\dfrac{(a - c)}{c} z {}_2F_1(a, b; c + 1; z) =
(1-z) {}_2F_1(a, b; c; z) - {}_2F_1(a, b - 1; c; z).$$
}

É bastante natural querer \textbf{generalizar} todo resultado, com as funções especiais não é diferente. Existem várias maneiras de apresentar uma generalização, por exemplo, aumentando o número de \textbf{variáveis independentes}; aumentando o número de \textbf{parâmetros} e, também, aumentando ambos, o número de \textbf{variáveis independentes e o número de parâmetros}. Aqui, vamos apenas generalizar aumentando o \textbf{número de parâmetros}, mantendo apenas \textbf{uma} variável independente. Justifica-se esta generalização, pois um pouco mais para a frente, vamos estudar as chamadas funções de Meijer, ou função $\mathbf{G}$ de Meijer e a função de Fox, ou função $\mathbf{H}$ de Fox.



\definicao{}{}{Sejam $p, q \in \mathbb{N}$, $\alpha_i$, com $i = 1, \ldots, p$ e $\beta_j \neq 0, -1, -2, \ldots$, com $j = 1, \ldots, q$. Definimos a série hipergeométrica generalizada através da série
$${}_pF_q
\left[\begin{array}{rl}
\alpha_1, \ldots, \alpha_p & \\
 & z \\
\beta_1, \ldots, \beta_q &
\end{array}\right]
= \sum_{k=0}^\infty \dfrac{(\alpha_1)_k \ldots (\alpha_p)_k z^k}{(\beta_1)_k \ldots (\beta_q)_k} \dfrac{z^k}{k!}
= {}_pF_q(\alpha_1, \ldots, \alpha_p; \beta_1, \ldots, \beta_q; z)$$
}

Admita que nenhum dos parâmetros do numerador é zero ou um inteiro negativo, caso contrário o problema da convergência deveria excluir os termos que tornassem zero o numerador, a série hipergeométrica generalizada é:

\begin{description}
\item (i) convergente para $|z| < \infty$, se $p \leq q$
\item (ii) convergente para $|z| < 1$, se $p = q + 1$
\item (iii) divergente para todo $z$, $z \neq 0$, se $p > q + 1$.
\end{description}

Um caso especial é a chamada série hipergeométrica de Kummer, ou série hipergeométrica confluente, denotada por ${}_1F_1(a; c; z)$, obtida no caso em que $p = q = 1$.

\definicao{}{}{Sejam $|z| < \infty$ e $p = q = 1$. Chama-se série hipergeométrica de Kummer a expressão
$${}_1F_1(a; c; z) = \sum_{k=0}^{\infty}
\dfrac{(a)_k z^k}{(c)_k k!}.$$
}

\exemplo{exem:aula03.11}{%11.
Introduza $z \rightarrow \dfrac{z}{b}$, com $b \neq 0$ na série hipergeométrica e tome o limite $b \rightarrow \infty$, a fim de obter
$$\lim_{b\to \infty} {}_2F_1 \left(a, b; c;\dfrac{z}{b}\right)
= \lim_{b\to\infty} \sum_{k=0}^{\infty} \dfrac{(a)_k(b)_k}{(c)_k k!}
\binom{z}{b}^k.$$
}

\solexemplo{Da Definição 2, podemos mostrar que
$$\lim_{b\to\infty} \left[(b)_k \left(\dfrac{z}{b}\right)^k\right]
= z^k$$

Logo, para $z$ limitado, obtemos:
$$\lim_{b\to\infty} {}_2F_1 \left(a, b; c;\dfrac{z}{b}\right)
= \sum_{k=0}^{\infty} \dfrac{(a)_k z^k}{(c)_k k!}
= {}_1F_1(a; c; z)$$
a chamada série hipergeométrica confluente.
}


\exercicio{}{%Do lar 21.
Seja $z \in \mathbb{C}$. Mostre que:

(a) $\dfrac{\ln(1 + z)}{z} = {}_2F_1(1, 1; 2;-z)$;

(b) $\cos(z) = {}_0F_1\left(-; \dfrac{1}{2};- \dfrac{z^{2}}{4}\right)$

(c) $e^z = {}_1F_1(a; a; z).$
}

\exercicio{}{%Do lar 22.
Utilize a representação integral para a função hipergeométrica para obter uma representação integral para a função hipergeométrica confluente, respeitadas
as condições de existência.
}

\section{Função hipergeométrica e casos particulares}

Como vamos ver adiante, a função hipergeométrica confluente, uma função envolvendo dois parâmetros, está relacionada com uma particular função de Mittag-Leffler. Visto que a função hipergeométrica confluente é obtida, com um conveniente processo de limite, a partir da função hipergeométrica, uma função envolvendo três parâmetros, vamos destacar alguns casos particulares, talvez os mais relevantes devido, por exemplo, à \textbf{geometria de um particular problema físico}. Ainda mais, uma vez que a função hipergeométrica é solução de uma equação diferencial ordinária, contendo três pontos singulares, incluindo o infinito, como já mencionamos, vamos considerar apenas a \textbf{expansão em termos de um só ponto}, como introduzimos a função hipergeométrica através de uma expansão em série na vizinhança do ponto $z_0 = 0$. Por fim, o destaque fica para as soluções polinomiais.

A partir da \ref{def:aula03.04}, vamos considerar casos particulares dos parâmetros $a$, $b$ e $c$, elencando, como já mencionado, aqueles mais relevantes. Sejam os parâmetros tais que $a = -n$, com $n = 0, 1, 2, \ldots, b = \alpha +\beta+n+1$ e $c = \alpha +1$, tais que $\alpha > -1$ e $\beta > -1$. Assim, temos os chamados \textbf{polinômios de Jacobi}, denotados por $P_{n}^{(\alpha ,\beta)}(x)$ e dados pela relação
$$P_{n}^{(\alpha ,\beta)}(x) = \dfrac{\Gamma(\alpha + n + 1)}{\Gamma(\alpha + 1) n!}{}_2F_1\left(-n, \alpha + \beta + n + 1; \alpha + 1;\dfrac{1 - x}{2}\right).$$

A partir desse resultado, consideremos, agora, os parâmetros $\alpha = \beta = \lambda -1/2$, com $\lambda > -1/2$. Com isso, obtemos os \textbf{polinômios de Gegenbauer}, denotados por $C^\lambda_n(x)$ e dados, para $\lambda \neq 0$, pela expressão
$$C^\lambda_n(x) = \dfrac{\Gamma(2\lambda + n)}{\Gamma(2\lambda ) n!} {}_2F_1\left(-n, n + 2\lambda ; \lambda + 1/2;\dfrac{1 - x}{2}\right).$$

No caso particular $\lambda = 0$, devemos usar o limite
\begin{equation}\label{eq:aula03.04}
C^0_n(x) = \lim_{\lambda \rightarrow 0} \dfrac{1}{\lambda} C^\lambda_n(x).
\end{equation}

Para concluir, a partir dos polinômios de Gegenbauer, tomemos $\lambda = 1$, a fim de obter os \textbf{polinômios de Chebyshev de segunda espécie}, denotados por $U_n(x)$ e dados, em termos da função hipergeométrica, pela seguinte expressão
$$U_n(x) = (n + 1) {}_2F_1\left(-n, n + 2; 3/2;\dfrac{1 - x}{2}\right).$$

Ainda mais, no caso em que $\lambda = 0$, obtemos os \textbf{polinômios de Chebyshev de primeira espécie}, denotados por $T_n(x)$ e dados, em termos da função hipergeométrica, por:
$$Tn(x) = {}_2F_1\left(-n, n + 1; 1;1 - x2\right).$$


Aqui, neste caso, levamos em conta o limite \autoref{eq:aula03.04}.

Por fim, para $\lambda = 1/2$, obtemos os clássicos polinômios de Legendre, denotados por $P_n(x)$ e dados, em termos da função hipergeométrica, por
$$P_n(x) = {}_2F_1\left(-n, n; 1/2;\dfrac{1 - x}{2}\right).$$

É importante notar que todos esses polinômios são soluções de uma equação diferencial ordinária de segunda ordem e, portanto, admitem uma outra solução linearmente independente, porém essa solução \textbf{não será um polinômio} e, sim, uma respectiva função.

\exemplo{exem:aula03.12}{%12.
Seja $z \in \mathbb{C}$. A forma mais geral de uma equação diferencial ordinária linear homogênea e de segunda ordem pode ser escrita na forma
$$\dfrac{d^2}{dz^2} w(z) + P(z) \dfrac{d}{dz} w(z) + Q(z) w(z) = 0,$$
onde $P(z)$ e $Q(z)$ são funções analíticas.

Note que poderíamos ter começado com uma equação diferencial contendo um coeficiente no termo de derivada segunda. Bastando, então, dividir por esse coeficiente, desde que diferente de zero, e obtemos a equação diferencial conforme mencionada. Ainda mais, nos pontos em que o coeficiente da derivada segunda é nulo, tais pontos devem ser tratados como singulares. Diante disso, vamos obter a equação hipergeométrica.
}

\solexemplo{
Com a imposição de termos três pontos singulares regulares, incluindo um ponto no infinito, somos levados ao que é conhecido como equação diferencial de Riemann-Papperitz, dada por
{\small
$$\begin{array}{rcl}
& w'' \\
+
\left[
\dfrac{1 - \alpha - \alpha'}{z - z_{1}}
+
\dfrac{1 - \beta - \beta'}{z - z_{2}}
+
\dfrac{1 - \gamma - \gamma'}{z - z_{3}}
\right]
& w' \\
+
\left[
\dfrac{\alpha \alpha'(z_{1} - z_{2})(z_{1} - z_{3})}{z - z_{1}}
+
\dfrac{\beta\beta'(z_{2} - z_{3})(z_{2} - z_{1})}{z - z_{2}}
+
\dfrac{\gamma\gamma'(z_{3} - z_{1})(z_{3} - z_{2})}{z - z_{3}}
\right]
\dfrac{1}{(z - z_{1})(z - z_{2})(z - z_{3})} & w(z)
= 0,
\end{array}$$}
onde $z_{1}$, $z_{2}$ e $z_{3}$ são os três pontos singulares regulares e $\alpha$, $\alpha'$, $\beta$, $\beta'$, $\gamma$ e $\gamma'$ são os expoentes, raízes das correspondentes equações indiciais, associados a estes três pontos, respectivamente e usamos a notação
$$\dfrac{d^2}{dz^{2}} w(z) = w'' \mbox{ e }\dfrac{d}{dz} w(z) = w'.$$

Visto que admitimos o infinito como ponto singular, emerge a condição envolvendo os expoentes
$$\alpha + \alpha' + \beta + \beta' + \gamma + \gamma' = 1$$
a chamada \textbf{condição de Riemann}. É costume indicar uma solução da equação através do símbolo de Papperitz, que deixa clara a notação envolvendo os expoentes e os pontos ingulares,
$$w(z) = \mathbf{P}
\left\{
\begin{array}{cccc}
z_{1} & z_{2} & z_{3} & \\
\alpha & \beta & \gamma & z \\
\alpha' & \beta' & \gamma' & 
\end{array}\right\}.$$

Essa notação equivale a dizer que $w(z)$ é uma solução da equação de Riemann-Papperitz, também conhecida como equação de Riemann, sendo $\alpha$ e $\alpha'$ os expoentes associados a $z_{1}$; $\beta$ e $\beta'$ os expoentes associados a $z_{2}$ e $\gamma$ e $\gamma'$ os expoentes associados a $z_{3}$. Essa equação goza de duas importantes propriedades envolvendo os
expoentes e os pontos singulares, a saber:

\textbf{P1}. A transformação bilinear 
$$y = \dfrac{ax+b}{cx+d},$$
com $x \neq -d/c$ preserva a forma da equação e altera a posição dos pontos singulares preservando os expoentes.

\textbf{P2}. A transformação da forma
$$w(z) = (z - z_{1})^{-\lambda} (z - z_{2})^{-\mu} (z - z_{3})^{-\nu} u(z),$$
com a condição $\lambda +\mu+\nu = 0$ também preserva a forma da equação, mantém os pontos singulares inalterados, mas aumenta os expoentes das quantidades $\lambda$, $\mu$ e $\nu$, respectivamente.

A partir de uma conveniente combinação destas duas transformações, podemos conduzir os três pontos singulares regulares $z_{1}$, $z_{2}$ e $z_{3}$ nos pontos $0$, $1$ e
$\infty$. Por fim, introduzindo a notação
$$\begin{array}{rcl}
\alpha + \beta + \gamma &=& a \\
\alpha + \beta' + \gamma &=& b \\
1 + \alpha - \alpha' &=& c,
\end{array}$$
obtemos a seguinte equação diferencial
$$z(1-z)\dfrac{d^2}{dz^{2}} w(z) + [c-(a+b+1)z] \dfrac{d}{dz} w(z) - ab w(z) = 0$$
que é a clássica equação hipergeométrica, sendo uma sua solução dada pela função conforme Definição 4. Ressaltamos que, nesta disciplina, vamos trabalhar apenas com $x \in \mathbb{R}$.
}


Como uma importante consequência, podemos afirmar que a solução de qualquer equação diferencial ordinária, linear, homogênea e de segunda ordem, com três pontos singulares regulares, \textbf{pode ser escrita em termos de uma solução da equação hipergeométrica}.

Note que a equação diferencial de Riemann-Papperitz permanece inalterada se trocarmos de lugar $z_{1}$, $z_{2}$ e $z_{3}$ na equação diferencial, ou se permutarmos entre si $\alpha$ com
$\alpha'$, ou $\beta$ com $\beta'$, a equação de Riemann permanece a mesma levando, porém, a uma equação hipergeométrica diferente. A primeira permutação pode ocorrer de $3! = 6$
maneiras diferentes, e a segunda possibilita $2 \times 2 = 4$ combinações de $\alpha's$ e $\beta's$, o que nos fornece um total de $24$ soluções diferentes, conhecidas como soluções de
Kummer. Como uma equação diferencial ordinária de segunda ordem só admite duas soluções linearmente independentes, existe entre essas $24$ soluções uma série de relações, conhecidas com o nome de relações de Kummer, que permitem obter outras soluções a partir de uma solução conhecida.

\exercicio{}{%Do lar 23.
Mostre que a função hipergeométrica, conforme apresentada na Definição 4, é uma solução da equação hipergeométrica.
}

\exercicio{}{%Do lar 24.
Obtenha uma segunda solução linearmente independente da equação hipergeométrica, em torno de $z_{1} = 0$.
}

\exercicio{}{%Do lar 25.
Determine a equação diferencial ordinária sendo uma solução dada por ${}_2F_1(1, b; b; x)$. Obtenha uma segunda solução linearmente independente da equação hipergeométrica.
}

\section{Função confluente e casos particulares}

Vamos agora partir da Definição 6. Note que essa função, dependente de dois parâmetros, também é solução de uma equação diferencial ordinária de segunda ordem com dois pontos singulares. Em analogia, às funções hipergeométricas, vamos considerar dois casos particulares dos parâmetros, mencionando apenas a relação com a função hipergeométrica confluente, a saber: as \textbf{funções de Whittaker}, denotadas por $M_{\nu ,\mu}(x)$ e os \textbf{polinômios de Laguerre}, denotados por $L^{(\alpha)}_{n}(x)$.

Sejam os parâmetros $a = \mu-\nu +1/2$ e $c = 1+2\mu$ na função hipergeométrica confluente, logo, temos:
$$M\nu ,\mu(x) = e^{\frac{-x}{2}} x^{\mu+\frac{1}{2}} {}_1F_1 (\mu - \nu + 1/2; 1 + 2\mu; x),$$
com a restrição $\mu \neq -1/2,-3/2, \ldots$.

Note que esta função, também com dois parâmetros, é apenas uma forma distinta de apresentar uma das soluções da respectiva equação diferencial ordinária, obtida a partir de uma conveniente mudança de variável dependente e os parâmetros como acima definidos.

Por outro lado, a fim de mencionarmos um caso particular envolvendo um parâmetro, os clássicos polinômios de Laguerre, utilizamos o seguinte limite
$$\lim_{\beta \to \infty} P^{(0, \beta)}_n \left(1-\dfrac{2}{\beta} x\right) \simeq L_n(x),$$
onde $$P^{(0,\beta)}_n\left(1 - \dfrac{2}{\beta} x\right)$$
são particulares polinômios de Jacobi.

Ainda mais, podemos escrever os polinômios de Laguerre em termos de uma função hipergeométrica confluente,
$$L_n(x) = \dfrac{1}{n!}{}_1F_1(-n; 1; x),$$
com $n = 0, 1, 2, \ldots$.

Por fim, a partir das funções hipergeométricas confluentes podemos ainda, como casos particulares dos parâmetros, obter outras funções e polinômios, dentre os quais mencionamos: os \textbf{polinômios de Hermite}, as \textbf{funções de Weber} e as \textbf{funções de Bessel}.

Antes de apresentarmos um exemplo específico, ressaltamos que essas funções especiais têm, além de um estudo puramente matemático, em particular estudando \textbf{ortogonalidade e funções geratrizes}, uma vasta classe de aplicações em problemas advindos da física matemática, dentre os quais mencionamos, apenas: o pêndulo simples, oscilador harmônico e o átomo de hidrogênio. Ainda mais, essas funções podem ser generalizadas, de uma maneira natural, admitindo $p$ termos no numerador e $q$ termos no denominador, conforme Definição 5, com notação dada pela expressão
$${}_pF_{q} (a_1, \ldots , a_p; b_1, \ldots, b_q; x) = {}_pF_{q}((a_p); (b_q); x) = \sum_{k=0}^{\infty} \dfrac{(a_1)_k \cdots (a_p)_k}{(b_1)_k \cdots (b_q)_k} \dfrac{x^k}{k!},$$
onde $(a_j)_k$ e $(b_j)_k$ são símbolos de Pochhammer.

\exemplo{exem:aula03.0}{%13.
Obter a forma explícita da equação hipergeométrica confluente. Em analogia ao processo de limite (confluência de duas singularidades) utilizado para introduzirmos a função hipergeométrica confluente, a partir da função hipergeométrica, vamos obter a equação hipergeométrica confluente, também conhecida como equação de Kummer ou mesmo equação de Tricomi, a partir da equação hipergeométrica.
}

\solexemplo{
Sejam $a, b, c \in \mathbb{R}$ satisfazendo as condições conforme Definição 4. Consideremos a equação diferencial hipergeométrica
$$x(1-x) \dfrac{d^2}{dx^2} w(x) + [c-(a+b+1)x]
\dfrac{d}{dx} w(x) - ab w(x) = 0,$$
na qual introduzimos a mudança de variável independente
$x \rightarrow x/b$, logo, já simplificando e rearranjando, obtemos a equação diferencial
$$x\left(1 -\dfrac{x}{b}\right) \dfrac{d^2}{dx^2} w(x) + \left[c -\left(\dfrac{a - 1}{b} + 1\right) x\right] \dfrac{d}{dx} w(x) - a w(x) = 0.$$
Tomando o limite $b \rightarrow \infty$, obtemos:
$$x \dfrac{d^2}{dx^2} w(x) + (c - x) \dfrac{d}{dx} w(x) - aw(x) = 0,$$
chamada \textbf{equação hipergeométrica confluente}.
}

\exercicio{}{%Do lar 26.(Função gama incompleta)
Expresse a \textbf{função gama incompleta}, definida por
$$\Gamma(\mu,z) = \int_{0}^{z} e^{-t} t^{\mu-1} dt,$$
em termos de uma função hipergeométrica confluente.
}


\exercicio{}{%Do lar 27.
Mostre que, satisfeitas as condições de existência, vale a relação
{}_1F_1(a; c; x) = e^x{}_1F_1(c - a; c; x).
}


Ver \cite{capelas2012funcoesespeciais} e \cite{capelas2014introducao}.

Antes de apresentar as funções hipergeométricas generalizadas e seus casos particulares, funções de Wright generalizadas, a clássica função de Wrigth, função de Meijer e função de Fox, é conveniente introduzir o conceito de integral de Mellin-Barnes.


%%%% 07 maio 21